\documentclass[11pt, oneside]{article}
%
%   get various packages
%
\usepackage[margin=1.0in]{geometry}                                     % adjust margins
\geometry{letterpaper}                                                  % ... or a4paper or a5paper or ... 
\usepackage{url}                                                        % need this to use URLs in bibtex
\usepackage{setspace}                                                   % need this for \setstrech{...}
\usepackage{scrextend}                                                  % need this for addmargin
\usepackage[export]{adjustbox}                                          % need this to get frame for includegraphics
\usepackage{bigints}                                                    % bigger integral symbol
%
%   tikz et al
%
\usepackage{tikz}
\usetikzlibrary{calc,patterns,angles,quotes,shapes,math,decorations,
                through,intersections,lindenmayersystems,backgrounds}    
\usepackage{pgfplots}
\usepackage{pgfplots}	
%
%	math stuff
%
\usepackage{amsmath,amsfonts,amssymb,amsthm}
\usepackage{mathtools}
\usepackage{commath}                                                    % get \norm{x}
\usepackage{fixmath}                                                    % get \mathbold
\usepackage{gensymb}                                                    % get \degree
\usepackage{circuitikz}                                                 % draw circuits
\usepackage{mathrsfs}

\usepackage{hyperref}
\usepackage{url}
\usepackage{subcaption}
\usepackage{authblk}
\usepackage{amsmath}
\usepackage{mathtools}
\usepackage{graphicx}
\usepackage[export]{adjustbox}
\usepackage{hyperref}
\usepackage{alltt}
\usepackage{color}
\usepackage[utf8]{inputenc}
\usepackage[english]{babel}
\usepackage{float}
\usepackage{bigints}
\usepackage{braket}
\usepackage{siunitx}
\usepackage{relsize}
\usepackage{multirow}
\usepackage{esvect}
%
%	watermarks
%
% \usepackage{draftwatermark}
% \SetWatermarkText{Draft}
% \SetWatermarkScale{5}
% \SetWatermarkLightness {0.9} 
% \SetWatermarkColor[rgb]{0.7,0,0}
%
%
\theoremstyle{definition}
\newtheorem{theorem}{Theorem}[section]
\newtheorem{definition}{Definition}[section]
\newtheorem{proposition}{Proposition}[section]
\newtheorem{lemma}{Lemma}[section]
\newtheorem{example}{Example}[section]
\newtheorem{remark}{Remark}[section]
%
%
%	so you can do e.g., \begin{bmatrix}[r] (or [c] or [l])
%
%
\makeatletter
\renewcommand*\env@matrix[1][c]{\hskip -\arraycolsep
  \let\@ifnextchar\new@ifnextchar
  \array{*\c@MaxMatrixCols #1}}
\makeatother
%
%
%
\newcommand{\argmax}{\operatornamewithlimits{argmax}}
\newcommand{\argmin}{\operatornamewithlimits{argmin}}
%
%	handy command
%
\newcommand*{\Scale}[2][4]{\scalebox{#1}{$#2$}}%
%
%
%
\title{random tests}
\author{David Meyer \\ dmm613@gmail.com}
\date{Last update: \today}


\begin{document}
\maketitle

\bigskip
\noindent
Bhaskara’s formula


{\huge
\begin{equation*}
\sin \theta^{\degree} \approx \dfrac{4\theta (180 - \theta)}{40500 - \theta(180-\theta)},
\text{ for $0 \leq \theta \leq 180$}
\end{equation*}
}

\bigskip
\noindent
Harshad numbers

{\huge
\begin{equation*}
\dfrac{6174}{6+1+7+4} = \dfrac{6174}{18} = 343
\end{equation*}
}


\bigskip
\noindent
Sine of 1,234,567,890 degrees

{\huge
\begin{equation*}
\sin (1,234,567,890\degree) = 1
\end{equation*}
}

\bigskip
\noindent
This is the probability that if 3 integers are chosen at random, 
no common factor will divide them all

{\huge
\begin{equation*}
\dfrac{1}{\zeta(3)} = \dfrac{1}{\frac{1}{1^3} + \frac{1}{2^3} + \frac{1}{3^3} + \frac{1}{4^3} + \cdots} \approx 83\%
\end{equation*}
}




\bigskip
\noindent
Mega Millions

{\Large
\begin{equation*}
\begin{array}{llll}
{\displaystyle {70 \choose 5}}
&=& \dfrac{70!}{(70-5)! \, 5!} 									
				&\qquad \qquad \mathrel{\#} {\displaystyle {n \choose k}} = \dfrac{n!}{(n-k)!\,k!} \\
[25pt]
&=& \dfrac{70\cdot 69\cdot68\cdot67\cdot66\cdot65!}{65! \, 5!} 	
				&\qquad \qquad \mathrel{\#} 70! = 70\cdot69\cdot68\cdot67\cdot66\cdot65! \\
[25pt]
&=& \dfrac{70\cdot69\cdot68\cdot67\cdot66}{5!} 					
				&\qquad \qquad \mathrel{\#} \text{cancel 65!}\\
[25pt]
&=& \dfrac{1,452,361,680}{120} 									
				&\qquad \qquad \mathrel{\#} 5! = 5 \cdot 4 \cdot 3 \cdot 2 \cdot 1 = 120 \\
[20pt]
&=&  12,103,014													
				&\qquad \qquad \mathrel{\#} {\displaystyle {70 \choose 5} = 12,103,014}
\end{array}
\end{equation*}}


\bigskip
\noindent
Schrödinger equation

{\normalsize
\begin{equation*}
H(t) \mid \psi(t)\rangle = i \hslash \, \dfrac{\partial}{\partial t} \mid \psi(t) \rangle
\end{equation*}
}

\newpage
\noindent
The Spiral of Theodorus

%
%	build a command that draws a picture
%	of the Spiral of Theodorus with parameter
%	n = the number of triangles
%
\newcommand*{\sqrtspiral}[2][scale=3]{
    \begin{tikzpicture}[#1]
        \def\sqrtlast{#2}
        \coordinate (A) at (0,0);								% side adjacent
        \coordinate (B) at (1cm,0);								% side opposite
        \draw (A) edge node[auto, swap] {1} (B);				% draw side adjacent for first triangle
        \foreach \n in {1,...,\sqrtlast}{						% draw \n triangles
            \ifthenelse{\equal{\n}{\sqrtlast}}					%
            {													%
                \def\currentcolor{red}							% draw the last triangle in red (\sqrtlast)
                \def\currentsqrt{n}								% draw the last triangle (root n)
            }													%
            {													%
                \def\currentcolor{black}						% current triangle
                \pgfmathtruncatemacro{\currentsqrt}{\n+1}		% hypotenuse is of length root (n+1)
            }													%
            \coordinate (C) at ($(B)!1cm!-90:(A)$);				% calculate hypotenuse
            \draw[\currentcolor] (A) edge node[fill=white]		% draw hypotenuse
            	{$\sqrt{\currentsqrt}$} (C);					% ...
            \draw[\currentcolor] (C) edge node[auto] {1} (B);	% draw the side opposite (of length 1)
            \coordinate (w) at ($(B)!4pt!-90:(A)$);				% build the coordinates for the right angle box
            \coordinate (z) at ($(B)!4pt!0:(A)$);				% ...
            \coordinate (t) at ($(w)!4pt!-90:(B)$);				% ...
            \draw (w) -- (t) -- (z);							% draw the right angle box
            \coordinate (B) at (C);								% update starting point
        }														% end \foreach
    \end{tikzpicture}											% end tikzpicture
}

\bigskip
\begin{figure}[H]
\centering
  \resizebox{0.60 \textwidth}{!} {								% resize figure if you want
	\sqrtspiral[scale=2]{14}									% draw the Spiral of Theodorus with n = 14
	}															% end resizebox
\caption{The Spiral of Theodorus}
\end{figure}



\bigskip
\bigskip
%
%
%
\section*{Acknowledgements}
%
%	LaTeX source on overleaf.com
%
\section*{\LaTeX \hspace{0.10 mm} Source}
% \url{}
%
%	get a bibliography
%
%	Note:.bib files go in ~/Library/texmf/bibtex/bib with TeXShop (MacTeX).
%	You can also use an absolute path, e.g. \bibliography{/Users/dmm/papers/bib/qc}
%
\bibliographystyle{plain}
\bibliography{ml}
%
%	done
%
\end{document} 

