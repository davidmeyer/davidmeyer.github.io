%
%	Use lualatex
%
% !TEX TS-program = LuaLaTeX
%
%	To get overleaf.com to use lualatex see
%	https://www.overleaf.com/learn/how-to/Changing_compiler.
%	Basically, change the compiler in Settings.
%
\documentclass{article}
%
%
%	LaTeX template
%
%	David Meyer
%	dmm613@gmail.com
%	08 Oct 2023
%
%
%   get various packages
%
\usepackage[margin=1.0in]{geometry}                                     % adjust margins
\geometry{letterpaper}                                                  % or a4paper or a5paper or ... 
\usepackage{url}                                                        % need this to use URLs in bibtex
\usepackage{setspace}                                                   % need this for \setstrech{...}
\usepackage{scrextend}                                                  % need this for addmargin
\usepackage[export]{adjustbox}                                          % need this to get frame for includegraphics
%
%   tikz et al
%
\usepackage{tikz}
\tikzset{>=latex}                                                       % default to LaTeX arrow head
\usetikzlibrary{calc,patterns,angles,quotes,shapes,math,decorations,
                through,intersections,lindenmayersystems,backgrounds,
                hobby,matrix,positioning}
\usepackage{tikz-cd}
\usepackage{tikzpeople}
\tikzcdset{every arrow/.append style={-latex}}							% commutative diagrams
\usepackage{circuitikz}                                                 % draw circuits    
\usepackage{pgfplots}
%
%	more math stuff
%
\usepackage{amsmath,amsfonts,amssymb,amsthm}
\usepackage{bm}
\usepackage{mathtools}
\usepackage{commath}                                                    % get \norm{x}
\usepackage{fixmath}                                                    % get \mathbold
\usepackage{gensymb}                                                    % get \degree
\usepackage{mathrsfs}
\usepackage{hyperref}
\usepackage{subcaption}
\usepackage{authblk}													% comment out if using beamer (stops \author{} from working in beamer)
\usepackage{graphicx}
\usepackage{hyperref}
\usepackage{alltt}
\usepackage{xcolor}
\usepackage{colortbl}													% \rowcolor{yellow!75} etc
\usepackage{float}
\usepackage{braket}
\usepackage{siunitx}
\usepackage{relsize}
\usepackage{multirow}
\usepackage{esvect}
\usepackage{enumitem}													% use characters instead of numbers in enumerate
\usepackage{changepage}													% needed for \begin{adjustwidth}{-3.25em}{-2.0em} (left justify)
\usepackage{bigints}
\usepackage{multirow}													% \multirow and \multicolumn
\usepackage{enumitem}													% change bullets in \begin{enumerate}
%
%	lualatex
%
%	Put this before \documentclass
%
%	!TEX TS-program = LuaLaTeX
%
%
\usepackage{luatex85,luamplib}
\mplibnumbersystem{double}
\mplibtextextlabel{enable}
%
%
%
%	Describe floating point parameters, \fpeval
%
\ExplSyntaxOn
\cs_set_eq:NN \fpeval \fp_eval:n
\ExplSyntaxOff
%
%	Get the x and y components out of a coordinate, e.g.
%
%	\coordinate (EP) at (8,5);
%	\gettikzxy{(EP)}{\slopex}{\slopey}
%
\makeatletter
\newcommand{\gettikzxy}[3]{%
  \tikz@scan@one@point\pgfutil@firstofone#1\relax
  \edef#2{\the\pgf@x}%
  \edef#3{\the\pgf@y}%
}
\makeatother
%
%
%	Watermarks
%
% \usepackage{draftwatermark}
% \SetWatermarkText{Draft}
% \SetWatermarkScale{5}
% \SetWatermarkLightness {0.9} 
% \SetWatermarkColor[rgb]{0.7,0,0}
%
%
%	theorems, definitions, etc
%
\theoremstyle{definition}
\newtheorem{theorem}{Theorem}[section]
\newtheorem{definition}{Definition}[section]
\newtheorem{proposition}{Proposition}[section]
\newtheorem{lemma}{Lemma}[section]
\newtheorem{example}{Example}[section]
\newtheorem{remark}{Remark}[section]
\newtheorem{notation}{Notation}[section]
%
%	For drawing matrix products 
%
\newcommand*{\vertbar}{\rule[-1ex]{0.5pt}{2.5ex}}
\newcommand*{\horzbar}{\rule[.5ex]{2.5ex}{0.5pt}}

%
%	The following code allows you to do
%
%	\begin{bmatrix}[r] (or [c] or [l])
%
\makeatletter
\renewcommand*\env@matrix[1][c]{\hskip -\arraycolsep
  \let\@ifnextchar\new@ifnextchar
  \array{*\c@MaxMatrixCols #1}}
\makeatother
%
%	make \arg{min,max}_{n \to \infty} work nicely
%
\newcommand{\argmax}{\operatornamewithlimits{argmax}}
\newcommand{\argmin}{\operatornamewithlimits{argmin}}
%
%	handy commands
%
\newcommand*{\Scale}[2][4]{\scalebox{#1}{$#2$}}
\DeclareMathOperator{\E}{\mathbb{E}}
\DeclareMathOperator{\bda}{\Big \downarrow}						% big down arrow
\newcommand{\veq}{\mathrel{\rotatebox{90}{$=$}}}
%
%	Title, author and date
%
\title{template.tex}
\author{David Meyer \\ 
	{\small \vspace{-2.75mm} \href{mailto:dmm613@gmail.com}{dmm613@gmail.com}}}
\date{Last Update: \today \\
	{\small \vspace{1.00mm} Initial Version: October 8, 2020}}
%
%
%
\begin{document}
\maketitle
%
%
%
\section{Introduction}
\label{section:introduction}
%
%
%	show a derivation/proof
%
% \begin{equation*}
% \begin{array}{llll}
% x
% &=& y				&\hspace{2em} \mathrel{\#} \text{comment or} \\
% &=& y				&\hspace{2em} \mathrel{\#} \text{comment} \\
% [10pt]
% \end{array}
% \end{equation*}
%
%
%
%	include an image
%
% \begin{figure}
% \center{\includegraphics[frame, scale=0.50] {images/X.png}}
% \caption{X}
% \label{fig:X}
% \end{figure}
%
%	How to put a box around a figure
%
% \begin{figure}[H]
%  \centering
%   \fbox{\begin{minipage}{34em}
%    \begin{equation*}
%     \begin{array}{rlrlrlr}
%       9^2      &=& 81           &\longrightarrow& 8 + 1           &=& 9	  \\
%       45^2     &=& 2025         &\longrightarrow& 20 + 25         &=& 45     \\
%       703^2    &=& 494209       &\longrightarrow& 494 + 209       &=& 703    \\
%       7777^2   &=& 60481729     &\longrightarrow& 6048 + 1729     &=& 7777   \\
%       857143^2 &=& 734694122449 &\longrightarrow& 734694 + 122449 &=& 857143 \\
%     \end{array}
%    \end{equation*}
%  \end{minipage}}
%  \caption{A Few Example Kaprekar Numbers}
%\end{figure}
%
%	
%	How to wrap (and resize) your tikzpicture in a figure
%
% \begin{figure}[H]
% \centering
%  \resizebox{0.40 \textwidth}{!} {		% resize the figure if you want
%    \begin{tikzpicture} 
%     .....
%    \end{tikzpicture}					% end tikzpicture
%  }									% end resizebox
% \caption{Some Figure}
% \label{fig:some_figure}
% \end{figure}
%
%	Build a figure that is resizable and framed with \fbox
% 
%                                                               % start building pascal's triangle
%       Parameters and settings                                 % start with the parameters and settings
%                                                               %
% \def \nrows {14}                                              % number of rows in the triangle
% \setlength {\fboxsep} {0.75em}								% get more space between fcolorbox and content
% \setlength {\fboxrule}{1.25pt}								% change fcolorbox line thickness
%                                                               %
%       Build the figure                                        % build the figure
%                                                               %
% \begin{figure}[H]                                             % begin the figure
%   \centering                                                  % center everything
%   \fcolorbox{black}{white} {									% put a color box around the figure
%     \resizebox{0.75 \textwidth}{!} {                          % resize the figure if you want
%     															%
%     <whatever>												% whatever goes in the figure
%																%
%      }                                                        % end resizebox
%   }                                                           % end fbox
%  \caption{X}                                     		   		% caption the figure
%   \label{fig:X}                                  			    % label the figure
% \end{figure}                                                  % end figure
% 
%
%
%
\section{Conclusions}
\label{section:conclusions}
%
%
%
\section*{Acknowledgements}
\label{section:acknowledgements}
%
%	LaTeX source on overleaf.com
%
\section*{\LaTeX \hspace{0.025 mm} Source}
\url{https://www.overleaf.com/read/jyppzhgdtrvg}
%
%	get a bibliography
%
%	Note:.bib files go in ~/Library/texmf/bibtex/bib with TeXShop (MacTeX).
%	You can also use an absolute path, e.g. \bibliography{/Users/dmm/papers/bib/qc}
%
\bibliographystyle{plain}
\bibliography{qc,ml}
%
%
%
% \section*{Appendix A}
%
%	done
%
\end{document} 

