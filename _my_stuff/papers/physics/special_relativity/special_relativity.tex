%
%	Use lualatex
%
% !TEX TS-program = LuaLaTeX
%
%	To get overleaf.com to use lualatex see
%	https://www.overleaf.com/learn/how-to/Changing_compiler.
%	Basically, change the compiler in Settings.
%
\documentclass{article}
%
%
%	LaTeX template
%
%	David Meyer
%	dmm613@gmail.com
%	08 Oct 2023
%
%
%   get various packages
%
\usepackage[margin=1.0in]{geometry}                                     % adjust margins
\geometry{letterpaper}                                                  % or a4paper or a5paper or ... 
\usepackage{url}                                                        % need this to use URLs in bibtex
\usepackage{setspace}                                                   % need this for \setstrech{...}
\usepackage{scrextend}                                                  % need this for addmargin
\usepackage[export]{adjustbox}                                          % need this to get frame for includegraphics
%
%   tikz et al
%
\usepackage{tikz}
\tikzset{>=latex}                                                       % default to LaTeX arrow head
\usetikzlibrary{calc,patterns,angles,quotes,shapes,math,decorations,
                through,intersections,lindenmayersystems,backgrounds,
                hobby,matrix}
\usepackage{tikz-cd}
\usepackage{tikzpeople}
\tikzcdset{every arrow/.append style={-latex}}
\usepackage{circuitikz}                                                 % draw circuits    
\usepackage{pgfplots}
%
%	more math stuff
%
\usepackage{amsmath,amsfonts,amssymb,amsthm}
\usepackage{bm}
\usepackage{mathtools}
\usepackage{commath}                                                    % get \norm{x}
\usepackage{fixmath}                                                    % get \mathbold
\usepackage{gensymb}                                                    % get \degree
\usepackage{mathrsfs}
\usepackage{hyperref}
\usepackage{subcaption}
\usepackage{authblk}							% comment out if using beamer (stops \author{} from working in beamer)
\usepackage{graphicx}
\usepackage{hyperref}
\usepackage{alltt}
\usepackage{xcolor}
\usepackage{colortbl}							% \rowcolor{yellow!75} etc
\usepackage{float}
\usepackage{braket}
\usepackage{siunitx}
\usepackage{relsize}
\usepackage{multirow}
\usepackage{esvect}
\usepackage{enumitem}							% use characters instead of numbers in enumerate
\usepackage{changepage}         					% needed for \begin{adjustwidth}{-3.25em}{-2.0em} (left justify)
\usepackage{bigints}
\usepackage{multirow}							% \multirow and \multicolumn
\usepackage{enumitem}							% change bullets in \begin{enumerate}
%
%	lualatex
%
%	Put this before \documentclass
%
%	!TEX TS-program = LuaLaTeX
%
%
\usepackage{luatex85,luamplib}
\mplibnumbersystem{double}
\mplibtextextlabel{enable}
%
%
%
%	Describe floating point parameters, \fpeval
%
\ExplSyntaxOn
\cs_set_eq:NN \fpeval \fp_eval:n
\ExplSyntaxOff
%
%	Get the x and y components out of a coordinate, e.g.
%
%	\coordinate (EP) at (8,5);
%	\gettikzxy{(EP)}{\slopex}{\slopey}
%
\makeatletter
\newcommand{\gettikzxy}[3]{%
  \tikz@scan@one@point\pgfutil@firstofone#1\relax
  \edef#2{\the\pgf@x}%
  \edef#3{\the\pgf@y}%
}
\makeatother
%
%
%	Watermarks
%
% \usepackage{draftwatermark}
% \SetWatermarkText{Draft}
% \SetWatermarkScale{5}
% \SetWatermarkLightness {0.9} 
% \SetWatermarkColor[rgb]{0.7,0,0}
%
%
%	theorems, definitions, etc
%
\theoremstyle{definition}
\newtheorem{theorem}{Theorem}[section]
\newtheorem{definition}{Definition}[section]
\newtheorem{proposition}{Proposition}[section]
\newtheorem{lemma}{Lemma}[section]
\newtheorem{example}{Example}[section]
\newtheorem{remark}{Remark}[section]
\newtheorem{notation}{Notation}[section]
%
%	For drawing matrix products 
%
\newcommand*{\vertbar}{\rule[-1ex]{0.5pt}{2.5ex}}
\newcommand*{\horzbar}{\rule[.5ex]{2.5ex}{0.5pt}}

%
%	The following code allows you to do
%
%	\begin{bmatrix}[r] (or [c] or [l])
%
\makeatletter
\renewcommand*\env@matrix[1][c]{\hskip -\arraycolsep
  \let\@ifnextchar\new@ifnextchar
  \array{*\c@MaxMatrixCols #1}}
\makeatother
%
%	make \arg{min,max}_{n \to \infty} work nicely
%
\newcommand{\argmax}{\operatornamewithlimits{argmax}}
\newcommand{\argmin}{\operatornamewithlimits{argmin}}
%
%	handy commands
%
\newcommand*{\Scale}[2][4]{\scalebox{#1}{$#2$}}
\DeclareMathOperator{\E}{\mathbb{E}}
\DeclareMathOperator{\bda}{\Big \downarrow}						% big down arrow
\newcommand{\veq}{\mathrel{\rotatebox{90}{$=$}}}
%
%	Title, author and date
%
\title{A Few Notes on Special Relativity}
\author{David Meyer \\ 
	{\small \vspace{-2.75mm} \href{mailto:dmm613@gmail.com}{dmm613@gmail.com}}}
\date{Last Update: \today \\
	{\small \vspace{1.00mm} Initial Version: December 23, 2024}}
%
%
%
\begin{document}
\maketitle
%
%
%
\section{Introduction}
\label{section:introduction}
TBD

\section{The Postulates of Special Relativity}
\label{section:postulates}

\begin{enumerate} 

\item The laws of physics are the same for all observers in
uniform motion relative to each other (i.e., in inertial
reference frames).  This means that no inertial observer can
detect a preferred or absolute state of motion. There is no
absolute "rest" frame; all motion is relative.

\item The speed of light in a vacuum is always constant,
regardless of the motion of the observer or the light
source. This speed is approximately $c=299,792,458$ m/s.
\end{enumerate}

\section{Time Dilation}
\label{section:timedialation}
Time dilation is an unusual phenomenon predicted by Einstein's
special theory of relativity.  It refers to the effect where time
appears to pass more slowly for an observer in motion relative to
a stationary observer. In simpler terms, a moving clock ticks
more slowly compared to a clock that is at rest, as observed by
someone who is not moving with the clock
\cite{physics:spacetime_hughes}.

\bigskip
\noindent
The key idea is that when an object moves at a speed close to the
speed of light relative to an observer, time for the moving
object (as seen by the observer) slows down. This effect becomes
more pronounced as the speed of the object approaches the speed
of light.

\bigskip
\noindent
Said another way, special relativity tells us that for an
observer $O$ in an inertial frame of reference, a clock that is
moving relative to the observer will be measured to tick more
slowly than a clock at rest in the observer's frame of reference
\cite{wikipedia:time_dilation}.

\bigskip
\noindent
To see why this happens, consider the following thought experiment. The 
experimental setup is shown in Figure \ref{figure:observer_at_rest}. 
\footnote{Assume here that the thickness of the source and the
detector are negligible so that the distance light travels is
$h$.} Here the observer, light source and detector are at rest in
inertial frame of reference $I$ (even though the train car is
moving with velocity $v$).


\begin{figure}[H]
  \centering
    \resizebox{0.65 \textwidth}{!} {                                    % resize the figure if you want
      \begin{tikzpicture}
%
%       Draw the train car
%
        \draw[thick] (0,0) rectangle (6,3);                             % main body
        \draw[thick] (0.5,-0.5) circle (0.5cm);                         % left wheel
        \draw[thick] (5.5,-0.5) circle (0.5cm);                         % right wheel
        \draw[draw=none] (6,3) -- (6,0);
%
%       Draw the observer inside
%
		\node[bob,mirrored,minimum size=1.0cm] at (4.55,1.10) {\footnotesize Observer $O$};
%
%       Draw the light source with smaller edges (rectangle)
%
        \draw[fill=yellow] (1.60,3) rectangle (2.4,2.7);                % light source as a rectangle
        \node[above] at (2.2,3.0) {Light Source};
%
%       Draw the detector
%
        \draw[fill=blue!50] (1.7,0.2) rectangle (2.3,0);                % detector completely inside the car
        \node[below] at (2,-0.05) {Detector};
%
%       Draw the light beam (vertical line)
%
        \draw[->,dashed, thick, red] (2,2.7) -- (2,0.2);                % vertical light beam
        \draw[<->, thick] (-0.75,3) -- (-0.75,0) node[midway, left] {\(h\)};
        \draw[dotted, thick] (0,3) -- (-1,3);                           % top dotted line
        \draw[dotted, thick] (0,0) -- (-1,0);                           % bottom dotted line
%
%       Draw the velocity vector
%
        \draw[->, thick] (6.5,1.5) -- (8.5,1.5) node[midway, above] {\(v\)}; 
%
%
%       Draw the ground
        \draw[thick] (-1,-1) -- (9,-1); % ground line
      \end{tikzpicture}
   }
 \caption{Observer $O$ at rest with respect to a light source and detector}
 \label{figure:observer_at_rest}
\end{figure}

\bigskip
\noindent
For observer at rest with respect to an inertial frame of
reference, the time taken for a beam of light to travel from the
light source to the detector is

\smallskip
\begin{equation*}
  \begin{array}{llll}
     h
     &=& c t 					
     		&\hspace{2em} \mathrel{\#} \text{$t$ is called 
		        the proper time \cite{wikipedia:proper_time}} \\                
[10pt]
	 &\Rightarrow& t = \dfrac{h}{c} 
	 		&\hspace{2em} \mathrel{\#} \text{solve for $t$}
  \end{array}
\end{equation*}

\medskip
\noindent
Now consider the same experimental setup as in Figure
\ref{figure:observer_at_rest} except this time the observer is on
the train landing. That is, the observer's frame of reference is
at rest with respect to the train car's frame of reference (which
is moving with velocity $v$).\footnote{Note that the velocity $v$
here is a the real constant representing a velocity confined to
the x-direction.}

\bigskip
\begin{figure}[H]
  \centering
  \resizebox{0.80 \textwidth}{!} {                                              % resize the figure if you want
     \begin{tikzpicture}
%
%       Draw a train car moved by delta x
%
        \draw[thick] (0,0) rectangle (6,3);                                     % main body
        \draw[thick] (0.5,-0.5) circle (0.5cm);                                 % left wheel
        \draw[thick] (5.5,-0.5) circle (0.5cm);                                 % right wheel
        \draw[draw=none] (6,3) -- (6,0);      									% label frame of reference
%
%
%       Draw an external observer outside the car, in the same position but relative to the moved train
%
		\node[bob,mirrored,minimum size=1.0cm] at (9.50,0.10) {\footnotesize Observer $O$};
%
%
%       Draw the light source attached to the ceiling
%
        \draw[fill=yellow] (2,3) rectangle (2.4,2.7);                           % light source as a rectangle
        \node[above] at (2.2,3.0) {Light Source};
%
%       Draw the detector moved ahead by delta x
%
        \draw[fill=blue!50] (3.7,0.2) rectangle (4.3,0);                        % detector ahead of the light source by delta x
        \node[above] at (3, -0.5) {$x^{\prime}$};
%
%       Draw the Light beam
%
%		light beam from the light source to the detector
%
        \draw[->,dashed, thick, red] (2.2,2.7) -- (4.0,0.2) node [midway,right,xshift=2.0mm] {$\color{black}{ct^{\prime}}$};   
        \draw[-,dashed, thin]       (2.2,2.7) -- (2.2,0.0) node [midway,left,xshift=-0.25mm,yshift=1.25mm] {$h$};                          
        \draw[dotted, thick] (2.2,-0) -- (2.2,-0.52);
        \draw[dotted, thick] (4.0,-0) -- (4.0,-0.52);
        \draw[<-]     (2.2,-0.25) -- (2.70,-0.25);
        \draw[->]     (3.35,-0.25) -- (4.0,-0.25);
%
%       Draw the height label (h) outside the car
%
        \draw[<->] (-0.50,3) -- (-0.50,0) node[midway, left] {\(h\)};
%
%       Draw the dotted lines parallel to the ground
%
        \draw[dotted, thick] (0,3) -- (-1,3); % top dotted line
        \draw[dotted, thick] (0,0) -- (-1,0); % bottom dotted line
%
%       Draw the velocity vector
%
        \draw[->] (6.5,1.5) -- (8.5,1.5) node[midway, above] {\(v\)}; 
%
%       Draw the ground
%
        \draw[thick] (-1,-1) -- (11,-1); % ground line
    \end{tikzpicture}
   }
 \caption{Observer $O$ at rest with respect to another inertial frame of reference}
 \label{figure:observer_motion}
\end{figure}

\bigskip
\noindent
The question now is how much time, call it $t^{\prime}$, passes
from the point of view of observer $O$ when the light travels
from the source to the detector. $x^{\prime}$ is the distance the
train moves forward (from the origin) in time $t^{\prime}$. We
also know that the speed of light in a vacuum is $c$ for all
observers, so the distance that light travels (from the source to
the detector) in this case is $c t^{\prime}$ (the hypotenuse of
the triangle shown in Figure \ref{figure:observer_motion}).

\medskip
\noindent
\remark{Note that in general, the coordinates in the "stationary"
frame of reference (call it $S$) are referred to as $(t,x,y,z)$,
while the corresponding coordinates in the "moving" frame of
reference (call it $S^{\prime}$) are denoted $(t^{\prime},x^{\prime}, 
y^{\prime},z^{\prime})$. Coordinates in $S^{\prime}$ are related to 
coordinates in $S$ by the Lorentz transformations 
\cite{einstein1905,lorentz1904electromagnetic}.
\label{remark:frames_of_reference}} 

\bigskip
\noindent
To compute $t^{\prime}$, consider the $h-x^{\prime}-c t^{\prime}$ 
triangle shown in Figure \ref{figure:observer_motion}, which is abstracted 
in Figure \ref{figure:triangle}. 

\begin{figure}[H]
  \centering
  \resizebox{0.35 \textwidth}{!} {                                                                      % resize the figure if you want
     \begin{tikzpicture} 
        \draw[->, line width=1pt] (-1.22,0) -- (4.22,0) node[right] {}; 
        \draw[->, line width=1pt] (0,-1.22) -- (0,5.22) node[above] {}; 
        \draw [line width=1pt,red] (1,1) -- (3,1) -- (1,4) -- cycle; 
        \draw[draw=none] (1,1) -- (1,4) node [midway,left] {$h$};
        \draw[draw=none] (1,1) -- (3,1) node [midway,below] {$x^{\prime}$};
        \draw[draw=none] (1,4) -- (3,1) node [midway,right,xshift=0.15cm] {$c t^{\prime}$};
        \end{tikzpicture}
     }
  \caption{Triangle formed by the motion of the train with respect to the observer $O$}
  \label{figure:triangle}
\end{figure}

\bigskip
\noindent
Using this triangle we can calculate $t^{\prime}$ using the
Pythagorean theorem:

\bigskip
\begin{equation*}
 \begin{array}{llll}
     (c t^{\prime})^2
     &=& h^2 + {x^{\prime}}^2 					
     		&\hspace{5.00em} \mathrel{\#} \text{by the Pythagorean theorem} \\
[10pt]
     &\Rightarrow& (c t^{\prime})^2 = {(ct)}^2 + {x^{\prime}}^2 					
     		&\hspace{5.00em} \mathrel{\#} h = c t \\
[15pt]
	&\Rightarrow& (c t^{\prime})^2 = (ct)^2 + (v t^{\prime})^2
			&\hspace{5.00em} \mathrel{\#} x^{\prime} = v t^{\prime} \\
[15pt]
	&\Rightarrow& c^2 {t^{\prime}}^2 = c^2t^2 + v^2 {t^{\prime}}^2
			&\hspace{5.00em} \mathrel{\#}  \text{square terms} \\
[15pt]
	&\Rightarrow& {t^{\prime}}^2 = t^2 + \dfrac{v^2}{c^2} {t^{\prime}}^2
			&\hspace{5.00em} \mathrel{\#}  \text{divide both sides by $c^2$} \\
[15pt]
	&\Rightarrow& {t^{\prime}}^2 - \dfrac{v^2}{c^2} {t^{\prime}}^2 = t^2 
			&\hspace{5.00em} \mathrel{\#}  \text{subtract $\dfrac{v^2}{c^2} 
			{t^{\prime}}^2$ from both sides} \\
[15pt]
	&\Rightarrow& {t^{\prime}}^2 \left (1 - \dfrac{v^2}{c^2} \right) = t^2 
			&\hspace{5.00em} \mathrel{\#} \text{factor out ${t^{\prime}}^2$} \\
[15pt]	
	&\Rightarrow& {t^{\prime}}^2 = \dfrac{t^2}{\left (1 - \dfrac{v^2}{c^2} \right)}
			&\hspace{5.00em} \mathrel{\#}  \text{divide both sides by $\left 
			(1 - \dfrac{v^2}{c^2}\right )$} \\
[35pt]
	&\Rightarrow& t^{\prime} = \dfrac{t}{\sqrt{1 - \dfrac{v^2}{c^2}}}
			&\hspace{5.00em} \mathrel{\#}  \text{take the square root of both sides}
  \end{array}
\end{equation*}


\bigskip
\noindent
So for observer $O$ the passage of time on the moving train slows
down by a factor of $\gamma = \dfrac{1}{\sqrt{1-\dfrac{v^2}{c^2}}}$.

\bigskip
\noindent
$\gamma$ is called the Lorentz factor (or sometimes the gamma
factor). Note also that $t^{\prime}$ can be written using the
Lorentz factor: $t^{\prime} = \gamma t$. The Lorentz factor is
sometimes written $\gamma(v)$ or $\gamma_{v}$ to emphasize $v$.
%
%
%
\section{Length Contraction}
\label{section:lenght_contraction}
%
%
%
\section{The Lorentz Transformations}
\label{section:the_lorentz_transformations}
%
%
%
\section{The Andromeda Paradox}
\label{section:the_andromeda_paradox}
%
%
%
\section{Spacetime Invariance}
\label{section:spacetime_invariance}
%
%
%
\section{Conclusions}
\label{section:conclusions}
%
%
%
\section*{Acknowledgements}
\label{section:acknowledgements}
%
%	LaTeX source on overleaf.com
%
\section*{\LaTeX \hspace{0.025 mm} Source}
\url{https://www.overleaf.com/read/xfghcdxjyzhg#675841}
%
%	get a bibliography
%
%	Note:.bib files go in ~/Library/texmf/bibtex/bib with TeXShop (MacTeX).
%	You can also use an absolute path, e.g. \bibliography{/Users/dmm/papers/bib/qc}
%
\bibliographystyle{plain}
\bibliography{physics}
%
%
%
% \section*{Appendix A}
%
%	done
%
\end{document} 

