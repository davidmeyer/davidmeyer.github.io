%
%	A bit on conservation of energy
%	
%	David Meyer
%	09/28/2022
%
%
\subsection{Conservation of Energy}
\label{subsec:conservation_of_energy}
Consider the case of a general force function $F(x)$. Here we define the kinetic 
energy of the system to be $\frac{1}{2}mv^{2}$. We also define the potential 
energy of the system, $V(x)$, to be the function

\begin{equation}
V(x) = - \int F(x) \, dx
\label{eqn:V(x)}
\end{equation}

\noindent
so that 

\begin{equation}
F(x) = -\dfrac{dV}{dx}
\label{eqn:F}
\end{equation}

\medskip
\noindent
Then the total energy of the system as a function of displacement and 
velocity, $E(x,v)$, is defined to be 


\begin{equation}
E(x,v) = \dfrac{1}{2}mv^{2} + V(x)
\label{eqn:E}
\end{equation}

\medskip
\noindent
One of the main reasons this energy function is important is that it is 
\emph{conserved}, meaning that its value along any trajectory is constant.
Said another way: For each trajectory $x(t)$ conforming to Newton’s 
Second Law, the system's total energy $E(x(t),\dot{x}(t))$ is independent 
of $t$.

\begin{theorem}
Suppose a particle's trajectory conforms to Newton’s Second Law 
in the form $F(x(t)) = m\ddot{x}(t)$ and let $V$ and $E$ be as 
in Equations (\ref{eqn:V(x)}) and (\ref{eqn:E}). Then the total 
energy of the system, $E(x(t),\dot{x}(t))$, is conserved. 

\bigskip
\noindent
{\bf Proof:} One way to see this is to show that 
$\dfrac{d}{dt}E(x(t),\dot{x}(t)) = 0$:

\begin{equation*}
\begin{array}{llll}
\dfrac{d}{dt} E(x,\dot{x})
&=& \dfrac{d}{dt} \bigg [\dfrac{1}{2}m(\dot{x}(t))^{2} + V(x(t)) \bigg ]	
		&\hspace{1em} \mathrel{\#} \text{Equation (\ref{eqn:E}) with $x = x(t)$ 
		                                 and $v = \dot{x}(t)$} \\	
[12pt]
&=& \dfrac{d}{dt} \bigg [ \dfrac{1}{2}m(\dot{x}(t))^{2} \bigg ] + \dfrac{d}{dt} V(x(t))
		&\hspace{1em} \mathrel{\#} \text{derivative is a linear operator \cite{wiki:linearity_of_differentiation}} \\
[12pt]
&=& m\dot{x}(t)\ddot{x}(t) + \dfrac{d}{dt} V(x(t))
		&\hspace{1em} \mathrel{\#} \text{by the chain and power rules \cite{wiki:chain_rule,wiki:power_rule}} \\
[12pt]
&=& m\dot{x}(t)\ddot{x}(t) + \dfrac{d}{dt} V(u)
		&\hspace{1em} \mathrel{\#} \text{let $u = x(t)$; this implies that 
		                                 $\dfrac{du}{dx} = 1 \Rightarrow du = dx$} \\
[12pt]
&=& m\dot{x}(t)\ddot{x}(t) + \left [\dfrac{d}{du}  V(u) \right ] \dot{x}(t)
		&\hspace{1em} \mathrel{\#} \text{by the chain rule: 
								         $\dfrac{d}{dt} V(u) = 
					                     \dfrac{dV}{du} \, \dfrac{du}{dt}$
					                     and 
							             $\dfrac{du}{dt}     = 
							             \dfrac{d}{dt} x(t)  = 
							             \dot{x}(t)$} \\
[12pt]
&=& m\dot{x}(t)\ddot{x}(t) + \left [ \dfrac{d}{dx}V(u) \right ] \dot{x}(t)
		&\hspace{1em} \mathrel{\#} du = dx \\
[12pt]
&=& m\dot{x}(t)\ddot{x}(t) + \left [ \dfrac{d}{dx}V(x(t)) \right ] \dot{x}(t)
		&\hspace{1em} \mathrel{\#} u = x(t) \\
[12pt]
&=& \dot{x}(t)\left [m \ddot{x}(t) + \dfrac{d}{dx}V(x(t)) \right ]
		&\hspace{1em} \mathrel{\#} \text{factor out $\dot{x}(t)$} \\
[12pt]
&=& \dot{x}(t)\Big [m \ddot{x}(t) - F(x(t)) \Big ]
		&\hspace{1em} \mathrel{\#} \text{$\dfrac{d}{dx}V(x(t)) = - F(x(t))$ 
		                                 (Equation (\ref{eqn:F}))} \\
[12pt]
&=& 0
		&\hspace{1em} \mathrel{\#} \text{since $m \ddot{x}(t) - F(x(t)) = 0$ 
		                                 by Newton's Second Law}
\end{array}
\end{equation*}

\bigskip
\noindent
Thus, the time derivative of the energy along any trajectory is zero which implies that
$E(x(t), \dot{x}(t))$ is independent of $t$. In summary, we may call the energy a conserved 
quantity (or constant of motion), since the particle neither gains nor loses energy as the 
particle moves according to Newton’s Second Law.
\end{theorem}

