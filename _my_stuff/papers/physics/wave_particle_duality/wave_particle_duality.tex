\documentclass{article}
%
%
%	wave_particle_duality.tex
%
%	David Meyer
%	dmm613@gmail.com
%	22 Sep 2022
%
%
%   get various packages
%
\usepackage[margin=1.0in]{geometry}                                     % adjust margins
\geometry{letterpaper}                                                  % or a4paper or a5paper or ... 
\usepackage{url}                                                        % need this to use URLs in bibtex
\usepackage{setspace}                                                   % need this for \setstrech{...}
\usepackage{scrextend}                                                  % need this for addmargin
\usepackage[export]{adjustbox}                                          % need this to get frame for includegraphics
%
%   tikz et al
%
\usepackage{tikz}
\usetikzlibrary{calc,patterns,angles,quotes,shapes,math,decorations,
                through,intersections,lindenmayersystems,backgrounds,
                hobby}
\tikzset{>=latex}														% default to LaTeX arrow head
\usepackage{circuitikz}                                                 % draw circuits    
\usepackage{pgfplots}
%
%	more math stuff
%
\usepackage{amsmath,amsfonts,amssymb,amsthm}
\usepackage{mathtools}
\usepackage{commath}                                                    % get \norm{x}
\usepackage{fixmath}                                                    % get \mathbold
\usepackage{gensymb}                                                    % get \degree
\usepackage{mathrsfs}
\usepackage{hyperref}
\usepackage{subcaption}
\usepackage{authblk}
\usepackage{graphicx}
\usepackage{hyperref}
\usepackage{alltt}
\usepackage{color}
\usepackage{float}
\usepackage{braket}
\usepackage{siunitx}
\usepackage{relsize}
\usepackage{multirow}
\usepackage{esvect}
%
%	watermarks
%
% \usepackage{draftwatermark}
% \SetWatermarkText{Draft}
% \SetWatermarkScale{5}
% \SetWatermarkLightness {0.9} 
% \SetWatermarkColor[rgb]{0.7,0,0}
%
%
%	theorems, definitions, etc
%
\theoremstyle{definition}
\newtheorem{theorem}{Theorem}[section]
\newtheorem{definition}{Definition}[section]
\newtheorem{proposition}{Proposition}[section]
\newtheorem{lemma}{Lemma}[section]
\newtheorem{example}{Example}[section]
\newtheorem{remark}{Remark}[section]
%
%	The following code allows you to do
%
%	\begin{bmatrix}[r] (or [c] or [l])
%
\makeatletter
\renewcommand*\env@matrix[1][c]{\hskip -\arraycolsep
  \let\@ifnextchar\new@ifnextchar
  \array{*\c@MaxMatrixCols #1}}
\makeatother
%
%	make \arg{min,max}_{n \to \infty} work nicely
%
\newcommand{\argmax}{\operatornamewithlimits{argmax}}
\newcommand{\argmin}{\operatornamewithlimits{argmin}}
%
%	handy commands
%
\newcommand*{\Scale}[2][4]{\scalebox{#1}{$#2$}}
\DeclareMathOperator{\E}{\mathbb{E}}
\DeclareMathOperator{\bda}{\Big \downarrow}						% big down arrow
\newcommand{\veq}{\mathrel{\rotatebox{90}{$=$}}}
%
%	Title, author and date
%
\title{A Few Notes on Wave-Particle Duality (etc)}
\author{David Meyer \\ \href{mailto:dmm613@gmail.com}
                            {dmm613@gmail.com}}
\date{Last update: \today}
%
%
%
\begin{document}
\maketitle
%
%
%
\section{Introduction}

\smallskip
\noindent
We know from basic electromagnetism
\cite{wiki:electromagnetic_radiation} that $c = \lambda \nu$,
where $c$ is the speed of light, $\lambda$ is the wavelength, and
$\nu$ is the frequency. Solving for $\nu$ we get


\bigskip
\begin{equation}
	\nu = \dfrac{c}{\lambda}
	\label{eqn:frequency}
\end{equation}

\bigskip
\noindent
We also know from Planck's work \cite{planck_energy} that
electromagnetic waves have energy $E_{\text{{\tiny P}}}$, where
$E_{\text{{\tiny P}}}$ is defined as


\smallskip
\begin{equation}
	E_{\text{{\tiny P}}} = h \nu
	\label{eqn:planck_energy}
\end{equation}


\bigskip
\noindent
and $h$ is Planck's constant \cite{wiki:plancks_constant}. 
Substituting Equation (\ref{eqn:frequency}) into Equation 
(\ref{eqn:planck_energy}) we get

\medskip
\begin{equation*}
	E_{\text{{\tiny P}}} = h \dfrac{c}{\lambda}
	\label{eqn:planck_c_over_lambda}
\end{equation*}

\medskip
\noindent
In addition, we know from Einstein
\cite{wiki:mass_energy_equivalence} that the energy of a mass
$m$, call it $E_{\text{{\tiny E}}}$, is


\begin{equation*}
	E_{\text{{\tiny E}}} = mc^2
	\label{eqn:E=mc^2}
\end{equation*}


{\setstretch{1.525}
\noindent
What de Broglie \cite{wiki:de_broglie} realized was that matter
must also behave like a wave\footnote{Einstein had already shown
that an electromagnetic wave can behave like a particle
\cite{wiki:photoelectric_effect}.}
\cite{wiki:matter_wave,wiki:wave_particle_duality} and so
$E_{\text{{\tiny P}}}$ must equal $E_{\text{{\tiny E}}}$.  This
implies that matter also has a wavelength: $\lambda =
\dfrac{h}{p}$, where $p := \norm{\mathbf{p}}$ is the Euclidean
Norm \cite{dmm:vector_calculus} (magnitude) of the momentum vector 
$\mathbf{p}$, and $\mathbf{p} = m \mathbf{v}$. $\lambda$ is called 
the \emph{de Broglie wavelength}.
\par}

\smallskip
\bigskip
\noindent
Here's one way to think about where $\lambda$ comes from:

\begin{equation*}
\begin{array}{llll}
h \dfrac{c}{\lambda}
&=& mc^2	
			&\hspace{5em} \mathrel{\#} \text{de Broglie's insight: 
				$E_{\text{{\tiny P}}} = E_{\text{{\tiny E}}}$} \\
[9pt]
&\Rightarrow& \dfrac{h}{\lambda} = mc
			&\hspace{5em} \mathrel{\#} \text{cancel $c$} \\
[9pt]
&\Rightarrow& \dfrac{h}{\lambda} = mv
			&\hspace{5em} \mathrel{\#} v := \norm{\mathbf{v}} = c \\
[9pt]
&\Rightarrow& \lambda = \dfrac{h}{m v}
			&\hspace{5em} \mathrel{\#} \text{solve for $\lambda$}\\
[9pt]
&\Rightarrow& \lambda = \dfrac{h}{p}
			&\hspace{5em} \mathrel{\#} \text{since $\mathbf{p} = m \mathbf{v}$ 
			we have $p := \norm{\mathbf{p}}     = 
			              \norm{m \mathbf{v}}   = 
			              |m| \norm{\mathbf{v}} = 
			              mv$} \\
[22pt] 
\end{array}
\end{equation*}

\subsection{Conservation of Energy in $\mathbb{R}^{1}$}
\label{subsec:conservation_of_energy}
The first thing to note here is that since we are in one-dimensional
space (motion is in $\mathbb{R}^{1}$), quantities such as displacement,
velocity, and acceleration, force, and energy are one-dimensional 
vectors (aka scalars). So we use, for example, $F = ma$ rather than 
$\mathbf{F} = m \mathbf{a}$.


\bigskip
\noindent
Next, we need one definition:

\smallskip
\begin{definition}
{\bf Trajectory:} A solution x(t) to the equation $F(x(t)) = m \ddot{x}(t)$,
Newton’s Second Law, is called a trajectory.
\end{definition}

\smallskip
\noindent
Now, consider the case of a general force function $F(x)$. Here we define 
the kinetic energy of a particle to be $\frac{1}{2}mv^{2}$. We also 
define the potential energy of a particle, $V(x)$, to be 

\smallskip
\begin{equation}
V(x) = - \int F(x) \, dx
\label{eqn:V(x)}
\end{equation}

\noindent
so that 

\begin{equation}
\dfrac{d}{dx}V(x) = -F(x)
\label{eqn:F}
\end{equation}

\bigskip
\noindent
Then the total energy of a particle as a function of displacement 
and velocity, $E(x,v)$, is defined to be 

\begin{equation}
E(x,v) = \dfrac{1}{2}mv^{2} + V(x)
\label{eqn:E}
\end{equation}

\medskip
\noindent
One of the main reasons this energy function is important is that it is 
\emph{conserved}, meaning that its value along any trajectory is constant.
Switching notation ($x = x(t)$ and $v = \dot{x}(t)$) and saying this in 
another way: An energy function is \emph{conserved} if, for each trajectory 
$x(t)$ conforming to Newton’s Second Law, a particle's total energy 
$E(x(t),\dot{x}(t))$ is independent of $t$.


\medskip
\begin{theorem}
\label{theorem:conservation_of_energy_in_R1}
Suppose a particle's trajectory conforms to Newton’s Second Law 
in the form $F(x(t)) = m\ddot{x}(t)$ and let $V$ and $E$ be as 
in Equations (\ref{eqn:V(x)}) and (\ref{eqn:E}). Then the total 
energy of the particle is conserved.

\medskip
{\setstretch{1.50}
\noindent
{\bf Proof.} One way to prove Theorem \ref{theorem:conservation_of_energy_in_R1}
is to show that the particle's total energy does not change with time, 
that is,$\dfrac{d}{dt}E(x(t),\dot{x}(t)) = 0$:
\par}
%
%
%
\begin{equation*}
\begin{array}{llll}
\dfrac{d}{dt} E(x(t),\dot{x}(t))
&=& \dfrac{d}{dt} \bigg [\dfrac{1}{2}m(\dot{x}(t))^{2} + V(x(t)) \bigg ]
		&\hspace{0em} \mathrel{\#} \text{definition of $E(x(t),\dot{x}(t))$ 
		                                 (Equation (\ref{eqn:E}))} \\		
[10pt]
&=& \dfrac{d}{dt} \bigg [ \dfrac{1}{2}m(\dot{x}(t))^{2} \bigg ] + \dfrac{d}{dt} V(x(t))
		&\hspace{0em} \mathrel{\#} \text{derivative is a linear 
		                                 operator \cite{wiki:linearity_of_differentiation}} \\
[10pt]
&=& m\dot{x}(t)\ddot{x}(t) + \dfrac{d}{dt} V(x(t))
		&\hspace{0em} \mathrel{\#} \text{chain \& power rules \cite{wiki:chain_rule,wiki:power_rule}:
										 $\dfrac{d}{dt} \left [  \dfrac{1}{2}m(\dot{x}(t))^{2} \right ] = 
										 m\dot{x}(t)\ddot{x}(t)$} \\
[10pt]
&=& m\dot{x}(t)\ddot{x}(t) + \dfrac{d}{dt} V(u)
		&\hspace{0em} \mathrel{\#} \text{let $u = x(t)$; this implies that
		                                 $\dfrac{du}{dt} = \dfrac{dx}{dt} = \dot{x}$
		                                 and 
		                                 $\dfrac{du}{dx} = 1$} \\
[10pt]
&=& m\dot{x}(t)\ddot{x}(t) + \left [\dfrac{d}{du}  V(u) \right ] \dot{x}(t)
		&\hspace{0em} \mathrel{\#} \text{chain rule: 
								         $\dfrac{dV}{dt} = 
					                     \dfrac{dV}{du} \dfrac{du}{dt}$
					                     and
							             $\dfrac{du}{dt} = \dot{x} 
							             \Rightarrow 
							             \dfrac{dV}{dt} = \left [\dfrac{dV}{du} \right ]\dot{x}$} \\
[10pt]
&=& m\dot{x}(t)\ddot{x}(t) + \left [ \dfrac{d}{dx}V(u) \right ] \dot{x}(t)
		&\hspace{0em} \mathrel{\#} \text{chain rule: $\dfrac{dV}{dx} = 
									     \dfrac{dV}{du} \dfrac{du}{dx}$ 
										 and 
										 $\dfrac{du}{dx} = 1
										 \, \Rightarrow
										 \dfrac{dV}{dx} = \dfrac{dV}{du}$}\\
[10pt]
&=& m\dot{x}(t)\ddot{x}(t) + \left [ \dfrac{d}{dx}V(x(t)) \right ] \dot{x}(t)
		&\hspace{0em} \mathrel{\#} \text{since $u = x(t)$} \\
[10pt]
&=& \dot{x}(t)\left [m \ddot{x}(t) + \dfrac{d}{dx}V(x(t)) \right ]
		&\hspace{0em} \mathrel{\#} \text{factor out $\dot{x}(t)$} \\
[10pt]
&=& \dot{x}(t)\Big [m \ddot{x}(t) - F(x(t)) \Big ]
		&\hspace{0em} \mathrel{\#} \text{$\dfrac{d}{dx}V(x(t)) = - F(x(t))$ 
		                                 (Equation (\ref{eqn:F}))} \\
[10pt]
&=& 0
		&\hspace{0em} \mathrel{\#} \text{Newton's Second Law: $m \ddot{x}(t) - F(x(t)) = 0$} 
\end{array}
\end{equation*}

\medskip
\noindent
So we see that the time derivative of the energy along any trajectory 
is zero, which implies that $E(x(t), \dot{x}(t))$ is independent of $t$. 
Energy is sometimes called a \emph{conserved quantity} (or \emph{constant 
of motion}) because a particle neither gains nor loses energy 
as it moves according to Newton’s Second Law.
\end{theorem}
%
%	Experiment with \input
%
% %
%	A bit on conservation of energy
%	
%	David Meyer
%	09/28/2022
%
%
\subsection{Conservation of Energy}
\label{subsec:conservation_of_energy}
Consider the case of a general force function $F(x)$. Here we define the kinetic 
energy of the system to be $\frac{1}{2}mv^{2}$. We also define the potential 
energy of the system, $V(x)$, to be the function

\begin{equation}
V(x) = - \int F(x) \, dx
\label{eqn:V(x)}
\end{equation}

\noindent
so that 

\begin{equation}
F(x) = -\dfrac{dV}{dx}
\label{eqn:F}
\end{equation}

\medskip
\noindent
Then the total energy of the system as a function of displacement and 
velocity, $E(x,v)$, is defined to be 


\begin{equation}
E(x,v) = \dfrac{1}{2}mv^{2} + V(x)
\label{eqn:E}
\end{equation}

\medskip
\noindent
One of the main reasons this energy function is important is that it is 
\emph{conserved}, meaning that its value along any trajectory is constant.
Said another way: For each trajectory $x(t)$ conforming to Newton’s 
Second Law, the system's total energy $E(x(t),\dot{x}(t))$ is independent 
of $t$.

\begin{theorem}
Suppose a particle's trajectory conforms to Newton’s Second Law 
in the form $F(x(t)) = m\ddot{x}(t)$ and let $V$ and $E$ be as 
in Equations (\ref{eqn:V(x)}) and (\ref{eqn:E}). Then the total 
energy of the system, $E(x(t),\dot{x}(t))$, is conserved. 

\bigskip
\noindent
{\bf Proof:} One way to see this is to show that 
$\dfrac{d}{dt}E(x(t),\dot{x}(t)) = 0$:

\begin{equation*}
\begin{array}{llll}
\dfrac{d}{dt} E(x,\dot{x})
&=& \dfrac{d}{dt} \bigg [\dfrac{1}{2}m(\dot{x}(t))^{2} + V(x(t)) \bigg ]	
		&\hspace{1em} \mathrel{\#} \text{Equation (\ref{eqn:E}) with $x = x(t)$ 
		                                 and $v = \dot{x}(t)$} \\	
[12pt]
&=& \dfrac{d}{dt} \bigg [ \dfrac{1}{2}m(\dot{x}(t))^{2} \bigg ] + \dfrac{d}{dt} V(x(t))
		&\hspace{1em} \mathrel{\#} \text{derivative is a linear operator \cite{wiki:linearity_of_differentiation}} \\
[12pt]
&=& m\dot{x}(t)\ddot{x}(t) + \dfrac{d}{dt} V(x(t))
		&\hspace{1em} \mathrel{\#} \text{by the chain and power rules \cite{wiki:chain_rule,wiki:power_rule}} \\
[12pt]
&=& m\dot{x}(t)\ddot{x}(t) + \dfrac{d}{dt} V(u)
		&\hspace{1em} \mathrel{\#} \text{let $u = x(t)$; this implies that 
		                                 $\dfrac{du}{dx} = 1 \Rightarrow du = dx$} \\
[12pt]
&=& m\dot{x}(t)\ddot{x}(t) + \left [\dfrac{d}{du}  V(u) \right ] \dot{x}(t)
		&\hspace{1em} \mathrel{\#} \text{by the chain rule: 
								         $\dfrac{d}{dt} V(u) = 
					                     \dfrac{dV}{du} \, \dfrac{du}{dt}$
					                     and 
							             $\dfrac{du}{dt}     = 
							             \dfrac{d}{dt} x(t)  = 
							             \dot{x}(t)$} \\
[12pt]
&=& m\dot{x}(t)\ddot{x}(t) + \left [ \dfrac{d}{dx}V(u) \right ] \dot{x}(t)
		&\hspace{1em} \mathrel{\#} du = dx \\
[12pt]
&=& m\dot{x}(t)\ddot{x}(t) + \left [ \dfrac{d}{dx}V(x(t)) \right ] \dot{x}(t)
		&\hspace{1em} \mathrel{\#} u = x(t) \\
[12pt]
&=& \dot{x}(t)\left [m \ddot{x}(t) + \dfrac{d}{dx}V(x(t)) \right ]
		&\hspace{1em} \mathrel{\#} \text{factor out $\dot{x}(t)$} \\
[12pt]
&=& \dot{x}(t)\Big [m \ddot{x}(t) - F(x(t)) \Big ]
		&\hspace{1em} \mathrel{\#} \text{$\dfrac{d}{dx}V(x(t)) = - F(x(t))$ 
		                                 (Equation (\ref{eqn:F}))} \\
[12pt]
&=& 0
		&\hspace{1em} \mathrel{\#} \text{since $m \ddot{x}(t) - F(x(t)) = 0$ 
		                                 by Newton's Second Law}
\end{array}
\end{equation*}

\bigskip
\noindent
Thus, the time derivative of the energy along any trajectory is zero which implies that
$E(x(t), \dot{x}(t))$ is independent of $t$. In summary, we may call the energy a conserved 
quantity (or constant of motion), since the particle neither gains nor loses energy as the 
particle moves according to Newton’s Second Law.
\end{theorem}


%
%
%
%	show a derivation/proof
%
% \begin{equation*}
% \begin{array}{llll}
% x
% &=& y				&\qquad \mathrel{\#} \text{comment or} \\
% &=& y				&\hspace{2em} \mathrel{\#} \text{comment} \\
% [10pt]
% \end{array}
% \end{equation*}
%
%
%
%	include an image
%
% \begin{figure}
% \center{\includegraphics[frame, scale=0.50] {images/X.png}}
% \caption{X}
% \label{fig:X}
% \end{figure}
%
\section{Conclusions}
%
%
%
\section{Acknowledgements}
%
%	LaTeX source on overleaf.com
%
\section*{\LaTeX \hspace{0.10 mm} Source}
% \url{}
%
%	get a bibliography
%
%	Note:.bib files go in ~/Library/texmf/bibtex/bib with TeXShop (MacTeX).
%	You can also use an absolute path, e.g. \bibliography{/Users/dmm/papers/bib/qc}
%
\bibliographystyle{plain}
\bibliography{qc}
%
%	done
%
\end{document} 

