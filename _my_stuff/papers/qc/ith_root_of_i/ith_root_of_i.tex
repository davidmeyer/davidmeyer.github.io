\documentclass[11pt, oneside]{article}   	% use "amsart" instead of "article" for AMSLaTeX format


% \usepackage{draftwatermark}
% \SetWatermarkText{Draft}
% \SetWatermarkScale{5}
% \SetWatermarkLightness {0.9} 
% \SetWatermarkColor[rgb]{0.7,0,0}


\usepackage{geometry}                   % See geometry.pdf to learn the layout options. There are lots.
\geometry{letterpaper}                  % ... or a4paper or a5paper or ... 
%\geometry{landscape}                   % Activate for for rotated page geometry
%\usepackage[parfill]{parskip}          % Activate to begin paragraphs with an empty line rather than an indent
\usepackage{graphicx}                   % Use pdf, png, jpg, or eps� with pdflatex; use eps in DVI mode
\usepackage{amssymb}
\usepackage{mathrsfs}
\usepackage{hyperref}
\usepackage{url}
\usepackage{subcaption}
\usepackage{authblk}
\usepackage{amsmath}
\usepackage{mathtools}
\usepackage{graphicx}
\usepackage[export]{adjustbox}
\usepackage{fixltx2e}
\usepackage{hyperref}
\usepackage{alltt}
\usepackage{color}
\usepackage[utf8]{inputenc}
\usepackage[english]{babel}
\usepackage{float}
\usepackage{bigints}
\usepackage{braket}
\usepackage{siunitx}

%
% so you can do e.g., \begin{bmatrix}[r] (or [c] or [l])
%

\makeatletter
\renewcommand*\env@matrix[1][c]{\hskip -\arraycolsep
  \let\@ifnextchar\new@ifnextchar
  \array{*\c@MaxMatrixCols #1}}
\makeatother

\newcommand{\argmax}{\operatornamewithlimits{argmax}}
\newcommand{\argmin}{\operatornamewithlimits{argmin}}

\begin{document}

\title {Is $\sqrt[i]{i}$ a Real Number?}
\author{David Meyer \\ dmm613@gmail.com}

\date{Last update: \today}							% Activate to display a given date or no date
\maketitle

Well, consider

\begin{equation*}
\begin{array}{lllll}
\sqrt[i]{i}
&=& i^{\frac{1}{i}}			
		&\qquad \qquad \qquad \qquad \mathbin{\#} \text{ $\sqrt[n]{x} = x^{\frac{1}{n}}$} \\
[12pt]
&=& i^{-i}
		&\qquad \qquad \qquad \qquad \mathbin{\#} \text{ $x^{\frac{1}{n}} = x^{-n}$} \\
[12pt]
&=& \dfrac{1}{i^{i}}
		&\qquad \qquad \qquad \qquad \mathbin{\#} \text{ $x^{-n} = \dfrac{1}{x^{n}}$} \\
[12pt]
&=& \dfrac{1}{e^{{(\frac{i\pi}{2})}^i}}
		&\qquad \qquad \qquad \qquad \mathbin{\#} \text{ $i = e^{\frac{i \pi}{2}}$ 
		                                                 \cite{notes:is_i_to_the_i_a_real_number}} \\
[12pt]
&=& \dfrac{1}{e^{\frac{i^2\pi}{2}}}
		&\qquad \qquad \qquad \qquad \mathbin{\#} \text{ ${(x^m)}^n = x^{mn}$}  \\
[12pt]
&=& \dfrac{1}{e^{-\frac{\pi}{2}}}
		&\qquad \qquad \qquad \qquad \mathbin{\#} \text{ $i^2 = -1$} \\
[12pt]
&=& e^{\frac{\pi}{2}}
		&\qquad \qquad \qquad \qquad \mathbin{\#} \text{ $\dfrac{1}{x^{-n}} = x^{n}$} \\
[12pt]
&\approx& 4.8105
		&\qquad \qquad \qquad \qquad \mathbin{\#} \text{ $e^{\frac{\pi}{2}} \in \mathbb{R} 
														 \Rightarrow \sqrt[i]{i} \in \mathbb{R}$}
\end{array}
\end{equation*}

\bigskip
\noindent
So apparently $\sqrt[i]{i}$ is a real number and in particular $\sqrt[i]{i} = e^{\frac{\pi}{2}} \approx 4.8105$.
%
%
%
% \section*{Acknowledgements}
%
%	LaTeX source on overleaf.com
%
% \section*{\LaTeX \hspace{0.10 mm} Source}
% \url{}
%
%	get a bibliography
%
%	Note:.bib files go in ~/Library/texmf/bibtex/bib with TeXShop (MacTeX).
%	You can also use an absolute path, e.g. \bibliography{/Users/dmm/papers/bib/qc}
%
\bibliographystyle{plain}
\bibliography{qc}
%
%	done
%
\end{document} 

