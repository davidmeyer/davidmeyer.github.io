
\documentclass{article}

% \usepackage{draftwatermark}
% \SetWatermarkText{Draft}
% \SetWatermarkScale{5}
% \SetWatermarkLightness {0.9} 
% \SetWatermarkColor[rgb]{0.7,0,0}


\usepackage{geometry}                		% See geometry.pdf to learn the layout options. There are lots.
\geometry{letterpaper}                   		% ... or a4paper or a5paper or ... 
%\geometry{landscape}                		% Activate for for rotated page geometryhttps://www.washingtonpost.com/world/europe/amid-impeachment-probe-gordon-sondland-is-overseeing-a-renovation-of-his-residence-that-has-cost-1-million-in-taxpayer-money/2019/10/16/d0eece92-ef86-11e9-bb7e-d2026ee0c199_story.html?tid=sm_tw
%\usepackage[parfill]{parskip}    		% Activate to begin paragraphs with an empty line rather than an indent
\usepackage{graphicx}				% Use pdf, png, jpg, or eps� with pdflatex; use eps in DVI mode
								% TeX will automatically convert eps --> pdf in pdflat						\label{thm:integral_domain}

								% TeX will automatically convert eps --> pdf in pdflatex		
\usepackage{amssymb}
\usepackage{amsmath}
\usepackage{amsthm}
\usepackage{mathrsfs}
\usepackage[hyphens,spaces,obeyspaces]{url}
\usepackage{url}
\usepackage{hyperref}
\usepackage{subcaption}
\usepackage{authblk}
\usepackage{mathtools}
\usepackage{graphicx}
\usepackage[export]{adjustbox}
\usepackage{fixltx2e}
\usepackage{hyperref}
\usepackage{alltt}
\usepackage{color}
\usepackage[utf8]{inputenc}
\usepackage[english]{babel}
\usepackage{float}
\usepackage{bigints}
\usepackage{braket}
\usepackage{siunitx}
\usepackage{mathtools}


\usepackage{tikz}
\usepackage{verbatim}
\usetikzlibrary{plotmarks}
\usepackage{pgfplots}
\usepackage{amsmath}
\usepackage{relsize}
 \usepackage[hyphenbreaks]{breakurl}
 
 \newtheorem{thm}{Theorem}[section]
% \newtheorem{defn}[thm]{Definition}
\theoremstyle{definition}
\newtheorem{definition}{Definition}[section]
\newtheorem{proposition}{Proposition}[section]
\newtheorem{lemma}{Lemma}[section]
\newtheorem{example}{Example}[section]

\newcommand{\argmax}{\operatornamewithlimits{argmax}}
\newcommand{\argmin}{\operatornamewithlimits{argmin}}

\title{$\sqrt{?} = \sqrt[3]{ x \sqrt[3] {x \sqrt[3]{x \sqrt[3]{x \cdots}}}}$}
\author{David Meyer \\ dmm613@gmail.com}
\date{Last update: \today}

\begin{document}
\maketitle

\noindent
Consider:

\begin{center}
\begin{equation*}
\begin{array}{llll}
\sqrt{y}
&:=&  \sqrt[3]{ x \sqrt[3] {x \sqrt[3]{x \sqrt[3]{x \cdots}}}}
				&\qquad \qquad \mathrel{\#} \text{define $y$} \\
[15pt]
&\Rightarrow& y^{\frac{1}{2}} = x^{\frac{1}{3}} \,  x^{\frac{1}{9}} \,  x^{\frac{1}{27}} \,  \cdots
				&\qquad \qquad \mathrel{\#} \text{rewrite radicals as fractions} \\
[15pt]
&\Rightarrow& y^{\frac{1}{2}} = x^{\frac{1}{3} + \frac{1}{9} + \frac{1}{27} + \cdots}
				&\qquad \qquad \mathrel{\#} \text{rewrite product as a sum ($x^{a}x^{b} = x^{a + b}$)}  \\
[12pt]
&\Rightarrow& y^{\frac{1}{2}} = x^{\sum\limits_{k = 1}^{\infty} {(\frac{1}{3})}^{k}}
				&\qquad \qquad \mathrel{\#} \sum\limits_{k = 1}^{\infty} {(\frac{1}{3})}^{k}  = \frac{1}{3} + \frac{1}{9} + \frac{1}{27} +  \cdots \\
[15pt]
&\Rightarrow& y^{\frac{1}{2}} = x^{\sum\limits_{k = 0}^{\infty} {(\frac{1}{3})}^{k} - 1}
				&\qquad \qquad \mathrel{\#}  \sum\limits_{k = 1}^{\infty} {(\frac{1}{3})}^{k}
				=  \sum\limits_{k = 0}^{\infty} {(\frac{1}{3})}^{k} \! -  {(\frac{1}{3})}^0 
				= \sum\limits_{k = 0}^{\infty} {(\frac{1}{3})}^{k} \! - 1 \\
[15pt]
&\Rightarrow& y^{\frac{1}{2}} = x^{\frac{1}{1- \frac{1}{3}}  - 1}
				&\qquad \qquad \mathrel{\#} \text{geometric series:} \sum\limits_{k = 0}^{\infty}  ar^k  = \dfrac{a}{1-r} \text{ for }  |r| < 1 \text{; here }  a = 1  \text{ and }  r = \frac{1}{3}  \\
[15pt]
&\Rightarrow& y^{\frac{1}{2}} = x^{\frac{3}{2}- 1}
				&\qquad \qquad \mathrel{\#}  \dfrac{1}{1 - \frac{1}{3}}  = \frac{3}{2} \\
[15pt]
&\Rightarrow& y^{\frac{1}{2}} = x^{\frac{1}{2}}
				&\qquad \qquad \mathrel{\#} \frac{3}{2} - 1 = \frac{1}{2} \\
[15pt]
&\Rightarrow& y  = x
				&\qquad \qquad \mathrel{\#} \text{square both sides}
\end{array}
\end{equation*}
\end{center}

\bigskip
\noindent
and so apparently $\mathlarger{\mathlarger{\sqrt{x} = \sqrt[3]{ x \sqrt[3] {x \sqrt[3]{x \sqrt[3]{x \cdots}}}}}}$

\end{document}
