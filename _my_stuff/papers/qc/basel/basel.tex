\documentclass[11pt, oneside]{article}   	% use "amsart" instead of "article" for AMSLaTeX format


% \usepackage{draftwatermark}
% \SetWatermarkText{Draft}
% \SetWatermarkScale{5}
% \SetWatermarkLightness {0.9} 
% \SetWatermarkColor[rgb]{0.7,0,0}


\usepackage{geometry}                           % See geometry.pdf to learn the layout options. There are lots.
\geometry{letterpaper}                          % ... or a4paper or a5paper or ... 
%\geometry{landscape}                           % Activate for for rotated page geometry
%\usepackage[parfill]{parskip}                  % Activate to begin paragraphs with an empty line rather than an indent
\usepackage{graphicx}                           % Use pdf, png, jpg, or eps� with pdflatex; use eps in DVI mode
\usepackage{amssymb}
\usepackage{mathrsfs}
\usepackage{hyperref}
\usepackage{url}
\usepackage{subcaption}
\usepackage{authblk}
\usepackage{amsmath}
\usepackage{mathtools}
\usepackage{graphicx}
\usepackage[export]{adjustbox}
\usepackage{fixltx2e}
\usepackage{hyperref}
\usepackage{alltt}
\usepackage{color}
\usepackage[utf8]{inputenc}
\usepackage[english]{babel}
\usepackage{float}
\usepackage{bigints}
\usepackage{braket}
\usepackage{siunitx}
\usepackage{relsize}
\usepackage{setspace}


%
% so you can do e.g., \begin{bmatrix}[r] (or [c] or [l])
%

\newcommand{\argmax}{\operatornamewithlimits{argmax}}
\newcommand{\argmin}{\operatornamewithlimits{argmin}}


\title{A Few Notes on the Basel Problem}
\author{David Meyer \\ dmm613@gmail.com}

\date{Last update: \today}							% Activate to display a given date or no date


\begin{document}
\maketitle

\section{Introduction}
The Basel problem \cite{wiki:basel} is an important problem in
number theory \cite{wiki:number_theory} that was first posed by
Pietro Mengoli in 1650 and solved by Leonhard Euler in 1734. The
Basel problem so is named for the Swiss city in whose university
two of the Bernoulli brothers successively served as professor of
mathematics (Jakob, 1687 - 1705, and Johann, 1705 - 1748).
Coincidentally Euler was born in Basel.

\bigskip
\noindent
The Basel problem asks whether the infinite sum

\begin{equation*}
\mathlarger{\sum\limits_{n = 1}^{\infty} \frac{1}{n^2} = \frac{1}{1^2}  + \frac{1}{2^2} + \frac{1}{3^2} + \cdots}
\end{equation*}

\bigskip
\noindent
has a closed form solution, that is, does it converge to a finite
number and if it does, what number does it converge to?  An
example of a related infinite series that does not converge is
the harmonic series \cite{wiki:harmonic}:

\begin{equation*}
\mathlarger{\sum\limits_{n = 1}^{\infty} \frac{1}{n} = \frac{1}{1} + \frac{1}{2} + \frac{1}{3} + \cdots = \infty}
\end{equation*}

\bigskip
\noindent
Interestingly both of the Bernoulli brothers found proofs that
the harmonic series diverges.

\bigskip
\noindent
The Basel problem resisted solution for some 84 years until the
then 26 year old Euler finally solved it.  Euler's surprising
solution is

\begin{equation*}
\mathlarger{\sum\limits_{n = 1}^{\infty} \frac{1}{n^2} = 
\frac{1}{1^2}  + \frac{1}{2^2} + \frac{1}{3^2} + \cdots = \frac{\pi^2}{6}}
\end{equation*}


\bigskip
\noindent
Note that both the harmonic series and the Basel problem are
instances of the famous Riemann zeta function \cite{wiki:zeta},
$\zeta(s)$.  In particular, the harmonic series is $\zeta(1)$ and
the Basel problem is $\zeta(2)$.

\section{Euler's Solution}
\label{sec:eulers_solution}
So how did Euler solve the Basel problem? The summary is that
Euler's extraordinary mathematical intuition led him to consider
sinthe Maclauren series \cite{wiki:taylor} for $\sin x$ and to
compare its coefficients to the coefficients of a polynomial
constructed from the zeros of $\sin x$.  In particular, he
reasoned that the coefficients of the Maclaurin series and the
polynomial must be equal for equal powers of $x$.

\bigskip
\noindent
This approach seems practical given that we know the Maclaurin
series for $\sin x$, we know the zeros of $\sin x$, and because
the Fundamental Theorem of Algebra \cite{wiki:ftoa} (and more
specifically the Weierstrass Factorization Theorem
\cite{wiki:weierstrass}) tells us that we can construct a
polynomial for $\sin x$ from its zeros\footnote{The terms "zero"
and "root" appear to be used interchangeably in mathematics
linterature.}.


\bigskip
\noindent
So how exactly did Euler arrive at his fantastic solution?

\subsection{Maclaurin (Taylor) Series for $\sin x$}
The story begins with the Maclaurin series \cite{wiki:taylor} for
$\sin x$. This series was well-known in Euler's time.  Note that
in the following discussion of the Maclaurin series for $\sin x$
we are using the $f^{(n)}$ notation for the $n^{\text{th}}$
derivative of $f$, noting that $f^{(0)} = f$.


\begin{equation*}
\begin{array}{lllll}
\sin x
&=& \sum\limits_{n=0}^{\infty} \frac{f^{(n)}(0)}{n!} x^n
        &\mathrel{\#} \text{definition of the Maclaurin series}  \\
[12pt]
&=& \frac{f^{(0)}(0)}{0!} x^0 +  \frac{f^{(1)}(0)}{1!} x^1 +
        \frac{f^{(2)}(0)}{2!} x^2 + \frac{f^{(3)}(0)}{3!} x^3 + \cdots
        &\mathrel{\#} \text{expand terms}                               \\
[12pt] 
&=& \frac{\sin 0}{0!}x^0 + \frac{\cos 0}{1!}x^1 +  \frac{-\sin
        0}{2!}x^2 +  \frac{-\cos 0}{3!}x^3 +  \frac{\sin 0}{4!} x^4 \cdots
        &\mathrel{\#} f(x) = \sin x \\ 
[12pt]
&=& 0 +  (\frac{1}{1!}) x^1 + 0 + (\frac{-1}{3!}) x^3+ 0 +
        (\frac{1}{5!}) x^5 + \cdots
        &\mathrel{\#} \text{the sequence is $\{0,1,0,-1,...\}$}   \\
[12pt]
&=&  x - \frac{x^3}{3!} + \frac{x^5}{5!}  - \frac{x^7}{7!} +
        \cdots
        &\mathrel{\#} \text{simplify} \\ 
\end{array}
\end{equation*}

\bigskip
\noindent
As mentioned above, one of Euler's key insights was to observe
that if we can find a polynomial for $\sin x$ then the
coefficients of terms of the same power in the Maclaurin series
and the polynomial must equal one another.

\bigskip
\noindent
So how did Euler find a polynomial for $\sin x$?

\subsection{Euler's Polynomial for $\sin x$}
We saw in Section \ref{sec:eulers_solution} that every
non-constant single-variable polynomial with complex coefficients
has at least one complex root and that a (possibly infinite)
polynomial can be constructed from the linear product of a
function's roots. Since we know that the roots of $\sin x$ are
$0, \pm \pi, \pm 2\pi, \hdots$ we can directly write down a
polynomial $P(x)$ for $\sin x$:

\begin{equation*}
\sin x =  A x (x - \pi)(x + \pi) (x - 2\pi)(x + 2\pi) (x - 3\pi) (x + 3\pi) \cdots
\end{equation*}

\bigskip
\noindent
The tricky part now is finding $A$. So how did Euler do it? 

\bigskip
\noindent
Euler noticed that for infinitesimally small values of $x$
$\sin x = x$, that is\footnote{Note that this limit occurs at
the zero root of $\sin x$, that is, $\sin 0 = 0$.}

\bigskip
\begin{equation*}
\lim_{x \to 0} \frac{\sin x}{x} = 1
\end{equation*}

\bigskip
\noindent
Continuing, we also know that

\begin{equation*}
\lim_{x \to 0} \dfrac{x (x - \pi)(x+\pi)(x - 2\pi)(x + 2\pi)(x - 3\pi)(x + 3\pi) \cdots}{x} = (-\pi^2)(-(2\pi)^2)(-(3\pi)^2) \cdots
\end{equation*}

\bigskip
\noindent
Together these imply that

\begin{equation*}
\begin{array}{lllll}
% {\displaystyle \lim_{x \to 0} \frac{\sin x}{x} = 1}
1 
&=&  A (x - \pi)(x + \pi) (x - 2\pi)(x+2\pi) \cdots
                &\qquad \qquad \mathrel{\#} {\displaystyle 1 =  \lim_{x \to 0} \frac{\sin x}{x} = \lim_{x \to 0}  \frac{P(x)}{x}} \\
[22pt]
&\Rightarrow& A = \dfrac{1}{(x - \pi)(x + \pi) (x - 2\pi)(x+2\pi)
\cdots }
                &\qquad \qquad \mathrel{\#} \text{solve for $A$} \\
[22pt]
&\Rightarrow& A = \dfrac{1}{(-\pi^2)(-(2\pi)^2) (-(3\pi)^2) \cdots } 
                &\qquad \qquad \mathrel{\#} \text{evaluate $A$ at the 0 root (set $x = 0)$} \\
[22pt]
&\Rightarrow& A = \dfrac{1}{(-\pi^2) (- 2^2 \pi^2) (-3^2 \pi^2) \cdots}
                &\qquad \qquad \mathrel{\#} \text{simplify}  
\end{array}
\end{equation*}

\bigskip
\noindent
So now we know that 

\begin{equation*}
\begin{array}{lllll}
\sin x 
&=&  A x (x - \pi)(x + \pi) (x - 2\pi)(x + 2\pi) (x - 3\pi) (x + 3\pi) \cdots                                                                                                                    &\mathrel{\#} \sin x = P(x) \\   
[22pt]                                                     
&=& \dfrac{x (x - \pi)(x+\pi)(x-2\pi)(x+2\pi)(x-3\pi)(x+3\pi) \cdots}{(-\pi^2)(-2^2\pi^2)(-3^2\pi^2) \cdots}                                                                &\mathrel{\#} A = ((-\pi^2) (- 2^2 \pi^2) (-3^2 \pi^2)  \cdots)^{-1}   \\
[22pt]
&=& \dfrac{x (x^2 - \pi^2)(x^2-4\pi^2)(x^2 - 9\pi^2) \cdots}{(-\pi^2)(-2^2\pi^2)(-3^2\pi^2) \cdots}                                                                          &\mathrel{\#} \text{multiply like terms}  \\
[22pt]
&=& x \cdot \Bigg [\dfrac{x^2 - \pi^2}{-\pi^2} \Bigg ] \cdot  \Bigg [\dfrac{x^2 - 4 \pi^2}{- 4 \pi^2} \Bigg ]  \cdot  \Bigg [\dfrac{x^2 - 9 \pi^2}{- 9 \pi^2} \Bigg ]  \cdots 
                                                                                                                                                                                                                              &\mathrel{\#} \text{collect terms, multiply squares} \\
[22pt]
&=& x \cdot \Bigg [\dfrac{\pi^2 - x^2}{\pi^2} \Bigg ] \cdot  \Bigg [\dfrac{4 \pi^2 -  x^2}{4 \pi^2} \Bigg ]  \cdot  \Bigg [\dfrac{9 \pi^2 - x^2}{9 \pi^2} \Bigg ]  \cdots 
                                                                                                                                                                                                                              &\mathrel{\#} \text{get rid of $-$ in the denominators} \\
[22pt]
&=& x \cdot  \Bigg [1 - \dfrac{x^2}{\pi^2} \Bigg ] \cdot \Bigg [1 - \dfrac{x^2}{4\pi^2} \Bigg ] \cdot  \Bigg [1 - \dfrac{x^2}{9\pi^2} \Bigg ] \cdots &\mathrel{\#} \text{simplify}   \\
[22pt]
&=& x \cdot  \Bigg [1 - \dfrac{x^2}{(1 \pi)^2} \Bigg ] \cdot \Bigg [1 - \dfrac{x^2}{(2\pi)^2} \Bigg ] \cdot  \Bigg [1 - \dfrac{x^2}{(3\pi)^2} \Bigg ] \cdots &\mathrel{\#} \text{put into a convenient form}   \\
[22pt]
&=& \mathlarger{ x \prod\limits_{n=1}^\infty \Bigg (1 - \frac{x^2}{(n\pi)^2} \Bigg ) }                                                                                              &\mathrel{\#} \text{Euler's polynomial for $\sin x$}
\end{array}
\end{equation*}

\bigskip
\noindent
So we see that Euler's polynomial $P(x)$ for $\sin x$ is 

\bigskip
\begin{equation}
\sin x = x \prod\limits_{n=1}^\infty \Bigg (1 - \frac{x^2}{(n\pi)^2} \Bigg )
\label{eqn:sine}
\end{equation}

\bigskip
\noindent
and so

\begin{equation*}
\frac{\sin x}{x} = \prod\limits_{n=1}^\infty \Bigg (1 - \frac{x^2}{(n\pi)^2} \Bigg )
\end{equation*}

\bigskip
\noindent
Now, we know that $\frac{P(x)}{x} = \frac{\sin x}{x}$. But we also know that the Maclaurin series for $\frac{\sin x}{x}$ is  

\bigskip
\begin{equation*}
\frac{\sin x }{x}= 1 - \frac{x^2}{3!} + \frac{x^4}{5!} - \frac{x^6}{7!} + \cdots
\end{equation*}


% \bigskip
% \noindent
% Euler also noticed that since the Maclaurin series for $\sin x$ is 
% 
% \begin{equation*}
% \sin x = \sum\limits_{n = 0}^\infty (-1)^2 \frac{x^{(2n+1)}} {(2n+1)!}
% \end{equation*}
% 
% \bigskip
% \noindent
% If we now multiply by $\frac{1}{x}$ term by term we get 
% 
% \bigskip
% \begin{equation*}
% \frac{\sin x}{x} = \sum\limits_{n = 1}^\infty (-1)^n \Bigg [\frac{1}{x} \Bigg ]\frac{x^{(2n+1)}} {(2n+1)!} = \sum\limits_{n = 1}^\infty (-1)^n \frac{x^{2n}} {(2n+1)!} 
% \end{equation*}

\bigskip
\noindent
Looking at the coefficients of the $x^2$ terms of
$\frac{\sin x}{x}$ we see that for the Maclauren series we have
$- \frac{1}{3!} = - \frac{1}{6}$ and for $\frac{P(x)}{x}$ we have

\bigskip
\begin{equation*}
\mathlarger{ - \bigg (\frac{1}{1^2 \pi^2} +  \frac{1}{2^2 \pi^2} +  \frac{1}{3^2 \pi^2} + \cdots \bigg ) 
= - \frac{1}{\pi^2} \bigg (\frac{1}{1^2} +  \frac{1}{2^2 } +  \frac{1}{3^2} + \cdots \bigg ) 
= - \frac{1}{\pi^2} \sum\limits_{n = 1}^\infty \frac{1}{n^2}}
\end{equation*}

\bigskip
\noindent
Since the coefficients of the $x^2$ terms of the Maclaurin series
and the polynomial $P(x)$ must be equal we see that

\begin{equation*}
- \dfrac{1}{6} = - \dfrac{1}{\pi^2} \sum\limits_{n=1}^\infty \dfrac{1}{n^2}
\end{equation*}

\bigskip
\noindent
Multiplying both sides by $- \pi^2$ we get

\begin{equation*}
\dfrac{\pi^2}{6} = \sum\limits_{n=1}^\infty \dfrac{1}{n^2}
\end{equation*}

\bigskip
\noindent
which amazingly is Euler's solution to the Basel problem. 


\section{Summary of Euler's Solution of the Basel Problem}
To summarize Euler's argument, first recall that Euler knew that
the Maclaurin series for the sine function was

\bigskip
\begin{equation*}
\sin x = x - \frac{x^3}{3!} + \frac{x^5}{5!} - \frac{x^7}{7!} + \cdots
\end{equation*}

\bigskip
\noindent
Now, if we divide both sides by $x$ we get\footnote{Recall that
we needed ${\displaystyle \lim_{x \to 0}} \frac{\sin x}{x} = 1$
to solve for $A$ in $P(x)$, the polynomial for $\sin x$.}

\bigskip
\begin{equation*}
\frac{\sin x }{x}= 1 - \frac{x^2}{3!} + \frac{x^4}{5!} - \frac{x^6}{7!} + \cdots
\end{equation*}


\bigskip
\noindent
We also know from the Weierstrass Factorization Theorem
\cite{wiki:weierstrass} that every entire function
\cite{wiki:entire_function} can be represented as a (possibly
infinite) product involving its zeroes. So Euler assumed that it
must be possible to represent $\sin x$ as an infinite product of
linear factors given by its roots. Using this machinery and
knowing that the zeros of $\sin x$ occur at $0, \pm \pi, \pm
2\pi, \hdots$ we see that

\begin{equation*}
\begin{array}{lllll}
\mathlarger{\frac{\sin x }{x}}
&=& \mathlarger{ \bigg (1 - \frac{x}{\pi} \bigg ) \bigg (1 + \frac{x}{\pi} \bigg )
                             \bigg (1 - \frac{x}{2\pi} \bigg ) \bigg (1 + \frac{x}{2 \pi} \bigg )
                             \bigg (1 - \frac{x}{3\pi} \bigg ) \bigg (1 + \frac{x}{3 \pi} \bigg )
                             \cdots} \\
[22pt]
&=& \mathlarger{ \bigg (1 - \frac{x^2}{\pi^2} \bigg ) 
                            \bigg (1 - \frac{x^2}{4 \pi^2} \bigg ) 
                            \bigg (1 - \frac{x^2}{9 \pi^2} \bigg ) 
                            \cdots}
 \end{array}
\end{equation*}

\bigskip
\noindent
Multiplying this product out and collecting the $x^2$ terms
(which we are allowed to do by Newton's Identities
\cite{wiki:newton_identies}), we see by induction that the $x^2$
coefficient of $\frac{\sin x}{x}$ is

\bigskip
\bigskip
\begin{equation*}
\mathlarger{ - \bigg (\frac{1}{\pi^2} +  \frac{1}{4 \pi^2} +  \frac{1}{9 \pi^2} + \cdots \bigg ) = - \frac{1}{\pi^2} \sum\limits_{n = 1}^\infty \frac{1}{n^2}}
\end{equation*}

\bigskip
\noindent
We also know that the coefficient of $x^2$ from Maclaurin series
for $\frac{\sin x}{x}$ is  

\begin{equation*}
- \dfrac{1}{3!} = - \dfrac{1}{6}
\end{equation*}

\bigskip
\noindent
One of Euler's many insights was that the coefficients in both
the Maclaurin series and the polynomial for $\frac{\sin x}{x}$
must be equal.  So for the $x^2$ term that means that

\bigskip
\begin{equation*}
- \dfrac{1}{6} = - \frac{1}{\pi^2} \sum\limits_{n = 1}^\infty \frac{1}{n^2}
\end{equation*}

\bigskip
\noindent
Multiplying both sides of this equation by $-\pi^2$ gives Euler's
solution to the Basel problem:  

\begin{equation*}
\dfrac{\pi^2}{6} = \sum\limits_{n = 1}^\infty \frac{1}{n^2}
\end{equation*}

\section{A Few Interesting Consequences}
One of the first interesting consequences of Euler's proof is
that when we plug $\frac{\pi}{2}$ into the product formula for
$\sin x$ (Equation \ref{eqn:sine}) we get the Wallis Product
\cite{wiki:wallis}. Why? Well, since $\sin \frac{\pi}{2} = 1$ 
and with a bit of rearranging we see that


\newpage
\begin{equation*}
\begin{array}{lllll}
\sin x
&=&  \mathlarger{x \prod\limits_{n=1}^\infty \Bigg (1 - \dfrac{x^2}{(n\pi)^2} \Bigg )}
		&\qquad \mathrel{\#}  \text{$\sin x = P(x)$ (Equation \ref{eqn:sine})} \\
[22pt]
&=& \mathlarger{x \prod\limits_{n=1}^\infty \Bigg (\dfrac{n^2\pi^2 - x^2}{n^2\pi^2} \Bigg )}                                   
		&\qquad \mathrel{\#}  \text{get a common denominator, multiply through} \\
[22pt]
&\Rightarrow& \mathlarger{\sin \frac{\pi}{2}= \dfrac{\pi}{2} \prod\limits_{n=1}^\infty \Bigg (\dfrac{n^2\pi^2 - \frac{\pi^2}{4}}{n^2\pi^2} \Bigg )}                                                                                                                                                                              		&\qquad \mathrel{\#}  \text{set $\mathlarger{x = \frac{\pi}{2}}$} \\
[22pt]
&\Rightarrow& \mathlarger{1 =  \dfrac{\pi}{2}  \prod\limits_{n=1}^\infty \Bigg ( \dfrac{ \pi^2 \big ( n^2 - \frac{1}{4} \big )} {\pi^2 n^2} \Bigg )}                                                                                                                                                                        		&\qquad \mathrel{\#} \text{$\sin \frac{\pi}{2} = 1 $, factor out $\pi^2$} \\
[22pt]
&\Rightarrow& \mathlarger{1 =  \dfrac{\pi}{2}  \prod\limits_{n=1}^\infty \Bigg ( \dfrac{ \big ( n^2 - \frac{1}{4} \big )} {n^2} \Bigg )}                                                                                                                                                                                 		&\qquad \mathrel{\#}  \text{cancel $\pi^2$} \\
[22pt]
&\Rightarrow&  \mathlarger{1 = \dfrac{\pi}{2}  \prod\limits_{n=1}^\infty \Bigg ( \dfrac{ \big ( \frac{4n^2 - 1}{4}  \big )} {n^2} \Bigg )}
		&\qquad \mathrel{\#}  \text{get a common denominator} \\
[22pt]
&\Rightarrow& \mathlarger{1 = \dfrac{\pi}{2} \prod\limits_{n=1}^\infty \bigg ( \frac {4n^2 - 1} {4n^2} \bigg )} 
		&\qquad \mathrel{\#}  \text{multiply right side by $\mathlarger{1= \frac{\frac{1}{n^2}}{\frac{1}{n^2}}}$} \\
[22pt]
&\Rightarrow&  \mathlarger{\dfrac{2}{\pi} = \prod\limits_{n=1}^\infty \bigg (\frac {4n^2 - 1} {4n^2} \bigg )}            
		&\qquad \mathrel{\#}  \text{multiply both sides by $\dfrac{2}{\pi}$} \\
[22pt]
% &\Rightarrow&  \mathlarger{\dfrac{2}{\pi} = \lim\limits_{a \to \infty} \prod\limits_{n=1}^a \bigg (\frac {4n^2 - 1} {4n^2} \bigg )}            
%		&\qquad \mathrel{\#}  \text{convert to limit} \\
% [22pt]
&\Rightarrow&  \mathlarger{\dfrac{\pi}{2} = \dfrac{1}{\prod\limits_{n=1}^\infty \Big (\frac{4n^2 - 1}{4n^2}\Big )}}          
		&\qquad \mathrel{\#} \text{take the reciprocal of both sides} \\
% [22pt]
% &\Rightarrow&  \mathlarger{\dfrac{\pi}{2} = \lim\limits_{a \to \infty} \dfrac{1}{\prod\limits_{n=1}^a \Big (\frac{4n^2 - 1}{4n^2}\Big )}}     
%		&\qquad \mathrel{\#} \text{quotient law for limits \cite{properties_of_limits}} \\
[22pt]
&\Rightarrow&  \mathlarger{\dfrac{\pi}{2} = \prod\limits_{n=1}^\infty \left (\dfrac{1}{\left (\frac{4n^2 - 1}{4n^2} \right )} \right )}
		&\qquad \mathrel{\#} \dfrac{1}{\prod\limits_{n=1}^\infty \Big (\frac{4n^2 - 1}{4n^2}\Big )} =
							 \prod\limits_{n=1}^\infty \left (\dfrac{1}{\Big (\frac{4n^2 - 1}{4n^2} \Big )} \right ) \\
[22pt]
% &\Rightarrow&  \mathlarger{\dfrac{\pi}{2} = \prod\limits_{n=1}^\infty \Bigg (\frac{4n^2}{4n^2 - 1}\Bigg )}   
%		&\qquad \mathrel{\#} \text{reciprocals again...} \\		 	
% [22pt]
&\Rightarrow&  \mathlarger{\dfrac{\pi}{2} = \prod\limits_{n=1}^\infty \Bigg (\frac {4n^2} {4n^2 - 1} \Bigg )}          
		&\qquad \mathrel{\#}  \dfrac{1}{\Big (\frac{4n^2 - 1}{4n^2} \Big )} = \frac {4n^2} {4n^2 - 1}  \\
[22pt]
&\Rightarrow&  \mathlarger{\dfrac{\pi}{2} = \prod\limits_{n=1}^\infty \bigg (\frac {(2n)(2n)} {(2n+1)(2n-1)} \bigg )}            
		&\qquad \mathrel{\#} \text{factor numerator and denominator} \\
		
\end{array}
\end{equation*}


\bigskip
\noindent
So we wind up with


\begin{equation*}
\frac{\pi}{2}  
= \prod\limits_{n =1}^\infty \Bigg  ( \frac{(2n)(2n)}{(2n-1)(2n+1)}  \Bigg ) 
= \Big ( \frac {2}{1} \cdot \frac {2}{3} \Big ) \cdot \Big ( \frac {4}{3} \cdot \frac {4}{5} \Big ) 
   \cdot \Big (\frac {6}{5} \cdot \frac {6}{7} \Big ) \cdot \Big (\frac{8}{7} \cdot \frac{8}{9}\Big ) \cdots 
\end{equation*}


\bigskip
\noindent
that is, the Wallis Product.

\section{Conclusions}
{\setstretch{2.25}
\doublespacing Today many proofs that ${\displaystyle \zeta(2) =
\sum\limits_{n=1}^\infty \frac{1}{n^2} = \frac{\pi^2}{6}}$
exist. Many of these proofs are considered, at least by some, to
be more rigorous than original Euler's simple and elegant
"proof". See for example \cite{proofwikibasels}. \par}

%
%
%
\section*{Acknowledgements}
Thanks to Tim Griffin for pointing out that one of the steps
in my derivation of the Wallis Product was shakey. 
%
%	LaTeX source on overleaf.com
%
\section*{\LaTeX \hspace{0.10 mm} Source}
\url{https://www.overleaf.com/read/xsnvysdcfcvd}
%
%	get a bibliography
%
%	Note:.bib files go in ~/Library/texmf/bibtex/bib with TeXShop (MacTeX).
%	You can also use an absolute path, e.g. \bibliography{/Users/dmm/papers/bib/qc}
%
\bibliographystyle{plain}
\bibliography{qc}
%
%	done
%
\end{document} 


