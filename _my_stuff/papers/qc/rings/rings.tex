\documentclass{article}
\usepackage{amsmath}
\usepackage{tcolorbox}
\usepackage{amssymb}
\usepackage{amsmath}
\usepackage{amsthm}
\usepackage{mathrsfs}
\usepackage{hyperref}
\usepackage{url}
\usepackage{subcaption}
\usepackage{authblk}
\usepackage{mathtools}
\usepackage{graphicx}
\usepackage[export]{adjustbox}
\usepackage{fixltx2e}
\usepackage{hyperref}
\usepackage{alltt}
\usepackage{color}
\usepackage[utf8]{inputenc}
\usepackage[english]{babel}
\usepackage{float}
\usepackage{bigints}
\usepackage{braket}
\usepackage{siunitx}

\theoremstyle{definition}
\newtheorem{thm}{Theorem}[section]
% \newtheorem{defn}[thm]{Definition}
\newtheorem{definition}{Definition}[section]

\newtcolorbox{mybox}{fontupper=\small}

\begin{document}

\begin{definition}
Let $I$ be a subset of a ring $R$. Then an additive subgroup of $R$ having the property that

\begin{equation*}
ra \in I \text{ for }  a \in I, \: r \in R
\end{equation*}

\bigskip
\noindent
is called a left ideal of $R$.  Similarly

\begin{equation*}
ar \in I \text{ for }  a \in I, \: r \in R
\end{equation*}

\bigskip
\noindent
is called a right ideal of $R$. If an ideal is both a right and a left ideal then we call it a two-sided ideal of $R$, or simply an ideal of $R$.
\end{definition}

\noindent
 We say that an ideal $I$ of $R$ is proper if $I \neq R$.  We say that is it non-trivial if $I \neq R$ and  $I \neq  0$.

\bigskip
\noindent
I've seen several different notations for ideals including, among others:  $rI = \{ri \mid r \in R, i \in I\}$ and 
$rI \subset I$ for $\forall r \in R$. 

\bigskip
\noindent
So in words: An \emph{ideal} $I$ is a subset of a ring $R$ such that

\begin{itemize}
\item $I$ is a subgroup of $R$ under addition (so $0 \in I$ and hence $I \neq \emptyset$) 
\item $I$  is not only closed under multiplication but also satisfies the \emph{stronger} property that it  
"absorbs" all of the elements of $R$ under multiplication: $\forall r \in R \Rightarrow rI \subset I$
\end{itemize}

\bigskip
\bigskip
\begin{mybox}
\begin{equation*}
\begin{array}{lllll}
&& \underline{\mathbf{Groups}} \\ 
\bullet&  \phi: G \xrightarrow{hom} H                            &\quad  \mathrel{\#} \text{$\phi$ is a group homomorphism: $G \simeq H$}                               \\
\bullet&  \ker \phi = N                                                    &\quad  \mathrel{\#} \text{$N$ is a \textbf{{\color{red} normal subgroup}} of $G$ ($N \lhd G$)}  \\
\bullet&  \text{if $\phi$ is onto then $H \simeq G/N$}   &\quad \mathrel{\#} \text{First Isomorphism Theorem for groups}                                                \\
\end{array}
\end{equation*}
\end{mybox}

\begin{mybox}
\begin{equation*}
\begin{array}{lllll}
&& \underline{\mathbf{Rings}} \\ 
\bullet&  \phi: R \xrightarrow{hom} S                           &\quad  \mathrel{\#} \text{$\phi$ is a ring homomorphism: $R \simeq S$}   \\
\bullet&  \ker \phi = I                                                     &\quad  \mathrel{\#} \text{$I$ is a two-sided \textbf{{\color{red} ideal}} in $R$ with $1 \notin I$}  \\
\bullet&  \text{if $\phi$ is onto then $S \simeq R/I$}   &\quad \mathrel{\#} \text{First Isomorphism Theorem for rings}
\end{array}
\end{equation*}
\end{mybox}

\bigskip
\bigskip
\noindent
\textbf{Notes}
\begin{itemize}
\item An ideal is the same thing as the kernel of a ring homomorphism $\phi$
\item Ideals are to rings as normal subgroups are to groups
\end{itemize}


\end{document}
