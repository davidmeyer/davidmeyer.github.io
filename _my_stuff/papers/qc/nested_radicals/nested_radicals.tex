\documentclass{article}
%
%
%	nested_radicals.tex
%
%	David Meyer
%	dmm613@gmail.com
%	22 Aug 2022
%
%
%   get various packages
%
\usepackage[margin=0.75in]{geometry}									% adjust margins
\geometry{letterpaper}                                                  % or a4paper or a5paper or ... 
\usepackage{url}                                                        % need this to use URLs in bibtex
\usepackage{setspace}                                                   % need this for \setstrech{...}
\usepackage{scrextend}                                                  % need this for addmargin
\usepackage[export]{adjustbox}                                          % need this to get frame for includegraphics
%
%   tikz et al
%
\usepackage{tikz}
\usetikzlibrary{calc,patterns,angles,quotes,shapes,math,decorations,
                through,intersections,lindenmayersystems,backgrounds}
\usepackage{circuitikz}                                                 % draw circuits    
\usepackage{pgfplots}
\usepackage{pgfplots}	
%
%	more math stuff
%
\usepackage{amsmath,amsfonts,amssymb,amsthm}
\usepackage{mathtools}
\usepackage{commath}                                                    % get \norm{x}
\usepackage{fixmath}                                                    % get \mathbold
\usepackage{gensymb}                                                    % get \degree
\usepackage{mathrsfs}
\usepackage{hyperref}
\usepackage{subcaption}
\usepackage{authblk}
\usepackage{graphicx}
\usepackage{hyperref}
\usepackage{alltt}
\usepackage{color}
\usepackage{float}
\usepackage{braket}
\usepackage{siunitx}
\usepackage{relsize}
\usepackage{multirow}
\usepackage{esvect}
%
%	watermarks
%
% \usepackage{draftwatermark}
% \SetWatermarkText{Draft}
% \SetWatermarkScale{5}
% \SetWatermarkLightness {0.9} 
% \SetWatermarkColor[rgb]{0.7,0,0}
%
%
%	theorems, definitions, etc
%
\theoremstyle{definition}
\newtheorem{theorem}{Theorem}[section]
\newtheorem{definition}{Definition}[section]
\newtheorem{proposition}{Proposition}[section]
\newtheorem{lemma}{Lemma}[section]
\newtheorem{example}{Example}[section]
\newtheorem{remark}{Remark}[section]
%
%	The following code allows you to do
%
%	\begin{bmatrix}[r] (or [c] or [l])
%
\makeatletter
\renewcommand*\env@matrix[1][c]{\hskip -\arraycolsep
  \let\@ifnextchar\new@ifnextchar
  \array{*\c@MaxMatrixCols #1}}
\makeatother
%
%	make \arg{min,max}_{n \to \infty} work nicely
%
\newcommand{\argmax}{\operatornamewithlimits{argmax}}
\newcommand{\argmin}{\operatornamewithlimits{argmin}}
%
%	also a handy command
%
\newcommand*{\Scale}[2][4]{\scalebox{#1}{$#2$}}%
%
%	Title, author and date
%
\title{A Bit on Ramanujan and Nested Radicals}
\author{David Meyer \\ \href{mailto:dmm613@gmail.com}
                            {dmm613@gmail.com}}
\date{Last update: \today}
%
%
%
\begin{document}
\maketitle
%
%
%
\section{Introduction}
In 1911, the Indian mathematical genius Srinivasa Ramanujan 
posed the following problem in the Journal of the Indian 
Mathematical Society \cite{ramanujan_roots}: What does the 
nested radical shown in Equation (\ref{eqn:3}) equal?


\bigskip
\begin{equation}
\sqrt{1+2 \sqrt{1+3 \sqrt{1+4  \sqrt{1+5 \sqrt{1 + \cdots}}}}}
\label{eqn:3}
\end{equation}

\medskip
\bigskip
\noindent
Having not received an answer for a few months, Ramanujan 
solved it himself. In these notes we look at Ramanujan’s 
solution to a more general form of this problem.

\bigskip
\noindent
Aside: I did a special case of this problem
in "What is hiding inside the number 3?"
\cite{notes:what_is_hiding_inside_the_number_3},
which is reproduced in Appendix A.

\section{Ramanujan's Approach}
What Ramanujan spotted was that for any non-negative 
integer $x$ we have

\medskip
\begin{equation*}
\begin{array}{llll}
x + 1
&=& \sqrt{(x+1)^2}		&\hspace{6em} \mathrel{\#} x+1 = \sqrt{(x+1)^2} \\
[5pt]
&=& \sqrt{1+2x+x^2}		&\hspace{6em} \mathrel{\#} (x+1)^2 = x^2+2x+1 \\
[5pt]
&=& \sqrt{1+x(x + 2)}	&\hspace{6em} \mathrel{\#} x^2+2x+1 = 1+x(x+2)
\end{array}
\end{equation*}

\bigskip
\noindent
Next, notice that rewriting $(x+2)$ as $(x+1)+1$
gives us

\begin{equation*}
\begin{array}{llll}
x + 1
&=& \sqrt{1+x((x+1)+1)}								
		&\hspace{4em} \mathrel{\#} (x+2) = (x+1)+1 \\
[10pt]
&=& \sqrt{1+x\sqrt{((x+1)+1)^2}}	
		&\hspace{4em}\mathrel{\#} (x+1)+1 = \sqrt{((x+1)+1)^2} \\
[10pt]
&=& \sqrt{1+x\sqrt{1+2(x+1)+(x+1)^2}}	
		&\hspace{4em} \mathrel{\#} ((x+1)+1)^2 = 1 + 
		\underbrace{2(x+1)+(x+1)^2}_{x^2+4x+3} \\
[10pt]
&=& \sqrt{1+x\sqrt{1+(x+1)(x+3)}}
		&\hspace{4em}\mathrel{\#} x^2+4x+3 = (x+1)(x+3)
\end{array}
\end{equation*}

\bigskip
\noindent
Continuing, we can rewrite $(x+3)$ as $(x+2)+1$ and so

\begin{equation*}
\begin{array}{llll}
x + 1
&=& \sqrt{1+x \sqrt{1+(x+1)((x+2)+1)}}								
		&\; \mathrel{\#} (x+3) = (x+2)+1 \\
[15pt]
&=& \sqrt{1+x\sqrt{1+(x+1)\sqrt{((x+2)+1)^2}}}
		&\; \mathrel{\#} (x+2)+1 = \sqrt{((x+2)+1)^2} \\
[15pt]
&=& \sqrt{1+x\sqrt{1+(x+1)\sqrt{1+2(x+2)+(x+2)^2}}}
		&\; \mathrel{\#} ((x+2)+1)^2 = 1 + 2(x+2)+(x+2)^2 \\
[15pt]
&=& \sqrt{1+x\sqrt{1+(x+1)\sqrt{1+x^2+6x+8}}}
		&\; \mathrel{\#} 2(x+2)+(x+2)^2 = x^2+6x+8 \\
[15pt]
&=& \sqrt{1+x\sqrt{1+(x+1)\sqrt{1 + (x+2)(x+4)}}}
		&\; \mathrel{\#} x^2+6x+8 = (x+2)(x+4) \\
[15pt]
&=& \sqrt{1+x\sqrt{1+(x+1)\sqrt{1+(x+2)\sqrt{((x+3)+1)^2}}}}
		&\; \mathrel{\#} (x+4) = (x+3)+1 = \sqrt{((x+3)+1)^2} \\
[15pt]
&=& \sqrt{1+x\sqrt{1+(x+1)\sqrt{1+(x+2)\sqrt{1+x^2+8x+15}}}}
		&\; \mathrel{\#} ((x+3)+1)^2 =  1 + x^2+8x+15 \\
[15pt]
&=& \sqrt{1+x\sqrt{1+(x+1)\sqrt{1+(x+2)\sqrt{1+(x+3)(x+5)}}}}
		&\; \mathrel{\#} x^2+8x+15 = (x+3)(x+5) \\
[12pt]
&=& \sqrt{1+x\sqrt{1+(x+1)\sqrt{1+(x+2)\sqrt{1+(x+3) \sqrt{((x+4)+1)^2}}}}}
		&\; \mathrel{\#} x+5 = (x+4)+1 = \sqrt{((x+4)+1)^2} 

\end{array}
\end{equation*}

\bigskip
\noindent
Now we can see that the general form of the expression is:

\bigskip
\begin{equation}
x+1 = \sqrt{1+x\sqrt{1+(x+1)\sqrt{1+(x+2)\sqrt{1+(x+3)\sqrt{1+(x+4)\sqrt{1 + (x+5)\sqrt{1+\hdots}}}}}}}
\label{eqn:continued_fraction}
\end{equation}

\bigskip
\noindent
Setting $x = 2$ in Equation (\ref{eqn:continued_fraction}) we get

\bigskip
\begin{equation*}
3 = \sqrt{1 + 2 \sqrt{1 + 3 \sqrt{1 + 4  \sqrt{1 + 5 \sqrt{1 + 6 \sqrt{1 + \cdots}}}}}}
\end{equation*}

\bigskip
\noindent
which is the result we saw in \cite{notes:what_is_hiding_inside_the_number_3}.
But now we have the general formula so we can plug in any positive integer, 
say $x = 124$ for example:

\bigskip
\begin{equation*}
125 = \sqrt{1 + 124 \sqrt{1 + 125 \sqrt{1 + 126  \sqrt{1 + 127 \sqrt{1 + \cdots}}}}}
\end{equation*}
%
%
%
\section*{Acknowledgements}
%
%	LaTeX source on overleaf.com
%
\section*{\LaTeX \hspace{0.10 mm} Source}
\url{https://www.overleaf.com/read/qwhvvhrzrgct}
%
%	get a bibliography
%
%	Note:.bib files go in ~/Library/texmf/bibtex/bib with TeXShop (MacTeX).
%	You can also use an absolute path, e.g. \bibliography{/Users/dmm/papers/bib/qc}
%
\bibliographystyle{plain}
\bibliography{qc}
%
%
%
\section*{Appendix A: What is hiding inside the number 3?}
\noindent
 Well, as we saw in \cite{notes:what_is_hiding_inside_the_number_3}:

\begin{center}
\begin{equation*}
\begin{array}{llll}
3 
&=&  \sqrt{9}
				&\qquad \qquad \qquad \mathrel{\#} 3 = \sqrt{9} \\
[6pt]
&=&  \sqrt{1 + 8}
				&\qquad \qquad \qquad \mathrel{\#} 9 = 1 + 8 \\
[6pt]
&=&  \sqrt{1 + 2*4}
				&\qquad \qquad \qquad \mathrel{\#} 8 = 2 * 4 \\
[6pt]
&=&  \sqrt{1 + 2 \sqrt{16}}
				&\qquad \qquad \qquad \mathrel{\#} 4 = \sqrt{16}\\
[6pt]
&=&  \sqrt{1 + 2 \sqrt{1 + 15}}
				&\qquad \qquad \qquad \mathrel{\#} 16 = 1 + 15 \\
[6pt]
&=&  \sqrt{1 + 2 \sqrt{1 + 3*5}}
				&\qquad \qquad \qquad \mathrel{\#} 15 = 3 * 5 \\
[6pt]
&=&  \sqrt{1 + 2 \sqrt{1 + 3 \sqrt{25}}}
				&\qquad \qquad \qquad \mathrel{\#} 5 = \sqrt{25} \\
[6pt]
&=&  \sqrt{1 + 2 \sqrt{1 + 3 \sqrt{1 + 24}}}
				&\qquad \qquad \qquad \mathrel{\#} 25 = 1 + 24 \\
[6pt]
&=&  \sqrt{1 + 2 \sqrt{1 + 3 \sqrt{1 + 4*6}}}
				&\qquad \qquad \qquad \mathrel{\#} 24 = 4 * 6 \\
[6pt]
&=&  \sqrt{1 + 2 \sqrt{1 + 3 \sqrt{1 + 4 \sqrt{36}}}}
				&\qquad \qquad \qquad \mathrel{\#} 6 = \sqrt{36}  \\
[6pt]
&=&  \sqrt{1 + 2 \sqrt{1 + 3 \sqrt{1 + 4 \sqrt{1 + 35}}}}
				&\qquad \qquad \qquad \mathrel{\#} 36 = 1 + 35 \\
[6pt]
&=&  \sqrt{1 + 2 \sqrt{1 + 3 \sqrt{1 + 4 \sqrt{1 + 5*7}}}}
				&\qquad \qquad \qquad \mathrel{\#} 35 = 5 * 7 \\
[6pt]
&=&  \sqrt{1 + 2 \sqrt{1 + 3 \sqrt{1 + 4 \sqrt{1 + 5 \sqrt{49}}}}}
				&\qquad \qquad \qquad \mathrel{\#} 7 = \sqrt{49} \\
[6pt]
&=& \cdots
				&\qquad \qquad \qquad \mathrel{\#} 49 = 1 + 48,\; 48 = 6*8, \; 8 = \sqrt{64}, \; \hdots 
\end{array}
\end{equation*}
\end{center}

\bigskip
\noindent
and so apparently $3 = \sqrt{1+2 \sqrt{1+3 \sqrt{1+4  \sqrt{1+5 \sqrt{1+6 \sqrt{1+\cdots}}}}}}$
%
%	done
%
\end{document} 


