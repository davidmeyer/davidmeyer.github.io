%
%	Use lualatex
%
% !TEX TS-program = LuaLaTeX
%
%	To get overleaf.com to use lualatex see
%	https://www.overleaf.com/learn/how-to/Changing_compiler.
%	Basically, change the compiler in Settings.
%
\documentclass{article}
%
%
%	mega_millions.tex
%
%	David Meyer
%	dmm613@gmail.com
%	27 Dec 2024
%
%
%   get various packages
%
\usepackage[margin=1.0in]{geometry}                                     % adjust margins
\geometry{letterpaper}                                                  % or a4paper or a5paper or ... 
\usepackage{url}                                                        % need this to use URLs in bibtex
\usepackage{setspace}                                                   % need this for \setstrech{...}
\usepackage{scrextend}                                                  % need this for addmargin
\usepackage[export]{adjustbox}                                          % need this to get frame for includegraphics
%
%   tikz et al
%
\usepackage{tikz}
\tikzset{>=latex}                                                       % default to LaTeX arrow head
\usetikzlibrary{calc,patterns,angles,quotes,shapes,math,decorations,
                through,intersections,lindenmayersystems,backgrounds,
                hobby,matrix}
\usepackage{tikz-cd}                                                    
\tikzcdset{every arrow/.append style={-latex}}
\usepackage{circuitikz}                                                 % draw circuits    
\usepackage{pgfplots}
%
%	more math stuff
%
\usepackage{amsmath,amsfonts,amssymb,amsthm}
\usepackage{bm}
\usepackage{mathtools}
\usepackage{commath}                                                    % get \norm{x}
\usepackage{fixmath}                                                    % get \mathbold
\usepackage{gensymb}                                                    % get \degree
\usepackage{mathrsfs}
\usepackage{hyperref}
\usepackage{subcaption}
\usepackage{authblk}													% comment out if using beamer (stops \author{} from working in beamer)
\usepackage{graphicx}
\usepackage{hyperref}
\usepackage{alltt}
\usepackage{xcolor}
\usepackage{colortbl}													% \rowcolor{yellow!75} etc
\usepackage{float}
\usepackage{braket}
\usepackage{siunitx}
\usepackage{relsize}
\usepackage{multirow}
\usepackage{esvect}
\usepackage{enumitem}													% use characters instead of numbers in enumerate
\usepackage{changepage}													% needed for \begin{adjustwidth}{-3.25em}{-2.0em} (left justify)
\usepackage{bigints}
\usepackage{multirow}													% \multirow and \multicolumn
\usepackage{enumitem}													% change bullets in \begin{enumerate}
%
%	lualatex
%
%	Put this before \documentclass
%
%	!TEX TS-program = LuaLaTeX
%
%
\usepackage{luatex85,luamplib}
\mplibnumbersystem{double}
\mplibtextextlabel{enable}
%
%
%
%	Describe floating point parameters, \fpeval
%
\ExplSyntaxOn
\cs_set_eq:NN \fpeval \fp_eval:n
\ExplSyntaxOff
%
%	Get the x and y components out of a coordinate, e.g.
%
%	\coordinate (EP) at (8,5);
%	\gettikzxy{(EP)}{\slopex}{\slopey}
%
\makeatletter
\newcommand{\gettikzxy}[3]{%
  \tikz@scan@one@point\pgfutil@firstofone#1\relax
  \edef#2{\the\pgf@x}%
  \edef#3{\the\pgf@y}%
}
\makeatother
%
%
%	Watermarks
%
% \usepackage{draftwatermark}
% \SetWatermarkText{Draft}
% \SetWatermarkScale{5}
% \SetWatermarkLightness {0.9} 
% \SetWatermarkColor[rgb]{0.7,0,0}
%
%
%	theorems, definitions, etc
%
\theoremstyle{definition}
\newtheorem{theorem}{Theorem}[section]
\newtheorem{definition}{Definition}[section]
\newtheorem{proposition}{Proposition}[section]
\newtheorem{lemma}{Lemma}[section]
\newtheorem{example}{Example}[section]
\newtheorem{remark}{Remark}[section]
\newtheorem{notation}{Notation}[section]
%
%	For drawing matrix products 
%
\newcommand*{\vertbar}{\rule[-1ex]{0.5pt}{2.5ex}}
\newcommand*{\horzbar}{\rule[.5ex]{2.5ex}{0.5pt}}

%
%	The following code allows you to do
%
%	\begin{bmatrix}[r] (or [c] or [l])
%
\makeatletter
\renewcommand*\env@matrix[1][c]{\hskip -\arraycolsep
  \let\@ifnextchar\new@ifnextchar
  \array{*\c@MaxMatrixCols #1}}
\makeatother
%
%	make \arg{min,max}_{n \to \infty} work nicely
%
\newcommand{\argmax}{\operatornamewithlimits{argmax}}
\newcommand{\argmin}{\operatornamewithlimits{argmin}}
%
%	handy commands
%
\newcommand*{\Scale}[2][4]{\scalebox{#1}{$#2$}}
\DeclareMathOperator{\E}{\mathbb{E}}
\DeclareMathOperator{\bda}{\Big \downarrow}						% big down arrow
\newcommand{\veq}{\mathrel{\rotatebox{90}{$=$}}}
%
%	Title, author and date
%
\title{Mega Millions}
\author{David Meyer \\ \href{mailto:dmm613@gmail.com}
                            {dmm613@gmail.com}}
\date{Last Update: \today \\
	 {\vspace{1.00mm} \small Initial Version: December 27, 2024}}
%
%
%
\begin{document}
\maketitle
%
%
%
\section{Introduction}
\label{section:introduction}
Mega Millions\textregistered \hspace{0.25mm} is a multi-state
lottery game played in 45 states plus the District of Columbia
and the U. S. Virgin Islands; a total of 47
jurisdictions. Section
\ref{section:probability_of_winning_the_jackpot} of this short
document describes the lottery rules and the probability of
winning the jackpot. Section
\ref{section:expected_value_of_a_ticket} describes the
expected value of a \$2 Mega Millions ticket.


\section{Probability of Winning the Jackpot}
\label{section:probability_of_winning_the_jackpot}
What are the odds of winning the Mega Millions jackpot?

\bigskip
\noindent
It turns out that the odds of winning are 1 in 302,575,350
\cite{mega_millions}. 

\bigskip
\noindent
Ok, but why? Let's see...

\smallskip
\begin{enumerate}
\item In the Mega Millions game there are 70 white balls,
numbered 1 to 70, in one drum. 5 balls are chosen from this drum
in the drawing. So there are

\begin{equation*}
       \binom{70}{5} = 12,103,014 
\end{equation*}

\bigskip
\noindent
ways to choose the 5 white balls. This calculation ignores order,
which doesn't matter in Mega Millions.

\item We also need to choose the Mega Ball. There are 25 Mega
Balls, numbered 1 to 25, in a different drum. 1 ball is chosen
from this drum in the drawing.  So there are

\smallskip
\begin{equation*}
       \binom{25}{1} = 25 
\end{equation*}

\medskip
\noindent
ways to choose the Mega Ball.

\item So the number of ways to choose 5 white balls and 1 Mega
Ball in the Mega Millions game is
 
\begin{equation*}
       12,103,014 \times 25 = 302,575,350
\end{equation*}

\smallskip
\item Putting this all together we see that the probability of
choosing the correct 5 white balls and the Mega Ball, call it
p(winning), is

\begin{equation*}
       \text{p(winning)} = \dfrac{1}{302,575,350} \approx 3.30 \times 10^{-9}
\end{equation*}
\end{enumerate}

\bigskip
\noindent
Said another way, the odds of winning are 1 in 302,575,350. 


\section{Expected Value of a Mega Millions Ticket}
\label{section:expected_value_of_a_ticket}
First, this is what we know:

\begin{enumerate}
\item Advertised Jackpot: $\$1.15 \times 10^9$ as of 12/27/2024
\item Probability of Winning: $\text{p(winning)} = \dfrac{1}{302,575,350}$
\item Lump Sum Option: Typically \~60\% of the advertised jackpot, so take 
	  the Lump sum to be \\ \\
      Lump sum = $(\$1.15 \times 10^9) \times 0.60 = \$690.00 \times 10^{6}$ 
      \vspace{0.25cm}
\item Taxes: Both Federal and potentially state taxes apply. Assume these numbers are
\begin{itemize}
\item Federal tax rate: ~37\%
\item State tax rates vary but average around 5\%
\item Combined tax rate: ~42\%
\end{itemize}

\medskip
Then the Post-tax lump sum = $\$690.00 \times 10^{6} \times (1.00-0.42) = \$400.20 \times 10^6$
\end{enumerate}

\subsection{Adjusted Lump Sum per Winner}
If the jackpot is split among $n$ winners, each winner receives:

\medskip
\begin{equation*}
\text{Post-Tax Lump Sum Share} = \dfrac{\text{Post-Tax Lump Sum}}{n} = \dfrac{\$400.20 \times 10^6}{n}
\end{equation*}

\bigskip
\noindent
For example:

\begin{itemize}
\item For one winner ($n=1$), the Post-Tax Lump Sum is $\$400.20 \times 10^6$
\item For two winners ($n=2$), the Post-Tax Lump Sum is $\$200.10 \times 10^6$
\item For three winners ($n=3$), the Post-Tax Lump Sum is $\$133.40 \times 10^6$
\item $\cdots$
\end{itemize}

\subsection{Expected Number of Winners}
The expected number of winners depends on $T$, the number of
tickets sold and the odds of winning. That is, the
expected number of winners is

\begin{equation*}
\text{Expected Number of Winners} = T \times \text{p(winning)} = \dfrac{T}{302,575,350}
\end{equation*}

\bigskip
\noindent
For example, suppose 500 million tickets are
sold\footnote{Apparently Mega Millions doesn't publish the total
number of tickets sold for any given drawing.}, so that $T = 500
\times 10^6$. Then 


\medskip
\begin{equation*}
\text{Expected Number of Winners} = \dfrac{500\times 10^6}{302,575,350} = 1.65
\vspace{3.25 cm}
\end{equation*}


\subsection{Post-Tax Lump Sum per Winner}
In our example there are 1.65 winners on average, so each winner receives:

\medskip
\begin{equation*}
\text{Post-Tax Lump Sum Share} = \dfrac{\$400.20 \times 10^6}{1.65} = \$242.50 \times 10^6
\end{equation*}


\subsection{Expected Value (jackpot only) of a Mega Millions Ticket}
The expected jackpot payout per ticket is $\text{Post-Tax Lump Sum Share} 
\times \text{Odds of Winning}$.


\bigskip
\noindent
Substituting the values for Post-Tax Lump Sum Share and the 
Odds of Winning we get

\medskip
\begin{equation*}
\dfrac{\$242.50 \times 10^6}{302,575,350} \approx \$0.80
\end{equation*}

\bigskip
\noindent
So given our assumptions the expected value of a Mega Millions 
lottery ticket is \$0.80.

\smallskip
\subsection{Net Expected Value of a \$2 Mega Millions Ticket}
Given our assumptions of $500 \times 10^6$ tickets sold, personal
taxes of $42\%$, and a jackpot of $\$1.15 \times 10^9$, we see
that the net expected value of a \$2 ticket is $\$0.80 - \$2 = -\$1.20$.
%
%
%
\section{Conclusions}
\label{section:conclusions}
Assuming a $\$1.15 \times 10^9$ jackpot with $500 \times 10^6$
tickets sold and a $42\%$ personal tax rate, the expected value
of a Mega Millions ticket is approximately \$0.80. Subtracting the \$2
ticket cost, the net expected value of a ticket is
approximately -1.20.
%
%
%
\section*{Acknowledgements}
\label{section:acknowledgements}
%
%	LaTeX source on overleaf.com
%
\section*{\LaTeX \hspace{0.025 mm} Source}
%\url{}
%
%	get a bibliography
%
%	Note:.bib files go in ~/Library/texmf/bibtex/bib with TeXShop (MacTeX).
%	You can also use an absolute path, e.g. \bibliography{/Users/dmm/papers/bib/qc}
%
\bibliographystyle{plain}
\bibliography{qc}
%
%
%
% \section*{Appendix A}
%
%	done
%
\end{document} 

