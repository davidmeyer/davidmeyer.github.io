\documentclass{article}
%
%
%	spiral_of_theodorus.tex
%
%	David Meyer
%	dmm613@gmail.com
%	18 Aug 2022
%
%
%   get various packages
%
\usepackage[margin=1.0in]{geometry}                                     % adjust margins
\geometry{letterpaper}                                                  % or a4paper or a5paper or ... 
\usepackage{url}                                                        % need this to use URLs in bibtex
\usepackage{setspace}                                                   % need this for \setstrech{...}
\usepackage{scrextend}                                                  % need this for addmargin
\usepackage[export]{adjustbox}                                          % need this to get frame for includegraphics
%
%   tikz et al
%
\usepackage{tikz}
\usetikzlibrary{calc,patterns,angles,quotes,shapes,math,decorations,
                through,intersections,lindenmayersystems,backgrounds}
\usepackage{circuitikz}                                                 % draw circuits    
\usepackage{pgfplots}
\usepackage{pgfplots}	
%
%	more math stuff
%
\usepackage{amsmath,amsfonts,amssymb,amsthm}
\usepackage{mathtools}
\usepackage{commath}                                                    % get \norm{x}
\usepackage{fixmath}                                                    % get \mathbold
\usepackage{gensymb}                                                    % get \degree
\usepackage{mathrsfs}
\usepackage{hyperref}
\usepackage{subcaption}
\usepackage{authblk}
\usepackage{graphicx}
\usepackage{hyperref}
\usepackage{alltt}
\usepackage{color}
\usepackage{float}
\usepackage{braket}
\usepackage{siunitx}
\usepackage{relsize}
\usepackage{multirow}
\usepackage{esvect}
%
%	watermarks
%
% \usepackage{draftwatermark}
% \SetWatermarkText{Draft}
% \SetWatermarkScale{5}
% \SetWatermarkLightness {0.9} 
% \SetWatermarkColor[rgb]{0.7,0,0}
%
%
%	theorems, definitions, etc
%
\theoremstyle{definition}
\newtheorem{theorem}{Theorem}[section]
\newtheorem{definition}{Definition}[section]
\newtheorem{proposition}{Proposition}[section]
\newtheorem{lemma}{Lemma}[section]
\newtheorem{example}{Example}[section]
\newtheorem{remark}{Remark}[section]
%
%	The following code allows you to do
%
%	\begin{bmatrix}[r] (or [c] or [l])
%
\makeatletter
\renewcommand*\env@matrix[1][c]{\hskip -\arraycolsep
  \let\@ifnextchar\new@ifnextchar
  \array{*\c@MaxMatrixCols #1}}
\makeatother
%
%	make \arg{min,max}_{n \to \infty} work nicely
%
\newcommand{\argmax}{\operatornamewithlimits{argmax}}
\newcommand{\argmin}{\operatornamewithlimits{argmin}}
%
%	also a handy command
%
\newcommand*{\Scale}[2][4]{\scalebox{#1}{$#2$}}%
%
%	Title, author and date
%
\title{The Spiral of Theodorus}
\author{David Meyer \\ \href{mailto:dmm613@gmail.com}
                            {dmm613@gmail.com}}
\date{Last Update: \today \\
	 {\vspace{1.00mm} \small Initial Version: August 18, 2022}}

%
%
%
\begin{document}
\maketitle
%
%
%
\section{A Bit About The Spiral}
\label{sec:spiral_of_theodorus}
The Spiral of Theodorus is a 
fantastic object that is composed of right
triangles placed edge-to-edge, where the sides of $n^{\text{th}}$
triangle are of length 1 and $\sqrt{n}$ and the hypotenuse is of
length $\sqrt{n+1}$.

\bigskip
\noindent
The spiral is named for Theodorus of Cyrene, the ancient Greek
mathematician who lived during the $5^{\text{th}}$ century BC and
who discovered the spiral \cite{wiki:theodorus_of_cyrene}. It is
thought that Theodorus used the spiral to prove that all of the 
square roots of the non-square integers from 3 to 17 are irrational 
\cite{wiki:spiral_of_theodorus}. Theodorus' original spiral stopped 
at $n = 16$, apparently because $\sqrt{16+1} = \sqrt{17}$ is the 
length of the hypotenuse of the last triangle in the spiral that
doesn't overlap; this is shown in Figure \ref{fig:the_spiral_of_theodorus}.

\bigskip
\noindent
Interestingly, in 1958 Erich Teuffel showed that no two hypotenuses
of the spiral will coincide regardless of how far the spiral
is continued \cite{wurzelschnecke}. Said another way, no two
hypotenuses of the spiral will coincide, no matter how big $n$
gets.

\bigskip
\noindent
In any event, the spiral is an incredible object.
%
%	Build a command that draws a picture
%	of the Spiral of Theodorus with parameter
%	n = the number of triangles.
%
\newcommand*{\sqrtspiral}[2][scale=3]{
    \begin{tikzpicture}[#1]														% start the tikzpicture
		\def\sqrtlast{#2}														% input parameters
%																				%																
%	Parameters																	%
%																				% 										% 
        \def \linewidth    {0.30mm}												% line width (make the lines thicker)
        \def \currentcolor {black}												% default to black
%																				%
%	First triangle																% draw first triangle
%																				%
        \coordinate (A) at (0,0);                                               % side adjacent
        \coordinate (B) at (1cm,0);                                             % side opposite
        \draw[line width = \linewidth,\currentcolor] (A) edge node[fill=white]	% draw first hypotenuse (of length \sqrt{1})
        	{$\sqrt{1}$} (B);													% first triangle
%																				%
%	Draw the n triangles														% draw the n triangles
%																				%
        \foreach \n in {1,...,\sqrtlast} {                                      % draw \n triangles
            \ifthenelse{\equal{\n}{\sqrtlast}}                                  % if it's the last triangle then make it red
            {                                                                   %
                \def \currentcolor {red}                                        % draw the last triangle in red (\sqrtlast)
            }                                                                   %
            {                                                                   %
                \def \currentcolor {black}                                      % otherwise make it black
            }                                                                   % end \ifthenelse 
            \pgfmathtruncatemacro{\currentsqrt}{\n+1}                           % hypotenuse is of length root (n+1)
%																				%
%	Draw everything																%
%																				%
            \coordinate (C) at ($(B)!1cm!-90:(A)$);                             % calculate hypotenuse
            \draw[line width=\linewidth,\currentcolor] (A) edge node[fill=white]% draw hypotenuse
            	{$\boldmath \sqrt{\currentsqrt}$} (C);							% ...
            \draw[line width=\linewidth, \currentcolor] (C) edge node[auto]		% draw the side opposite (of length 1)
            	{$\boldmath 1$} (B);											%
            \coordinate (w) at ($(B)!4pt!-90:(A)$);                             % build the coordinates for the right angle box
            \coordinate (z) at ($(B)!4pt!0:(A)$);                               % ...
            \coordinate (t) at ($(w)!4pt!-90:(B)$);                             % ...
            \draw [line width=\linewidth,\currentcolor] (w) -- (t) -- (z);		% draw the right angle box
            \coordinate (B) at (C);                                             % update starting point
        }                                                                       % end \foreach
    \end{tikzpicture}                                                           % end tikzpicture
}                                                                               % end newcommand
%
%	draw the spiral wrapped in a figure
%
%	Parameters:
%		\numberoftriangles (aka n)
%
\def \numberoftriangles {17}													% number of triangles (aka n)
%
%
%
\bigskip
\begin{figure}[H]																% wrap the spiral in a figure
        \centering																% center the whole thing
        \resizebox{0.60 \textwidth}{!} {                                        % resize figure if you want
        	$\boldmath{\sqrtspiral[scale=2.75]{\numberoftriangles}}$			% draw the spiral
        }                                                                       % end resizebox
\caption{The Spiral of Theodorus ($n = \numberoftriangles$)}
\label{fig:the_spiral_of_theodorus}
\end{figure}
%
%	Draw a bigger spiral
%
\def \numberoftriangles {128}													% 128 triangles
%
%
%
\section{A Spiral of Theodorus with $n = \numberoftriangles$}
%
%
%
\bigskip
\begin{figure}[H]																% wrap the spiral in a figure
        \centering																% center the whole thing
        \resizebox{0.75 \textwidth}{!} {                                        % resize figure if you want
        	$\boldmath{\sqrtspiral[scale=2.75]{\numberoftriangles}}$			% draw the spiral
        }                                                                       % end resizebox
\caption{A spiral with $n = \numberoftriangles$}
\end{figure}
%
%
%
%
\bigskip
\section*{Acknowledgements}
%
%	LaTeX source on overleaf.com
%
\section*{\LaTeX \hspace{0.10 mm} Source}
\url{https://www.overleaf.com/read/scsptwzqxzdt}
%
%	get a bibliography
%
%	Note:.bib files go in ~/Library/texmf/bibtex/bib with TeXShop (MacTeX).
%	You can also use an absolute path, e.g. \bibliography{/Users/dmm/papers/bib/qc}
%
\bibliographystyle{plain}
\bibliography{qc}
%
%	done
%
\end{document} 

