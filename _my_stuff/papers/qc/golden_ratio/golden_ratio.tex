\documentclass{article}
%
%
%	An Interesting Integral Involving The Golden Ratio
%
%	David Meyer
%	dmm613@gmail.com
%	23 Jan 2022
%
%
%   get various packages
%
\usepackage[margin=1.0in]{geometry}                                     % adjust margins
\geometry{letterpaper}                                                  % or a4paper or a5paper or ... 
\usepackage{url}                                                        % need this to use URLs in bibtex
\usepackage{setspace}                                                   % need this for \setstrech{...}
\usepackage{scrextend}                                                  % need this for addmargin
\usepackage[export]{adjustbox}                                          % need this to get frame for includegraphics
%
%   tikz et al
%
\usepackage{tikz}
\usetikzlibrary{calc,patterns,angles,quotes,shapes,math,decorations,
                through,intersections,lindenmayersystems,backgrounds}
\usepackage{circuitikz}                                                 % draw circuits    
\usepackage{pgfplots}
\usepackage{pgfplots}	
%
%	more math stuff
%
\usepackage{amsmath,amsfonts,amssymb,amsthm}
\usepackage{mathtools}
\usepackage{commath}                                                    % get \norm{x}
\usepackage{fixmath}                                                    % get \mathbold
\usepackage{gensymb}                                                    % get \degree
\usepackage{mathrsfs}
\usepackage{hyperref}
\usepackage{subcaption}
\usepackage{authblk}
\usepackage{graphicx}
\usepackage{hyperref}
\usepackage{alltt}
\usepackage{color}
\usepackage{float}
\usepackage{braket}
\usepackage{siunitx}
\usepackage{relsize}
\usepackage{multirow}
\usepackage{esvect}
\usepackage{bigints}
%
%	watermarks
%
% \usepackage{draftwatermark}
% \SetWatermarkText{Draft}
% \SetWatermarkScale{5}
% \SetWatermarkLightness {0.9} 
% \SetWatermarkColor[rgb]{0.7,0,0}
%
%
%	theorems, definitions, etc
%
\theoremstyle{definition}
\newtheorem{theorem}{Theorem}[section]
\newtheorem{definition}{Definition}[section]
\newtheorem{proposition}{Proposition}[section]
\newtheorem{lemma}{Lemma}[section]
\newtheorem{example}{Example}[section]
\newtheorem{remark}{Remark}[section]
%
%	The following code allows you to do
%
%	\begin{bmatrix}[r] (or [c] or [l])
%
\makeatletter
\renewcommand*\env@matrix[1][c]{\hskip -\arraycolsep
  \let\@ifnextchar\new@ifnextchar
  \array{*\c@MaxMatrixCols #1}}
\makeatother
%
%	make \arg{min,max}_{n \to \infty} work nicely
%
\newcommand{\argmax}{\operatornamewithlimits{argmax}}
\newcommand{\argmin}{\operatornamewithlimits{argmin}}
%
%	also a handy command
%
\newcommand*{\Scale}[2][4]{\scalebox{#1}{$#2$}}%
%
%	Title, author and date
%
\title{An Interesting Integral Involving The Golden Ratio $\phi$}
\author{David Meyer \\ \href{mailto:dmm613@gmail.com}{dmm613@gmail.com}}
\date{Last update: \today}
%
%
%
\begin{document}
\maketitle
%
%
%
\noindent
Consider the following integral:

{\Large
\begin{equation}
\ln \phi = \int_{0}^{\frac{1}{2}} \dfrac{1}{\sqrt{x^{2} + 1}} \; dx
\label{eqn:lnphi}
\end{equation}
}

\bigskip
{\setstretch{1.25}
\noindent
To see how Equation (\ref{eqn:lnphi}) works, first 
parameterize it with 
$x = \tan y$ and $\sqrt{x^{2}+1} = \sec y$. Then \\
$dx = \sec^2 y \, dy$, $y = \arctan x$, and 
$\sec y = \sec (\arctan x) = \sqrt{x^{2}+1}$. 
So \par}


\begin{equation*}
\begin{array}{rcll} 
{\displaystyle \int_{0}^{\frac{1}{2}} \dfrac{1}{\sqrt{x^{2} + 1}} \;  dx} 
&=& {\displaystyle \int_{0}^{\frac{1}{2}} \dfrac{\sec^2 y}{\sec y} \;  dy}			
		&\; \mathrel{\#} \text{use above parameterization} \\
[10pt]
&=&{\displaystyle \int_{0}^{\frac{1}{2}} \sec y} \;  dy								
		&\; \mathrel{\#} \dfrac{\sec^2 y}{\sec y} = \sec y \\
[10pt]
&=&{\displaystyle \int_{\arctan 0}^{\arctan \frac{1}{2}} \sec y} \;  dy			
		&\; \mathrel{\#} \text{parameterization $\Rightarrow y = \arctan x$} \\
[10pt]
&=&{\displaystyle \int_{0}^{\arctan \frac{1}{2}} \sec y} \; dy			
		&\; \mathrel{\#} \arctan 0 = 0 \\
[8pt]
&=&{\displaystyle \ln | \sec y + \tan y | \Big |_{0}^{\arctan \frac{1}{2}}}			
		&\; \mathrel{\#} \text{integral of $\sec y \; dy$ \cite{wiki:integral_of_secant} 
		                       and the FToC \cite{ftoc}} \\
[14pt]
&=& \ln | \sec (\arctan \frac{1}{2}) + \tan (\arctan \frac{1}{2}) |  - \ln |\sec 0 + \tan 0 |
		&\; \mathrel{\#} \text{expand previous line} \\
[14pt]
&=& \ln | \sec (\arctan \frac{1}{2}) + \tan (\arctan \frac{1}{2}) |  - \ln |1 + 0 |
		&\; \mathrel{\#} \text{$\sec 0 = 1$ and $\tan 0 = 0$} \\
[14pt]
&=& \ln | \sec (\arctan \frac{1}{2}) + \tan (\arctan \frac{1}{2}) |  - 0
		&\; \mathrel{\#} \ln | 1 + 0| = \ln 1 = 0 \\
[14pt]
&=& \ln | \sec (\arctan \frac{1}{2}) + \frac{1}{2} |
		&\; \mathrel{\#} \tan(\arctan x) = x \\
[10pt]
&=& \ln \Big | \sqrt{\left (\frac{1}{2} \right )^{\mathsmaller 2} + 1} + \frac{1}{2} \Big |
		&\; \mathrel{\#} \text{$\sec({\arctan x}) = \sqrt{x^2 +1}$} \\
[10pt]
&=& \ln \Big | \sqrt{\frac{1}{4} + 1} + \frac{1}{2} \Big |
		&\; \mathrel{\#} {\left ( \frac{1}{2} \right ) }^{\mathsmaller 2} = \frac{1}{4} \\
[10pt]
&=& \ln \Big | \sqrt{\frac{5}{4}} + \frac{1}{2} \Big |
		&\; \mathrel{\#} \sqrt{\frac{1}{4} + 1} = \sqrt{\frac{1}{4} + \frac{4}{4}} = \sqrt{\frac{5}{4}} \\
[10pt]
&=& \ln \Big | \frac{1 + \sqrt{5}}{2} \Big |
		&\; \mathrel{\#} \sqrt{\frac{5}{4}} + \frac{1}{2} = \frac{\sqrt{5}}{2} + \frac{1}{2} = \frac{1 + \sqrt{5}}{2} \\
[10pt]
&=& \ln \phi
		&\; \mathrel{\#} \phi =  \frac{1 + \sqrt{5}}{2} \; \cite{wiki:golden_ratio}
\end{array}
\end{equation*}
%
%
%
\section*{Acknowledgements}
Paul Masson (@paulmasson@mathstodon.xyz) pointed out that a faster way to 
get the result is to recognize that the integral is the inverse hyperbolic 
sine and then use its logarithmic form \cite{inverse_hyperbolic_sine}. So 
for $x \in \mathbb{R}$ we have:

\bigskip
\begin{equation}
\bigintssss \hspace{-0.5em}  \dfrac{1}{\sqrt{x^{2} + 1}} \; dx = \sinh^{-1} x = 
\ln \left (x + \sqrt{x^{2} + 1} \right )
\label{eqn:ihs}
\end{equation}

\bigskip
\noindent
Next, notice that 


\begin{equation}
\ln \left (x + \sqrt{x^{2} + 1} \right ) = \ln c \Rightarrow x + \sqrt{x^{2} + 1} = c
\label{eqn:c}
\end{equation}


\bigskip
\noindent
Then the upper limit of integration for the integral
in Equation (\ref{eqn:ihs}) in terms of $c$ is

\medskip
\begin{equation*}
\begin{array}{llll}
x + \sqrt{x^{2} + 1} 
&=& c &\hspace{1em} \mathrel{\#} \text{Equation (\ref{eqn:c})} \\
[6pt]
&\Rightarrow& x^2 + 2x\sqrt{x^2+1} + x^2+1= c^2 
			&\hspace{1em} \mathrel{\#} \text{square both sides} \\
[8pt]
&\Rightarrow& 2x^2 + 2x\sqrt{x^2+1} + 1 = c^2 
			&\hspace{1em} \mathrel{\#} \text{collect terms} \\
[8pt]
&\Rightarrow& 2x^2 + 2x\sqrt{x^2+1} = c^2 -1  
			&\hspace{1em} \mathrel{\#} \text{subtract 1 from both sides} \\
[8pt]
&\Rightarrow& 2x (x + \sqrt{x^2+1}) = c^2 -1  
			&\hspace{1em} \mathrel{\#} \text{factor out $2x$} \\
[8pt]
&\Rightarrow& 2xc = c^2 -1  
			&\hspace{1em} \mathrel{\#} x + \sqrt{x^{2}+1} = c \\
[4pt]
&\Rightarrow& x = \dfrac{c^2 -1}{2c}  
			&\hspace{1em} \mathrel{\#} \text{solve for $x$} \\
\end{array}
\end{equation*}

\bigskip
\noindent
So now we know that


\begin{equation}
\int_{0}^{\frac{c^2 -1}{2c}} \dfrac{1}{\sqrt{x^{2} + 1}} \; dx = \ln c
\label{eqn:integral_equals_ln_c}
\end{equation}

\bigskip
\bigskip
\noindent
Equation (\ref{eqn:integral_equals_ln_c}) holds for $c \in \mathbb{Z}, 
\, \mathbb{R}, \, \mathbb{C}$, $\hdots$ [@paulmasson@mathstodon.xyz].

\bigskip
\noindent
If we set $x = \frac{1}{2}$ in Equation (\ref{eqn:c})
then $c = x + \sqrt{x^2+1} = \frac{1}{2} + \sqrt{\frac{5}{4}} = 
\frac{1 + \sqrt{5}}{2} = \phi$. Alternatively, we can see that 
$c = \phi$ when the upper limit of integration in
Equation (\ref{eqn:ihs}) equals $\frac{1}{2}$, 
since

\medskip
\begin{equation*}
\begin{array}{llll}
\dfrac{c^2 -1}{2c}
&=& \dfrac{1}{2}			
				&\hspace{1em} \mathrel{\#} \text{set the upper limit of integration 
					$\left ( \frac{c^2 -1}{2c} \right )$ to $\frac{1}{2}$} \\
[10pt]
&\Rightarrow& \dfrac{c^2 -1}{c}	= 1 		
				&\hspace{1em} \mathrel{\#} \text{multiply both sides by 2} \\
[15pt]
&\Rightarrow& c^2 -1 = c 		
				&\hspace{1em} \mathrel{\#} \text{multiply both sides by $c$} \\
[15pt]
&\Rightarrow& c^2 -c -1 = 0		
				&\hspace{1em} \mathrel{\#} \text{$c^2-c-1$ is $\phi$'s 
								minimal polynomial} \\
\end{array}
\end{equation*}

\bigskip
\noindent
Here we can conclude that $c = \phi$, since $c^2 -c -1$ is $\phi$'s 
minimal polynomial and thus has $\phi$ as it's positive root \cite{wiki:golden_ratio}. 
Checking this numerically we see that 

\begin{equation*}
\begin{array}{llll}
c^2 -c -1 
&=& 0
				&\hspace{1em} \mathrel{\#} \text{$\phi$'s minimal polynomial} \\
[6pt]
&\Rightarrow& c = \dfrac{-(-1) \pm \sqrt{(-1)^2 - 4 (1) (-1)}}{2(1)}	
				&\hspace{1em} \mathrel{\#} \text{solve for $c$ using the quadric 
								formula \cite{wiki:quadradic_formula} } \\
[6pt]
&\Rightarrow& c = \dfrac{1 \pm \sqrt{5}}{2}	
				&\hspace{1em} \mathrel{\#} \text{simplify} \\
[6pt]
&\Rightarrow& c = \dfrac{1 + \sqrt{5}}{2}	
				&\hspace{1em} \mathrel{\#} \text{positive root} \\
[10pt]
&\Rightarrow& c = \phi	
				&\hspace{1em} \mathrel{\#} \phi \coloneqq \frac{1 + \sqrt{5}}{2} \\

\end{array}
\end{equation*}


\bigskip
\noindent 
@deilann@tech.lgbt also notes that $c^2-c-1$ is (or at least should be :-)) immediately 
identifiable as the golden ratio's quadratic form and $\phi$'s minimal polynomial which 
has $\phi$ and the negative inverse of $\phi$ as roots and so is "not sure solving it 
in full is truly necessary once you've gotten there".
%
%	LaTeX source on overleaf.com
%
\section*{\LaTeX \hspace{0.10 mm} Source}
\url{https://www.overleaf.com/read/mkjdjwtmnzjd}
%
%	get a bibliography
%
%	Note:.bib files go in ~/Library/texmf/bibtex/bib with TeXShop (MacTeX).
%	You can also use an absolute path, e.g. \bibliography{/Users/dmm/papers/bib/qc}
%
\bibliographystyle{plain}
\bibliography{qc}
%
%
%
\section*{Appendix A}
This was my first attempt at proving Equation (\ref{eqn:ihs}):

\bigskip
\noindent
Equation (\ref{eqn:lnphi}) holds for a particular choice of the 
upper endpoint of the integral in Equation (\ref{eqn:ihs}). In 
particular, Equation (\ref{eqn:lnphi}) holds when the upper endpoint of the 
integral equals $\frac{1}{2}$. More specifically:

\begin{equation*}
\begin{array}{llll} 
{\displaystyle \int_{0}^{\frac{1}{2}} \dfrac{1}{\sqrt{x^{2} + 1}} \;  dx} 
&=&	\ln \left (x + \sqrt{x^{2} + 1} \right ) \bigg |_{0}^{\frac{1}{2}}
		&\hspace{1em} \mathrel{\#} \text{Equation (\ref{eqn:ihs}) and the FToC} \\
[15pt]
&=& \ln \left ( \frac{1}{2} + \sqrt{\left (\frac{1}{2} \right )^{2} + 1} \right ) -
	\ln \left ( 0 + \sqrt{0^{2} + 1} \right )
		&\hspace{1em} \mathrel{\#} f(x) \Big |_{a}^{b} \coloneqq f(b) - f(a)\\
[15pt]
&=& \ln \left ( \frac{1}{2} + \sqrt{\frac{5}{4}} \right ) -
	\ln \left ( 0 + \sqrt{0^{2} + 1} \right )
		&\hspace{1em} \mathrel{\#} \sqrt{\left (\frac{1}{2} \right )^{2} + 1} = 
									\sqrt{\frac{5}{4}} \\
[15pt]
&=& \ln \left ( \frac{1}{2} + \sqrt{\frac{5}{4}} \right ) - \ln 1
		&\hspace{1em} \mathrel{\#} 0 + \sqrt{0^{2} + 1} = 1 \\
[15pt]
&=& \ln \left ( \frac{1}{2} + \sqrt{\frac{5}{4}} \right )
		&\hspace{1em} \mathrel{\#} \ln 1 = 0 \\
[15pt]
&=& \ln \left ( \frac{1 + \sqrt{5}}{2} \right )
		&\hspace{1em} \mathrel{\#} \frac{1}{2} + \sqrt{\frac{5}{4}} = 
									\frac{1 + \sqrt{5}}{2} \\
[15pt]
&=& \ln \phi
		&\hspace{1em} \mathrel{\#} \phi \coloneqq \frac{1 + \sqrt{5}}{2}
\end{array}
\end{equation*}
$\blacksquare$
%
%	done
%
\end{document} 