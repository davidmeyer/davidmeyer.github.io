
\documentclass{article}

% \usepackage{draftwatermark}
% \SetWatermarkText{Draft}
% \SetWatermarkScale{5}
% \SetWatermarkLightness {0.9} 
% \SetWatermarkColor[rgb]{0.7,0,0}


\usepackage{geometry}                		% See geometry.pdf to learn the layout options. There are lots.
\geometry{letterpaper}                 		% ... or a4paper or a5paper or ... 
%\geometry{landscape}                		% Activate for for rotated page geometryhttps://www.washingtonpost.com/world/europe/amid-impeachment-probe-gordon-sondland-is-overseeing-a-renovation-of-his-residence-that-has-cost-1-million-in-taxpayer-money/2019/10/16/d0eece92-ef86-11e9-bb7e-d2026ee0c199_story.html?tid=sm_tw
%\usepackage[parfill]{parskip}    		% Activate to begin paragraphs with an empty line rather than an indent
\usepackage{graphicx}				% Use pdf, png, jpg, or eps� with pdflatex; use eps in DVI mode
						% TeX will automatically convert eps --> pdf in pdflat						\label{thm:integral_domain}

						% TeX will automatically convert eps --> pdf in pdflatex		
\usepackage{amssymb}
\usepackage{amsmath}
\usepackage{amsthm}
\usepackage{mathrsfs}
\usepackage[hyphens,spaces,obeyspaces]{url}
\usepackage{url}
\usepackage{hyperref}
\usepackage{subcaption}
\usepackage{authblk}
\usepackage{mathtools}
\usepackage{graphicx}
\usepackage[export]{adjustbox}
\usepackage{fixltx2e}
\usepackage{hyperref}
\usepackage{alltt}
\usepackage{color}
\usepackage[utf8]{inputenc}
\usepackage[english]{babel}
\usepackage{float}
\usepackage{bigints}
\usepackage{braket}
\usepackage{siunitx}
\usepackage{mathtools}


\usepackage{tikz}
\usepackage{verbatim}
\usetikzlibrary{plotmarks}
\usepackage{pgfplots}
\usepackage{amsmath}
\usepackage{relsize}
 \usepackage[hyphenbreaks]{breakurl}
 \usepackage{setspace}

 
 \newtheorem{thm}{Theorem}[section]
% \newtheorem{defn}[thm]{Definition}
\theoremstyle{definition}
\newtheorem{definition}{Definition}[section]
\newtheorem{proposition}{Proposition}[section]
\newtheorem{lemma}{Lemma}[section]
\newtheorem{example}{Example}[section]

\newcommand{\argmax}{\operatornamewithlimits{argmax}}
\newcommand{\argmin}{\operatornamewithlimits{argmin}}

\title{$\sqrt{2 + \sqrt{2 + \sqrt{2 + \sqrt{2 + \cdots}}}}  \; = \; ?$}
\author{David Meyer \\ dmm@1-4-5.net}
\date{Last update: \today}

\begin{document}
\maketitle

\section{Introduction}
One way to think about this question is the following: 

\begin{equation*}
\begin{array}{llll}
S
&=&  \sqrt{2 + \sqrt{2 + \sqrt{2 + \sqrt{2 + \cdots}}}} 
				&\qquad \qquad \mathrel{\#} \text{define $S$} \\
[5pt]
&\Rightarrow& S^2 = 2 + \sqrt{2 + \sqrt{2 + \sqrt{2 + \sqrt{2 + \cdots}}}} 
				&\qquad \qquad \mathrel{\#} \text{square both sides} \\
[2pt]
&\Rightarrow& S^2 = 2 + S 
				&\qquad \qquad \mathrel{\#} S = \sqrt{2 + \sqrt{2 + \sqrt{2 + \sqrt{2 + \cdots}}}}  \\
[7pt]
&\Rightarrow& S^2 -S - 2 = 0 
				&\qquad \qquad \mathrel{\#} \text{collect terms} \\
[5pt]
&\Rightarrow& (S - 2)(S + 1) = 0
				&\qquad \qquad \mathrel{\#} \text{factor} \\
[5pt]
&\Rightarrow& S \in \{2,-1\}
				&\qquad \qquad \mathrel{\#} \text{solve for $S$} \\
[5pt]
&\Rightarrow& S = 2
				&\qquad \qquad \mathrel{\#} \text{positive root of $S$ (what about the negative root?)} 

\end{array}
\end{equation*}

\smallskip
\noindent
So apparently $ \sqrt{2 + \sqrt{2 + \sqrt{2 + \sqrt{2 + \cdots}}}}  = 2$.


%
%	LaTeX source on overleaf.com
%
% \section*{\LaTeX}
%
%	get a bibliography
%
%	Note:.bib files go in ~/Library/texmf/bibtex/bib with TeXShop (MacTeX).
%	You can also use an absolute path, e.g. \bibliography{/Users/dmm/papers/bib/qc}
%
% \bibliographystyle{plain}
% \bibliography{qc}
%
%	done
%
\end{document}
