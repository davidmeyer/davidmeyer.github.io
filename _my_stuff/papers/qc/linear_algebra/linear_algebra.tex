\documentclass{article}
%
%
%	linear_algebra.tex
%
%	David Meyer
%	dmm613@gmail.com
%	25 Mar 2023
%
%
%   get various packages
%
\usepackage[margin=1.0in]{geometry}                                     % adjust margins
\geometry{letterpaper}                                                  % or a4paper or a5paper or ... 
\usepackage{url}                                                        % need this to use URLs in bibtex
\usepackage{setspace}                                                   % need this for \setstrech{...}
\usepackage{scrextend}                                                  % need this for addmargin
\usepackage[export]{adjustbox}                                          % need this to get frame for includegraphics
%
%   tikz et al
%
\usepackage{tikz}
\usetikzlibrary{calc,patterns,angles,quotes,shapes,math,decorations,
                through,intersections,lindenmayersystems,backgrounds,
                hobby}
\tikzset{>=latex}                                                       % default to LaTeX arrow head
\usepackage{circuitikz}                                                 % draw circuits
\usepackage{pgfplots}
%
%	more math stuff
%
\usepackage{amsmath,amsfonts,amssymb,amsthm}
\usepackage{bm}
\usepackage{mathtools}
\usepackage{commath}                                                    % get \norm{x}
\usepackage{fixmath}                                                    % get \mathbold
\usepackage{gensymb}                                                    % get \degree
\usepackage{mathrsfs}
\usepackage{hyperref}
\usepackage{subcaption}
\usepackage{authblk}
\usepackage{graphicx}
\usepackage{hyperref}
\usepackage{alltt}
\usepackage{xcolor}
\usepackage{float}
\usepackage{braket}
\usepackage{siunitx}
\usepackage{relsize}
\usepackage{multirow}
\usepackage{esvect}
\usepackage{enumitem}													% use characters instead of numbers in enumerate
%
%	Describe floating point parameters, \fpeval
%
\ExplSyntaxOn
\cs_set_eq:NN \fpeval \fp_eval:n
\ExplSyntaxOff
%
%	Get the x and y components out of a (tikz) coordinate, e.g.
%
%	\coordinate (EP) at (8,5);
%	\gettikzxy{(EP)}{\x}{\y}
%
\makeatletter
\newcommand{\gettikzxy}[3]{%
  \tikz@scan@one@point\pgfutil@firstofone#1\relax
  \edef#2{\the\pgf@x}%
  \edef#3{\the\pgf@y}%
}
\makeatother
%
%
%	Watermarks
%
% \usepackage{draftwatermark}
% \SetWatermarkText{Draft}
% \SetWatermarkScale{5}
% \SetWatermarkLightness {0.9} 
% \SetWatermarkColor[rgb]{0.7,0,0}
%
%
%	theorems, definitions, etc
%
\theoremstyle{definition}
\newtheorem{theorem}{Theorem}[section]
\newtheorem{definition}{Definition}[section]
\newtheorem{proposition}{Proposition}[section]
\newtheorem{lemma}{Lemma}[section]
\newtheorem{example}{Example}[section]
\newtheorem{remark}{Remark}[section]
%
%	For drawing matrix products 
%
\newcommand*{\vertbar}{\rule[-1ex]{0.5pt}{2.5ex}}
\newcommand*{\horzbar}{\rule[.5ex]{2.5ex}{0.5pt}}
%
%
%	The following code allows you to do
%
%	\begin{bmatrix}[r] (or [c] or [l])
%
\makeatletter
\renewcommand*\env@matrix[1][c]{\hskip -\arraycolsep
  \let\@ifnextchar\new@ifnextchar
  \array{*\c@MaxMatrixCols #1}}
\makeatother
%
%	make \arg{min,max}_{n \to \infty} work nicely
%
\newcommand{\argmax}{\operatornamewithlimits{argmax}}
\newcommand{\argmin}{\operatornamewithlimits{argmin}}
%
%	handy commands
%
\newcommand*{\Scale}[2][4]{\scalebox{#1}{$#2$}}
\DeclareMathOperator{\E}{\mathbb{E}}
\DeclareMathOperator{\bda}{\Big \downarrow}									% big down arrow
\newcommand{\veq}{\mathrel{\rotatebox{90}{$=$}}}
%
%	Title, author and date
%
\title{A Few Notes on Linear Algebra}
\author{David Meyer \\ \href{mailto:dmm613@gmail.com}
                            {dmm613@gmail.com}}
\date{Last update: \today}
%
%
%
\begin{document}
\maketitle
%
%
%
\section{Introduction}
\label{sec:introduction}
%
%
%
\section{Linear Subspaces}
\label{sec:linear_subspaces}

\begin{definition} {\bf Linear Subspace:} Let $U \subseteq
\mathbb{R}^{n}$ with $U \neq \emptyset$.  Then $U$ is called a
\emph{linear subspace} of $\mathbb{R}^{n}$ if $U$ is closed under
linear combination \cite{wiki:linear_subspace}. More specifically, 
let $\mathbf{u}^{(1)},\mathbf{u}^{(2)},\hdots, \mathbf{u}^{(k)}$ be 
vectors in $U$ and let $\lambda_{1},\lambda_{2},\hdots, \lambda_{k}$ 
be scalars in $\mathbb{R}$. Then for $U$ to be a linear subspace we 
need that

\begin{equation*}
\sum\limits_{j=1}^{k} \lambda_{j} \mathbf{u}^{(j)} \in U
\label{eqn:linear_subspace}
\end{equation*}

\medskip
\noindent
That is, all linear combinations of vectors in $U$ and scalars in
$\mathbb{R}$ are also in $U$.
\end{definition}


\subsection{Characteristics of Linear Subspaces}
$U \subseteq \mathbb{R}^{n}$ is a (linear) subspace iff

\begin{enumerate}[label=(\alph*)]
\item $\mathbf{0} \in U$ 
\item $\mathbf{u} \in U$ and $\lambda \in \mathbb{R} 
	   \Rightarrow \lambda \mathbf{u} \in U$ 
\item $\mathbf{u}, \mathbf{v} \in U \Rightarrow 
       \mathbf{u}+\mathbf{v} \in U$
\end{enumerate}

\medskip
\noindent
For example, consider a line $U \subseteq \mathbb{R}^{2}$.  
Then for all $\mathbf{u}, \mathbf{v} \in U$ and 
$\lambda \in \mathbb{R}$ we have $\lambda \mathbf{u} \in U$ 
and $\mathbf{u}+\mathbf{v} \in U$. Thus $U$ is a linear 
subspace of $\mathbb{R}^{2}$. This is shown (hopefully) in 
Figure \ref{fig:linear_subspace}.
%
%	draw the linear subspace U
%
\bigskip
\begin{figure}[H]
  \centering
  \resizebox{0.47 \textwidth}{!} {                      % resize figure here if you want
    \begin{tikzpicture}[scale=0.60]                     % can also scale the tikzpicture here
       \coordinate (O)          at (0,0);               % origin
       \coordinate (Xstart)     at (-1,0);              % x-axis 
       \coordinate (Xend)       at (10,0);              % x-axis
       \coordinate (Ystart)     at (0,-1);              % y-axis
       \coordinate (Yend)       at (0,8);               % y-axis
       \coordinate (EP)         at (8,5);               % endpoint of U
%
%
%
	   \gettikzxy{(EP)}{\x}{\y}							% get the x and y components of EP (in units of pt)
%
%
%		\x and \y (\the\pgf@x and \the\pgf@y) are in units of pt. 
%		The pt units are Anglo-American printers' points, where 72.27pt is about 1in.
%
%                     
       \def \slope {\fpeval{\y/\x}};					% the line U goes through the origin so \slope = m = y/x
%														% \y/\x divides out the factor of 72.27pt/in
%
%
%		Coordinates are of the form (x,\slope*x). Since the line U goes through
%		the origin y = mx for some point on the line (I'm using EP = (8,5)).
%		All of this implies that m = y/x = 5/8 = \slope
%
%		So coordinates on the line U are of the form (x,(\slope*x)).
%
       \coordinate (SP)         at (-2,\slope*-2);      % start point (U)
       \coordinate (uEP)        at (2,\slope*2);        % end point (u)
       \coordinate (vEP)        at (3,\slope*3);        % end point (v)
       \coordinate (u+vEP)      at (6,\slope*6);        % end point (u+vEP)
%	
%	draw axes and linear subspace U
%
       \draw[thin,->,black] (Xstart) -- (Xend) node[right] {$x$};       % x axis
       \draw[thin,->,black] (Ystart) -- (Yend) node[above] {$y$};       % y axis
       \draw[thick,magenta] (SP)     -- (EP)   node [right,xshift=1.50mm,yshift=1.50mm] 
					 {$\color{black} {\color{magenta} U} \subseteq \mathbb{R}^{2}$}; 
%
%	draw vectors u, v and u+v
%    
		\draw[thick,green,->](O) -- (uEP) node
                     [right,below,xshift=2.00mm,yshift=0.50mm] 
                     {$\color{green}\mathbf{u}$}; 
		\draw[thick,red,->](uEP) -- (vEP) node
                     [right,xshift=0.25mm,yshift=-1.0mm]
                     {${\color{red}\mathbf{v}}
                       {\color{black}=} 
                       {\color{black} \lambda}
                       {\color{green}\mathbf{u}}$};  
		\draw[thick,blue,->](vEP) -- (u+vEP) node
                     [right,xshift=0.25mm,yshift=-1.25mm]  
                     {${\color{blue}\mathbf{w}}
                       {\color{black}=}
                       {\color{green}\mathbf{u}}
                       {\color{black} +}
                       {\color{red}\mathbf{v}}$}; 
   \end{tikzpicture}
 }
\caption{The line $U$ is a linear subspace of $\mathbb{R}^{2}$}
\label{fig:linear_subspace}
\end{figure}
%
%
%
\section{Linear Maps}
\label{sec:linear_maps}
In linear algebra a linear map is a mapping $\mathbf{V}\to
\mathbf{W}$ between two vector spaces which preserves the
operations of vector addition and scalar
multiplication. Interestingly, the same names and the same
definition are also used for the more general case of modules
over a ring (need module homomorphism cite).
%
%
%
\begin{definition}
\label{def:linear_map}
{\bf Linear Map}: A function $f:\mathbb{R}^{n} \rightarrow
\mathbb{R}^{m}$ is called \emph{linear} if for all
$\mathbf{x},\mathbf{y} \in \mathbb{R}^{n}$ and $\lambda \in
\mathbb{R}$:

\begin{enumerate}[label=(\alph*)]
\item $f(\mathbf{x}+\mathbf{y}) = f(\mathbf{x})+f(\mathbf{y})$		
		\hspace{3.50em} \# vector addition is \emph{distributive}
\label{enumerate:distributive}

\item $f(\lambda \mathbf{x}) = \lambda f(\mathbf{x})$				
		\hspace{7.30em} \# scalar multiplication is
                \emph{compatible} (aka homogeneity) 
\label{enumerate:compatible}
\end{enumerate}


\noindent
Note that in Definition \ref{def:linear_map}
\ref{enumerate:distributive} the vector addition on the left-hand
side is in $\mathbb{R}^{n}$ while on the right-hand side the
vector addition is in $\mathbb{R}^{m}$. Similarly, in Definition
\ref{def:linear_map} \ref{enumerate:compatible} the scalar
multiplication on the left-hand side is in $\mathbb{R}^{n}$ while
the scalar multiplication on the right-hand side is in
$\mathbb{R}^{m}$.
\end{definition}

\subsection{Examples}

\begin{enumerate}
\item $f: \mathbb{R} \rightarrow \mathbb{R}$ by $f(x) = x$

\medskip
This is linear because 
\begin{itemize}
\item $f(x+y) = x + y = f(x) + f(y)$  
\item $f(\lambda x) = \lambda x = \lambda f(x)$	
\end{itemize}

\medskip
\item $f: \mathbb{R} \rightarrow \mathbb{R}$ by $f(x) = x^2$

\medskip
This is not linear because $f(3\cdot 1) = f(3) = 9$ and
for $f$ a linear map we have $f(3 \cdot 1) = 3 \cdot 
f(1) = 3 \cdot 1 = 3$. But $9 \neq 3$ so $f$ is not linear.

\medskip
\item $f: \mathbb{R} \rightarrow \mathbb{R}$ by $f(x) = x+1$

\medskip
This is not linear because $f(0 \cdot 1) = 1$ but 
$0\cdot f(1) = 0$ and $1 \neq 0$.

\smallskip
\noindent
Aside: $f(x) = x+1$ looks linear but under this definition
(\ref{enumerate:distributive} and \ref{enumerate:compatible}
above) $f$ is not linear.
\end{enumerate}

\subsection{Relationship between linear maps and group homomorphism}
As we saw above, in linear algebra, a linear map is a function
between two vector spaces that preserves the vector space
structure, i.e., it preserves addition and scalar
multiplication. A group homomorphism, on the other hand, is a
function between two groups that preserves the group structure,
i.e., it preserves the group operation.


\subsubsection{Group Homomorphism}
Let $(G,\diamond)$ and $(H, \circ)$ be groups. Common usage is to
use $G$ to refer to $(G,\diamond)$. Similarly, $H$ will refer to
$(H, \circ)$. Then a mapping $\phi: G \rightarrow H$ is called a
homomorphism iff

\medskip
\begin{equation*}
\phi(x \diamond y) = \phi(x) \circ \phi(y) \;\; \forall x,y \in G
\end{equation*}

\bigskip
\noindent
Essentially, a homomorphism $\phi: G \rightarrow H$ is a way of
exploring the structure of $H$ by varying $G$ using structure
preserving transformations. Specifically, $\phi$ preserves the
group structure.


\bigskip
\noindent
\begin{example} Define a map 

\begin{equation*}
\phi : G \rightarrow H
\end{equation*}

\bigskip
\noindent
where $G = \mathbb{Z}$ and $H = \mathbb{Z}_2 =
\mathbb{Z}/2\mathbb{Z}$ is the standard group of order two. Then
define $\phi: \mathbb{Z} \rightarrow \mathbb{Z}_2$ by the rule

\begin{equation*}
  \phi(x) =
    \begin{cases}
      0 & \text{if $x$ is even} \\
      1 & \text{if $x$ is odd}  \\
    \end{cases}       
\end{equation*}

\bigskip
\noindent
It is not too hard to check that $\phi$ is a homomorphism. 
Suppose that $x$ and $y$ are two integers. Then there are 
four cases:

\begin{itemize}
 \item $x$ and $y$ are both even
 
 In this case $\phi(x+y) = 0$ (even + even = even). Here
 $\phi(x)+\phi(y) = 0+0 = 0$ so $\phi(x+y) = \phi(x) + \phi(y) =
 0 + 0 = 0$.
 
 \item $x$ and $y$ are both odd
  
In this case $\phi(x+y) = 0$ (odd + odd = even). Here $\phi(x) +
\phi(y) = 1 + 1 = 2 \mod 2 = 0$, so $\phi(x+y) = \phi(x) +
\phi(y) = 1 + 1 = 2 \mod 2 = 0$.

 \item $x$ is even and $y$ is odd or  $x$ is odd and  $y$ is even
 
In this case one is even and the other is odd and $x + y$ is odd.
Here $\phi(x+y) = 1$ and $\phi(x)+\phi(y) = 1+0 = 1$ so
$\phi(x+y) = \phi(x)+\phi(y)$.
 \end{itemize}
 
\smallskip
\noindent
Thus $\phi$ is a homomorphism. Note that in this example
$\diamond = +$ (normal addition in $\mathbb{Z}$) and $\circ = +$
(addition mod 2 in $\mathbb{Z}_2$).
\end{example}



\noindent
Not surprisingly, there is a close relationship between linear
maps and group homomorphisms because any linear map between two
vector spaces can be seen as a group homomorphism between the
additive groups of those vector spaces. Specifically, if we have
a linear map $f: \mathbf{V} \to \mathbf{W}$ between two vector
spaces $\mathbf{V}$ and $\mathbf{W}$ over the same field
$\mathbf{F}$, then we can define a group homomorphism from
$\phi:(\mathbf{V},+) \to (\mathbf{W},+)$ by sending each vector
$\mathbf{v} \in \mathbf{V}$ to $f(\mathbf{v}) \in \mathbf{W}$.
This is a group homomorphism because $f$ preserves addition (its
a linear map, Definition \ref{def:linear_map}
\ref{enumerate:distributive} above) and hence it preserves the
group structure.

\bigskip
\noindent
On the other hand, any group homomorphism between two additive
groups of vector spaces can be seen as a linear map between those
vector spaces. Specifically, if we have a group homomorphism
$\phi: (\mathbf{V}, +) \to (\mathbf{W}, +)$ between two additive
groups of vector spaces, then we can define a linear map from $f:
\mathbf{V}$ to $\mathbf{W}$ by sending each vector $\mathbf{v}
\in \mathbf{V}$ to $\phi(\mathbf{v}) \in \mathbf{W}$. This is a
linear map because $\phi$ preserves vector addition and scalar
multiplication and therefore it preserves the structure of the
underlying vector spaces.

\bigskip
\noindent
So basically, every linear map $f$ between two vector spaces over
the same field can be thought of as a group homomorphism $\phi$
between the additive groups of those vector spaces, and every
group homomorphism between two groups can be thought of as a
linear map between the vector spaces over the same field that are
associated with those groups. So a linear map is a homomorphism of 
vector spaces. However, if the vector spaces have additional structure, 
for example a ring, an algebra, or a lie algebra, then a linear map 
is not always a homomorphism of those additional structures.
%
%
%
\section{Matrices Induce Linear Maps}
Consider a matrix $\bm{A} \in \mathbb{R}^{m \times n}$ and
$\mathbf{x} \in \mathbb{R}^{n}$, and define $f_{\bm{A}}:
\mathbb{R}^{n} \rightarrow \mathbb{R}^{m}$ by $\mathbf{x}
\mapsto \bm{A}\mathbf{x}$.

\medskip
\begin{proposition}
$f_{\bm{A}}$ is a linear map

\bigskip
\noindent
Ok, but why?  Here is one way to look at it:
$f_{\bm{A}}$ a linear map means

\medskip
\begin{itemize}
\item $f_{\bm{A}}(\mathbf{x}+\mathbf{y}) = f_{\bm{A}}(\mathbf{x})+
		f_{\bm{A}}(\mathbf{y}) \Rightarrow		
		\bm{A}(\mathbf{x} + \mathbf{y}) = \bm{A}\mathbf{x} + \bm{A}\mathbf{y}$
		\hspace{4.00em} \# $f_{\bm{A}}$ is distributive (Definition \ref{def:linear_map} 
		\ref{enumerate:distributive})

\item $f_{\bm{A}}(\lambda \mathbf{x}) = \lambda f_{\bm{A}}(\mathbf{x}) \Rightarrow				
		\bm{A}(\lambda \mathbf{x}) = \lambda \bm{A}\mathbf{x}$
		\hspace{11.25em} \# $f_{\bm{A}}$ is compatible (Definition \ref{def:linear_map} 
		\ref{enumerate:compatible})
\end{itemize}

\noindent
Here's an example: Let $\bm{A} \in \mathbb{R}^{m \times 2}$ with $\bm{A} = 
	\left (
		\hspace{-0.50em}
		\begin{array}{cccc}
			\vertbar & \vertbar \\
			a_{1}    & a_{2}    \\
			\vertbar & \vertbar 
		\end{array}
		\hspace{-0.50em}
	\right )$ 
and let $\bm{x} = \left ( 
					\hspace{-0.50em}
					\begin{array}{cccc}
						x_1 \\
						x_2
					\end{array} 
					\hspace{-0.50em}
					\right )$
and 
$\bm{y} = \left ( 
					\hspace{-0.50em}
					\begin{array}{cccc}
						y_1 \\
						y_2
					\end{array} 
					\hspace{-0.50em}
					\right )$
be vectors in $\mathbb{R}^{2}$. Then


\begin{equation*}
\begin{array}{llll}
\bm{A}(\mathbf{x}+\mathbf{y})
&=&
	\left (
		\hspace{-0.50em}
		\begin{array}{cccc}
			\vertbar & \vertbar \\
			a_{1}    & a_{2}    \\
			\vertbar & \vertbar 
		\end{array}
		\hspace{-0.50em}
	\right )
	\hspace{-0.25em}
	\left (
	\hspace{-0.25em}
		\left (
			\hspace{-0.50em}
			\begin{array}{cccc}
				x_1 \\
				x_2
			\end{array}
			\hspace{-0.50em}
		\right ) 
	+
		\left (
			\hspace{-0.50em}
			\begin{array}{cccc}
				y_1 \\
				y_2
			\end{array}
			\hspace{-0.50em}
		\right )
	\hspace{-0.25em}
	\right )                                                &\hspace{3.00em}\mathrel{\#} \text{definitions of $\bm{A}$, $\mathbf{x}$, and $\mathbf{y}$} \\
[25pt]
&=&
	\left (
		\hspace{-0.50em}
		\begin{array}{cccc}
			\vertbar & \vertbar \\
			a_{1}    & a_{2}    \\
			\vertbar & \vertbar 
		\end{array}
		\hspace{-0.50em}
	\right )
	\hspace{-0.25em}
	\Bigg (
		\hspace{-0.50em}
		\begin{array}{cccc}
			x_1 + y_1\\
			x_2 + y_2
		\end{array}
		\hspace{-0.50em}
	\Bigg ) 												&\hspace{3.00em}\mathrel{\#} \text{matrix (vector) addition in $\mathbb{R}^{2}$} \\
[25pt]
&=&
	\left (
		\hspace{-0.50em}
		\begin{array}{cccc}
			\vertbar \\
			a_{1}    \\
			\vertbar 
		\end{array}	
		\hspace{-0.50em}
	\right )
	\Big (
		\hspace{-0.50em}
		\begin{array}{cccc}
			x_1 + y_1 
		\end{array}
		\hspace{-0.50em}
	\Big ) 
	+
	\left (
		\hspace{-0.50em}
		\begin{array}{cccc}
			\vertbar \\
			a_{2}    \\
			\vertbar 
		\end{array}
		\hspace{-0.50em}
	\right )
	\Big ( 
		\hspace{-0.50em}
		\begin{array}{cccc}
			x_2 + y_2
		\end{array}
		\hspace{-0.50em}
	\Big ) 												&\hspace{3.00em}\mathrel{\#} \text{definition of the matrix product} \\
[25pt]
&=&
	\left (
		\hspace{-0.50em}
		\begin{array}{cccc}
			\vertbar \\
			a_{1}    \\
			\vertbar 
		\end{array}	
		\hspace{-0.50em}
	\right )
	x_1
	+
	\left (
		\hspace{-0.50em}
		\begin{array}{cccc}
			\vertbar \\
			a_{1}    \\
			\vertbar 
		\end{array}	
		\hspace{-0.50em}
	\right )
	y_1
	+
	\left (
		\hspace{-0.50em}
		\begin{array}{cccc}
			\vertbar \\
			a_{2}    \\
			\vertbar 
		\end{array}	
		\hspace{-0.50em}
	\right )
	x_2
	+
	\left (
		\hspace{-0.50em}
		\begin{array}{cccc}
			\vertbar \\
			a_{2}    \\
			\vertbar 
		\end{array}	
		\hspace{-0.50em}
	\right )
	y_2 												&\hspace{3.00em}\mathrel{\#} \text{matrix product distributes over addition} \\
[25pt]
&=&
	\left (
		\hspace{-0.50em}
		\begin{array}{cccc}
			\vertbar \\
			a_{1}    \\
			\vertbar 
		\end{array}	
		\hspace{-0.50em}
	\right )
	x_1
	+
	\left (
		\hspace{-0.50em}
		\begin{array}{cccc}
			\vertbar \\
			a_{2}    \\
			\vertbar 
		\end{array}	
		\hspace{-0.50em}
	\right )
	x_2
	+
	\left (
		\hspace{-0.50em}
		\begin{array}{cccc}
			\vertbar \\
			a_{1}    \\
			\vertbar 
		\end{array}	
		\hspace{-0.50em}
	\right )
	y_1
	+
	\left (
		\hspace{-0.50em}
		\begin{array}{cccc}
			\vertbar \\
			a_{2}    \\
			\vertbar 
		\end{array}	
		\hspace{-0.50em}
	\right )
	y_2													&\hspace{3.00em}\mathrel{\#} \text{vector addition is commutative} \\
[25pt]
&=&
	\left (
		\hspace{-0.50em}
		\begin{array}{cccc}
			\vertbar & \vertbar \\
			a_{1}    & a_{2}    \\
			\vertbar & \vertbar 
		\end{array}
		\hspace{-0.50em}
	\right )
	\hspace{-0.25em}
		\Bigg (
			\hspace{-0.50em}
			\begin{array}{cccc}
				x_1 \\
				x_2
			\end{array}
			\hspace{-0.50em}
		\Bigg )
		+
		\left (
		\hspace{-0.50em}
		\begin{array}{cccc}
			\vertbar & \vertbar \\
			a_{1}    & a_{2}    \\
			\vertbar & \vertbar 
		\end{array}
		\hspace{-0.50em}
	\right )
	\hspace{-0.25em}
		\Bigg (
			\hspace{-0.50em}
			\begin{array}{cccc}
				y_1 \\
				y_2
			\end{array}
			\hspace{-0.50em}
		\Bigg )											&\hspace{3.00em}\mathrel{\#} \text{definition of the matrix product} \\
[25pt]
&=&
	\left (
		\hspace{-0.50em}
		\begin{array}{cccc}
			\vertbar & \vertbar \\
			a_{1}    & a_{2}    \\
			\vertbar & \vertbar 
		\end{array}
		\hspace{-0.50em}
	\right )
		\bm{x}		+
		\left (
		\hspace{-0.50em}
		\begin{array}{cccc}
			\vertbar & \vertbar \\
			a_{1}    & a_{2}    \\
			\vertbar & \vertbar 
		\end{array}
		\hspace{-0.50em}
		\right )
		\bm{y}											&\hspace{3.00em}\mathrel{\#}\bm{x} = \Bigg (
																							\hspace{-0.50em}
																							\begin{array}{cccc}
																								x_1 \\
																								x_2
																							\end{array}
																							\hspace{-0.50em}
																							\Bigg )
																							\text{ and } 
																							\bm{y} = \Bigg (
																							\hspace{-0.50em}
																							\begin{array}{cccc}
																								y_1 \\
																								y_2
																							\end{array}
																							\hspace{-0.50em}
																							\Bigg ) \\
[20pt]
&=&
\bm{A}\mathbf{x} + \bm{A}\mathbf{y}						&\hspace{3.00em}\mathrel{\#} \bm{A} = \left (
																							\hspace{-0.50em}
																							\begin{array}{cccc}
																								\vertbar & \vertbar \\
																								a_{1}    & a_{2}    \\
																								\vertbar & \vertbar 
																							\end{array}
																							\hspace{-0.50em}
																							\right ) \\
\end{array}
\end{equation*}

\bigskip
\noindent
So $\bm{A}(\mathbf{x}+\mathbf{y}) =
\bm{A}\mathbf{x}+\bm{A}\mathbf{y}$. Here we see that the matrix
$\bm{A}$, which is just a table of real (or complex) numbers,
induces the abstract linear map $f_{\bm{A}}$ (and vice versa)
\cite{wiki:linear_map}.
\end{proposition}

\bigskip
\noindent
Suppose $f$ is a linear map and let $\bm{x} = \left
( \hspace{-0.35em} \begin{array}{cccc} x_1, x_2,\hdots,
x_n \end{array} \hspace{-0.35em} \right)$ be a vector in
$\mathbb{R}^{n}$. Then define the canonical unit vectors in
$\mathbb{R}^{n}$ as follows:

\begin{definition}
\label{def:canonical_unit_vectors}
{\bf Canonical Unit Vectors:} The canonical unit vectors in 
$\mathbb{R}^{n}$ are the vectors $\mathbf {e}_{i}$ 


\medskip
\begin{equation*}
B = \{\mathbf {e} _{i}: 1 \leq i \leq n\}
\end{equation*}

\bigskip
\noindent
where $\mathbf{e}_i$ is the $i^\text{th}$ unit vector, that is,
it has a one in the $i^\text{th}$ coordinate (position) and zeros
everywhere else\footnote{The canonical unit vectors look
surprisingly like the one-hot encoding used in machine learning
\cite{wiki:one-hot}.}.  The set $B$ forms a basis, sometimes
called the canonical basis, for the vector space
$\mathbb{R}^{n}$.

\bigskip
\noindent
The basis vectors $\mathbf{e}_i$ have column vector format

\begin{flalign*}
\mathbf{e}_{1} = 
\left (
\begin{array}{llll} 
1 \\
0 \\
\vdots \\
0 \\
0
\end{array}
\right )
\!\! , \hspace{0.25em}
\mathbf{e}_{2} = 
\left ( 
\begin{array}{llll}
0 \\
1 \\
\vdots \\
0 \\
0
\end{array}
\right )
\!\! ,
\hdots
\!\! , \hspace{0.25em}
\mathbf{e}_{n-1} =
\left (
\begin{array}{llll} 
0 \\
0 \\
\vdots \\
1 \\
0
\end{array} 
\right )
\!\! , \hspace{0.25em}
\mathbf{e}_{n} =
\left (
\begin{array} {llll} 
0 \\
0 \\
\vdots \\
0 \\
1
\end{array}
\right )
\end{flalign*}
\end{definition}

\bigskip
\noindent
Here is an interesting observation about linear maps and basis vectors:

\begin{equation*}
\begin{array}{llll}
f(\bm{x})
&=& f(x_1 \bm{e_1} + x_2 \bm{e_2} + \hdots + x_n \bm{e_n})				
		&\hspace{1.25em} \mathrel{\#} \text{$\bm{x}$ is a linear combination 
		of basis vectors} \\
[5pt]
&=& f(x_1 \bm{e_1}) + f(x_2 \bm{e_2}) + \hdots + f(x_n \bm{e_n})
		&\hspace{1.25em} \mathrel{\#} \text{$f$ a linear map $\Rightarrow$ 
		$f$ is distributive (Definition \ref{def:linear_map} 
		\ref{enumerate:distributive})} \\
[5pt]
&=& x_1 f(\bm{e_1}) + x_2 f(\bm{e_2}) + \hdots + x_n f(\bm{e_n})
		&\hspace{1.25em} \mathrel{\#} \text{$f$ a linear map $\Rightarrow$ 
		$f$ is compatible (Definition \ref{def:linear_map} 
		\ref{enumerate:compatible})} \\
\end{array}
\end{equation*}

\medskip
\noindent
Since $f(\bm{x}) = x_1 f(\bm{e_1}) + x_2 f(\bm{e_2}) + \hdots + x_n f(\bm{e_n})$
apparently in order to understand $f$ you only have to know how $f$ maps the
basis vectors.																						
%
%
%
\section{Conclusions}
\label{sec:conclusions}
%
%
%
\section{Acknowledgements}
\label{sec:acknowledgements}
%
%	LaTeX source on overleaf.com
%
\section*{\LaTeX \hspace{0.025 mm} Source}
\url{https://www.overleaf.com/read/pmsddwthrchw}
%
%
%	get a bibliography
%
%	Note:.bib files go in ~/Library/texmf/bibtex/bib with TeXShop (MacTeX).
%	You can also use an absolute path, e.g. \bibliography{/Users/dmm/papers/bib/qc}
%
\bibliographystyle{plain}
\bibliography{qc}
%
%
%
\section*{Appendix A: Draw Some Matrices}

\bigskip
\begin{equation*}
\begin{array}{llll}
A^{n \times m} 
&=& \left (
		\begin{array}{ccc}
		\horzbar & a_{1} & \horzbar \\
		\horzbar & a_{2} & \horzbar \\
				 & \vdots    &          \\
		\horzbar & a_{n} & \horzbar
	\end{array}
	\right )
\end{array}
\end{equation*}

\bigskip

\begin{equation*}
\begin{array}{llll}
A^{T}
&=& \left (
	\begin{array}{cccc}
		\vertbar & \vertbar &        & \vertbar \\
		a_{1}    & a_{2}    & \ldots & a_{n}    \\
		\vertbar & \vertbar &        & \vertbar 
	\end{array}
	\right )
\end{array}
\end{equation*}


\bigskip
\setlength{\extrarowheight}{1ex}


\begin{equation*}
\begin{array}{llll}
B^{n \times k} 
&=& \left (
		\begin{array}{ccc}
		\horzbar & b_{1} & \horzbar \\
		\horzbar & b_{2} & \horzbar \\
				 & \vdots    &          \\
		\horzbar & b_{n} & \horzbar
	\end{array}
	\right )
\end{array}
\end{equation*}

\begin{equation*}
\begin{array}{llll}
AB := A^{T}B
&=&
\left(
	\begin{array}{cccc}
		\vertbar & \vertbar &        & \vertbar \\
		a_{1}    & a_{2}    & \ldots & a_{n}    \\
		\vertbar & \vertbar &        & \vertbar 
	\end{array}
	\right )
	\left(
		\begin{array}{ccc}
		\horzbar & b_{1} & \horzbar \\
		\horzbar & b_{2} & \horzbar \\
				 & \vdots    &          \\
		\horzbar & b_{n} & \horzbar
	\end{array}
	\right ) \\

&=&
	\left (
		\begin{array}{cccc}
		a_{1}^{T} b_{1} & a_{1}^{T} b_{2} & \hdots & a_{1}^{T} b_{k} \\
		a_{2}^{T} b_{1} & a_{2}^{T} b_{2} & \hdots & a_{2}^{T} b_{k} \\
		\vdots          & \vdots          & \ddots & \vdots       										 \\
		a_{m}^{T} b_{1} & a_{m}^{T} b_{2} & \hdots & a_{m}^{T} b_{k}
	\end{array}
	\right ) \\ \\

&=&
	\left(
		\begin{array}{cccc}
		\langle a_{1}, b_{1} \rangle & \langle  a_{1}, b_{2} \rangle & \hdots & \langle a_{1}, b_{k} \rangle \\
		\langle a_{2}, b_{1} \rangle & \langle a_{2}, b_{2} \rangle  & \hdots & \langle a_{2}, b_{k} \rangle \\
		\vdots                       & \vdots                        & \ddots & \vdots       										 \\
		\langle a_{m}, b_{1} \rangle & \langle a_{m}, b_{2} \rangle  & \hdots & \langle a_{m}, b_{k} \rangle
	\end{array}
	\right ) \\
 \end{array}
\end{equation*}



\newpage
\begin{equation*}
\begin{array}{llll}
A^{T} 
&=& \left(
		\begin{array}{ccc}
		\horzbar & a^{T}_{1} & \horzbar \\
		\horzbar & a^{T}_{2} & \horzbar \\
				 & \vdots    &          \\
		\horzbar & a^{T}_{n} & \horzbar
	\end{array}
	\right )
\end{array}
\end{equation*}
%
%	done
%
\end{document} 

