\documentclass[11pt, oneside]{article}   	% use "amsart" instead of "article" for AMSLaTeX format


% \usepackage{draftwatermark}
% \SetWatermarkText{Draft}
% \SetWatermarkScale{5}
% \SetWatermarkLightness {0.9} 
% \SetWatermarkColor[rgb]{0.7,0,0}


\usepackage{geometry}                           % See geometry.pdf to learn the layout options. There are lots.
\geometry{letterpaper}                          % ... or a4paper or a5paper or ... 
%\geometry{landscape}                           % Activate for for rotated page geometry
%\usepackage[parfill]{parskip}                  % Activate to begin paragraphs with an empty line rather than an indent
\usepackage{graphicx}                           % Use pdf, png, jpg, or eps� with pdflatex; use eps in DVI mode
                                                % TeX will automatically convert eps --> pdf in pdflat
                                                % TeX will automatically convert eps --> pdf in pdflatex		
\usepackage{amssymb}
\usepackage{mathrsfs}
\usepackage{hyperref}
\usepackage{url}
\usepackage{subcaption}
\usepackage{authblk}
\usepackage{amsmath}
\usepackage{mathtools}
\usepackage{graphicx}
\usepackage[export]{adjustbox}
\usepackage{fixltx2e}
\usepackage{hyperref}
\usepackage{alltt}
\usepackage{color}
\usepackage[utf8]{inputenc}
\usepackage[english]{babel}
\usepackage{float}
\usepackage{bigints}
\usepackage{braket}
\usepackage{siunitx}
\usepackage{relsize}
\usepackage{setspace}


%
% so you can do e.g., \begin{bmatrix}[r] (or [c] or [l])
%

\newcommand{\argmax}{\operatornamewithlimits{argmax}}
\newcommand{\argmin}{\operatornamewithlimits{argmin}}


\title{Euler's Product Formula and the Riemann Zeta Function}
\author{David Meyer \\ dmm613@gmail.com}

\date{Last update: \today}							% Activate to display a given date or no date


\begin{document}
\maketitle

\section{Introduction}
In the 1730s Leonhard Euler proved one of the most important
theorems in number theory, known today as Euler's product formula
\cite{euler_product_formula}.  Euler's product formula exposes
the deep relationship between the prime numbers and the Riemann
zeta function \cite{wiki:zeta}. Euler's proof appeared in his
thesis titled \emph{Variae observationes circa series infinitas}
(Various Observations about Infinite Series), which was published
by the St. Petersburg Academy in 1737
\cite{mactutor:history_of_calculus}. The discovery of the product
formula was an important advancement in number theory and is
perhaps the main reason why (or at least one of the main reasons
why) the zeta function plays such a central role in the study of
prime numbers.

\medskip
\noindent
The Euler product formula is

\bigskip
\begin{equation}
\zeta (s) = \prod _{p{\text{ prime}}}{\frac {1}{1-p^{-s}}}
\label{eqn:euler_product_zeta}
\end{equation}

\bigskip
\noindent
Here the left hand side is the famous Riemann zeta function 
$\zeta (s)$: 

\bigskip
\begin{equation*}
\zeta (s) = \sum _{n=1}^{\infty } \frac {1}{n^{s}}
          = 1 + \frac {1}{2^{s}} + \frac {1}{3^{s}} + \frac
          {1}{4^{s}}  + \frac {1}{5^{s}}  + \cdots  
\end{equation*}

\bigskip
\noindent
and the product on the right hand side extends over all prime
numbers $p$: 

\bigskip
\begin{equation*}
\prod _{p {\text{ prime}}}  \frac {1}{1-p^{-s}} = \frac
{1}{1-2^{-s}} \cdot \frac {1}{1-3^{-s}} \cdot \frac {1}{1-5^{-s}}
\cdots  \frac {1}{1-p^{-s}} \cdots  
\end{equation*}


\bigskip
\noindent
So how did Euler discover the amazing relationship 
shown in Equation (\ref{eqn:euler_product_zeta})?


\section{Euler's Proof of the Product Formula}
The proof presented here (amazingly) makes use only of simple
algebra and is the method by which Euler originally discovered
the product formula.  Euler's proof is quite clever and takes
advantage of a sieving process that is analogous to the ancient
Sieve of Eratosthenes \cite{wiki:sieve_of_eratosthenes}. The
proof is iterative and proceeds as follows:


\bigskip
\noindent
Euler's proof starts with the zeta function\footnote{Euler first
studied $\zeta(s)$ as a real function; Riemann was the first to
view it as a complex function \cite{nichols2016}.}, Equation
(\ref{equation1}), as the input to the first iteration of the
proof.

\bigskip
\begin{equation}
\zeta(s) = \sum _{n=1}^{\infty } \frac {1}{n^{s}} = 1 + \frac
{1}{2^{s}} + \frac {1}{3^{s}} + \frac{1}{4^s} + \frac{1}{5^s} +
\cdots     
\label{equation1}  
\end{equation}

\bigskip
\noindent
The first step of the algorithm is to multiply both sides of the
input, Equation (\ref{equation1}), by the second term on its hand
right side, $\frac{1}{2^s}$. This gives us

\bigskip
\begin{equation}
\frac {1}{2^{s}} \zeta (s) = \frac {1}{2^{s}} + \frac {1}{4^{s}}
+ \frac {1}{6^{s}} + \frac {1}{8^{s}} + \frac {1}{10^{s}} +
\cdots  
\label{equation2}
\end{equation}

\bigskip
\noindent
The next step is to subtract Equation (\ref{equation2}) from
Equation (\ref{equation1}), which gives us a new equation,
Equation (\ref{equation3}). This new equation will serve as the
input to the next iteration.

\bigskip
\begin{equation}
\left (1- \frac {1}{2^{s}} \right ) \zeta (s) = 1 + \frac
{1}{3^{s}} + \frac {1}{5^{s}} + \frac {1}{7^{s}} + \frac
{1}{9^{s}} + \frac {1}{11^{s}} + \frac {1}{13^{s}} + \cdots  
\label{equation3}
\end{equation}

\bigskip
\noindent
Notice that this process knocks out all of the terms on the right
hand side of the input equation (Equation (\ref{equation1}))
which have a factor of 2 in the denominator. These are the terms
that are "sieved" in this iteration of the algorithm.


\bigskip
\noindent
Now we repeat the previous steps with their output (Equation
(\ref{equation3})) as input.  Accordingly we multiply Equation
(\ref{equation3}) by the the second term on its right hand side,
$\frac{1}{3^s}$, which gives us

\bigskip
\begin{equation}
\frac {1}{3^{s}} \left (1 - \frac {1}{2^{s}} \right ) \zeta (s) =
\frac {1}{3^{s}} + \frac {1}{9^{s}} + \frac {1}{15^{s}} + \frac
{1}{21^{s}} + \frac {1}{27^{s}} + \frac {1}{33^{s}} + \cdots  
\label{equation4}
\end{equation}

\bigskip
\noindent
Subtracting Equation (\ref{equation4}) from Equation
(\ref{equation3}) we get a new equation, Equation
(\ref{equation5}):

\bigskip
\begin{equation}
\left ( 1 - \frac {1}{3^{s}} \right ) \left ( 1 - \frac
{1}{2^{s}} \right ) \zeta (s) = 1 + \frac {1}{5^{s}} + \frac
{1}{7^{s}} + \frac {1}{11^{s}} + \frac {1}{13^{s}} + \frac
{1}{17^{s}} + \cdots  
\label{equation5}
\end{equation}

\bigskip
\noindent
This iteration knocks out all of the terms on the right hand side
of Equation (\ref{equation3}) which have a factor of 3 in the
denominator. The next iteration will knock out all of the terms
on the right hand side that have a factor of 5 in the
denominator, yielding

\bigskip
\begin{equation*}
\left ( 1 - \frac {1}{5^{s}} \right )  \left ( 1 - \frac
{1}{3^{s}} \right ) \left ( 1 - \frac {1}{2^{s}} \right ) \zeta
(s) = 1 + \frac {1}{7^{s}} + \frac {1}{11^{s}} + \frac
{1}{13^{s}} + \frac {1}{17^{s}} + \cdots  
\end{equation*}

\medskip
\bigskip
\noindent
What we see is that this algorithm "sieves" the right hand sides
of these successive equations while building up the product of
$\left ( 1 - \frac{1}{p^{s}} \right)$ terms for prime $p$ on the
left hand side.

\bigskip
\noindent
If we repeat this process infinitely we wind up with  

\begin{equation*}
\cdots \left ( 1 - \frac {1}{13^{s}} \right ) \left ( 1 - \frac
{1}{11^{s}} \right ) \left ( 1 - \frac {1}{7^{s}} \right ) \left
( 1 - \frac {1}{5^{s}} \right )  \left ( 1 - \frac {1}{3^{s}}
\right ) \left ( 1 - \frac {1}{2^{s}} \right ) \zeta (s) = 1 
\end{equation*}

\bigskip
\noindent
Solving for $\zeta(s)$ we get

\bigskip
\begin{equation*}
\zeta (s) = \frac {1}{\left ( 1 - \frac {1}{2^{s}} \right ) \left
( 1 - \frac {1}{3^{s}} \right ) \left ( 1 - \frac {1}{5^{s}}
\right ) \left ( 1 - \frac {1}{7^{s}} \right )  
 \left ( 1 - \frac {1}{11^{s}} \right ) \left ( 1 - \frac {1}{13^{s}} \right )\cdots}
\end{equation*}

\bigskip
\bigskip
\noindent
which can be written in a perhaps more familiar (and more
compact) way as an infinite product over all of the primes:

\bigskip
\begin{equation*}
\zeta (s) = \prod _{p{\text{ prime}}} \frac {1}{1-p^{-s}}
\end{equation*}

\bigskip
\noindent
which is Euler's product formula for the zeta
function\footnote{This result is frequently referred to simply as
"Euler's product formula".}.

\bigskip
\noindent
Finally, to make this proof rigorous we need to show that when
$\Re (s) > 1$ the right hand side of the sieved equations
approaches 1.  Fortunately for us this follows immediately from
the convergence of the Dirichlet series for $\zeta(s)$
\cite{wiki:dirichlet_series}. $\blacksquare$

\section{Conclusions}
Euler's product formula is one of the most important results in
the history of number theory and paved the way for Dirichlet,
Riemann and others to the fuse arithmetic and analysis into
analytic number theory \cite{wiki:analytic_number_theory}.  In
fact, so important is Euler's product formula in the history of
number theory that John Derbyshire, in his 2003 book \emph{Prime
Obsession: Bernhard Riemann and the Greatest Unsolved Problem in
Mathematics} \cite{book:prime_obsession}, calls Euler's result
the "Golden Key" to indicate it's significance in the development
of analytic number theory and number theory in general.

\bigskip
\noindent
As an aside, Mathologer has a very nice video on Euler's product
formula that has some fancy animations that make it easy to see
how the sieve works \cite{mathologer:eulers_product_formula}.
%
%
%
\section*{Acknowledgements}
Thanks to Matthew Anderson who found a typo in 
Equation (\ref{equation1}).

%
%	LaTeX source on overleaf.com
%
\section*{\LaTeX}
\url{https://www.overleaf.com/read/hvqdtdyftqpb}
%
%	get a bibliography
%
%	Note:.bib files go in ~/Library/texmf/bibtex/bib with TeXShop (MacTeX).
%	You can also use an absolute path, e.g. \bibliography{/Users/dmm/papers/bib/qc}
%
\bibliographystyle{plain}
\bibliography{qc}
%
\end{document} 

