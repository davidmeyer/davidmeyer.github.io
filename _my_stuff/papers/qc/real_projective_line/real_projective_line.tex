%
%	Use lualatex
%
% !TEX TS-program = LuaLaTeX
%
%	To get overleaf.com to use lualatex see
%	https://www.overleaf.com/learn/how-to/Changing_compiler.
%	Basically, change the compiler in Settings.
%
%
\documentclass{article}
%
%
%	LaTeX template
%
%	projective_geometry.tex
%
%	David Meyer
%	dmm613@gmail.com
%	10 Mar 2024
%
%
%   get various packages
%
\usepackage[margin=1.0in]{geometry}                                     % adjust margins
\geometry{letterpaper}                                                  % or a4paper or a5paper or ... 
\usepackage{url}                                                        % need this to use URLs in bibtex
\usepackage{setspace}                                                   % need this for \setstrech{...}
\usepackage{scrextend}                                                  % need this for addmargin
\usepackage[export]{adjustbox}                                          % need this to get frame for includegraphics
%
%   tikz et al
%
\usepackage{tikz}
\usetikzlibrary{calc,patterns,angles,quotes,shapes,math,decorations,
                through,intersections,lindenmayersystems,backgrounds,
                hobby}
\tikzset{>=latex}                                                       % default to LaTeX arrow head
\usepackage{circuitikz}                                                 % draw circuits    
\usepackage{pgfplots}
%
%	more math stuff
%
\usepackage{amsmath,amsfonts,amssymb,amsthm}
\usepackage{bm}
\usepackage{mathtools}
\usepackage{commath}                                                    % get \norm{x}
\usepackage{fixmath}                                                    % get \mathbold
\usepackage{gensymb}                                                    % get \degree
\usepackage{mathrsfs}
\usepackage{hyperref}
\usepackage{subcaption}
\usepackage{authblk}                                                    % comment out if using beamer (stops \author{} from working in beamer)
\usepackage{graphicx}
\usepackage{hyperref}
\usepackage{alltt}
\usepackage{xcolor}
\usepackage{colortbl}                                                   % \rowcolor{yellow!75} etc
\usepackage{float}
\usepackage{braket}
\usepackage{siunitx}
\usepackage{relsize}
\usepackage{multirow}
\usepackage{esvect}
\usepackage{enumitem}                                                   % use characters instead of numbers in enumerate
\usepackage{changepage}                                                 % needed for \begin{adjustwidth}{-3.25em}{-2.0em} (left justify)
\usepackage{bigints}
\usepackage{multirow}                                                   % \multirow and \multicolumn
\usepackage{luacode}
%
%	lualatex
%
%	Put this before \documentclass
%
%	!TEX TS-program = LuaLaTeX
%
%
\usepackage{luatex85,luamplib}
% \mplibnumbersystem{double}
% \mplibtextextlabel{enable}
%
%
%
%	Describe floating point parameters, \fpeval
%
\ExplSyntaxOn
\cs_set_eq:NN \fpeval \fp_eval:n
\ExplSyntaxOff
%
%	Get the x and y components out of a coordinate, e.g.
%
%	\coordinate (EP) at (8,5);
%	\gettikzxy{(EP)}{\slopex}{\slopey}
%
\makeatletter
\newcommand{\gettikzxy}[3]{%
  \tikz@scan@one@point\pgfutil@firstofone#1\relax
  \edef#2{\the\pgf@x}%
  \edef#3{\the\pgf@y}%
}
\makeatother
%
%
%	Watermarks
%
% \usepackage{draftwatermark}
% \SetWatermarkText{Draft}
% \SetWatermarkScale{5}
% \SetWatermarkLightness {0.9} 
% \SetWatermarkColor[rgb]{0.7,0,0}
%
%
%	theorems, definitions, etc
%
\theoremstyle{definition}
\newtheorem{theorem}{Theorem}[section]
\newtheorem{definition}{Definition}[section]
\newtheorem{proposition}{Proposition}[section]
\newtheorem{lemma}{Lemma}[section]
\newtheorem{example}{Example}[section]
\newtheorem{remark}{Remark}[section]
%
%	For drawing matrix products 
%
\newcommand*{\vertbar}{\rule[-1ex]{0.5pt}{2.5ex}}
\newcommand*{\horzbar}{\rule[.5ex]{2.5ex}{0.5pt}}

%
%	The following code allows you to do
%
%	\begin{bmatrix}[r] (or [c] or [l])
%
\makeatletter
\renewcommand*\env@matrix[1][c]{\hskip -\arraycolsep
  \let\@ifnextchar\new@ifnextchar
  \array{*\c@MaxMatrixCols #1}}
\makeatother
%
%	make \arg{min,max}_{n \to \infty} work nicely
%
\newcommand{\argmax}{\operatornamewithlimits{argmax}}
\newcommand{\argmin}{\operatornamewithlimits{argmin}}
%
%	handy commands
%
\newcommand*{\Scale}[2][4]{\scalebox{#1}{$#2$}}
\DeclareMathOperator{\E}{\mathbb{E}}
\DeclareMathOperator{\bda}{\Big \downarrow}						% big down arrow
\newcommand{\veq}{\mathrel{\rotatebox{90}{$=$}}}
%
%	Title, author and date
%
\title{A Bit on the Real Projective Line}
\author{David Meyer \\ \href{mailto:dmm613@gmail.com}
                            {dmm613@gmail.com}}
\date{Last update: \today}
%
%
%
\begin{document}
\maketitle
%
%
%	A few parameters
%
\setlength			{\fboxsep} {0.75em}						% space between fcolorbox and content
\setlength			{\fboxrule}{0.50pt}						% fcolorbox line thickness
\def \scalefactor	{0.50}									% resizebox scale factor
\def \twidth		{\textwidth}							% resizebox height = \scalefactor * \w
%
%
%
\section{The Real Projective Line}
\label{section:real_projective_line}

\vspace{2em}

%
%	Scale image if you like
%
\def \scalefactor {1.00}
%
%	Draw the picture
%
\begin{minipage}[c]{0.50\textwidth}
\begin{figure}[H]
 \centering
 \resizebox{\scalefactor \textwidth}{!} {
    \begin{tikzpicture}[scale=3]
%
%	Draw the axes
%
       \draw[-] (-1.75,0) -- (1.75,0) node[below,xshift=-0.15cm]               {$x$};
       \draw[]  (1.50,0)              node[above,xshift=0.25cm]                {$\lambda[1,0]$};
       \draw[-] (0,-0.25) -- (0,1.50) node[right,xshift=0.20cm,yshift=-0.20cm] {$y$};
       \draw[]  (0,1.50)              node[above]                              {$\lambda[0,1]$};
%
%	Draw the y = 1 line
%
       \draw [-] (-1.75,1) -- (1.75,1);
%
%	Draw the y = x line
%
       \draw [-] (-1.00,-1.00) -- (1.25,1.25) node[above,xshift=-0.25cm] {$\lambda[x,1] = \lambda [\cos \theta, \sin \theta]$};
%
%	Intersection of y = 1 and y = x
%
       \fill [red] (1,1) circle (0.75pt);
       \node [below,yshift=-0.1cm,xshift=0.25cm] at (1,1) {$1,\frac{\pi}{4}$};
%
%	Draw theta
%
       \draw [->] (0.3,0.3) arc (45:90:0.42);
       \node[] at (0.10,0.25) {$\theta$};
% 
%	Draw upper half of unit circle in blue
%
      \draw [blue,thin]  (1,0)  arc (0:180:1);
      \draw [dashed,thin](-1,0) arc (180:360:1);

%
%	Put some black points on the unit circle
%
      \fill [black] (1,0)  circle (0.75pt); 
      \fill [black] (-1,0) circle (0.75pt); 
      \fill [black] (0,1)  circle (0.75pt);
%
%	Draw a dot at the origin
%
      \fill [red] (0,0) circle (0.75pt);
%
%	Draw a few coordinates of interest
% 
      \draw (1,0)  node[below] {$(1,0)$};
      \draw (-1,0) node[below] {$(-1,0)$};
      \draw (0,1)  node[below,yshift=-0.10cm] {$(0,1)$};
    \end{tikzpicture}														% end tikzpicture
   }																		% end resizebox
  \caption{Real Projective Line Setup}
  \label{table:real_projective_line_setup}
\end{figure}
\end{minipage}

\def \scalefactor {0.75}

\begin{minipage}[c]{0.50\textwidth}
\begin{figure}[H]
 \centering
 \resizebox{\scalefactor \textwidth}{!} {
   \begin{tikzpicture}[scale=3]
     \draw (0,0) circle (1);												% draw unit circle
     %\node[right] at (1,1) {$x,\theta$};									% annotiations
     \node[right] at (1,0) {$1,\frac{\pi}{4}$};
     \node[above] at (0,1) {$0,0$};
     \node[left] at (-1,0) {$-1,-\frac{\pi}{4}$};
     \node[below,yshift=-0.15cm] at (0,-1) {$-\infty,-\frac{\pi}{2}; \infty,\frac{\pi}{2}$};
   \end{tikzpicture}
  }
 \caption{The Real Projective Line is a Circle: $(x,\theta)$}
 \label{table:real_projective_line_is_a_circle}
\end{figure}
\end{minipage}

\vspace{2.0cm}

%
%	Experiments with \directlua{.}
%
%	Pass variable defined in LaTeX (\num) to the Lua interpreter
%
\def \num {36}												% the parameter \num is defined in LaTeX

% \luaexec{tex.sprint(string.format('\%.3f', math.pi))}

\directlua{													% call the Lua interpreter 
  local result = math.sqrt(\num)							% result is defined in the Lua interpreter
  tex.sprint("The square root of ", \num, " is: ", result)	% pass \num to the Lua interpreter
}

% $\pi = \directlua{tex.sprint(math.pi)}$


\vspace{1.75em}
%
%
%
\begin{figure}[H]											% wrap the MetaPost figure in a LaTeX figure
  \centering												% center the whole thing
    \fcolorbox{black}{white} {								% make the frame black
       \resizebox{\scalefactor \twidth}{!} {				% resize the figure if you want (change \scalefactor)
          \begin{mplibcode}									% build the figure in MetaPost 
             u = 100;										% set diameter
             if u <= 0:										% diameter must be > 0
               message "u <= 0";							% supposedly this gets written on the console and then LaTeX exits
 			 else:											% u > 0
               beginfig(1)									% begin MetaPost figure
                 draw fullcircle scaled u withcolor blue;	% draw a blue circle with diameter u
                 pickup pencircle scaled 1.50;				% set the pen to circular shape (pencircle) with width scaled to 1.50 units
%															%
%	Draw dots and annotations at 0, pi/2, pi and 3pi/2		%
%															%
                 draw (u/2,0); draw(0,u/2); draw(-u/2,0); draw(0,-u/2);
                 label.rt(textext("$\;\theta = 0$") scaled .75, (u/2,0));
                 label.top(textext("$\theta = \displaystyle{\frac{\pi}{2}}$") scaled .75, (0,u/2));
                 label.lft(textext("$\theta = \pi\;$") scaled .75, (-u/2,0));				
                 label.bot(textext("$\theta =\displaystyle{\frac{3\pi}{2}}$") scaled .75, (0,-u/2));
               endfig;										% end MetaPost figure
             fi;											% end if u <= 0: 
          \end{mplibcode}									% end MetaPost code
       } 													% end resizebox
    } 														% end fcolorbox
  \label{figure:annotated_circle}
  \caption{Annotated Circle}
\end{figure}
%
%
%
\vspace{1.50em}
\section*{Acknowledgements}
\label{section:acknowledgements}
%
%	LaTeX source on overleaf.com
%
\section*{\LaTeX \hspace{0.025 mm} Source}
%\url{}
%
%	get a bibliography
%
%	Note:.bib files go in ~/Library/texmf/bibtex/bib with TeXShop (MacTeX).
%	You can also use an absolute path, e.g. \bibliography{/Users/dmm/papers/bib/qc}
%
\bibliographystyle{plain}
\bibliography{qc}
%
%
%
% \section*{Appendix A}
%
%	done
%
\end{document} 

