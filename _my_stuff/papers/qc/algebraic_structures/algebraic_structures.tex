\documentclass[11pt, oneside]{article}
%
%   get various packages
%
\usepackage[margin=1.0in]{geometry}                                     % adjust margins
\geometry{letterpaper}                                                  % ... or a4paper or a5paper or ... 
\usepackage{url}                                                        % need this to use URLs in bibtex
\usepackage{setspace}                                                   % need this for \setstrech{...}
\usepackage{scrextend}                                                  % need this for addmargin
\usepackage[export]{adjustbox}                                          % need this to get frame for includegraphics
%
%   tikz et al
%
\usepackage{tikz}
\usetikzlibrary{calc,patterns,angles,quotes,shapes,math,decorations,
                through,intersections,lindenmayersystems,backgrounds}    
\usepackage{circuitikz}                                                 % draw circuits
\usepackage{pgfplots}
\usepackage{pgfplots}	
%
%	more math stuff
%
\usepackage{amsmath,amsfonts,amssymb,amsthm}
\usepackage{mathtools}
\usepackage{commath}                                                    % get \norm{x}
\usepackage{fixmath}                                                    % get \mathbold
\usepackage{gensymb}                                                    % get \degree
\usepackage{mathrsfs}
\usepackage{hyperref}
\usepackage{subcaption}
\usepackage{authblk}
\usepackage{graphicx}
\usepackage{hyperref}
\usepackage{alltt}
\usepackage{color}
\usepackage{float}
\usepackage{braket}
\usepackage{siunitx}
\usepackage{relsize}
\usepackage{multirow}
\usepackage{esvect}
%
%	watermarks
%
% \usepackage{draftwatermark}
% \SetWatermarkText{Draft}
% \SetWatermarkScale{5}
% \SetWatermarkLightness {0.9} 
% \SetWatermarkColor[rgb]{0.7,0,0}
%
%
\theoremstyle{definition}
\newtheorem{theorem}{Theorem}[section]
\newtheorem{definition}{Definition}[section]
\newtheorem{proposition}{Proposition}[section]
\newtheorem{lemma}{Lemma}[section]
\newtheorem{example}{Example}[section]
\newtheorem{remark}{Remark}[section]
%
%
%	so you can do e.g., \begin{bmatrix}[r] (or [c] or [l])
%
%
\makeatletter
\renewcommand*\env@matrix[1][c]{\hskip -\arraycolsep
  \let\@ifnextchar\new@ifnextchar
  \array{*\c@MaxMatrixCols #1}}
\makeatother
%
%
%
\newcommand{\argmax}{\operatornamewithlimits{argmax}}
\newcommand{\argmin}{\operatornamewithlimits{argmin}}
%
%	handy command
%
\newcommand*{\Scale}[2][4]{\scalebox{#1}{$#2$}}%
%
%	Title, author and date
%
\title{A Note on Algebraic Structures}
\author{David Meyer \\ dmm613@gmail.com}
\date{Last Update: \today \\
	 {\vspace{1.00mm} \small Initial Version: May 8, 2018}}
%
\begin{document}
\maketitle
%
%
%
\section{A Few Algebraic Structures and Their Features}

\setlength{\tabcolsep}{0.60em}
\begin{center}
\begin{table}[H]
\scalebox{0.75}{
\begin{tabular}{l | c | c | c | c | c | c}
{{\large \bf Structure}}        & $\textbf{\large ABO}^{\ref{item:abo}}$            & {\large \bf Identity}
                                & {\large \bf Inverse}                              & {\large \bf $\text{Distributive}^{\ref{item:distributive}}$}
                                & {\large\bf $\text{Commutative}^{\ref{item:commutative}}$}   & {\bf Comments} \\
\hline\hline					% separate column names from rows
Semigroup                       & \checkmark & no         & no         & N/A        & no    & $(S,\circ)$ \\
Monoid                          & \checkmark & \checkmark & no         & N/A        & no    & Semigroup with identity $\in S$ \\
Group                           & \checkmark & \checkmark & \checkmark & N/A        & no    
 											 & Monoid with inverses: $a \in S\backslash\{0\} \Rightarrow a^{-1} \in S$ \\
Abelian Group                   & \checkmark & \checkmark & \checkmark & N/A        & \checkmark $(\circ)$ & Commutative group \\
$\text{Ring}_{+}$               & \checkmark & \checkmark & \checkmark & N/A        & \checkmark $(+)$     & Abelian group under $+$ \\
$\text{Ring}_{*}$               & \checkmark & yes/no     & no         & \checkmark & no                   & Monoid under $*$ \\
Division Ring                   & \checkmark & \checkmark $(+,*)$      &\checkmark $(+,*)$  & \checkmark
                                             & \checkmark $(+)$  & Ring with multiplicative inverses \\
Field                           & \checkmark & \checkmark $(+,*)$      & \checkmark $(+,*)$ & \checkmark   & \checkmark $(+,*)$
                                             & Division ring with commutative multiplication \\
Module		                    & \checkmark & \checkmark $(+,*)$      & \checkmark $(+)$   & \checkmark   & \checkmark $(+)$
                                             & Abelian group under $+$, scalars $\in$ Ring \\
Vector Space                    & \checkmark & \checkmark $(+,*)$      & \checkmark $(+)$   & \checkmark   & \checkmark $(+)$
                                             & Abelian group under $+$, scalars $\in$ Field \\
Algebra over a Ring            & yes/no & \checkmark $(+,*)$      & \checkmark $(+)$   & \checkmark   & \checkmark $(+)$
                                             & Module with bilinear  $\text{product}^{\ref{item:bilinear_map}}$\\
Algebra over a Field            & yes/no & \checkmark $(+,*)$      & \checkmark $(+)$   & \checkmark   & \checkmark $(+)$
                                             & Vector space with bilinear product
\end{tabular}}
\caption{A Few Algebraic Structures and Their Features}
\label{tab:algebraic_structures}
\end{table}
\end{center}

\noindent
\subsection{Definitions}

\begin{enumerate}
\item \textbf{ABO:} Associative Binary Operation 
\label{item:abo}
\begin{itemize}
\item $(x \circ y) \circ z = x \circ  (y \circ z)$  
      for all $x, y, z \in S$
\item $x \circ y \in S$ for all $x, y \in S$
      ($S$ is closed under $\circ$)
\end{itemize}

\item \textbf{Distributive:} Distributive Property 
\label{item:distributive}
\begin{itemize}
\item Left Distributive Property:  $x * (y+z )= (x*y) + (x*z)$ for
                                   all $x, y, z \in S$
\item Right Distributive Property: $(y + z) * x = (y*x) + (z*x)$
                                   for all $x, y, z \in S$
\item $*$ is \emph{distributive}   over $+$ if $*$ is left and 
                                   right distributive
\end{itemize}

\item \textbf{Commutative:} Commutative Property
\label{item:commutative}
\begin{itemize}
\item $x \circ y = y \circ x {\mbox{ for all }}x,y\in S$
\end{itemize}

\item \textbf{Bilinear Map:} 
\label{item:bilinear_map}
A bilinear map is a function combining elements of two vector
spaces to yield an element of a third vector space, and is linear
in each of its arguments \cite{wikipedia:bilinear_map}.  Matrix
multiplication is an example.

\bigskip
\noindent
More specifically, a bilinear map is a function \( B: V \times W
\to Z \) such that for all \( v_1, v_2 \in V \), \( w_1, w_2 \in
W \), and scalars \( \alpha \in \mathbb{F} \):

\begin{equation*}
   B(\alpha v_1 + v_2, w) = \alpha B(v_1, w) + B(v_2, w)
\end{equation*}

and

\vspace{-0.85em}
\begin{equation*}
   B(v, \alpha w_1 + w_2) = \alpha B(v, w_1) + B(v, w_2)
\end{equation*}

\bigskip
\noindent
Note that it may be the case that $V=W=Z$.
\end{enumerate}



\noindent
\section{Notes}
\begin{itemize}
\item Table \ref{tab:algebraic_structures} implies that $\text{F}
\subset \text{R} \subset \text{G} \subset \text{M} \subset \text{SG}$.
\item Whether or not a ring has a multiplicative identity 
      seems to depend on the field of study. 

\smallskip
\noindent
      In general the definition of a ring $R$ doesn't require a
      multiplicative inverse in $R$ ($a^{-1} \notin R$ for all $a
      \in R$) or that multiplication be commutative in
      $R$. Specifically: $R$ is an Abelian group under $+$ but we
      don't require that multiplication be commutative (while
      $a+b = b+a$ for all $a,b \in R$, we don't require that $ab
      = ba$ for all $a,b \in R$). These are perhaps the main ways
      in which a ring differs from a field. In addition, as
      mentioned above in some cases $R$ need not include a
      multiplicative identity $(1 \notin R)$.
\item $\text{VS} \subset \text{Module}$ since the scalars in a
      module come from a ring as opposed to a field like we find
      in vector spaces and $\text{F} \subset \text{R}$
\cite{module_theory_blyth}.
\end{itemize}
%
%
%
% \section*{Acknowledgements}
%
%	LaTeX source on overleaf.com
%
\section*{\LaTeX \hspace{0.10 mm} Source}
\url{https://www.overleaf.com/read/fcfcnyxmgzwv}
%
%	get a bibliography
%
%	Note:.bib files go in ~/Library/texmf/bibtex/bib with TeXShop (MacTeX).
%	You can also use an absolute path, e.g. \bibliography{/Users/dmm/papers/bib/qc}
%
\bibliographystyle{plain}
\bibliography{qc}
%
%	done
%
\end{document} 
