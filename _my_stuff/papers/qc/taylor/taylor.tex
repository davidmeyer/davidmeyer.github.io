\documentclass[11pt, oneside]{article}   	% use "amsart" instead of "article" for AMSLaTeX format


\usepackage{geometry}                		% See geometry.pdf to learn the layout options. There are lots.
\geometry{letterpaper}                   		% ... or a4paper or a5paper or ... 
%\geometry{landscape}                		% Activate for for rotated page geometryhttps://www.washingtonpost.com/world/europe/amid-impeachment-probe-gordon-sondland-is-overseeing-a-renovation-of-his-residence-that-has-cost-1-million-in-taxpayer-money/2019/10/16/d0eece92-ef86-11e9-bb7e-d2026ee0c199_story.html?tid=sm_tw
%\usepackage[parfill]{parskip}    		% Activate to begin paragraphs with an empty line rather than an indent
\usepackage{graphicx}				% Use pdf, png, jpg, or eps� with pdflatex; use eps in DVI mode
								% TeX will automatically convert eps --> pdf in pdflat						\label{thm:integral_domain}

								% TeX will automatically convert eps --> pdf in pdflatex		
\usepackage{amssymb}
\usepackage{amsmath}
\usepackage{amsthm}
\usepackage{mathrsfs}
\usepackage[hyphens,spaces,obeyspaces]{url}
\usepackage{url}
\usepackage{hyperref}
\usepackage{subcaption}
\usepackage{authblk}
\usepackage{mathtools}
\usepackage{graphicx}
\usepackage[export]{adjustbox}
\usepackage{fixltx2e}
\usepackage{hyperref}
\usepackage{alltt}
\usepackage{color}
\usepackage[utf8]{inputenc}
\usepackage[english]{babel}
\usepackage{float}
\usepackage{bigints}
\usepackage{braket}
\usepackage{siunitx}
\usepackage{mathtools}


\usepackage{tikz}
\usepackage{verbatim}
\usetikzlibrary{plotmarks}
\usepackage{pgfplots}
\usepackage{amsmath}
\usepackage{relsize}
\usepackage[hyphenbreaks]{breakurl}
\usepackage{setspace}

 
 \theoremstyle{definition}
 \newtheorem{thm}{Theorem}[section]
% \newtheorem{defn}[thm]{Definition}
\theoremstyle{definition}
\newtheorem{definition}{Definition}[section]
\newtheorem{proposition}{Proposition}[section]
\newtheorem{lemma}{Lemma}[section]
\newtheorem{example}{Example}[section]
\theoremstyle{remark}
\newtheorem{remark}{Remark}

\newcommand{\argmax}{\operatornamewithlimits{argmax}}
\newcommand{\argmin}{\operatornamewithlimits{argmin}}



\title{A Few Notes On The Taylor Series}
\author{David Meyer \\ dmm@1-4-5.net}

\date{Last update: \today}							% Activate to display a given date or no date

\begin{document}
\maketitle

\section{Introduction}

\section{Taylor Series}
\label{sec:taylor_series}

\section{Taylor Inequality}
Suppose that the function $f(x)$ is infinitely differentiable (smooth) at $x=a$. Then as we saw in Section \ref{sec:taylor_series}, 
the Taylor series for $f(x)$, centered at $x=a$, is

\medskip
\begin{equation}
T(x) = \sum\limits _{k=0}^{\infty}{\frac {f^{(k)}(a)}{k!}}(x-a)^{k}
\label{eqn:taylor_series}
\end{equation}


\bigskip
\noindent
We can divide $T(x)$ into a Taylor polynomial of degree $n$,  denoted $T_n(x)$,  and an infinite series $R_n(x)$, called the \emph{Taylor remainder}, such that  

\medskip
\begin{equation}
T(x) = T_n(x) + R_n(x)
\label{eqn:T(x)}
\end{equation}

\bigskip
\noindent
Here  $T_n(x) =  \sum\limits _{k=0}^{n}{\frac {f^{(k)}(a)}{k!}}(x-a)^{k}$ and  $R_n(x) =  \!\! \sum\limits _{k=n+1}^{\infty}{\frac {f^{(k)}(a)}{k!}}(x-a)^{k}$. 

\bigskip
\noindent
Understanding how $R_n(x)$ relates to $T(x)$, that is, $R_n(x) = T(x) - T_n(x)$,  will allow us to understand, among other things, 
how good of an approximation $T(x)$ is to $f(x)$, whether or not $T(x)$ converges,  and to what.

\bigskip
\noindent
BTW, a useful form of Equation \ref{eqn:T(x)} to think of $R_n(x)$ as the difference between $f(x)$ and the Taylor
polynomial of degree $n$ for $f(x)$ centered at $a$. That is



\bigskip
\begin{equation*}
\begin{array}{lllll}
f(x)
&=& \sum\limits _{k=0}^{\infty}{\frac {f^{(k)}(a)}{k!}}(x-a)^{k}                                                                                                                    &\mathrel{\#} \text{Taylor series of $f(x)$}                                    \\  
[12pt]
&\Rightarrow& T_n(x) =  \sum\limits _{k=0}^{n}{\frac {f^{(k)}(a)}{k!}}(x-a)^{k}                                                                                          &\mathrel{\#} \text{definition of the Taylor polynomial $T_n(x)$}  \\  
[12pt]
&\Rightarrow& R_n(x) = f(x) - T_n(x)                                                                                                                                                       &\mathrel{\#} \text{definition of the Taylor remainder  $R_n(x)$}  \\  
[12pt]
&\Rightarrow& R_n(x) =  \sum\limits _{k=0}^{\infty}{\frac {f^{(k)}(a)}{k!}}(x-a)^{k} - \sum\limits _{k=0}^{n}{\frac {f^{(k)}(a)}{k!}}(x-a)^{k} &\mathrel{\#} \text{expand $f(x)$ and $T_n(x)$}                           \\
[12pt]
&\Rightarrow& R_n(x) = \frac{f^{(n+1)}(a)}{(n+1)!} (x-a)^{(n+1)}                                                                                                             &\mathrel{\#} \text{see comment below}
\end{array}
\end{equation*}

\bigskip
\noindent
\doublespace
The last line above, $R_n(x) = \sum\limits _{k=n+1}^{\infty}{\frac {f^{(k)}(a)}{k!}}(x-a)^{k} =\frac{f^{(n+1)}(a)}{(n+1)!} (x-a)^{(n+1)}$, follows 
because for $k > n+1$ $f^{(k)}(a) = 0$ ($f^{(k)}(x)$ for $k > n+1$ has a $(x-a)^k$ term which is zero at $x = a$).
\singlespace

\bigskip
\noindent
So now we can see that the Taylor remainder $R_n(x)$ is a kind of error in our Taylor polynomial $T_n(x)$'s  estimate of $f(x)$: 

\begin{equation}
R_n(x) = f(x) - T_n(x) = \frac{f^{(n+1)}(a)}{(n+1)!} (x-a)^{(n+1)}
\label{eqn:R(x)}
\end{equation}




\bigskip
\noindent
We can use Equations \ref{eqn:T(x)} and \ref{eqn:R(x)} to state the Taylor Remainder Theorem:

\bigskip
\noindent


\begin{thm}
{\bf Taylor Remainder Theorem:}  If $|f^{(n)}| \leq M$ for $|x - a| \leq d$ then the remainder $R_n(x)$ of the
Taylor series $T(x)$ satisfies the inequality

\begin{equation*}
\big | R_n(x) \big |  \leq \frac{M}{(n+1)!} |x -a|^{n+1}
\end{equation*}
\end{thm}
\bigskip
\noindent
for $|x-a| \leq d$, where $M$ and $d$ are constants.

\bigskip
\noindent
{\bf Sketch of Proof: }

\newpage
\section*{Acknowledgements}

%\newpage
\bibliographystyle{plain}
\bibliography{/Users/dmm/papers/bib/qc}
\end{document}

\end{document} 
 
