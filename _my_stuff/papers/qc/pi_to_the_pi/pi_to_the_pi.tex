\documentclass[14pt, oneside]{article}
%
%   get various packages
%
\usepackage{geometry}                                   % See geometry.pdf to learn the layout options
\geometry{margin=0.50in}
\geometry{letterpaper}                                  % ... or a4paper or a5paper or ... 
\usepackage{setspace}
\usepackage{mathtools}                                  % \coloneqq
\usepackage{graphicx}
%
%	math stuff
%
\usepackage{amssymb}
\usepackage{amsmath}
\usepackage{amsthm}
\usepackage{mathrsfs}
\usepackage[hyphens,spaces,obeyspaces]{url}
\usepackage{url}
\usepackage{hyperref}
\usepackage{subcaption}
\usepackage{authblk}
\usepackage{mathtools}
\usepackage{graphicx}
\usepackage[export]{adjustbox}
\usepackage{fixltx2e}
\usepackage{hyperref}
\usepackage{alltt}
\usepackage{color}
\usepackage[utf8]{inputenc}
\usepackage[english]{babel}
\usepackage{float}
\usepackage{bigints}
\usepackage{braket}
\usepackage{siunitx}
\usepackage{mathtools}
\usepackage{xcolor}
\usepackage[hyphenbreaks]{breakurl}
\usepackage{relsize}



\title{{\Huge $\pi^\pi = \; ?$}}
\author{David Meyer \\ \href{mailto:dmm@1-4-5.net}{dmm@1-4-5.net}}
\date{Last update: \today}


\begin{document}
\maketitle

\large
\noindent
Can we find an expression for $\pi^\pi$? The first thing 
we might do is to consider what we know about $x^x$. We 
do know that $a^x = e^{x \ln a}$ for positive $a$, since

\medskip
\begin{equation*}
\begin{array}{llll}
y
&=& a^x											&\qquad \qquad \qquad \qquad \mathrel{\#} \text{define $y$} \\
[10pt]
&\Rightarrow& \ln y = \ln a^x					&\qquad \qquad \qquad \qquad \mathrel{\#} \text{take the log of both sides} \\
[10pt]
&\Rightarrow& \ln y = x \ln a 					&\qquad \qquad \qquad \qquad \mathrel{\#} \text{power rule for logarithms} \\
[10pt]
&\Rightarrow& e^{\ln y} = e^{x \ln a}			&\qquad \qquad \qquad \qquad \mathrel{\#} \text{exponentiate both sides} \\
[10pt]
&\Rightarrow& y = e^{x \ln a}					&\qquad \qquad \qquad \qquad \mathrel{\#} e^{\ln y} = y \\
[10pt]
&\Rightarrow& a^x = e^{x \ln a}					&\qquad \qquad \qquad \qquad \mathrel{\#} y = a^x
\end{array}
\end{equation*}

\smallskip
\bigskip
\noindent
We can use the same reasoning to show that $x^x = e^{x \ln
x}$ for $x > 0$. Then setting $x = \pi$ we get

{\Large
\begin{equation}
{\displaystyle \pi^{\pi} = e^{\pi \ln \pi}}
\label{eqn:pi_to_the_pi}
\end{equation}}

\noindent
All good, but what is $e^{\pi \ln \pi}$? We can use a Maclaurin
series to evaluate this expression as follows:

\medskip
\begin{equation*}
\begin{array}{llll}
e^x
&=& {\displaystyle \sum\limits_{n=0}^{\infty} \dfrac{x^n}{n!}}
	&\qquad \qquad \qquad \mathrel{\#} \text{Maclaurin series for $e^x$} \\
[12pt]
&\Rightarrow& {\displaystyle e^{\pi \ln \pi} = \sum\limits_{n=0}^{\infty} \dfrac{(\pi \ln\pi)^n}{n!}}	
		&\qquad \qquad \qquad \mathrel{\#} \text{set $x = \pi \ln \pi$} \\
[15pt]
&\Rightarrow& {\displaystyle e^{\pi \ln \pi} =\sum\limits_{n=0}^{\infty} \dfrac{\pi^n \ln^n\pi}{n!}} 		
	&\qquad \qquad \qquad \mathrel{\#} \text{simplify} \\
[15pt]
&\Rightarrow& {\displaystyle \pi^\pi = \sum\limits_{n=0}^{\infty} \dfrac{\pi^n \ln^n\pi}{n!}}													
	&\qquad \qquad \qquad \mathrel{\#} \text{$e^{\pi \ln \pi} = \pi^\pi$ (Equation (\ref{eqn:pi_to_the_pi}))}
 
\end{array}
\end{equation*}


\medskip
\bigskip
\noindent
So we get the cool result that

{\Large
\begin{equation*}
{\mathlarger \pi^\pi = \sum\limits_{n=0}^{\infty} \dfrac{\pi^n \ln^n\pi}{n!}}
\end{equation*}}

\bigskip
\noindent
Next question: is $\pi^\pi$ rational or irrational? 

\end{document}

