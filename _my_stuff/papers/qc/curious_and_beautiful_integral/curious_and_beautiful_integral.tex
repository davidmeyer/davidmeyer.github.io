\documentclass{article}
%
%
%	curious_and_beautiful_integral.tex
%
%	David Meyer
%	dmm613@gmail.com
%	08 Oct 2023
%
%
%   get various packages
%
\usepackage[margin=0.65in]{geometry}                                     % adjust margins
\geometry{letterpaper}                                                  % or a4paper or a5paper or ... 
\usepackage{url}                                                        % need this to use URLs in bibtex
\usepackage{setspace}                                                   % need this for \setstrech{...}
\usepackage{scrextend}                                                  % need this for addmargin
\usepackage[export]{adjustbox}                                          % need this to get frame for includegraphics
%
%   tikz et al
%
\usepackage{tikz}
\usetikzlibrary{calc,patterns,angles,quotes,shapes,math,decorations,
                through,intersections,lindenmayersystems,backgrounds,
                hobby}
\tikzset{>=latex}                                                       % default to LaTeX arrow head
\usepackage{circuitikz}                                                 % draw circuits    
\usepackage{pgfplots}
%
%	more math stuff
%
\usepackage{amsmath,amsfonts,amssymb,amsthm}
\usepackage{bm}
\usepackage{mathtools}
\usepackage{commath}                                                    % get \norm{x}
\usepackage{fixmath}                                                    % get \mathbold
\usepackage{gensymb}                                                    % get \degree
\usepackage{mathrsfs}
\usepackage{hyperref}
\usepackage{subcaption}
\usepackage{authblk}													% comment out if using beamer (stops \author{} from working in beamer)
\usepackage{graphicx}
\usepackage{hyperref}
\usepackage{alltt}
\usepackage{xcolor}
\usepackage{colortbl}													% \rowcolor{yellow!75} etc
\usepackage{float}
\usepackage{braket}
\usepackage{siunitx}
\usepackage{relsize}
\usepackage{multirow}
\usepackage{esvect}
\usepackage{enumitem}													% use characters instead of numbers in enumerate
\usepackage{changepage}													% needed for \begin{adjustwidth}{-3.25em}{-2.0em} (left justify)
\usepackage{bigints}
\usepackage{multirow}													% \multirow and \multicolumn
%
%	lualatex
%
%	Put this before \documentclass
%
%	!TEX TS-program = LuaLaTeX
%
%
% \usepackage{luatex85,luamplib}
% \mplibnumbersystem{double}
% \mplibtextextlabel{enable}
%
%
%
%	Describe floating point parameters, \fpeval
%
\ExplSyntaxOn
\cs_set_eq:NN \fpeval \fp_eval:n
\ExplSyntaxOff
%
%	Get the x and y components out of a coordinate, e.g.
%
%	\coordinate (EP) at (8,5);
%	\gettikzxy{(EP)}{\slopex}{\slopey}
%
\makeatletter
\newcommand{\gettikzxy}[3]{%
  \tikz@scan@one@point\pgfutil@firstofone#1\relax
  \edef#2{\the\pgf@x}%
  \edef#3{\the\pgf@y}%
}
\makeatother
%
%
%	Watermarks
%
% \usepackage{draftwatermark}
% \SetWatermarkText{Draft}
% \SetWatermarkScale{5}
% \SetWatermarkLightness {0.9} 
% \SetWatermarkColor[rgb]{0.7,0,0}
%
%
%	theorems, definitions, etc
%
\theoremstyle{definition}
\newtheorem{theorem}{Theorem}[section]
\newtheorem{definition}{Definition}[section]
\newtheorem{proposition}{Proposition}[section]
\newtheorem{lemma}{Lemma}[section]
\newtheorem{example}{Example}[section]
\newtheorem{remark}{Remark}[section]
%
%	For drawing matrix products 
%
\newcommand*{\vertbar}{\rule[-1ex]{0.5pt}{2.5ex}}
\newcommand*{\horzbar}{\rule[.5ex]{2.5ex}{0.5pt}}

%
%	The following code allows you to do
%
%	\begin{bmatrix}[r] (or [c] or [l])
%
\makeatletter
\renewcommand*\env@matrix[1][c]{\hskip -\arraycolsep
  \let\@ifnextchar\new@ifnextchar
  \array{*\c@MaxMatrixCols #1}}
\makeatother
%
%	make \arg{min,max}_{n \to \infty} work nicely
%
\newcommand{\argmax}{\operatornamewithlimits{argmax}}
\newcommand{\argmin}{\operatornamewithlimits{argmin}}
%
%	handy commands
%
\newcommand*{\Scale}[2][4]{\scalebox{#1}{$#2$}}
\DeclareMathOperator{\E}{\mathbb{E}}
\DeclareMathOperator{\bda}{\Big \downarrow}						% big down arrow
\newcommand{\veq}{\mathrel{\rotatebox{90}{$=$}}}
%
%	Title, author and date
%
\title{A Curious Integral: ${\displaystyle \int\limits_{0}^{1} x \cdot 
	   		\sqrt{x \cdot \sqrt[3]{x \cdot \sqrt[4]{x \cdots}}}} \; dx = 
			\dfrac{1}{e}$}
\author{David Meyer \\ \href{mailto:dmm613@gmail.com}
                            {dmm613@gmail.com}}
\date{Last update: \today}
%
%
%
\begin{document}
\maketitle
%
%
%
\section{Ok, but why?}
\label{sec:introduction}

\bigskip
{\Large Consider:}

\begin{center}
      \begin{math}								% turn on math mode
      \fontsize{11.40}{8}
	   \begin{array}{llll}						% make it an array
	       {\displaystyle \int\limits_{0}^{1} x \cdot 
	   		\sqrt{x \cdot \sqrt[3]{x \cdot \sqrt[4]{x \cdots}}}} \; dx
	   &=& {\displaystyle \int\limits_{0}^{1} x^{\frac{1}{1}} \cdot x^{\frac{1}{1 \cdot 2}} \cdot 
	   		x^{\frac{1}{1 \cdot 2 \cdot 3}} \cdot x^{\frac{1}{1 \cdot 2 \cdot 3 \cdot 4}}}
			\cdots \; dx
			&\hspace{0.25em} \mathrel{\#} \text{radical $\rightarrow$ exponent notation} \\
[18pt]
		&=& {\displaystyle \int\limits_{0}^{1} x^{\big [ \frac{1}{1} + \frac{1}{1 \cdot 2} + 
	   		\frac{1}{1 \cdot 2 \cdot 3} + \frac{1}{1 \cdot 2 \cdot 3 \cdot 4} + 
			\cdots \big ]}} \; dx
			&\hspace{0.25em} \mathrel{\#} x^a \cdot x^b = x^{a+b} \\
[18pt]
		&=& {\displaystyle \int\limits_{0}^{1} x^{\left [\sum\limits_{n=1}^{\infty} \frac{1}{n!} \right ]} \; dx}
			&\hspace{0.25em} \mathrel{\#} \frac{1}{1} +\frac{1}{1\cdot 2} + \frac{1}{1\cdot 2 \cdot 3} + 
			\frac{1}{1 \cdot 2\cdot3\cdot 4} + \cdots = 
			\sum\limits_{n=1}^{\infty} \frac{1}{n!} \\
[18pt]
		&=& {\displaystyle \int\limits_{0}^{1} x^{\left [ \left ( \sum\limits_{n=0}^{\infty} \frac{1}{n!} \right ) 
											\, - \, 1 \right ]} \; dx}
			&\hspace{0.25em} \mathrel{\#} \sum\limits_{n=1}^{\infty} \frac{1}{n!} = 
											\left (\sum\limits _{n=0}^{\infty} \frac{1}{n!} \right ) - \dfrac{1}{0!} =
											\left (\sum\limits _{n=0}^{\infty} \frac{1}{n!} \right ) - 1\\
[18pt]
		&=& {\displaystyle \int\limits_{0}^{1} x^{e-1}} \; dx
			&\hspace{0.25em} \mathrel{\#}  \sum\limits _{n=0}^{\infty} \frac{1}{n!} = e  
											\Rightarrow \left (\sum\limits _{n=0}^{\infty} \frac{1}{n!} \right ) - 1 
											= e - 1 \; \cite{wikipedia:representations_of_e} \\
[18pt]
		&=& \dfrac{x^{(e-1)+1}}{(e-1)+1} \bigg |_{0}^{1} 
			&\hspace{0em} \mathrel{\#} \text{by the power rule \cite{wiki:power_rule}
												and the FToC \cite{ftoc}} \\
[18pt]
		&=& \dfrac{x^{e}}{e} \bigg |_{0}^{1}
			&\hspace{0.25em} \mathrel{\#} (e - 1) + 1 = e \\
[18pt]
		&=& \dfrac{1^{e}}{e} - \dfrac{0^{e}}{e}
			&\hspace{0.25em} \mathrel{\#} \text{evaluate at the endpoints} \\
[18pt]
		&=& \dfrac{1}{e} - \dfrac{0}{e}
			&\hspace{0.25em} \mathrel{\#} 1^e = 1 \text{ and } 0^e = 0 \\
[12pt]
		&=& \dfrac{1}{e}
			&\hspace{0.25em} \mathrel{\#} {\displaystyle \int\limits_{0}^{1} x \cdot 
	   		\sqrt{x \cdot \sqrt[3]{x \cdot \sqrt[4]{x \cdots}}}}
			\; dx = \dfrac{1}{e} 

	   \end{array}
	  \end{math}
\end{center}


\section{Conclusions}
\label{sec:conclusions}
%
%
%
\section*{Acknowledgements}
\label{sec:acknowledgements}
%
%	LaTeX source on overleaf.com
%
\section*{\LaTeX \hspace{0.025 mm} Source}
\url{https://www.overleaf.com/read/ydvsjrpsjphm#b24ca3}
%
%	get a bibliography
%
%	Note:.bib files go in ~/Library/texmf/bibtex/bib with TeXShop (MacTeX).
%	You can also use an absolute path, e.g. \bibliography{/Users/dmm/papers/bib/qc}
%
\bibliographystyle{plain}
\bibliography{qc}
%
%
%
% \section*{Appendix A}
%
%	done
%
\end{document} 

