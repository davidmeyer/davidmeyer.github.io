\documentclass[11pt, oneside]{article}
%
%   get various packages
%
\usepackage[margin=1.0in]{geometry}                                     % adjust margins
\geometry{letterpaper}                                                  % ... or a4paper or a5paper or ... 
\usepackage{url}                                                        % need this to use URLs in bibtex
\usepackage{setspace}                                                   % need this for \setstrech{...}
\usepackage{scrextend}                                                  % need this for addmargin
\usepackage[export]{adjustbox}                                          % need this to get frame for includegraphics
\usepackage{bigints}                                                    % bigger integral symbol
%
%   tikz et al
%
\usepackage{tikz}
\usetikzlibrary{calc,patterns,angles,quotes,shapes,math,decorations,
                through,intersections,lindenmayersystems,backgrounds}    
\usepackage{pgfplots}
\usepackage{pgfplots}	
%
%	math stuff
%
\usepackage{amsmath,amsfonts,amssymb,amsthm}
\usepackage{mathtools}
\usepackage{commath}                                                    % get \norm{x}
\usepackage{fixmath}                                                    % get \mathbold
\usepackage{gensymb}                                                    % get \degree
\usepackage{circuitikz}                                                 % draw circuits
\usepackage{mathrsfs}

\usepackage{hyperref}
\usepackage{url}
\usepackage{subcaption}
\usepackage{authblk}
\usepackage{amsmath}
\usepackage{mathtools}
\usepackage{graphicx}
\usepackage[export]{adjustbox}
\usepackage{hyperref}
\usepackage{alltt}
\usepackage{color}
\usepackage[utf8]{inputenc}
\usepackage[english]{babel}
\usepackage{float}
\usepackage{bigints}
\usepackage{braket}
\usepackage{siunitx}
\usepackage{relsize}
\usepackage{multirow}
\usepackage{esvect}



%
%	watermarks
%
% \usepackage{draftwatermark}
% \SetWatermarkText{Draft}
% \SetWatermarkScale{5}
% \SetWatermarkLightness {0.9} 
% \SetWatermarkColor[rgb]{0.7,0,0}
%
%
\theoremstyle{definition}
\newtheorem{theorem}{Theorem}[section]
\newtheorem{definition}{Definition}[section]
\newtheorem{proposition}{Proposition}[section]
\newtheorem{lemma}{Lemma}[section]
\newtheorem{example}{Example}[section]
\newtheorem{remark}{Remark}[section]
%
%
%	so you can do e.g., \begin{bmatrix}[r] (or [c] or [l])
%
%
\makeatletter
\renewcommand*\env@matrix[1][c]{\hskip -\arraycolsep
  \let\@ifnextchar\new@ifnextchar
  \array{*\c@MaxMatrixCols #1}}
\makeatother
%
%
%
\newcommand{\argmax}{\operatornamewithlimits{argmax}}
\newcommand{\argmin}{\operatornamewithlimits{argmin}}
%
%	handy command
%
\newcommand*{\Scale}[2][4]{\scalebox{#1}{$#2$}}%
%
%
%
\title{What is hiding inside the number 2?}
\author{David Meyer \\ dmm613@gmail.com}
\date{Last update: \today}


\begin{document}
\maketitle

\noindent
Consider

\begin{equation*}
\begin{array}{llll}
2 
&=& \sqrt{4}											&\qquad \qquad \mathrel{\#} 4 = 2^2 \\
[10pt]
&=& \sqrt{2+2}											&\qquad \qquad \mathrel{\#} 4 = 2+2 \\
[10pt]
&=& \sqrt{2+ \sqrt{4}}									&\qquad \qquad \mathrel{\#} 4 = 2^2 \\
[10pt]
&=& \sqrt{2+ \sqrt{2+2}}								&\qquad \qquad \mathrel{\#} 4 = 2+2 \\
[10pt]
&=& \sqrt{2+ \sqrt{2+\sqrt{4}}}							&\qquad \qquad \mathrel{\#} 4 = 2^2 \\
[10pt]
&=& \sqrt{2+ \sqrt{2+\sqrt{2+2}}}						&\qquad \qquad \mathrel{\#} 4 = 2+2 \\
[5pt]
&=& \sqrt{2+ \sqrt{2+\sqrt{2+\sqrt{4}}}}				&\qquad \qquad \mathrel{\#} 4 = 2^2 \\
[5pt]
&=& \sqrt{2+ \sqrt{2+\sqrt{2+\sqrt{2+2}}}}				&\qquad \qquad \mathrel{\#} 4 = 2+2 \\
[5pt]
&=& \sqrt{2+ \sqrt{2+\sqrt{2+\sqrt{2+\sqrt{4}}}}}		&\qquad \qquad \mathrel{\#} 4 = 2^2 \\
[10pt]
&=& \cdots
\end{array}
\end{equation*}

\noindent
and so apparently 

\begin{equation*}
2 =  \sqrt{2+ \sqrt{2+\sqrt{2+\sqrt{2+\sqrt{2 + \cdots}}}}}
\end{equation*}
%
%
%
\section*{Acknowledgements}
Thanks to Bruce Mah who pointed out that since $4 = 2 + 2 = 2 \cdot2$
we can also write

\begin{equation}
2 = \sqrt{2 \cdot \sqrt{2 \cdot \sqrt{2 \cdot \sqrt{2 \cdot \sqrt{2 \cdot \cdots}}}}}
\label{eqn:mah}
\end{equation}

\noindent
Amazing.

\bigskip
\noindent
Dave Neary also pointed out that if you take the log of both sides 
of Equation (\ref{eqn:mah}) you get

\bigskip
\begin{equation}
1 = \dfrac{1}{2} + \dfrac{1}{4} + \dfrac{1}{8} + \cdots
\label{eqn:neary}
\end{equation}

\bigskip
\noindent
Here we can notice that the right-hand side (RHS) 
of Equation (\ref{eqn:neary}) is a geometric series 
\cite{wiki:geometric_series} with $a = \frac{1}{2}$ 
and $r = \frac{1}{2}$. Since this geometric series 
converges to $\frac{a}{1-r}$, we see that

\begin{equation*}
\begin{array}{llll}
1 
&=& \dfrac{1}{2} + \dfrac{1}{4} + \dfrac{1}{8} + \cdots
		&\qquad \mathrel{\#} \text{take the log of both sides of Equation (\ref{eqn:mah})} \\
[10pt]
&=& \dfrac{\frac{1}{2}}{1-\frac{1}{2}}
		&\qquad \mathrel{\#} \text{the RHS is a geometric series that 
		                           converges to $\frac{a}{1-r}$ with $a=r=\frac{1}{2}$} \\
[12pt]
&=& \dfrac{\frac{1}{2}}{\frac{1}{2}}
		&\qquad \mathrel{\#} \text{simplify} \\
[12pt]
&=& 1
		&\qquad \mathrel{\#} \text{amazing} \\
\end{array}
\end{equation*}



%
%	LaTeX source on overleaf.com
%
% \section*{\LaTeX \hspace{0.10 mm} Source}
% \url{}
%
%	get a bibliography
%
%	Note:.bib files go in ~/Library/texmf/bibtex/bib with TeXShop (MacTeX).
%	You can also use an absolute path, e.g. \bibliography{/Users/dmm/papers/bib/qc}
%
\bibliographystyle{plain}
\bibliography{qc}
%
%	done
%
\end{document} 

