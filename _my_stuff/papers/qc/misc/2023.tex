\documentclass{article}
%
%
%	LaTeX template
%
%	David Meyer
%	dmm613@gmail.com
%	15 July 2022
%
%
%   get various packages
%
\usepackage[margin=1.0in]{geometry}                                     % adjust margins
\geometry{letterpaper}                                                  % or a4paper or a5paper or ... 
\usepackage{url}                                                        % need this to use URLs in bibtex
\usepackage{setspace}                                                   % need this for \setstrech{...}
\usepackage{scrextend}                                                  % need this for addmargin
\usepackage[export]{adjustbox}                                          % need this to get frame for includegraphics
%
%   tikz et al
%
\usepackage{tikz}
\usetikzlibrary{calc,patterns,angles,quotes,shapes,math,decorations,
                through,intersections,lindenmayersystems,backgrounds,
                hobby}
\tikzset{>=latex}                                                       % default to LaTeX arrow head
\usepackage{circuitikz}                                                 % draw circuits    
\usepackage{pgfplots}
%
%	more math stuff
%
\usepackage{amsmath,amsfonts,amssymb,amsthm}
\usepackage{mathtools}
\usepackage{commath}                                                    % get \norm{x}
\usepackage{fixmath}                                                    % get \mathbold
\usepackage{gensymb}                                                    % get \degree
\usepackage{mathrsfs}
\usepackage{hyperref}
\usepackage{subcaption}
\usepackage{authblk}
\usepackage{graphicx}
\usepackage{hyperref}
\usepackage{alltt}
\usepackage{color}
\usepackage{float}
\usepackage{braket}
\usepackage{siunitx}
\usepackage{relsize}
\usepackage{multirow}
\usepackage{esvect}
%
%	watermarks
%
% \usepackage{draftwatermark}
% \SetWatermarkText{Draft}
% \SetWatermarkScale{5}
% \SetWatermarkLightness {0.9} 
% \SetWatermarkColor[rgb]{0.7,0,0}
%
%
%	theorems, definitions, etc
%
\theoremstyle{definition}
\newtheorem{theorem}{Theorem}[section]
\newtheorem{definition}{Definition}[section]
\newtheorem{proposition}{Proposition}[section]
\newtheorem{lemma}{Lemma}[section]
\newtheorem{example}{Example}[section]
\newtheorem{remark}{Remark}[section]
%
%	The following code allows you to do
%
%	\begin{bmatrix}[r] (or [c] or [l])
%
\makeatletter
\renewcommand*\env@matrix[1][c]{\hskip -\arraycolsep
  \let\@ifnextchar\new@ifnextchar
  \array{*\c@MaxMatrixCols #1}}
\makeatother
%
%	make \arg{min,max}_{n \to \infty} work nicely
%
\newcommand{\argmax}{\operatornamewithlimits{argmax}}
\newcommand{\argmin}{\operatornamewithlimits{argmin}}
%
%	handy commands
%
\newcommand*{\Scale}[2][4]{\scalebox{#1}{$#2$}}
\DeclareMathOperator{\E}{\mathbb{E}}
\DeclareMathOperator{\bda}{\Big \downarrow}						% big down arrow
\newcommand{\veq}{\mathrel{\rotatebox{90}{$=$}}}
%
%	Title, author and date
%
\title{template.tex}
\author{David Meyer \\ \href{mailto:dmm613@gmail.com}
                            {dmm613@gmail.com}}
\date{Last update: \today}
%
%
%
\begin{document}
\maketitle
%
%
%
\section{Introduction}
\label{sec:introduction}
Did you know that the first 10 natural numbers can be combined 
using arithmetic operators to make 2023? 


\bigskip
{\large
\begin{center}
\begin{equation*}
\Bigg [ \left (\dfrac{10 \times 9 \times 8 \times 7 \times 6}{5} \right ) \div 3 \Bigg ] + 4 + 2 + 1 = 2023
\end{equation*}
\end{center}}

\bigskip
%
%
%	show a derivation/proof
%
% \begin{equation*}
% \begin{array}{llll}
% x
% &=& y				&\qquad \mathrel{\#} \text{comment or} \\
% &=& y				&\hspace{2em} \mathrel{\#} \text{comment} \\
% [10pt]
% \end{array}
% \end{equation*}
%
%
%
%	include an image
%
% \begin{figure}
% \center{\includegraphics[frame, scale=0.50] {images/X.png}}
% \caption{X}
% \label{fig:X}
% \end{figure}
%
%	How to put a box around a figure
%
% \begin{figure}[H]
%  \centering
%   \fbox{\begin{minipage}{34em}
%    \begin{equation*}
%     \begin{array}{rlrlrlr}
%       9^2      &=& 81           &\longrightarrow& 8 + 1           &=& 9	  \\
%       45^2     &=& 2025         &\longrightarrow& 20 + 25         &=& 45     \\
%       703^2    &=& 494209       &\longrightarrow& 494 + 209       &=& 703    \\
%       7777^2   &=& 60481729     &\longrightarrow& 6048 + 1729     &=& 7777   \\
%       857143^2 &=& 734694122449 &\longrightarrow& 734694 + 122449 &=& 857143 \\
%     \end{array}
%    \end{equation*}
%  \end{minipage}}
%  \caption{A Few Example Kaprekar Numbers}
%\end{figure}
%
%	
%	How to wrap (and resize) your tikzpicture in a figure
%
% \begin{figure}[H]
% \centering
%  \resizebox{0.40 \textwidth}{!} {		% resize the figure if you want
%    \begin{tikzpicture} 
%     .....
%    \end{tikzpicture}					% end tikzpicture
%  }									% end resizebox
% \caption{Some Figure}
% \label{fig:some_figure}
% \end{figure}
%
%
%
\section{Conclusions}
\label{sec:conclusions}
%
%
%
\section{Acknowledgements}
\label{sec:acknowledgements}
%
%	LaTeX source on overleaf.com
%
\section*{\LaTeX \hspace{0.10 mm} Source}
\url{https://www.overleaf.com/read/jyppzhgdtrvg}
%
%	get a bibliography
%
%	Note:.bib files go in ~/Library/texmf/bibtex/bib with TeXShop (MacTeX).
%	You can also use an absolute path, e.g. \bibliography{/Users/dmm/papers/bib/qc}
%
\bibliographystyle{plain}
\bibliography{qc,ml}
%
%	done
%
\end{document} 

