
\documentclass{article}

% \usepackage{draftwatermark}
% \SetWatermarkText{Draft}
% \SetWatermarkScale{5}
% \SetWatermarkLightness {0.9} 
% \SetWatermarkColor[rgb]{0.7,0,0}


\usepackage{geometry}   
\geometry{letterpaper}
\usepackage{graphicx}	
\usepackage{amssymb}
\usepackage{amsmath}
\usepackage{amsthm}
\usepackage{mathrsfs}
\usepackage[hyphens,spaces,obeyspaces]{url}
\usepackage{url}
\usepackage{hyperref}
\usepackage{subcaption}
\usepackage{authblk}
\usepackage{mathtools}
\usepackage{graphicx}
\usepackage[export]{adjustbox}
\usepackage{fixltx2e}
\usepackage{hyperref}
\usepackage{alltt}
\usepackage{color}
\usepackage[utf8]{inputenc}
\usepackage[english]{babel}
\usepackage{float}
\usepackage{bigints}
\usepackage{braket}
\usepackage{siunitx}
\usepackage{mathtools}


\usepackage{tikz}
\usepackage{verbatim}
\usetikzlibrary{plotmarks}
\usepackage{pgfplots}
\usepackage{amsmath}
\usepackage{relsize}
 \usepackage[hyphenbreaks]{breakurl}
 
 \newtheorem{thm}{Theorem}[section]
% \newtheorem{defn}[thm]{Definition}
\theoremstyle{definition}
\newtheorem{definition}{Definition}[section]
\newtheorem{proposition}{Proposition}[section]
\newtheorem{lemma}{Lemma}[section]
\newtheorem{example}{Example}[section]

\newcommand{\argmax}{\operatornamewithlimits{argmax}}
\newcommand{\argmin}{\operatornamewithlimits{argmin}}

\title{What is hiding inside the number 3?}
\author{David Meyer \\ dmm613@gmail.com}
\date{Last update: \today}

\begin{document}
\maketitle

\noindent
So what is hiding inside the number 3? Well, consider:

\begin{center}
\begin{equation*}
\begin{array}{llll}
3 
&=&  \sqrt{9}
				&\qquad \qquad \qquad \mathrel{\#} 3 = \sqrt{9} \\
[6pt]
&=&  \sqrt{1 + 8}
				&\qquad \qquad \qquad \mathrel{\#} 9 = 1 + 8 \\
[6pt]
&=&  \sqrt{1 + 2*4}
				&\qquad \qquad \qquad \mathrel{\#} 8 = 2 * 4 \\
[6pt]
&=&  \sqrt{1 + 2 \sqrt{16}}
				&\qquad \qquad \qquad \mathrel{\#} 4 = \sqrt{16}\\
[6pt]
&=&  \sqrt{1 + 2 \sqrt{1 + 15}}
				&\qquad \qquad \qquad \mathrel{\#} 16 = 1 + 15 \\
[6pt]
&=&  \sqrt{1 + 2 \sqrt{1 + 3*5}}
				&\qquad \qquad \qquad \mathrel{\#} 15 = 3 * 5 \\
[6pt]
&=&  \sqrt{1 + 2 \sqrt{1 + 3 \sqrt{25}}}
				&\qquad \qquad \qquad \mathrel{\#} 5 = \sqrt{25} \\
[6pt]
&=&  \sqrt{1 + 2 \sqrt{1 + 3 \sqrt{1 + 24}}}
				&\qquad \qquad \qquad \mathrel{\#} 25 = 1 + 24 \\
[6pt]
&=&  \sqrt{1 + 2 \sqrt{1 + 3 \sqrt{1 + 4*6}}}
				&\qquad \qquad \qquad \mathrel{\#} 24 = 4 * 6 \\
[6pt]
&=&  \sqrt{1 + 2 \sqrt{1 + 3 \sqrt{1 + 4 \sqrt{36}}}}
				&\qquad \qquad \qquad \mathrel{\#} 6 = \sqrt{36}  \\
[6pt]
&=&  \sqrt{1 + 2 \sqrt{1 + 3 \sqrt{1 + 4 \sqrt{1 + 35}}}}
				&\qquad \qquad \qquad \mathrel{\#} 36 = 1 + 35 \\
[6pt]
&=&  \sqrt{1 + 2 \sqrt{1 + 3 \sqrt{1 + 4 \sqrt{1 + 5*7}}}}
				&\qquad \qquad \qquad \mathrel{\#} 35 = 5 * 7 \\
[6pt]
&=&  \sqrt{1 + 2 \sqrt{1 + 3 \sqrt{1 + 4 \sqrt{1 + 5 \sqrt{49}}}}}
				&\qquad \qquad \qquad \mathrel{\#} 7 = \sqrt{49} \\
[6pt]
&=& \cdots
				&\qquad \qquad \qquad \mathrel{\#} 49 = 1 + 48,\; 48 = 6*8, \; 8 = \sqrt{64}, \; \hdots 
\end{array}
\end{equation*}
\end{center}

\bigskip
\noindent
and so apparently $3 = \sqrt{1 + 2 \sqrt{1 + 3 \sqrt{1 + 4  \sqrt{1 + 5 \sqrt{1 + 6 \sqrt{64}}}}}} \cdots$

\end{document}
