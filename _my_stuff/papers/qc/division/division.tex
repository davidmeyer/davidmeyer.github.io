\documentclass{article}
%
%
%	division.tex
%
%	David Meyer
%	dmm613@gmail.com
%	15 July 2022
%
%
%   get various packages
%
\usepackage[margin=1.0in]{geometry}                                     % adjust margins
\geometry{letterpaper}                                                  % or a4paper or a5paper or ... 
\usepackage{url}                                                        % need this to use URLs in bibtex
\usepackage{setspace}                                                   % need this for \setstrech{...}
\usepackage{scrextend}                                                  % need this for addmargin
\usepackage[export]{adjustbox}                                          % need this to get frame for includegraphics
%
%   tikz et al
%
\usepackage{tikz}
\usetikzlibrary{calc,patterns,angles,quotes,shapes,math,decorations,
                through,intersections,lindenmayersystems,backgrounds,
                hobby}
\tikzset{>=latex}                                                       % default to LaTeX arrow head
\usepackage{circuitikz}                                                 % draw circuits    
\usepackage{pgfplots}
%
%	more math stuff
%
\usepackage{amsmath,amsfonts,amssymb,amsthm}
\usepackage{bm}
\usepackage{mathtools}
\usepackage{commath}                                                    % get \norm{x}
\usepackage{fixmath}                                                    % get \mathbold
\usepackage{gensymb}                                                    % get \degree
\usepackage{mathrsfs}
\usepackage{hyperref}
\usepackage{subcaption}
\usepackage{authblk}													% comment out if using beamer (stops \author{} from working in beamer)
\usepackage{graphicx}
\usepackage{hyperref}
\usepackage{alltt}
\usepackage{xcolor}
\usepackage{float}
\usepackage{braket}
\usepackage{siunitx}
\usepackage{relsize}
\usepackage{multirow}
\usepackage{esvect}
\usepackage{enumitem}													% use characters instead of numbers in enumerate
\usepackage{changepage}													% needed for \begin{adjustwidth}{-3.25em}{-2.0em} (left justify)
%
%	Describe floating point parameters, \fpeval
%
\ExplSyntaxOn
\cs_set_eq:NN \fpeval \fp_eval:n
\ExplSyntaxOff
%
%	Get the x and y components out of a coordinate, e.g.
%
%	\coordinate (EP) at (8,5);
%	\gettikzxy{(EP)}{\slopex}{\slopey}
%
\makeatletter
\newcommand{\gettikzxy}[3]{%
  \tikz@scan@one@point\pgfutil@firstofone#1\relax
  \edef#2{\the\pgf@x}%
  \edef#3{\the\pgf@y}%
}
\makeatother
%
%
%	Watermarks
%
% \usepackage{draftwatermark}
% \SetWatermarkText{Draft}
% \SetWatermarkScale{5}
% \SetWatermarkLightness {0.9} 
% \SetWatermarkColor[rgb]{0.7,0,0}
%
%
%	s, definitions, etc
%
\theoremstyle{definition}
\newtheorem{theorem}{Theorem}[section]
\newtheorem{definition}{Definition}[section]
\newtheorem{proposition}{Proposition}[section]
\newtheorem{lemma}{Lemma}[section]
\newtheorem{example}{Example}[section]
\newtheorem{remark}{Remark}[section]

%	For drawing matrix products 
%
\newcommand*{\vertbar}{\rule[-1ex]{0.5pt}{2.5ex}}
\newcommand*{\horzbar}{\rule[.5ex]{2.5ex}{0.5pt}}

%
%	The following code allows you to do
%
%	\begin{bmatrix}[r] (or [c] or [l])
%
\makeatletter
\renewcommand*\env@matrix[1][c]{\hskip -\arraycolsep
  \let\@ifnextchar\new@ifnextchar
  \array{*\c@MaxMatrixCols #1}}
\makeatother
%
%	make \arg{min,max}_{n \to \infty} work nicely
%
\newcommand{\argmax}{\operatornamewithlimits{argmax}}
\newcommand{\argmin}{\operatornamewithlimits{argmin}}
%
%	handy commands
%
\newcommand*{\Scale}[2][4]{\scalebox{#1}{$#2$}}
\DeclareMathOperator{\E}{\mathbb{E}}
\DeclareMathOperator{\bda}{\Big \downarrow}						% big down arrow
\newcommand{\veq}{\mathrel{\rotatebox{90}{$=$}}}
%
%	Title, author and date
%
\title{A Few Notes on the Division Algorithm}
\author{David Meyer \\ \href{mailto:dmm613@gmail.com}
                            {dmm613@gmail.com}}
\date{Last update: \today}
%
%
%
\begin{document}
\maketitle
%
%
%
\section{Introduction}
\label{sec:introduction}

TBD

\section{Fields and the Division Algorithm}

\begin{definition}
Let $f$ and $g$ be polynomials in $F[x$]. Then we say that $f$
divides $g$, or $g$ is divisible by $f$ if there is a polynomial
$h$ with $g = fh$.
\end{definition}

\noindent
For integers, the greatest common divisor of two integers $a$ and
$b$ is the largest integer dividing both $a$ and $b$. This
definition doesn't quite work for polynomials. In particular,
while we cannot talk about "largest" polynomial in the same
manner as we do for integers, we can talk about the degree of a
polynomial.

\bigskip
\noindent
Recall that the degree of a nonzero polynomial $f$ is the largest
integer $m$ for which the coefficient $a_m$ of $x^m$ is
nonzero. For example, if $f(x) = a_nx^n + \hdots + a_1x+a_0$ with
$a_n \neq 0$, then the degree of $f(x)$, written $\deg f$, is
$n$. The degree function allows us to measure size of
polynomials.

\bigskip
\noindent
There is one extra complication with this definition of $\deg
f$. Consider for example that any polynomial of the form $ax^2$
with $a \neq 0$ divides $x^2$ and $x^3$. Thus, there isn't a
unique polynomial of highest degree that divides a pair of
polynomials. To get around this uniqueness problem the convention
is to consider the \emph{monic} polynomials, that is, those
polynomials whose leading coefficient is $1$. For example, $x^2$
is the monic polynomial of degree $2$ that divides both $x^2$ and
$x^3$, while $5x^2$ is not monic.

\bigskip
\noindent
Finally, I'll use $ab$ to represent $a \cdot b$ where the meaning
is unambiguous.


\begin{definition}
Let $f$ and $g$ be polynomials over $F$, where $f$ and $g$ are
not both zero. Then a greatest common divisor of $f$ and $g$,
$\gcd(f,g)$, is a monic polynomial of largest degree that divides
both $f$ and $g$.
\label{def:gcd}
\end{definition}


\noindent
Unfortunately, there is a problem with Definition \ref{def:gcd}
which has to do with uniqueness. In particular, could there be
more than one greatest common divisor of a pair of polynomials?
As we will se below, the answer turns out to be no.

\bigskip
\noindent
Note that the main reason for assuming that the coefficients of
our polynomials lie in a field is to ensure that the division
algorithm is valid. Why?  Because division is well behaved in
integral domains (where there are no zero divisors), and because
every field is an integral domain and every finite integral
domain is a field. But why is that?


\begin{theorem}
Every field $F$ is an integral domain.
\label{theorem:integral_domain}
\end{theorem}

\noindent
\textbf{Proof.} Recall that if $F$ is a field then each non-zero
$r \in F$ has an inverse $r^{-1}$.  So suppose $r,s \in F$ and $r
\neq 0$ such that $r \cdot s = 0$. Then the claim is that $s =
0$.  Ok, but why? Consider


\begin{equation*}
\begin{array}{rcll} 
r \cdot s
&=& 0                                                   
			&\quad \mathrel{\#} \text{assumption with $r \neq 0$} \\
[5pt]
&\Rightarrow& r^{-1} \cdot (r \cdot s) = r^{-1} \cdot 0 
			&\quad \mathrel{\#} \text{multiply both sides by $r^{-1}$} \\
[5pt]
&\Rightarrow&  r^{-1} \cdot (r \cdot s) =  0            
			&\quad \mathrel{\#} x \cdot 0 = 0 \\
[5pt]
&\Rightarrow&  (r^{-1} \cdot r) \cdot s =  0            
			&\quad \mathrel{\#} \text{multiplication is associative} \\
[5pt]
&\Rightarrow&  1 \cdot s = 0                            
			&\quad \mathrel{\#} r^{-1} \cdot r = 1 \\
[5pt]
&\Rightarrow&  s = 0                                    
			&\quad \mathrel{\#} 1 \cdot x = x
\end{array}
\end{equation*}

\bigskip
\noindent
So $r$ is not a zero divisor. But every non-zero element $r$ of
the field $F$ has an inverse ($r$ is a "unit") so $F$ has no zero
divisors and is by definition an integral domain. $\blacksquare$

\bigskip
\begin{theorem}
Every finite integral domain $R$ is a field.
\label{theorem:finite_id_is_a_field}
\end{theorem}

\noindent
\textbf{Proof.}  The proof is based on the fact that since $R$ is
an integral domain it has a cancellation law (or equivalently,
$R$ has no zero divisors).  Having a cancellation law means that

\smallskip
\begin{equation}
ab = ac \Rightarrow b = c
\label{eqn:cancellation_law}
\end{equation}

\medskip
\noindent
Ok, but why is $R$ having a cancellation law equivalent to $R$ 
being an integral domain? Consider Lemma \ref{lemma:zero_divisors}:

\bigskip
\begin{lemma}
$R$ has a cancellation law iff $R$ is an integral domain.
\label{lemma:zero_divisors}
\end{lemma}


\noindent
\textbf{Proof.} Consider the $\Rightarrow$ and $\Leftarrow$ cases:

\begin{itemize}
\item [($\Rightarrow$)] Suppose $R$ has a cancellation law 
with $a,b \in R$, $a \neq 0$ and assume that $ab = 0$. Then

\vspace{-0.25cm}															% why is there too much vertical space here?
\begin{equation*}
\begin{array}{rcll} 
ab
&=& 0 								
			&\hspace{2em} \mathrel{\#} \text{by assumption} \\
[5pt]
&\Rightarrow& a \cdot b = a \cdot 0 
			&\hspace{2em} \mathrel{\#} 0 = a \cdot 0 \\
[5pt]
&\Rightarrow& b = 0 				
			&\hspace{2em} \mathrel{\#} \text{(cancel 
			$a \Rightarrow a$ is not a zero divisor) 
			$\Rightarrow R$ is an integral domain} \\
\end{array}
\end{equation*}

\medskip
\noindent
Thus $R$ has a cancellation law $\Rightarrow$ $R$ is an integral 
domain.

\medskip
\noindent
\item [($\Leftarrow$)] Suppose $R$ is an integral domain with $a,b,c \in R$, 
$a \neq 0$ and assume that $ab = ac$. Then 

\begin{equation*}
\begin{array}{rcll} 
ab
&=& ac                                                  
			&\hspace{2em} \mathrel{\#} \text{by assumption} \\
[5pt]
&\Rightarrow& ab-ac = 0
			&\hspace{2em} \mathrel{\#} \text{subtract $ac$ from both sides} \\
[5pt]
&\Rightarrow& a\cdot(b-c) = 0
			&\hspace{2em} \mathrel{\#} \text{factor out $a$} \\
[5pt]
&\Rightarrow& b-c = 0
			&\hspace{2em} \mathrel{\#} \text{$R$ has no zero divisors and 
				$a \neq 0 \Rightarrow b-c$ must equal zero} \\
[5pt]
&\Rightarrow& b = c 
			&\hspace{2em} \mathrel{\#} \text{$(ab=ac \Rightarrow b = c)
			\Rightarrow R$ has a cancellation law}
\end{array}
\end{equation*}

\medskip
\noindent
Thus $R$ is an integral domain $\Rightarrow$ $R$ has a cancellation 
law. $\blacksquare$
\end{itemize}


\medskip
\noindent
Now, to see why any finite integral domain $R$ is a field, consider 
$R = \{r, r^2, r^3, \hdots, r^n\}$ where $r^k \neq 0$ for $1 \leq k
\leq n$.  Since $R$ is finite we will have $r^k = r^l$ for some
$k$ and $l$ such that $k > l$. Then

\bigskip
\begin{equation*}
\begin{array}{rlll} 
r^k 
&=& r^l                                                 
			&\hspace{2em} \mathrel{\#} \text{$R$ is a finite integral domain} \\
[5pt]
&\Rightarrow& r \cdot r^{k-1} = r \cdot r^{l -1}        
			&\hspace{2em} \mathrel{\#} \text{factor out $r$} \\
[5pt]
&\Rightarrow& r^{k-1} = r^{l -1}                        
			&\hspace{2em}\mathrel{\#} \text{use cancellation law (cancel $r$, 
				Equation (\ref{eqn:cancellation_law})}) \\
[5pt]
&\Rightarrow& r \cdot r^{k-2} = r \cdot r^{l -2}        
			&\hspace{2em}\mathrel{\#} \text{factor out $r$} \\
[5pt]
&\Rightarrow& r^{k-2} = r^{l -2}                        
			&\hspace{2em}\mathrel{\#} \text{use cancellation law (cancel $r$, 
				Equation (\ref{eqn:cancellation_law}))} \\
[5pt]
&\vdots                                                 
			&&\hspace{2em}\mathrel{\#} \text{iterate $l - 1$ times} \\ 
[5pt]   
&\Rightarrow& r^{k - l + 1} = r^1                       
			&\hspace{2em}\mathrel{\#} \text{...} \\ 
[5pt]      
&\Rightarrow& r \cdot r^{k - l} = r \cdot r^0           
			&\hspace{2em}\mathrel{\#} \text{factor out $r$} \\
[5pt] 
&\Rightarrow& r^{k-l} = r^{0}                           
			&\hspace{2em}\mathrel{\#} \text{use cancellation law (cancel $r$, 
				Equation (\ref{eqn:cancellation_law}))} \\
[5pt]
&\Rightarrow& r^{k-l} = 1                               
			&\hspace{2em}\mathrel{\#} r^0 = 1
\end{array}
\end{equation*}

\bigskip
\noindent
So $ r^{k-l} = 1 $. If $k-l = 1$ then $r$ a unit since $r^{k-l} =
r^{1} = 1$ so $r^{-1}$ is $\frac{1}{r}$. Otherwise $k-l > 1$ and
$r^{k-l} = 1 \Rightarrow r^{k-l-1} = \frac{1}{r}$.  So $r^{-1} =
r^{k-l -1}$ and every $r \neq 0 \in R$ has an inverse. Thus every
non-zero $r \in R$ is a unit and so $R$ is a
field. $\blacksquare$

\bigskip
\noindent
So every field is an integral domain and every finite integral
domain is a field and so division behaves reasonably in these
cases.  


\bigskip
\noindent
A few conventions: set $\deg 0 = -\infty$ and $-\infty + -\infty
= - \infty$ and $-\infty +n = -\infty$ for $n \in \mathbb{Z}$.

\begin{lemma}
Let $F$ be a field and let $f$ and $g$ be polynomials over
$F$. Then $\deg fg = \deg f  + \deg g$. 
\label{lemma:degree}
\end{lemma}

\noindent
\textbf{Proof.} If either $f = 0$ or $g = 0$, then the equality
$\deg fg = \deg f + \deg g$ is true by our convention above. So,
suppose that $f \neq 0$ and $g \neq 0$. Write $f = a_n x^n +
\hdots + a_0$ and $g = b_m x^m + \hdots + b_0$ with $a_n \neq 0$
and $b_m \neq 0$.  Therefore, $\deg f = n$ and $\deg g = m$. The
definition of polynomial multiplication yields

\smallskip
\begin{equation*}
fg = (a_nb_m)x^{n+m} +(a_nb_{m-1} + a_{n-1} b_m)  x^{n+m-1} +
\cdots +a_0b_0 
\end{equation*}

\bigskip
\noindent
Since every field is an integral domain and since the
coefficients come from a field we know that there are no zero
divisors and so we can conclude that $a_nb_m \neq 0$, and so
$\deg fg = n + m = \deg f+ \deg g$, as desired. $\blacksquare$

\begin{definition}
The characteristic of a field $F$, $\text{char}(F)$, is the
smallest positive integer $n$ such that 

\medskip
\begin{equation*}
n \cdot 1 = \underbrace{1 + 1 + \cdots + 1}_{n \text{ times}} = 0
\end{equation*}
\end{definition}

\bigskip
\noindent
The characteristic has all kinds of interesting properties. For
example 

\begin{lemma}
Let $F$ be a field. Then
\begin{enumerate}
\item If $\text{char}(F) > 0$  then $\text{char} (F)$ is prime
\label{char:1}
\item If $F$ is a finite field then $\text{char} (F)  >  0$
\label{char:2}
\item If $F$ is a finite field then $\text{char} (F)$ is prime
\end{enumerate}
\end{lemma}

\bigskip
\noindent
\textbf{Proof.} 

\begin{enumerate}
\item Assume that $n$ is not prime so that there exists a
nontrivial factorization of $\text{char}(F) = n = p \cdot
q$. Then 

\begin{equation*}
\begin{array}{rcll}
0
&=& n \cdot 1                           
			&\qquad \mathrel{\#} \text{definition of $\text{char}(F)$} \\
[5pt]
&=& (p \cdot q) \cdot 1                 
			&\qquad \mathrel{\#} n = p \cdot q \\
[5pt]
&=& p \cdot (q \cdot 1)                 
			&\qquad \mathrel{\#} \text{multiplication is associative} \\
[5pt]
&=& (p \cdot 1) \cdot (q \cdot 1)       &\qquad \mathrel{\#} p = p \cdot 1
\end{array}
\end{equation*}

So $(p \cdot 1) \cdot (q \cdot 1) = 0$. But we know that $F$ is a
field and fields have no zero divisors so either $(p \cdot 1) =
0$ or $(q \cdot 1) = 0$. But this contradicts the assumption that
$\text{char}(F)$ was the smallest positive integer $n$ such that
$n \cdot 1 = 0$ since $p, q \neq 0$ and $n = p \cdot q$. So $n =
\text{char}(F)$ must have been prime.
\label{enumerate:1}

Another way to think about this proof is to notice that if $F$ is
a field of characteristic $n$ then the elements
$\{0,1,2,\hdots,,n-1\}$ of $F$ obey the rules for addition and
multiplication modulo $n$. Therefore there is a copy of
$\mathbb{Z}_n$ inside of $F$. Since $\mathbb{Z}_n$ has zero
divisors when $n$ is not prime (and a field is an integral domain
and so has no zero divisors, see 
\ref{theorem:integral_domain}), it follows that the characteristic of
a field must be prime.

Thus every finite field $F$ must have characteristic $p$ for some
prime $p$ and the elements $\{0,1,2,\hdots,p-1\}$ form a copy of
$\mathbb{Z}_p$ inside of $F$. This copy of $\mathbb{Z}_p$ is
known as the \emph{prime subfield of F}.

\item Since $F$ is finite not every element can be unique. In
particular, consider a finite field $F = \{f, f^2, f^3, \hdots,
f^n\}$ where $f^k \neq 0$ for $1 \leq k \leq n$.  Since $F$ is
finite we will have $(f^k \cdot 1) = (f^l \cdot 1)$ for some $k$
and $l$ where $k > l$.  Then $(f^k - f^l ) \cdot 1 = 0$ and so
$\text{char}(F) > 0$.
\label{enumerate:2}

\item Immediate by \ref{enumerate:1}. and \ref{enumerate:2}. $\blacksquare$
\end{enumerate}


\begin{proposition} \textbf{(Division Algorithm)}
Let $F$ be a field and let $f$ and $g$ be polynomials over $F$
with $f$ nonzero. Then there are unique polynomials $q$ and $r$
with $g = qf + r$ 
with $\deg r  < \deg f$.
\end{proposition}

\noindent
\textbf{Proof.}  Let $\mathcal{S} = \{t \in F[x] \mid t = g - qf
\text{ for some } q \in F[x] \}$. Now, we know $\mathcal{S} \neq
\emptyset$ since $g \in \mathcal{S}$ (see Equation (\ref{eqn:f})
below). We also know by the well ordering property of the
integers \cite{well_ordering_principle} that there is a
polynomial $r$ of least degree in $\mathcal{S}$. Then by
definition there is a $q \in F[x]$ with $r = g - qf$ and so $g =
qf +r$.

\bigskip
\noindent
We need to show that $\deg r < \deg f$. So suppose that $\deg r
\geq \deg f$, say with $n = \deg f$ and $m = \deg r$.  If $f =
a_nx^n + \hdots +a_0$ and $r = r_mx^m + \hdots +r_0$ with $a_n
\neq 0$ and $r_m \neq 0$, then by the method of long division of
polynomials, we see that $r = (r_m a_n^{-1})x^{m - n} f +
r^\prime$ with $\deg r^\prime < m = \deg r$. But then


\begin{equation*}
\begin{array}{rcll}
g
&=& qf + r                                      
			&\quad \mathrel{\#} g \in \mathcal{S} \text{ and well ordering principle} \\
[5pt]
&=& qf + (r_m a_n^{-1}) x^{m - n}f + r^\prime   
			&\quad \mathrel{\#}  r = (r_m a_n^{-1})x^{m - n} f + r^\prime \\
[5pt]
&=& (q + r_m a_n^{-1} x^{m - n}) f + r^\prime   
			&\quad \mathrel{\#} \text{factor out $f$} \\
[5pt]
&\Rightarrow& r^\prime \in \mathcal{S}             
			&\quad \mathrel{\#} r^\prime = g - (q + r_m a_n^{-1} x^{m - n}) f \in \mathcal{S}
\end{array}
\end{equation*}

\bigskip
\noindent
Since $\deg r^\prime < \deg r$, $r^\prime \in \mathcal{S} $
contradicts our choice of $r$ (the assumption that $r$ was the
polynomial of least degree in $\mathcal{S}$).  Therefore $\deg r
\geq \deg f$ is false so $\deg r < \deg f$ is true and this
proves existence of $q$ and $r$.

\bigskip
\noindent
We still have to show the uniqueness of $q$ and $r$. To do this,
suppose that $g = qf +r$ and $g = q^{\prime} f + r^\prime$ for
some polynomials $q,q^\prime, r,r^\prime \in F[x]$, and with both
$\deg r < \deg f$ and $\deg r^\prime < \deg f$. Then $qf +r =
q^{\prime} f +r^{\prime}$, so $(q - q^{\prime})f = r^\prime -
r$. Taking degrees and using Lemma \ref{lemma:degree}, we have

\begin{equation}
\deg (q - q^{\prime}) + \deg f  = \deg (r^\prime - r)
\label{eqn:qminusqprime}
\end{equation}

\bigskip
\noindent
Since $\deg r < \deg f$ and $\deg r^\prime < \deg f$, we have
$\deg (r^\prime - r) < 0$. However, if $\deg(q^\prime - q) \geq
0$, we get a contradiction of Equation (\ref{eqn:qminusqprime}). 
In particular, by our convention the only way for this to happen 
is for $\deg(q^\prime - q) = \deg(r^\prime - r) = - \infty$, and 
the only way for that to happen is if $q^\prime - q = 0 = r^\prime 
- r$. So $q^\prime = q$ and $r^\prime =r$, which shows
uniqueness. $\blacksquare$

\bigskip
\begin{proposition}
Let $F$ be a field and let $f$ and $g$ be polynomials over $F$,
with not both $f$ and $g$ zero. Then $\gcd(f, g)$ exists and is
unique.  Furthermore, there are polynomials $h$ and $k$ with
$\gcd(f, g) = hf + kg$.
\end{proposition}

\noindent
\textbf{Proof.}  Here we have to show that $\gcd(f, g)$ exists
and that it is unique. To show that $\gcd(f, g)$ exists, let
$\mathcal{S} = \{hf + kg : h, k \in F [x]\}$. Then $\mathcal{S}$
contains nonzero polynomials $f$ and $g$ (at least), since for
some $f^\prime \in \mathcal{S}$

\begin{equation}
\begin{array}{rcll}
f^\prime
&=& hf +  kg                    
		&\quad \mathrel{\#} \text{definition of elements of } \mathcal{S} \\
[5pt]
&=& 1 \cdot f + 0 \cdot g       
		&\quad \mathrel{\#} \text{let $h = 1$ and $k = 0$}  \\
[5pt]
&=& f                           
		&\quad \mathrel{\#} \text{so $f \in \mathcal{S}$}
\end{array}
\label{eqn:f}
\end{equation}

\bigskip
\noindent
Likewise for $g$ (let $h = 0$ and $k = 1$). So $f,g \in
\mathcal{S}$. But this means that there is a nonzero polynomial
$d \in \mathcal{S}$ of smallest degree by the well ordering
principle \cite{well_ordering_principle}, so we can write $d = hf
+ kg$ for some $h, k \in F [x]$. By dividing by the leading
coefficient of $d$, we may assume that $d$ is monic without
changing the condition that $d$ is the polynomial of least degree
in $\mathcal{S}$.

\bigskip
\noindent
Now the claim is that $d = \gcd(f,g)$. To show that $d$ is a
common divisor of $f$ and $g$, first consider $f$. By the
division algorithm, we can write $f$ as $f = qd + r$ for some
polynomials $q$ and $r$ with $\deg r < \deg d$. Then

\begin{equation}
\begin{array}{rcll}
f
&=& qd +  r                             
			&\mathrel{\#} f \in \mathcal{S} \text{ by Equation (\ref{eqn:f})}, 
				\mathcal{S} \text{ has a division algorithm} \\
[5pt]
&\Rightarrow& r = f  - qd                 
			&\mathrel{\#} \text{solve for $r$} \\
[5pt]
&\Rightarrow& r = f - q(hf + kg)           
			&\mathrel{\#} d = hf + kg \\
[5pt]
&\Rightarrow& r = f - qhf  - qkg           
			&\mathrel{\#} \text{arithmetic} \\
[5pt]
&\Rightarrow& r = (1 - qh)f  + (-qk)g      
			&\mathrel{\#} F[x] \text{ is closed under */+ so } r  \text{ is of the form } hf + kg \\
[5pt]
&\Rightarrow&  r \in \mathcal{S}           
			&\mathrel{\#} r \text{ is of the form } hf + kg \text{ so } r \in \mathcal{S} 
\end{array}
\label{eqn:division}
\end{equation}


\bigskip
\noindent
So $r \in \mathcal{S}$. Now, if $r \neq 0$ we have contradiction
since we assumed that $\deg(r) < \deg(d)$ and that $d$ was the
polynomial of lowest degree in $\mathcal{S}$. Thus $r$ must equal
$0$, which shows that $f = qd$, and so $d$ divides
$f$. Similarly, $d$ divides $g$ and so $d$ is a common divisor of
$f$ and $g$.

\bigskip
\noindent
If $e$ is any other common divisor of $f$ and $g$, then e divides
any combination of $f$ and $g$; in particular, $e$ divides $d= hf
+ kg $.  This forces $\deg(e) \leq \deg(d)$ by Lemma
\ref{lemma:degree}. Thus, $d$ is the monic polynomial of largest
degree that divides $f$ and $g$, so $d$ is a greatest common
divisor of $f$ and $g$.  This shows that $\gcd(f,g)$ exists.

\bigskip
\noindent
To show that $\gcd(f,g)$ is unique, suppose that $d$ and
$d^\prime$ are both monic common divisors of $f$ and $g$ of
largest degree.  By Equation (\ref{eqn:division}), we can write
both $d$ and $d^\prime$ as combinations of $f$ and $g$. In
addition, the argument above shows that $d$ divides $d^\prime$
and vice-versa. If $d^\prime = ad$ and $d = bd^\prime$, then $d =
bd^\prime = bad$. Taking degrees shows that $\deg(ba) = 0$, which
means that $a$ and $b$ are both constants. However since $d$ and
$d^\prime$ are monic, for $d^\prime = ad$ to be monic it must be
that $a = 1$. Thus, $d^\prime = ad = d$.  This shows that the
greatest common divisor is unique, completing the proof. 
%
%
%
\section{Conclusions}
\label{sec:conclusions}
%
%
%
\section{Acknowledgements}
\label{sec:acknowledgements}
Thanks to @binary\_operation@mathstodon.xyz for 
the insightful comments on Lemma \ref{lemma:zero_divisors}.
%
%	LaTeX source on overleaf.com
%
\section*{\LaTeX \hspace{0.025 mm} Source}
\url{https://www.overleaf.com/read/tffzttzxhsyw}
%
%	get a bibliography
%
%	Note:.bib files go in ~/Library/texmf/bibtex/bib with TeXShop (MacTeX).
%	You can also use an absolute path, e.g. \bibliography{/Users/dmm/papers/bib/qc}
%
\bibliographystyle{plain}
\bibliography{qc}
%
%	done
%
\end{document} 
