\documentclass{article}
%
%
%	LaTeX template
%
%	David Meyer
%	dmm613@gmail.com
%	08 Oct 2023
%
%
%   get various packages
%
\usepackage[margin=1.0in]{geometry}                                     % adjust margins
\geometry{letterpaper}                                                  % or a4paper or a5paper or ... 
\usepackage{url}                                                        % need this to use URLs in bibtex
\usepackage{setspace}                                                   % need this for \setstrech{...}
\usepackage{scrextend}                                                  % need this for addmargin
\usepackage[export]{adjustbox}                                          % need this to get frame for includegraphics
%
%   tikz et al
%
\usepackage{tikz}
\usetikzlibrary{calc,patterns,angles,quotes,shapes,math,decorations,
                through,intersections,lindenmayersystems,backgrounds,
                hobby}
\tikzset{>=latex}                                                       % default to LaTeX arrow head
\usepackage{circuitikz}                                                 % draw circuits    
\usepackage{pgfplots}
%
%	more math stuff
%
\usepackage{amsmath,amsfonts,amssymb,amsthm}
\usepackage{bm}
\usepackage{mathtools}
\usepackage{commath}                                                    % get \norm{x}
\usepackage{fixmath}                                                    % get \mathbold
\usepackage{gensymb}                                                    % get \degree
\usepackage{mathrsfs}
\usepackage{hyperref}
\usepackage{subcaption}
\usepackage{authblk}                                                    % comment out if using beamer (stops \author{} from working in beamer)
\usepackage{graphicx}
\usepackage{hyperref}
\usepackage{alltt}
\usepackage{xcolor}
\usepackage{colortbl}                                                   % \rowcolor{yellow!75} etc
\usepackage{float}
\usepackage{braket}
\usepackage{siunitx}
\usepackage{relsize}
\usepackage{multirow}
\usepackage{esvect}
\usepackage{enumitem}                                                   % use characters instead of numbers in enumerate
\usepackage{changepage}                                                 % needed for \begin{adjustwidth}{-3.25em}{-2.0em} (left justify)
\usepackage{bigints}
%
%	lualatex
%
%	Put this before \documentclass
%
%	!TEX TS-program = LuaLaTeX
%
%
% \usepackage{luatex85,luamplib}
% \mplibnumbersystem{double}
% \mplibtextextlabel{enable}
%
%
%
%	Describe floating point parameters, \fpeval
%
\ExplSyntaxOn
\cs_set_eq:NN \fpeval \fp_eval:n
\ExplSyntaxOff
%
%	Get the x and y components out of a coordinate, e.g.
%
%	\coordinate (EP) at (8,5);
%	\gettikzxy{(EP)}{\slopex}{\slopey}
%
\makeatletter
\newcommand{\gettikzxy}[3]{%
  \tikz@scan@one@point\pgfutil@firstofone#1\relax
  \edef#2{\the\pgf@x}%
  \edef#3{\the\pgf@y}%
}
\makeatother
%
%
%	Watermarks
%
% \usepackage{draftwatermark}
% \SetWatermarkText{Draft}
% \SetWatermarkScale{5}
% \SetWatermarkLightness {0.9} 
% \SetWatermarkColor[rgb]{0.7,0,0}
%
%
%	theorems, definitions, etc
%
\theoremstyle{definition}
\newtheorem{theorem}{Theorem}[section]
\newtheorem{definition}{Definition}[section]
\newtheorem{proposition}{Proposition}[section]
\newtheorem{lemma}{Lemma}[section]
\newtheorem{example}{Example}[section]
\newtheorem{remark}{Remark}[section]
%
%	For drawing matrix products 
%
\newcommand*{\vertbar}{\rule[-1ex]{0.5pt}{2.5ex}}
\newcommand*{\horzbar}{\rule[.5ex]{2.5ex}{0.5pt}}

%
%	The following code allows you to do
%
%	\begin{bmatrix}[r] (or [c] or [l])
%
\makeatletter
\renewcommand*\env@matrix[1][c]{\hskip -\arraycolsep
  \let\@ifnextchar\new@ifnextchar
  \array{*\c@MaxMatrixCols #1}}
\makeatother
%
%	make \arg{min,max}_{n \to \infty} work nicely
%
\newcommand{\argmax}{\operatornamewithlimits{argmax}}
\newcommand{\argmin}{\operatornamewithlimits{argmin}}
%
%	handy commands
%
\newcommand*{\Scale}[2][4]{\scalebox{#1}{$#2$}}
\DeclareMathOperator{\E}{\mathbb{E}}
\DeclareMathOperator{\bda}{\Big \downarrow}						% big down arrow
\newcommand{\veq}{\mathrel{\rotatebox{90}{$=$}}}
%
%
%
\title{Fermat's Christmas Theorem}
\author{David Meyer \\ dmm613@gmail.com}
%
\date{Last update: \today}
%
%
%
\begin{document}
\maketitle

\section{Fermat's Christmas Theorem}

\noindent
Fermat's Christmas Theorem is a beautiful and simply stated
theorem \cite{wiki:christmas_theorem}. The theorem (aka Fermat's
theorem on sums of two squares) states that an odd prime number
$p$ can be expressed as

\bigskip
\begin{center}
\scalebox{1.25} {$p = r^{2} + s^{2}$}
\end{center}
\bigskip

\noindent
where $r,s \in \mathbb{N}$, if and only if $p \equiv 1 \textrm{
(mod $4$)}$. That is, the theorem holds iff $p = 4n + 1$ for some
$n \in \mathbb{N}$.

\bigskip
\noindent
For example, the primes 5, 13, 17, 29, 37 and 41 are all
congruent to 1 modulo 4 and can be expressed as sums of two
squares in the following ways:

\begin{equation*}
\begin{array}{rcll} 
5   &=& 1^{2} + 2^{2}   \\
13  &=& 2^{2} + 3^{2}   \\
17  &=& 1^{2} + 4^{2}   \\
29  &=& 2^{2} + 5^{2}   \\
37  &=& 1^{2} + 6^{2}   \\
41  &=& 4^{2} + 5^{2}
\end{array}
\end{equation*}


\bigskip
\noindent
On the other hand, the primes 3, 7, 11, 19, 23 and 31 are all
congruent to 3 modulo 4 and none of them can be expressed as the
sum of two squares. This is the easier part of the theorem since
it follows immediately from the observation that all squares are
congruent to 0 or 1 modulo 4.

\bigskip
\noindent
The prime numbers $p$ for which Fermat's Christmas Theorem holds
are called Pythagorean primes.  See
\cite{wiki:pythagorean_primes} for more on Pythagorean primes.

\bigskip
\noindent
A variety of proofs of Fermat's Christmas Theorem can be found in
\cite{wiki:christmas_theorem_proofs}.
%
%
%
\section{Why is it called Fermat's Christmas Theorem?}
This theorem is called Fermat's Christmas Theorem 
because Fermat announced a proof of the theorem in a letter 
to Mersenne dated December 25, 1640 \cite{wiki:christmas_theorem}. And of course, Fermat 
didn't include a proof in his letter.
%
%
%
\section*{Acknowledgements}
\label{sec:acknowledgements}
%
%	LaTeX source on overleaf.com
%
\section*{\LaTeX \hspace{0.025 mm} Source}
% \url{}
\newpage				% why?
%
%	get a bibliography
%
%	Note:.bib files go in ~/Library/texmf/bibtex/bib with TeXShop (MacTeX).
%	You can also use an absolute path, e.g. \bibliography{/Users/dmm/papers/bib/qc}
%
\bibliographystyle{plain}
\bibliography{qc}
%
%
%
% \section*{Appendix A}
%
%	done
%
\end{document} 


