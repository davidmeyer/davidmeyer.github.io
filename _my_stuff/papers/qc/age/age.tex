\documentclass[11pt, oneside]{article}          % use "amsart" instead of "article" for AMSLaTeX format


\usepackage{geometry}                           % See geometry.pdf to learn the layout options. There are lots.
\geometry{letterpaper}                          % ... or a4paper or a5paper or ... 
%\geometry{landscape}                           % Activate for for rotated page geometry
%\usepackage[parfill]{parskip}                  % Activate to begin paragraphs with an empty line rather than an indent
\usepackage{graphicx}                           % Use pdf, png, jpg, or eps� with pdflatex; use eps in DVI mode
%
%
%
\usepackage{amssymb}
\usepackage{amsmath}
\usepackage{amsthm}
\usepackage{mathrsfs}
\usepackage[hyphens,spaces,obeyspaces]{url}
\usepackage{url}
\usepackage{hyperref}
\usepackage{subcaption}
\usepackage{authblk}
\usepackage{mathtools}
\usepackage{graphicx}
\usepackage[export]{adjustbox}
\usepackage{fixltx2e}
\usepackage{hyperref}
\usepackage{alltt}
\usepackage{color}
\usepackage[utf8]{inputenc}
\usepackage[english]{babel}
\usepackage{float}
\usepackage{bigints}
\usepackage{braket}
\usepackage{siunitx}
\usepackage{mathtools}
\usepackage{setspace}



\usepackage{tikz}
\usepackage{verbatim}
\usetikzlibrary{plotmarks}
\usepackage{pgfplots}
\usepackage{amsmath}
\usepackage{relsize}
\usepackage[hyphenbreaks]{breakurl}
 
\newtheorem{thm}{Theorem}[section]
\theoremstyle{definition}
\newtheorem{definition}{Definition}[section]
\newtheorem{proposition}{Proposition}[section]
\newtheorem{lemma}{Lemma}[section]
\newtheorem{example}{Example}[section]

\newcommand{\argmax}{\operatornamewithlimits{argmax}}
\newcommand{\argmin}{\operatornamewithlimits{argmin}}





\title{What Is The Difference In Our Ages?}
\author{David Meyer \\ dmm613@gmail.com}
\date{Last update: \today}

\begin{document}
\maketitle

\section{Introduction}
Suppose your age is $x_t$ and my age is $y_t$ at some time $t$.
Then in $n$ years your age is $x_t + n$ and my age is $y_t +
n$. This implies that as time goes by (as measured by $n \in
\mathbb{N}$) the difference in our ages vanishes!

\bigskip
\noindent
Why? Consider that 

\begin{equation*}
\centering
\lim\limits_{n \to \infty} \Bigg [ \frac{x_t + n}{y_t + n} \Bigg ]  = 1
\end{equation*}

\bigskip
\noindent
which is another way of saying the same thing. This result is
reassuring since it pretty much models our common experience. 
But why is this true? As we will see in Section
\ref{subsec:putting_it_all_together}, this "ratio of our ages"
sequence is a \emph{convergent sequence} in $\mathbb{R}$ with
limit one.


\subsection{A More General Formulation}
More generally consider the function $f_a(n) = a + n$ 
where $a,n \in \mathbb{N}$. Then 

\bigskip
\begin{equation}
\centering
\lim\limits_{n \to \infty} \Bigg [ \frac{f_a(n)}{f_b(n)} \Bigg ]  = 1
\label{eqn:lim}
\end{equation}

\bigskip
\noindent
for $a,b,n \in \mathbb{N}$. 

\bigskip
\noindent
Equation (\ref{eqn:lim}) is also frequently
written using the following alternate 
notation: 

\medskip
\begin{equation*}
f_a(n) \thicksim f_b(n)
 \end{equation*}

\bigskip
\noindent
where the $\thicksim$ symbol means that the ratio of its two
arguments tends towards $1$ as its arguments tend toward
$\infty$. Said another way:

\medskip
\begin{equation*}
f(x) \thicksim g(x) \Rightarrow \lim\limits_{x \to \infty} \dfrac{f(x)}{g(x)} = 1
\end{equation*}

\smallskip
\section{Proof Using Sequences}
Recall our problem setup: Suppose your age at some time 
$t$ is $x_t$ and my age at that same time is $y_t$. Then 
in $n$ years your age will be $x_t + n$ and my age will 
be $y_t + n$, for $n \in \mathbb{N}$. 

\bigskip
\noindent
Now consider the "ratio of our ages" sequence:

\medskip
\begin{equation*}
(a_n)_{n \in \mathbb{N}} = \left (\dfrac{x_t +n}{y_t + n}\right )_{n \in \mathbb{N}}
\end{equation*}

\bigskip
\noindent 
The notation $(a_n)_{n \in \mathbb{N}}$ means that $a$ is a map from
$\mathbb{N}$ to $\mathbb{R}$, namely

\begin{equation*}
a: \mathbb{N} \to \mathbb{R}
\end{equation*}

\bigskip
\noindent
That is, $(a_n)_{n \in \mathbb{N}}$ is a sequence of real numbers.

\bigskip
\noindent 
\begin{definition}{\bf Convergent Sequence: }  We say that a sequence 
$(a_n)_{n \in \mathbb{N}}$ is \emph{convergent} to $a \in
\mathbb{R}$ if for all $\epsilon$ greater that zero there exists
an $N$ in $\mathbb{N}$ such that if $n \in \mathbb{N}$ is greater
than or equal to $N$ then $\mid a_n - a \mid < \epsilon$. Put
more concisely

\begin{equation*}
\forall \epsilon > 0 \;\; \exists N \in \mathbb{N} \;\; 
 \forall n \geq N : \;\; \mid a_n - a \mid \; < \epsilon
\end{equation*}

\bigskip
\noindent
Here $a$ is called the \emph{limit} of the sequence $(a_n)_{n \in
\mathbb{N}_{0}}$ and we write ${\displaystyle \lim_{n \to \infty} a_n
= a}$ (or sometimes $a_n \xrightarrow{n \to \infty}{} a$).
\end{definition}


\noindent
So now we want to ask the obvious question: what happens to the
ratio of our ages sequence as $n$ goes to $\infty$? In other
words, does this sequence converge, and if it does, to what
value?

\bigskip
\noindent
All of this means that we want to know if this limit 

\bigskip
\begin{equation*}
\lim_{n \to \infty} \left (\dfrac{x_t +n}{y_t + n}\right )_{n \in \mathbb{N}}
\end{equation*}

\bigskip
\noindent
exists and if it does, what is its value.


\bigskip
\noindent
As we will see in Section \ref{subsec:putting_it_all_together},
we do have some machinery we can use to evaluate this limit such
as the algebraic rules for limits \cite{properties_of_limits} and
the fact that the limit is a linear operator \cite{fa_mit}.

\subsection{Putting It All Together}
\label{subsec:putting_it_all_together}
We can see that the "ratio of our ages" sequence 
$\left (\dfrac{x_t +n}{y_t + n}\right )_{n \in \mathbb{N}}$ 
is indeed a convergent sequence with limit equal to
one, since

\begin{equation*}
\begin{array}{lllll}
{\displaystyle \lim_{n \to \infty} \left (\dfrac{x_t + n}{y_t + n}\right )}
&=& \dfrac{{\displaystyle \lim_{n \to \infty} \left (x_t + n \right )}} {{\displaystyle  \lim_{n \to \infty} \left (y_t + n \right)}}
					&\qquad \qquad \qquad \mathrel{\#} \text{quotient rule for limits \cite{properties_of_limits}} \\
[25pt]
&=& \dfrac{{\displaystyle \lim_{n \to \infty} \left (x_t + n \right )}\left (\frac{1}{n}\right )} 
          {{\displaystyle \lim_{n \to \infty} \left (y_t + n \right)} \left ( \frac{1}{n} \right )}
					&\qquad \qquad \qquad \mathrel{\#} \text{multiply the fraction by $1 
					= \frac{\left ( \frac{1}{n} \right )}{\left ( \frac{1}{n} \right )}$}\\
[25pt]
&=& \dfrac{{\displaystyle \lim_{n \to \infty} \left (\frac{x_t}{n} + \frac{n}{n} \right )}} 
           {{\displaystyle \lim_{n \to \infty} \left (\frac{y_t}{n} + \frac{n}{n} \right)}}
					&\qquad \qquad \qquad \mathrel{\#} \text{multiply through by $\frac{1}{n}$}\\
[25pt]
&=& \dfrac{{\displaystyle \lim_{n \to \infty}  \left (\frac{x_t}{n} + 1 \right )}} 
           {{\displaystyle \lim_{n \to \infty} \left (\frac{y_t}{n} + 1 \right)}}
					&\qquad \qquad \qquad \mathrel{\#} \frac{n}{n} = 1\\
[25pt]
&=& \dfrac{{\displaystyle \lim_{n \to \infty}   \frac{x_t}{n} + \lim_{n \to \infty} 1}} 
           {{\displaystyle \lim_{n \to \infty}  \frac{y_t}{n} + \lim_{n \to \infty} 1}}
					&\qquad \qquad \qquad \mathrel{\#} \text{limit is a linear operator \cite{fa_mit}} \\
[25pt]
&=& \dfrac{{\displaystyle \lim_{n \to \infty}  \frac{x_t}{n} + 1}} 
           {{\displaystyle \lim_{n \to \infty} \frac{y_t}{n} + 1}}
					&\qquad \qquad \qquad \mathrel{\#} \text{${\displaystyle \lim_{n \to \infty} c 
					= c}$ for constant $c$ \cite{properties_of_limits}} \\
[25pt]
&=& \dfrac{0 + 1} {0 + 1}
					&\qquad \qquad \qquad \mathrel{\#} 
					{\displaystyle \lim_{n \to \infty} \frac{c}{n} = c \lim_{n \to \infty} \frac{1}{n} = 0} 
					\text{ for constant $c$ \cite{limits_of_sequences}} \\
[15pt]
&=& \dfrac{1}{1}	&\qquad \qquad \qquad \mathrel{\#} \text{simplify} \\
[15pt]
&=& 1				&\qquad \qquad \qquad \mathrel{\#} \text{apparently we're all roughly the same age}
\end{array}
\end{equation*}


\section{Conclusions}
So I guess that while in our short lives our ages appear to vary widely
across our life spans, when we look at a bigger picture that difference
disappears.
%
%
%
\section*{Acknowledgements}
%                                                                                                                                          
%       LaTeX source on overleaf.com                                                                                                       
%                                                                                                                                          
\section*{\LaTeX \hspace{0.10 mm} Source}
% \url{}
%                                                                                                                                          
%       get a bibliography                                                                                                                 
%                                                                                                                                          
%       Note:.bib files go in ~/Library/texmf/bibtex/bib with TeXShop (MacTeX).                                                            
%       You can also use an absolute path, e.g. \bibliography{/Users/dmm/papers/bib/qc}                                                    
%                                                                                                                                          
\bibliographystyle{plain}
\bibliography{qc}
%                                                                                                                                          
%       done                                                                                                                               
%                                                                                                                                          
\end{document}
