\documentclass{article}
%
%
%	fourier_series.tex
%
%	David Meyer
%	dmm613@gmail.com
%	08 Oct 2023
%
%
%   get various packages
%
\usepackage[margin=1.0in]{geometry}                                     % adjust margins
\geometry{letterpaper}                                                  % or a4paper or a5paper or ... 
\usepackage{url}                                                        % need this to use URLs in bibtex
\usepackage{setspace}                                                   % need this for \setstrech{...}
\usepackage{scrextend}                                                  % need this for addmargin
\usepackage[export]{adjustbox}                                          % need this to get frame for includegraphics
%
%   tikz et al
%
\usepackage{tikz}
\usetikzlibrary{calc,patterns,angles,quotes,shapes,math,decorations,
                through,intersections,lindenmayersystems,backgrounds,
                hobby}
\tikzset{>=latex}                                                       % default to LaTeX arrow head
\usepackage{circuitikz}                                                 % draw circuits    
\usepackage{pgfplots}
%
%	more math stuff
%
\usepackage{amsmath,amsfonts,amssymb,amsthm}
\usepackage{bm}
\usepackage{mathtools}
\usepackage{commath}                                                    % get \norm{x}
\usepackage{fixmath}                                                    % get \mathbold
\usepackage{gensymb}                                                    % get \degree
\usepackage{mathrsfs}
\usepackage{hyperref}
\usepackage{subcaption}
\usepackage{authblk}                                                    % comment out if using beamer (stops \author{} from working in beamer)
\usepackage{graphicx}
\usepackage{hyperref}
\usepackage{alltt}
\usepackage{xcolor}
\usepackage{colortbl}                                                   % \rowcolor{yellow!75} etc
\usepackage{float}
\usepackage{braket}
\usepackage{siunitx}
\usepackage{relsize}
\usepackage{multirow}
\usepackage{esvect}
\usepackage{enumitem}                                                   % use characters instead of numbers in enumerate
\usepackage{changepage}                                                 % needed for \begin{adjustwidth}{-3.25em}{-2.0em} (left justify)
\usepackage{bigints}
\usepackage{multirow}                                                   % \multirow and \multicolumn
%
%	lualatex
%
%	Put this before \documentclass
%
%	!TEX TS-program = LuaLaTeX
%
%
% \usepackage{luatex85,luamplib}
% \mplibnumbersystem{double}
% \mplibtextextlabel{enable}
%
%
%
%	Describe floating point parameters, \fpeval
%
\ExplSyntaxOn
\cs_set_eq:NN \fpeval \fp_eval:n
\ExplSyntaxOff
%
%	Get the x and y components out of a coordinate, e.g.
%
%	\coordinate (EP) at (8,5);
%	\gettikzxy{(EP)}{\slopex}{\slopey}
%
\makeatletter
\newcommand{\gettikzxy}[3]{%
  \tikz@scan@one@point\pgfutil@firstofone#1\relax
  \edef#2{\the\pgf@x}%
  \edef#3{\the\pgf@y}%
}
\makeatother
%
%
%	Watermarks
%
% \usepackage{draftwatermark}
% \SetWatermarkText{Draft}
% \SetWatermarkScale{5}
% \SetWatermarkLightness {0.9} 
% \SetWatermarkColor[rgb]{0.7,0,0}
%
%
%	theorems, definitions, etc
%
\theoremstyle{definition}
\newtheorem{theorem}{Theorem}[section]
\newtheorem{definition}{Definition}[section]
\newtheorem{proposition}{Proposition}[section]
\newtheorem{lemma}{Lemma}[section]
\newtheorem{example}{Example}[section]
\newtheorem{remark}{Remark}[section]
%
%	For drawing matrix products 
%
\newcommand*{\vertbar}{\rule[-1ex]{0.5pt}{2.5ex}}
\newcommand*{\horzbar}{\rule[.5ex]{2.5ex}{0.5pt}}

%
%	The following code allows you to do
%
%	\begin{bmatrix}[r] (or [c] or [l])
%
\makeatletter
\renewcommand*\env@matrix[1][c]{\hskip -\arraycolsep
  \let\@ifnextchar\new@ifnextchar
  \array{*\c@MaxMatrixCols #1}}
\makeatother
%
%	make \arg{min,max}_{n \to \infty} work nicely
%
\newcommand{\argmax}{\operatornamewithlimits{argmax}}
\newcommand{\argmin}{\operatornamewithlimits{argmin}}
%
%	handy commands
%
\newcommand*{\Scale}[2][4]{\scalebox{#1}{$#2$}}
\DeclareMathOperator{\E}{\mathbb{E}}
\DeclareMathOperator{\bda}{\Big \downarrow}						% big down arrow
\newcommand{\veq}{\mathrel{\rotatebox{90}{$=$}}}
%
%	Title, author and date
%
\title{A Few Notes on the Fourier Series}
\author{David Meyer \\ \href{mailto:dmm613@gmail.com}
                            {dmm613@gmail.com}}
\date{Last Update: \today \\
	 {\vspace{1.00mm} \small Initial Version: October 3, 2019}}
%
%
%
\begin{document}
\maketitle
%
%
%
\section{Introduction}
\label{sec:introduction}

\section{Vector Spaces: Linear Algebra vs. the Fourier Series}
\label{section:vector_spaces}
Here we will use the convention that the constant $c$ denotes 
the constant function $f(x) = c$, for all $x$, when the context 
indicates that $c$ is a function. On the other hand, the notation
$c \cdot c$ denotes the scalar multiplication of the scalar value 
$c$ with itself. Consequentially, I will use {\bf 1} to represent 
the constant function $f(x) = 1$ and use context to disambiguate 
the function $f(x)$ from the scalar value $1$. For example, in 
Equation (\ref{eqn:1}), $\langle {\bf 1},{\bf 1} \rangle = \langle 
f(x),f(x) \rangle$, while the notation $1 \cdot 1$ represents the scalar 
multiplication of the scalar value 1 with itself.

\setlength{\tabcolsep}{15pt}
\renewcommand{\arraystretch}{3.00}


\smallskip
\bigskip
\begin{table}[H]
 \Large
 \centering
 \resizebox{0.85 \textwidth}{!} {
   \begin{tabular}{|l|c|c|} 
     \hline
     {\Huge \bf What} & {\Huge \bf Linear Algebra} & {\Huge \bf Fourier Series} \\ 
     \hline
     \hline
     {\LARGE \bf Vector Space}                  & $\boldsymbol{\mathbb{R}}^{n}$ 
                                                & Piecewise smooth $2\pi$-periodic functions on $\mathbb{R}$ \\
[10pt]
     \hline
     {\LARGE \bf Inner Product}                 & $\langle {\bf u},{\bf v} \rangle =
                                                        {\displaystyle \sum\limits_{i=1}^{n} u_{i}v_{i}}$
                                                & $\langle f(t),g(t) \rangle = {\displaystyle \frac{1}{\pi} 
                                                        \int\limits_{-\pi}^{\pi} f(t)g(t) \, dt}$ \\
[10pt]
     \hline
     {\LARGE \bf Orthonormal Basis $\left (\mathbb{R}^3 \right )$} 
                                                & $\{(1,0,0),(0,1,0),(0,0,1)\}$ 
                                                & $\{{\bf 1}, \cos mt, \sin nt\}, \, m,n \in \mathbb{N}\backslash\{0\}$ \\
[10pt]
     \hline
     {\LARGE \bf Representation of a Vector in the Basis} 		
                                                & ${\displaystyle {\mathbf x} = \sum\limits_{i=1}^{n} a_{i}\mathbf{e}_{i}}$ 
                                                & $f(t) = {\displaystyle \dfrac{a_{0}}{2} + \sum\limits_{n=1}^{\infty} 
                                                        a_{n} \cos nt + \sum\limits_{n=1}^{\infty} b_{n} \sin nt}$  \\
[10pt]
     \hline
     {\LARGE \bf Coefficients are Projections} 
                                                & $a_{i} = \langle \mathbf{x},\mathbf{e}_{i} \rangle$
                                                & \begin{tabular}{lll}
                                                        \multirow{3}{*} {}
                                                                & $a_{0} = \langle f(t),1 \rangle$       \vspace{-1.10cm} \\
                                                                & $a_{m} = \langle f(t),\cos mt \rangle$ \vspace{-1.10cm} \\
                                                                & $b_{m} = \langle f(t),\sin mt \rangle$
                                                        \end{tabular} \\
     \hline
  \end{tabular}
 }
 \caption{Vector Spaces: Linear Algebra vs. Fourier Series}
 \label{table:vector_spaces}
\end{table}

\smallskip
\subsection{Orthonormal Bases and the Fourier Series}
Some people (specifically Rahul Narain (@narain@mathstodon.xyz))
feel that the orthonormal basis for the Fourier series should be
$\{{\bf 1},\sqrt{2} \cos mt,\sqrt{2} \sin nt\}, \, m,n \in \mathbb{N}
\backslash \{0\}$. 

\bigskip
\noindent
Ok, but why is Narain arguing this? My guess is as follows:
First, we know that the inner product of a vector $\mathbf{u}$ 
with itself equals 1 ($\langle \mathbf{u},\mathbf{u} \rangle = 
1$). However, for the Fourier series (where {\bf 1} represents the 
constant function $f(t) = 1$) we have $\langle {\bf 1},{\bf 1}
\rangle = 2$. Why? We can see this in Equation (\ref{table:vector_spaces}) 
and in a bit more detail in Figure \ref{figure:fourier_inner_product}.

\begin{equation}
	\langle {\bf 1},{\bf 1} \rangle											= 
		\frac{1}{\pi}\int\limits_{-\pi}^{\pi} 1 \cdot 1 \, dt	= 
		\frac{1}{\pi}\int\limits_{-\pi}^{\pi} dt				= 
		\frac{1}{\pi}  \, t \, \bigg |^{\pi}_{-\pi}				= 
		\frac{1}{\pi} \left (\pi - (-\pi) \right )				= 
		\frac{1}{\pi} 2\pi										= 
		2
\label{eqn:1}
\end{equation}


\bigskip
\begin{figure}[H]
  \fcolorbox{black}{white} {								% make the frame blue
    \begin{math}
      \begin{array}{llll}
        \langle {\bf 1},{\bf 1} \rangle
        &=& {\displaystyle \frac{1}{\pi}\int\limits_{-\pi}^{\pi} f(t) f(t) \, dt}
     	   &\hspace{2em} \mathrel{\#} \text{since {\bf 1} represents $f(t) = 1$ and the 
										 definition of the inner product 
										 (Table \ref{table:vector_spaces})} \\
        &=& {\displaystyle \frac{1}{\pi}\int\limits_{-\pi}^{\pi} 1 \cdot 1 \, dt}
     	   &\hspace{2em} \mathrel{\#} \text{since in $f(t) = 1$ for all 
										 $t \in \mathbb{R}$} \\
	   &=& {\displaystyle \frac{1}{\pi}\int\limits_{-\pi}^{\pi} dt}
	  	   &\hspace{2em} \mathrel{\#} \text{since $1 \cdot 1 = 1$} \\
	   &=& {\displaystyle \frac{1}{\pi}  \, t \, \bigg |^{\pi}_{-\pi}}
		   &\hspace{2em} \mathrel{\#} \text{by the FToC \cite{ftoc}} \\
	   &=& {\displaystyle \frac{1}{\pi} \left (\pi - (-\pi) \right )}
		   &\hspace{2em} \mathrel{\#} \text{evaluate at the end points} \\
	   &=& {\displaystyle \frac{1}{\pi} 2\pi}
		   &\hspace{2em} \mathrel{\#} \text{simplify} \\
	   &=& {\displaystyle 2}
		   &\hspace{2em} \mathrel{\#} \langle 1,1 \rangle = 2 \; 
								   \cite{quora:fourier_basis_functions}
     \end{array}
   \end{math}
 }
 \caption{Value of $\langle 1,1 \rangle$ for Fourier Series}
 \label{figure:fourier_inner_product}
\end{figure}

\bigskip
\noindent
On the other hand, $\langle \cos nt, \cos nt \rangle =
{\displaystyle \dfrac{1}{\pi} \int\limits_{-\pi}^{\pi} \cos^2 nt
\,dt = \dfrac{1}{\pi} \pi = 1}$. In the same way, $\langle \sin
nt, \sin nt \rangle = 1$.
  
 
\bigskip
\noindent
 We can also see that the vectors in the basis $\{{\bf 1}, \cos mt,
 \sin nt\}, m,n \in \mathbb{N}\backslash\{0\}$ are 
 orthogonal:
 
\begin{equation*}
  \langle {\bf 1},\sin nt \rangle = {\displaystyle
  \dfrac{1}{\pi}\int\limits_{-\pi}^{\pi} 1 \sin nt \, dt} =
  {\displaystyle \dfrac{1}{\pi}\int\limits_{-\pi}^{\pi} \sin nt
  \, dt}  = 0
\end{equation*} 
  
\smallskip
\noindent
Similarly, $\langle {\bf 1},\cos nt \rangle = \langle \cos mt,\sin nt
\rangle = 0$.


\bigskip
\noindent
Note that some authors set ${\displaystyle \langle {\bf 1},
{\bf 1} \rangle = \int\limits_{0}^{2\pi} dt = 2 \pi}$
\cite{quora:fourier_basis_functions}.
 


%
%
%	show a derivation/proof
%
% \begin{equation*}
% \begin{array}{llll}
% x
% &=& y				&\hspace{2em} \mathrel{\#} \text{comment or} \\
% &=& y				&\hspace{2em} \mathrel{\#} \text{comment} \\
% [10pt]
% \end{array}
% \end{equation*}
%
%
%
%	include an image
%
% \begin{figure}
% \center{\includegraphics[frame, scale=0.50] {images/X.png}}
% \caption{X}
% \label{fig:X}
% \end{figure}
%
%	How to put a box around a figure
%
% \begin{figure}[H]
%  \centering
%   \fbox{\begin{minipage}{34em}
%    \begin{equation*}
%     \begin{array}{rlrlrlr}
%       9^2      &=& 81           &\longrightarrow& 8 + 1           &=& 9	  \\
%       45^2     &=& 2025         &\longrightarrow& 20 + 25         &=& 45     \\
%       703^2    &=& 494209       &\longrightarrow& 494 + 209       &=& 703    \\
%       7777^2   &=& 60481729     &\longrightarrow& 6048 + 1729     &=& 7777   \\
%       857143^2 &=& 734694122449 &\longrightarrow& 734694 + 122449 &=& 857143 \\
%     \end{array}
%    \end{equation*}
%  \end{minipage}}
%  \caption{A Few Example Kaprekar Numbers}
%\end{figure}
%
%	
%	How to wrap (and resize) your tikzpicture in a figure
%
% \begin{figure}[H]
% \centering
%  \resizebox{0.40 \textwidth}{!} {		% resize the figure if you want
%    \begin{tikzpicture} 
%     .....
%    \end{tikzpicture}					% end tikzpicture
%  }									% end resizebox
% \caption{Some Figure}
% \label{fig:some_figure}
% \end{figure}
%
%	Build a figure that is resizable and framed with \fbox
% 
%                                                               % start building pascal's triangle
%       Parameters and settings                                 % start with the parameters and settings
%                                                               %
% \def \nrows {14}                                              % number of rows in the triangle
% \setlength {\fboxsep} {0.75em}								% get more space between fcolorbox and content
% \setlength {\fboxrule}{1.25pt}								% change fcolorbox line thickness
%                                                               %
%       Build the figure                                        % build the figure
%                                                               %
% \begin{figure}[H]                                             % begin the figure
%   \centering                                                  % center everything
%   \fcolorbox{black}{white} {									% put a color box around the figure
%     \resizebox{0.75 \textwidth}{!} {                          % resize the figure if you want
%     															%
%     <whatever>												% whatever goes in the figure
%																%
%      }                                                        % end resizebox
%   }                                                           % end fbox
%  \caption{X}                                     		   		% caption the figure
%   \label{fig:X}                                  			    % label the figure
% \end{figure}                                                  % end figure
% 
%
%
%
\section{Conclusions}
\label{sec:conclusions}
%
%
%
\section*{Acknowledgements}
\label{sec:acknowledgements}
Thanks to Dave Neary (@dneary@mathstodon.xyz), Grégoire 
Locqueville (@glocq@mathstodon.xyz) and Alex (@alexmath@tech.lgbt) 
for their insightful comments on the meaning of the symbol 1 in 
the Fourier series basis.
%
%	LaTeX source on overleaf.com
%
\section*{\LaTeX \hspace{0.025 mm} Source}
\url{https://www.overleaf.com/read/mtpfwbpmwcpg#b65e5d}
%
%	get a bibliography
%
%	Note:.bib files go in ~/Library/texmf/bibtex/bib with TeXShop (MacTeX).
%	You can also use an absolute path, e.g. \bibliography{/Users/dmm/papers/bib/qc}
%
\bibliographystyle{plain}
\bibliography{qc}
%
%
%
% \section*{Appendix A}
%
%	done
%
\end{document} 

