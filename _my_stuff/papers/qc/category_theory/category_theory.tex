\documentclass{article}
%
%
%	category_theory.tex
%
%	David Meyer
%	dmm613@gmail.com
%	26 Apr 2024
%
%
%   get various packages
%
\usepackage[margin=0.90in]{geometry}                                    % adjust margins
\geometry{letterpaper}                                                  % or a4paper or a5paper or ... 
\usepackage{url}                                                        % need this to use URLs in bibtex
\usepackage{setspace}                                                   % need this for \setstrech{...}
\usepackage{scrextend}                                                  % need this for addmargin
\usepackage[export]{adjustbox}                                          % need this to get frame for includegraphics
%
%   tikz et al
%
\usepackage{tikz}
\tikzset{>=latex}                                                       % default to LaTeX arrow head
\usetikzlibrary{calc,patterns,angles,quotes,shapes,math,decorations,
                through,intersections,lindenmayersystems,backgrounds,
                hobby,matrix}
\usepackage{tikz-cd}                                                    % commutative diagrams
\tikzcdset{every arrow/.append style={-latex}}                          % default to LaTeX arrow head in tikz-cd (and others)
\usepackage{circuitikz}                                                 % draw circuits    
\usepackage{pgfplots}

%
%	more math stuff
%
\usepackage{amsmath,amsfonts,amssymb,amsthm}
\usepackage{bm}
\usepackage{mathtools}
\usepackage{commath}                                                    % get \norm{x}
\usepackage{fixmath}                                                    % get \mathbold
\usepackage{gensymb}                                                    % get \degree
\usepackage{mathrsfs}
\usepackage{hyperref}
\usepackage{subcaption}
\usepackage{authblk}                                                    % comment out if using beamer (stops \author{} from working in beamer)
\usepackage{graphicx}
\usepackage{hyperref}
\usepackage{alltt}
\usepackage{xcolor}
\usepackage{colortbl}                                                   % \rowcolor{yellow!75} etc
\usepackage{float}
\usepackage{braket}
\usepackage{siunitx}
\usepackage{relsize}
\usepackage{multirow}
\usepackage{esvect}
\usepackage{enumitem}                                                   % use characters instead of numbers in enumerate
\usepackage{changepage}                                                 % needed for \begin{adjustwidth}{-3.25em}{-2.0em} (left justify)
\usepackage{bigints}
\usepackage{multirow}                                                   % \multirow and \multicolumn
\usepackage{enumitem}                                                   % change bullets in \begin{enumerate}
%
%	lualatex
%
%	Put this before \documentclass
%
%	!TEX TS-program = LuaLaTeX
%
%
% \usepackage{luatex85,luamplib}
% \mplibnumbersystem{double}
% \mplibtextextlabel{enable}
%
%
%
%	Describe floating point parameters, \fpeval
%
\ExplSyntaxOn
\cs_set_eq:NN \fpeval \fp_eval:n
\ExplSyntaxOff
%
%	Get the x and y components out of a coordinate, e.g.
%
%	\coordinate (EP) at (8,5);
%	\gettikzxy{(EP)}{\slopex}{\slopey}
%
\makeatletter
\newcommand{\gettikzxy}[3]{%
  \tikz@scan@one@point\pgfutil@firstofone#1\relax
  \edef#2{\the\pgf@x}%
  \edef#3{\the\pgf@y}%
}
\makeatother
%
%
%	Watermarks
%
% \usepackage{draftwatermark}
% \SetWatermarkText{Draft}
% \SetWatermarkScale{5}
% \SetWatermarkLightness {0.9} 
% \SetWatermarkColor[rgb]{0.7,0,0}
%
%
%	theorems, definitions, etc
%
\theoremstyle{definition}
\newtheorem{theorem}{Theorem}[section]
\newtheorem{definition}{Definition}[section]
\newtheorem{proposition}{Proposition}[section]
\newtheorem{lemma}{Lemma}[section]
\newtheorem{example}{Example}[section]
\newtheorem{remark}{Remark}[section]
\newtheorem{notation}{Notation}[section]

%
%	For drawing matrix products 
%
\newcommand*{\vertbar}{\rule[-1ex]{0.5pt}{2.5ex}}
\newcommand*{\horzbar}{\rule[.5ex]{2.5ex}{0.5pt}}

%
%	The following code allows you to do
%
%	\begin{bmatrix}[r] (or [c] or [l])
%
\makeatletter
\renewcommand*\env@matrix[1][c]{\hskip -\arraycolsep
  \let\@ifnextchar\new@ifnextchar
  \array{*\c@MaxMatrixCols #1}}
\makeatother
%
%	make \arg{min,max}_{n \to \infty} work nicely
%
\newcommand{\argmax}{\operatornamewithlimits{argmax}}
\newcommand{\argmin}{\operatornamewithlimits{argmin}}
%
%	handy commands
%
\newcommand*{\Scale}[2][4]{\scalebox{#1}{$#2$}}
\DeclareMathOperator{\E}{\mathbb{E}}
\DeclareMathOperator{\bda}{\Big \downarrow}									% big down arrow
\newcommand{\veq}{\mathrel{\rotatebox{90}{$=$}}}
%
%	Title, author and date
%
\title{A Few Notes on Category Theory}
\author{David Meyer \\ \href{mailto:dmm613@gmail.com}
                            {dmm613@gmail.com}}
\date{Last Update: \today \\
	 {\vspace{1.00mm} \small Initial Version: April 4, 2024}}
%
%
%
%
\begin{document}
\maketitle
%
%
%
\section{Introduction}
\label{section:introduction}
A category is a collection of objects together with a collection
of composable morphisms between these objects.  Category
theorists use the language "collection of objects" or "collection
of morphisms" to sidestep some set-theoretic issues, including:
if instead "the set of objects" was used then in the category of
sets the set of objects would we be the set of all sets, which we
know by Russell's Paradox does not exist
\cite{russells_paradox}. As a result the vague word "collection"
might be interpreted as meaning "class" (in the ZF set theory
sense \cite{zf_set_theory}).


\bigskip
\noindent
In any event, a more formal definition of a category is: 

\bigskip
\begin{definition} {\bf Category}: 
In category theory, a category is defined by a set of axioms that
describe its fundamental properties and behaviors. These are the
components of a category:

\begin{enumerate} [label=(\roman*).]

\item {\bf Objects}: A category consists of a collection of
objects. These objects can be anything: sets, groups, spaces,
etc. For example, in the category of sets, the objects are sets
themselves. If the collection of objects in a category is a set 
then the category is called a "small" category.

\noindent
The collection of objects in a category $\mathcal{C}$ is denoted
simply as $\mathcal{C}$, or sometimes
$\operatorname{Ob}(\mathcal{C})$.  Note that we refer to a
generic category as $\mathcal{C}$, while bold font is used to
indicate a specific category. For example, the category of sets
is called {\bf Set}.

\item {\bf Morphisms (Arrows)}: Between every two objects in a
category, there exists a collection of morphisms (also called
maps or arrows). Morphisms can represent various mathematical
structures like functions, homomorphisms, transformations, etc.

{\setstretch{1.25}
\noindent
The collection of morphisms from $A$ to $B$ in $\mathcal{C}$ is
denoted by $\mathcal{C}(A,B)$, or sometimes
$\operatorname{Hom}_{\mathcal{C}}(A,B)$. For $A,B \in
\mathcal{C}$, a morphism $f$ from $A$ to $B$ is denoted as $f \in
\mathcal{C}(A,B)$, $f: A \to B$, or sometimes $A \xrightarrow{f}
B$. The latter form is used in commutative diagrams; see Section
\ref{subsubsection:diagrams}. All of these forms represent a
morphism $f$ with domain $A$ (sometimes denoted at
$\operatorname{dom}(f)$) and codomain $B$ (sometimes denoted as
$\operatorname{codom}(f)$). \par}

\medskip
\noindent
If the collection of morphisms between every pair of objects in 
a category is a set then the category is called "locally small".
Said another way, if for all $A,B \in \mathcal{C}$ the collection
of morphisms $\mathcal{C}(A,B)$ is a set then $\mathcal{C}$ is
locally small.
 
\item {\bf Identity Morphisms}: For each object $A$ in the
category, there exists an identity morphism denoted as
$\text{id}_A$ or simply $1_A$. The identity morphism
$\text{id}_A: A \to A$ is the morphism that "does nothing" when
composed with other morphisms. It serves as an identity element
in the category.
 
\item {\bf Composition of Morphisms}: Morphisms in a category can
be composed. If $f: A \to B$ and $g: B \to C$ are morphisms in
the category, then there exists a composite morphism denoted as
$g \circ f: A \to C$, which represents the composition of $f$
followed by $g$. One consequence of this requirement: categories 
are closed under $\circ$.
 
\item {\bf Associativity of Composition}: Composition of
morphisms is associative. That is, if $f: A \to B$, $g: B \to C$,
and $h: C \to D$ are morphisms in the category, then $h \circ (g
\circ f) = (h \circ g) \circ f$.
\end{enumerate}

\medskip
\noindent
These axioms capture the essential structure of a category and
form the basis for studying relationships and structures across
various mathematical domains using category theory.

\label{definition:category}
\end{definition}

\subsection{Morphisms}
\label{subsection:morphisms}

\subsubsection{Commutative Diagrams}
\label{subsubsection:commutative_diagrams}
%
%	Draw the function composition triangle
%
\begin{center}
   \begin{tikzcd} [column sep={45pt},
                   row    sep={35pt}]
        A \arrow[r,"f"] \arrow[d, swap, "g \circ f"] & B \arrow[ld,"g"] \\
        C 
   \end{tikzcd}
\end{center}


\subsubsection{Aside: groups, monoids, and categories}
Here's a crazy (and beautiful) fact: A group is essentially the
same thing as a category that has only one object and in which
all the morphisms are isomorphisms.

\bigskip
\noindent
Ok, but why? 

\bigskip
\noindent
Well, first consider a category $\mathcal{C}$ with just one
object. Call that object $A$. That is, the class
$\operatorname{Ob}(\mathcal{C})$ contains only one object,
namely, $A$. Since $\mathcal{C}$ only has one object all of
$\mathcal{C}$'s morphisms are in
$\operatorname{Hom}_{\mathcal{C}} (A,A)$.


\bigskip
\noindent
Here the category $\mathcal{C}$ consists of

\begin{itemize}
  \item the class of objects $\mathcal{C}$ consisting of the single object $A$ 
  \item the class of morphisms $\mathcal{C}(A,A)$ consisting of isomorphisms $f: A \to A$
  \item an associative composition function $\circ:
        \mathcal{C}(A,A) \times \mathcal{C}(A,A) \to
        \mathcal{C}(A,A)$  
  \item a two-sided unit $1_{A}$
\end{itemize}

\medskip
\noindent
Interestingly, these four conditions would make
$\mathcal{C}(A,A)$ into a group except for inverses. However,
saying that every morphism in $\mathcal{C}$ is an isomorphism
implies that every element of $\mathcal{C}(A,A)$ has an inverse
(with respect to $\circ$). Hence $(\mathcal{C}(A,A), \circ)$ is a
group.

\bigskip
\noindent
So if $(G,\cdot)$ is the group $(\mathcal{C}(A,A),\circ)$ then we
have this correspondence:

\begin{center}
  \begin{table}[H]
    \begin{tabular}{l|r}
      \multicolumn{1}{c}{\centering {\bf Category}} & \multicolumn{1}{c}{\centering {\bf Group}} \\ 
      \hline\hline
      Category $\mathcal{C}$ with single object $A$ & Corresponding group $G$ \\
      Morphisms in $\mathcal{C}$                    & Elements of $G$ \\
      $\circ \in \mathcal{C}$                       & $\cdot \; \in G$ \\
      $1_{A} \in \mathcal{C}$                       & $1 \in G$ \\
      \hline \hline
    \end{tabular}
   \end{table}
\end{center}

\noindent
The diagram of $\mathcal{C}$ looks something like

\medskip
\begin{tikzcd}
    A \arrow[loop above, distance=1cm, in=120, out=60]{} 
      \arrow[loop left,  distance=1cm, in=150, out=210]{} 
      \arrow[loop below, distance=1cm, in=240, out=300]{}
      \arrow[loop right, distance=1cm, in=30,  out=330]{} 
\end{tikzcd}

\medskip
\noindent
where the arrows represent the different $A \to A$ morphisms in
$\mathcal{C}$. These $A \to A$ morphisms are the elements of the
group $G$.

\bigskip
\noindent
Summary: A group is a category with the special properties that 
all of the morphisms are invertible and there is only one object.

\bigskip
\noindent
By a similar argument, a category $\mathcal{C}$ with one object,
call it $A$, is the same thing as a monoid. This is because
unlike a group, every element of a monoid is not required to have
an inverse (and hence the morphisms in $\mathcal{C}(A,A)$ are not
required to be isomorphisms). Here we define the composition 
$a \circ b$ to be the product $ab$. Then for $\mathcal{C}$ to be 
a monoid all that is required is that there be an identity element 
and that the operation be associative. 

\subsubsection{Monomorphisms}
\label{subsubsection:monomorphisms}
Before looking at the definition of monomorphism, it is useful to
look at the definition of injectivity for sets:

\medskip
\begin{definition} {\bf Injectivity}: A function $f:B \to C$ is 
said to be injective if for all $b_{1}, b_{2} \in B$ we 
have 

\begin{equation*}
f(b_{1}) = f(b_{2}) \Rightarrow b_{1} = b_{2}
\end{equation*}

\label{definition:injectivity}
\end{definition}

\noindent
This standard definition of an injective mapping (aka 1:1
mapping) is not suitable for category theory because, among other
things, it looks inside the objects. This is an issue because
objects in category theory have "nothing inside them", that is,
they have no internal structure. In the case of the Definition
\ref{definition:injectivity}, $b_{1}$ and $b_{2}$ are "inside"
the object $B$.

\bigskip
\noindent
For functions between sets, injectivity is an important
concept. But for maps in an arbitrary category, injectivity might
not make sense. So how can we define a similar but more general
concept for categories using only morphisms? Consider the
following commutative diagram:

\smallskip
%
%	Use shift left and shift right to separate the 
%	two arrows between A and B
%
\begin{center}
  \begin{tikzcd}[column sep={30pt}]
     A \arrow[r,"g_{1}",shift left] \arrow[r,"g_{2}",swap,shift right] &
     B \arrow[r,"f"]                                                   & 
     C 
  \end{tikzcd}
\end{center}

\smallskip
\noindent
Since this diagram commutes we know that $f \circ g_{1} = f \circ
g_{2}$.  $f$ is called a monomorphism if it adheres to Definition
\ref{definition:monomorphism}.

\bigskip
\begin{definition} {\bf Monomorphism}: Consider a category
$\mathcal{C}$ with morphisms $g \in \mathcal{C}(A,B)$ and $f \in
\mathcal{C}(B,C)$. Then a morphism $f \in \mathcal{C}(B,C)$ is
called a monomorphism if for all $g_{1},g_{2} \in
\mathcal{C}(A,B)$ we have


\begin{equation}
f \circ g_{1} = f \circ g_{2} \Rightarrow g_{1} = g_{2} 
\label{equation:monomorphism}
\end{equation}

\medskip
\noindent
This property is also called left cancellation.
\label{definition:monomorphism}
\end{definition}


\medskip
\noindent
Although monomorphism is defined solely in terms of morphisms
(and not elements), since the above diagram commutes ($f \circ
g_{1} = f \circ g_{2}$) both $g_{1}$ and $g_{2}$ must map to
elements in $A$ to the same element in $B$. This suggests that
monomorphism is at least in part about retaining uniqueness (or
the lack thereof) in the domain in the codomain (the same is true
for injectivity).  We can see this since we know by Equation
(\ref{equation:monomorphism}) and the law of contraposition
\cite{wikipedia:contraposition} that
  
\begin{equation}
g_{1} \neq g_{2} \Rightarrow f \circ g_{1} \neq f \circ g_{2} 
\label{equation:contraposition}
\end{equation}

\bigskip
\noindent
One way to interpret Equation (\ref{equation:contraposition}) is
that different elements in the domain will be different in the
codomain. That is, the uniqueness of $g_{1}$ and $g_{2}$ is
preserved by $f$. The same kind of argument can be made for
injectivity.


\bigskip
\noindent
Finally, note that while injectivity seems to be the same thing
as monomorphism (and injective functions are monomorphisms in
{\bf Set}), monomorphism and injectivity are not necessarily the
same.  Said another way, monomorphism is the generalized-element
analogue of injectivity.

\subsubsection{Epimorphisms}
\label{subsubsection:epimorphisms}

\subsubsection{Isomorphisms}
\label{subsubsection:isomorphisms}

\subsubsection{Functors}
\label{subsubsection:functors}

\vspace{1.0em}
\begin{example}
Let $\mathcal{C}$ be the category {\bf Set} with $A,B,S \in 
\mathcal{C}$ and $f: A \to B$. Then there is a 
functor $- \times S$ defined as follows:


\bigskip
\begin{math}
   - \times S =
   \begin{cases} 
      A \times S                                        
      		\hspace{10.0em} \mathrel{\#} \text{takes $A$ to its cartesian product with $S$} \\
      f \times \text{id}_{S}: A \times S \to B \times S 
      		\hspace{1.55em} \mathrel{\#} \text{by $(a,s) \mapsto (f(a),s)$}
   \end{cases}
\end{math}
\label{example:functor}
\end{example}


\subsection{Natural Transformations}
\label{subsection:naturl_transformations}

\begin{definition} 
\label{definition:natural_transformation}
{\bf Natural Transformation}: Let $F,G: \mathcal{C} \to
\mathcal{D}$ be functors from a category $\mathcal{C}$ to 
a category $\mathcal{D}$. A natural transformation $\eta$ 
from $F$ to $G$ is a family of morphisms in $D$, one for 
each object $A$ in $\mathcal{C}$, such that

\begin{equation*}
\eta_{A}:F(A)\to G(A)
\end{equation*}
\end{definition}

\medskip
\noindent
The morphism $\eta_{A}$ is called the component of 
$\eta$ at object $A$. The "naturality condition"
is expressed as a commuting square. Specifically,
for any $f \in \mathcal{C}(A,B)$ the following diagram
commutes:
%
%
%
\medskip
%
%	Draw the picture
%
%
%          		 F(f)
%		   F(A) ------> F(B)
%	  		|            |
%	  η_{A} |            |  η_{B}
%	        v            v
%	       G(A) ------> G(B)
%                G(f)
%
%
\begin{center}
  \begin{tikzcd} [column sep={45pt},
                  row    sep={35pt}]
    F(A) \arrow[r,"F(f)"] \arrow[d, swap, "\eta_{A}"] & F(B) \arrow[d, "\eta_{B}"] \\
    G(A) \arrow[r, swap, "G(f)"]                      & G(B)
  \end{tikzcd}
\end{center}


\medskip
\noindent
Here $F(f)$ and $G(f)$ 
are the mappings $f$ under the functors $F$ and $G$ 
(respectively), where $F(f) \in \mathcal{D}(F(A),F(B))$ 
and $G(f) \in \mathcal{D}(G(A),G(B))$. $\eta_{A}$ and 
$\eta_{B}$ are natural transformations with $\eta_{A} 
\in \mathcal{D}(F(A),G(A))$ and $\eta_{B} \in 
\mathcal{D}(F(B),G(B))$.

\bigskip
\noindent
Commutativity here means that going from $F(A)$ to $G(B)$ via
$F(f)$ and $\eta_{B}$ is the same as going from $F(A)$ to $G(B)$
via $\eta_{A}$ and $G(f)$.  More specifically, commutativity
requires that for any $f \in \mathcal{C}(A,B)$ we have 

\begin{equation}
G(f) \circ \eta_{A} = \eta_{B} \circ F(f)
\label{equation:commutivity}
\end{equation}

\bigskip
\noindent
Note: I have seen Equation (\ref{equation:commutivity}) 
abbreviated as $G \eta = \eta F$.

\medskip
\noindent
\begin{example}
Consider the functor $- \times S$ defined in Example
\ref{example:functor}. For any function $g : S \to T$

\begin{equation*}
\text{id}\_ \times g : - \times S \to - \times T 
\end{equation*}

\smallskip
\noindent
is a natural transformation. To verify this, pick two objects 
$A,B$ and a morphism $f : A \to B$. Then we need to show 
that the following diagram commutes:

\medskip
%
%	Draw the picture
%
%
\begin{center}
  \begin{tikzcd} [column sep={45pt},
                  row    sep={40pt}]
    A \times S \arrow[r,"f \times \text{id}_{S}"] \arrow[d, swap, "\text{id}_{A} \times g"] & 
    		B \times S \arrow[d, "\text{id}_{B} \times g"] \\
    A \times T \arrow[r, swap, "f \times \text{id}_{T}"]                                    & 
    		B \times T
  \end{tikzcd}
\end{center}

\bigskip
\noindent
To show that this diagram commutes, pick a pair $(a,s)$ in the
top left corner.  Then if we go right then down we see that
$(a,s) \mapsto (f(a),s) \mapsto (f(a),g(s))$.  On the other hand,
if we go down then right we see that $(a,s) \mapsto (a, g(s))
\mapsto (f (a), g(s))$.  Since both paths between $A \times S$
and $B \times T$ map $(a,b)$ to $(f(a),g(s))$, the diagram
commutes and hence $\text{id}\_ \times g$ a natural
transformation \cite{intro_category_theory_notes}.
\end{example}

\begin{remark}
On interesting thing is that a natural transformation $\eta$ 
kind of commutes the two functors. That is, for functors 
$F,G: \mathcal{C} \to \mathcal{D}$, if $\eta : F \to G$ is a 
natural transformation then for any $f \in \mathcal{C}(A,B)$ 
we have $G(f) \circ \eta_{A} = \eta_{B} \circ F(f)$ 
(Equation (\ref{equation:commutivity})). If we drop some detail 
we see that $G \eta = \eta F$.
\label{remark:natural_transformation_commutes}
\end{remark}
%
%
%
\section{Conclusions}
\label{section:conclusions}
%
%
%
\section*{Acknowledgements}
\label{section:acknowledgements}
Thanks to John Carlos Baez (@johncarlosbaez@mathstodon.xyz) and
Andrew Stacey (@loopspace@mathstodon.xyz) for their helpful
comments on earlier versions of this document.
%
%	LaTeX source on overleaf.com
%
\section*{\LaTeX \hspace{0.025 mm} Source}
\url{https://www.overleaf.com/read/wnptmrwwfjgv#a36a79}
%
%	get a bibliography
%
%	Note:.bib files go in ~/Library/texmf/bibtex/bib with TeXShop (MacTeX).
%	You can also use an absolute path, e.g. \bibliography{/Users/dmm/papers/bib/qc}
%
\bibliographystyle{plain}
\bibliography{qc}
%
%
%
\section*{Appendix A: A Few Algebraic Structures}
\begin{center}
\begin{table}[H]
\scalebox{0.75}{
\begin{tabular}{l | c | c | c | c |  c | c}
{{\large \bf Structure}}        & $\textbf{\large ABO}^1$              & {\large \bf Identity}
                                & {\large \bf Inverse}                 & {\large \bf $\text{Distributive}^2$}
                                & {\large\bf $\text{Commutative}^3$}   & {\bf Comments} \\
\hline\hline					% separate column names from rows
Semigroup                       & \checkmark & no         & no         & N/A        & no    & $(S,\circ)$ \\
Monoid                          & \checkmark & \checkmark & no         & N/A        & no    & Semigroup plus identity $\in S$ \\
Group                           & \checkmark & \checkmark & \checkmark & N/A        & no    & Monoid plus inverse $\in S$ \\
Abelian Group                   & \checkmark & \checkmark & \checkmark & N/A        & \checkmark $(\circ)$ & Commutative group \\
$\text{Ring}_{+}$               & \checkmark & \checkmark & \checkmark & \checkmark & \checkmark $(+)$     & Abelian group under $+$ \\
$\text{Ring}_{*}$               & \checkmark & yes/no     & no         & \checkmark & no                   & Monoid under $*$ \\
$\text{Field}_{(+,*)}$          & \checkmark & \checkmark $(+,*)$      & \checkmark $(+,*)$ & \checkmark   & \checkmark $(+,*)$
                                             & Abelian group under $+$ and $*$  \\
Vector Space                    & \checkmark & \checkmark $(+,*)$      & \checkmark $(+)$   & \checkmark   & \checkmark $(+)$
                                             & Abelian group under $+$, scalars $\in$ Field \\
Module                          & \checkmark & \checkmark $(+,*)$	   & \checkmark $(+)$   & \checkmark   & \checkmark $(+)$
                                             & Abelian group under $+$, scalars $\in$ Ring

\end{tabular}}
\caption{A Few Algebraic Structures and Their Features}
\label{table:algebraic_structures}
\end{table}
\end{center}

\noindent
where

\begin{center}
\begin{enumerate}
\item \textbf{ABO:} Associative Binary Operation 
\begin{itemize}
\item $(x \circ y) \circ z = x \circ  (y \circ z)$  
      for all $x, y, z \in S$
\item $x \circ y \in S$ for all $x, y \in S$
      ($S$ is closed under $\circ$)
\end{itemize}

\item \textbf{Distributive:} Distributive Property 
\begin{itemize}
\item Left Distributive Property:  $x * (y+z )= (x*y) + (x*z)$ for
                                   all $x, y, z \in S$
\item Right Distributive Property: $(y + z) * x = (y*x) + (z*x)$
                                   for all $x, y, z \in S$
\item $*$ is \emph{distributive}   over $+$ if $*$ is left and 
                                   right distributive
\end{itemize}

\item \textbf{Commutative:} Commutative Property
\begin{itemize}
\item $x \circ y = y \circ x {\mbox{ for all }}x,y\in S$
\end{itemize}
\end{enumerate}
\end{center}
%
%	done
%
\end{document} 

