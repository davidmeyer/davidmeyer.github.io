\documentclass{article}
%
%
%	infinite_sum_a_over_b.tex
%
%	David Meyer
%	dmm613@gmail.com
%	31 Aug 2022
%
%
%   get various packages
%
\usepackage[margin=1.0in]{geometry}                                     % adjust margins
\geometry{letterpaper}                                                  % or a4paper or a5paper or ... 
\usepackage{url}                                                        % need this to use URLs in bibtex
\usepackage{setspace}                                                   % need this for \setstrech{...}
\usepackage{scrextend}                                                  % need this for addmargin
\usepackage[export]{adjustbox}                                          % need this to get frame for includegraphics
%
%   tikz et al
%
\usepackage{tikz}
\usetikzlibrary{calc,patterns,angles,quotes,shapes,math,decorations,
                through,intersections,lindenmayersystems,backgrounds}
\usepackage{circuitikz}                                                 % draw circuits    
\usepackage{pgfplots}
\usepackage{pgfplots}	
%
%	more math stuff
%
\usepackage{amsmath,amsfonts,amssymb,amsthm}
\usepackage{mathtools}
\usepackage{commath}                                                    % get \norm{x}
\usepackage{fixmath}                                                    % get \mathbold
\usepackage{gensymb}                                                    % get \degree
\usepackage{mathrsfs}
\usepackage{hyperref}
\usepackage{subcaption}
\usepackage{authblk}
\usepackage{graphicx}
\usepackage{hyperref}
\usepackage{alltt}
\usepackage{color}
\usepackage{float}
\usepackage{braket}
\usepackage{siunitx}
\usepackage{relsize}
\usepackage{multirow}
\usepackage{esvect}
%
%	watermarks
%
% \usepackage{draftwatermark}
% \SetWatermarkText{Draft}
% \SetWatermarkScale{5}
% \SetWatermarkLightness {0.9} 
% \SetWatermarkColor[rgb]{0.7,0,0}
%
%
%	theorems, definitions, etc
%
\theoremstyle{definition}
\newtheorem{theorem}{Theorem}[section]
\newtheorem{definition}{Definition}[section]
\newtheorem{proposition}{Proposition}[section]
\newtheorem{lemma}{Lemma}[section]
\newtheorem{example}{Example}[section]
\newtheorem{remark}{Remark}[section]
%
%	The following code allows you to do
%
%	\begin{bmatrix}[r] (or [c] or [l])
%
\makeatletter
\renewcommand*\env@matrix[1][c]{\hskip -\arraycolsep
  \let\@ifnextchar\new@ifnextchar
  \array{*\c@MaxMatrixCols #1}}
\makeatother
%
%	make \arg{min,max}_{n \to \infty} work nicely
%
\newcommand{\argmax}{\operatornamewithlimits{argmax}}
\newcommand{\argmin}{\operatornamewithlimits{argmin}}
%
%	handy commands
%
\newcommand*{\Scale}[2][4]{\scalebox{#1}{$#2$}}
\DeclareMathOperator{\E}{\mathbb{E}}
\DeclareMathOperator{\bda}{\Big \downarrow}						% big down arrow
\newcommand{\veq}{\mathrel{\rotatebox{90}{$=$}}}

%
%	Title, author and date
%
\title{What does the series $S = \sum\limits_{n = 1}^\infty {\left (\frac{a}{b} \right )}^{n}$ 
       converge to?}
\author{David Meyer \\ \href{mailto:dmm613@gmail.com}
                            {dmm613@gmail.com}}
\date{Last update: \today}
%
%
%
\begin{document}
\maketitle
%
%
%
\bigskip
\section{First up: does this series converge?}
Here we'll use the ratio test for convergence \cite{wiki:ratio_test}
and so we want to think of $S$ as

\bigskip
\begin{equation}
S = \sum\limits_{n = 1}^{\infty} a_n
\label{eqn:S}
\end{equation}

\medskip
\noindent
where $a_n = {\left ( \frac{a}{b} \right )}^n.$

\bigskip
\noindent
The usual form of the ratio test uses the limit 
$L = \lim\limits_{n\to \infty}\left|{\frac {a_{{n+1}}}{a_{n}}}\right|$.
The ratio test tells us that 

\medskip
\begin{enumerate}
\item If $L < 1$ then the series converges absolutely.
\label{enumerate:less_than_one}
\item If $L > 1$ then the series diverges.
\item If $L = 1$ (or the limit doesn't exist) then the test is 
      inconclusive.
\end{enumerate}

\bigskip
\noindent
To apply the ratio test we want to compute the following
limit:

\bigskip
\begin{equation*}
L = \lim _{n\to \infty}\left| \dfrac{\hphantom{^15}						% \hpantom is trying to center the numerator and denominator
								 \left ( \frac{a}{b}\right )^{n+1}} 
                                {\left ( \frac{a}{b} \right )^{n}} \right|
\end{equation*}

\bigskip
\noindent
Since $\lim\limits_{n\to \infty} c = c$ for constant $c$ and since 
$\dfrac{a}{b}$ is a constant with respect to $n$ we see that the limit 
$L$ is

\bigskip
\begin{equation*}
L = \lim _{n\to \infty}\left| \dfrac{\hphantom{^15}						% \hpantom is trying to center the numerator and denominator
								\left ( \frac{a}{b}\right )^{n+1}} 
                                {\left ( \frac{a}{b} \right )^{n}} \right|
                                = \lim _{n\to \infty}\left| \dfrac{a}{b} \right |
                                = \dfrac{a}{b}
\end{equation*}

\bigskip
\noindent
If $a < b$ then $\dfrac{a}{b} < 1$ and so by clause \ref{enumerate:less_than_one}
of the ratio test we know that $S$ converges absolutely.


\section{Ok, $S$ converges. What does it converge to?}
\label{section:what_does_S_converge_to}

\bigskip
\noindent
Since we know that $S$ converges absolutely when $a < b$,
here is one way to think about the question:  
      
\begin{equation*}
\begin{array}{lllll}
S
&=& \sum\limits_{n = 1}^\infty {\left (\frac{a}{b} \right )}^n
			&\hspace{5em} \mathrel{\#} \text{definiton of $S$ (Equation (\ref{eqn:S}))} \\  
[15pt]
&=& {\left (\frac{a}{b} \right )}^1
        + {\left (\frac{a}{b} \right )}^2 + {\left (\frac{a}{b} \right )}^3 + \cdots
			&\hspace{5em} \mathrel{\#} \text{expand $S$} \\  
[15pt]
&\Rightarrow& \big (\frac{b}{a} \big ) \cdot S = \big (\frac{b}{a} \big ) \cdot \left
        [{\left (\frac{a}{b} \right )}^1 + {\left (\frac{a}{b} \right )}^2 + 
        {\left (\frac{a}{b} \right )}^3 + \; \cdots \right ]
                &\hspace{5em} \mathrel{\#} \text{multiply both sides by 
                                       $\big (\frac{b}{a} \big )$} \\ 
[15pt]
&\Rightarrow& \big (\frac{b}{a} \big ) \cdot S = 1 + \left [{\left (\frac{a}{b} \right )}^1
        + {\left (\frac{a}{b} \right )}^2 + {\left (\frac{a}{b} \right )}^3 + \; \cdots \right ]
                &\hspace{5em}  \mathrel{\#} \text{multiply through on right side} \\
[15pt]
&\Rightarrow&  \big (\frac{b}{a} \big ) \cdot S = 1 + S
                &\hspace{5em}  \mathrel{\#} \text{definition of $S$} \\
[15pt]
&\Rightarrow&  \big (\frac{b}{a} \big ) \cdot S - S = 1
                &\hspace{5em}  \mathrel{\#} \text{subtract $S$ from both sides} \\
[15pt]
&\Rightarrow&  \big (\frac{b}{a} \big ) \cdot S - \left (\frac{a}{a} \right ) \cdot S = 1
                &\hspace{5em}  \mathrel{\#} \text{multiply $S$ by $1 = \frac{a}{a}$} \\
[15pt]
&\Rightarrow&  S \cdot \left [\frac{b}{a}  - \frac{a}{a} \right ] = 1
                &\hspace{5em}  \mathrel{\#} \text{factor out $S$} \\
[15pt]
&\Rightarrow& S \cdot \left [ \frac{b \:-\: a} {a} \right ] = 1
                &\hspace{5em}  \mathrel{\#} \text{simplify} \\ 
[15pt] 
&\Rightarrow& S  = \frac{a}{b \:-\: a}
                &\hspace{5em}  \mathrel{\#} \text{multiply both sides
                 by $ \frac{a}{b \:-\: a} $} 
\end{array}
\end{equation*}

\bigskip
\noindent
So $S = \sum\limits_{n = 1}^\infty {\left (\frac{a}{b} \right
)}^n = \frac{a}{b \:-\: a}$, where $a,b \in \mathbb{N}$ and $a <
b$.

\bigskip
\noindent
For example, if we let $a =1$ and $b =2$ then $\sum\limits_{n =
1}^\infty {\left (\frac{1}{2} \right )}^n = \frac{1}{2 \:-\: 1} =
1$. Similarly, if $a =1$ and $b = 3$ then $\sum\limits_{n =
1}^\infty {\left (\frac{1}{3} \right )}^n = \frac{1}{3 \:-\: 1} =
\frac{1}{2}$.


\section{$S$ is a Geometric Series}
As pointed out by John Carlos Baez
(@johncarlosbaez@mathstodon.xyz), $S$ does not depend on $a$ or
$b$, but rather only on $x$.

\bigskip
\noindent
More specifically, we can see that $S = \sum\limits_{n=1}^{\infty} x^n$
is a geometric series with a first term of $x$ and a 
common ratio of $x$. The general form of a geometric series is given by
\cite{wiki:geometric_series}:

\begin{equation*}
S = a + ar + ar^2 + ar^3 + \ldots 
\end{equation*}

\medskip
\noindent
In this case, $a = x$ is the first term of the series, and $r =
x$ is the common ratio.

\bigskip
\noindent
The sum of an infinite geometric series is well known and can be
calculated using the formula:

\begin{equation}
S = \frac{a}{1 - r}
\label{equation:geometric_series_closed_form}
\end{equation}

\smallskip
\noindent
for $|r| < 1$. If we then substitute $a = x$ and $r = x$ into Equation 
(\ref{equation:geometric_series_closed_form}) we get:

\begin{equation}
S = \frac{x}{1 - x}
\label{equation:s}
\end{equation}

\medskip
\noindent
So we can see that for $S$ to converge, we need $|x| <
1$. If $|x| \geq 1$ the series diverges and does not have a
finite sum.

\bigskip
\noindent
Summary: $S = \sum\limits_{n=1}^{\infty} x^n = \frac{x}{1 - x}$
for $|x| < 1$. Otherwise, as we saw above, the series does not
converge.

\bigskip
\noindent
Finally, if $x = \frac{a}{b}$ then for $|x| < 1$

\begin{equation*}
\begin{array}{lllll}
S 
&=& \dfrac{x}{1 - x} 						&\hspace{5em} \mathrel{\#} \text{Equation (\ref{equation:s})} \\ 
[12pt]
&=& \dfrac{\frac{a}{b}}{1 - \frac{a}{b}}	&\hspace{5em} \mathrel{\#} \text{set $x = \frac{a}{b}$} \\ 
[12pt]
&=& \dfrac{\frac{a}{b}}{\frac{b -a}{b}} 	&\hspace{5em} \mathrel{\#} \text{get a common denominator} \\ 
[12pt]
&=& \dfrac{a}{b-a} 							&\hspace{5em} \mathrel{\#} \text{multiply by $1 = \dfrac{\frac{b}{1}}{\frac{b}{1}}$} \\ 
\end{array}
\end{equation*}

\medskip
\noindent
So we see that if we set $x = \frac{a}{b}$ then $S = \frac{a}{b-a}$ when $a < b$. 
This is the result that we saw in Section \ref{section:what_does_S_converge_to}.


%
%
% \section{Conclusions}
%
%
\section*{Acknowledgements}
Thanks again to John Carlos Baez (@johncarlosbaez@mathstodon.xyz)
for pointing out that the sum really doesn't depend on the
fraction $\frac{a}{b}$ but rather depends on the value of the
variable $x$.
%
%	LaTeX source on overleaf.com
%
\section*{\LaTeX \hspace{0.10 mm} Source}
% \url{}
%
%	get a bibliography
%
%	Note:.bib files go in ~/Library/texmf/bibtex/bib with TeXShop (MacTeX).
%	You can also use an absolute path, e.g. \bibliography{/Users/dmm/papers/bib/qc}
%
\bibliographystyle{plain}
\bibliography{qc}
%
%	done
%
\end{document} 

