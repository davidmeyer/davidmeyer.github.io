\documentclass[11pt, oneside]{article}   	% use "amsart" instead of "article" for AMSLaTeX format


% \usepackage{draftwatermark}
% \SetWatermarkText{Draft}
 % \SetWatermarkScale{5}
% \SetWatermarkLightness {0.85} 
% \SetWatermarkColor[rgb]{0.7,0,0}


\usepackage{geometry}                		% See geometry.pdf to learn the layout options. There are lots.
\geometry{letterpaper}                   		% ... or a4paper or a5paper or ... 
%\geometry{landscape}                		% Activate for for rotated page geometry
%\usepackage[parfill]{parskip}    		% Activate to begin paragraphs with an empty line rather than an indent
\usepackage{graphicx}				% Use pdf, png, jpg, or eps� with pdflatex; use eps in DVI mode
								% TeX will automatically convert eps --> pdf in pdflatex		
\usepackage{amssymb}
\usepackage{mathrsfs}
\usepackage{hyperref}
\usepackage{url}
\usepackage{authblk}
\usepackage{amsmath}
\usepackage{graphicx}
\usepackage{fixltx2e}
\usepackage{hyperref}
\usepackage{alltt}
\usepackage{color}
\usepackage{bigints}

\newcommand{\argmax}{\operatornamewithlimits{argmax}}
\newcommand{\argmin}{\operatornamewithlimits{argmin}}

\title{Notes on NEURAL ARCHITECTURE SEARCH WITH REINFORCEMENT LEARNING \\
Barret Zoph, Quoc V. Le}
\author{David Meyer \\ dmm@\{1-4-5.net,uoregon.edu,brocade.com,...\}}
\begin{document}
\maketitle
%\section{}
%\subsection{}

Notes
\begin{itemize}
\item $\pi_\theta(s,a) = P[A_t = a | S_t = s, \theta] = P_{(a_1{:T};\theta_c)}$ (policy parameterized by $\theta$)
\item $J(\theta_c) = \mathbb{E}_{P(a_1:T;  \theta_c)} [R]$  (hook the RNN loss function to the RL reward)
\item $J_1(\theta) = V^{\pi_\theta}(s_1) = \mathbb{E}_{\pi_\theta}[v_1] = \mathbb{E}_{P(a_1:T;  \theta_c)}[R]$ (episodic environments)
\item $\nabla_{\theta_{c}} J(\theta_c) = \sum\limits_{t = 1}^{T} E_{P(a_1:T;\theta_c)}
\big[ \nabla_{\theta_c} \log P(a_t | a_{(t-1):1};\theta_c)R \big] $ (REINFORCE policy gradient)
\item $a_t$ is the predicted action ($a$) and $a_{(t-1):1}$ is the state $s$ up to step $t-1$ encoded in the RNN

\end{itemize}

\section{Computing the gradient analytically}

First, we assume that the policy $\pi_\theta$ is differentiable wherever it is non-zero (this is a softer requirement than requiring $\pi_\theta$ be differentiable \emph{everywhere}). In addition, we know the gradient: $\nabla_\theta J(\theta)$. In this case, let  $p(\mathbf{x};\theta)$ be the likelihood parametrized by $\theta$ and let $\log p(\mathbf{x}; \theta)$ be the \emph{log likelihood}. Then

\begin{flalign}
y & = p(\mathbf{x}; \theta) \qquad \qquad \qquad \; \;  \; \; \; \mathbin{\#} \text{definition; see above} \\
z &= \log y = \log p(\mathbf{x}; \theta)  \qquad  \; \; \mathbin{\#} \text{definition; z is the  log likelihood} \\
\frac{dz}{d \theta} &= \frac{dz}{dy} \cdot \frac{dy}{d\theta}  \qquad \qquad \quad \; \; \; \; \; \; \; \mathbin{\#} \text{chain rule definition} \\
\frac{dz}{dy} &=  \frac{1}{p(\mathbf{x};\theta)}  \quad \qquad \qquad \qquad  \;  \mathbin{\#} \frac{\log (X)}{dX} \approx \frac{1}{X} \\
 \frac{dy}{d\theta} &= \frac{d \; p(\mathbf{x};\theta)}{d\theta} = \nabla_\theta p(\mathbf{x};\theta) \quad   \mathbin{\#} \text{definition (chain rule, again)}\\
\frac{dz}{d \theta} & = \frac{dz}{dy} \cdot \frac{dy}{d\theta} = \frac{\nabla_\theta \; p(\mathbf{x};\theta)}{p(\mathbf{x};\theta)} 
\quad   \mathbin{\#} \text{chain rule}\\
&= \nabla_\theta \log p(\mathbf{x};\theta) \qquad \qquad \; \; \mathbin{\#} \text{using the identity }\nabla_{\theta} \log (w) = \frac{1}{w} \nabla_\theta w  
\end{flalign}

\bigskip
\noindent
and setting $w = p(\mathbf{x};\theta)$. Here $\nabla_\theta \log p(\mathbf{x};\theta)$ is known as the score or sometimes the \emph{Fischer} information. So the \emph{log derivative trick} (sometimes \emph{likelihood ratio}) is 
\begin{align*}
\nabla_\theta \log p(\mathbf{x};\theta)   = \frac{\nabla_\theta \; p(\mathbf{x};\theta)}{p(\mathbf{x};\theta)}
\end{align*}

\bigskip
\noindent
Setting  $\pi_\theta(s,a) = p(\mathbf{x};\theta)$ we see that

\begin{align}
\label{eqn:ld_identity}
\nabla_\theta \pi_\theta(s,a) &= \pi_\theta(s,a) \frac{\nabla_\theta \; \pi_\theta(s,a)}{\pi_\theta(s,a)} \\
                                             &= \pi_\theta(s,a)  \nabla_\theta \log \pi_\theta(s,a)    \qquad \qquad \quad \; \; \; \mathbin{\#} 
\label{eqn:ld_identity}
\text{log derivative trick}                 
\end{align}

\bigskip
\noindent
and the score function is $\nabla_\theta \log \pi_\theta(s,a)$. 

\bigskip
\noindent
Now, since here $\pi_\theta(s,a) = P(a_{1:T};\theta_c)$, we have  
 
 \bigskip
 \noindent
 
\begin{align}
\nabla_{\theta_{c}} J(\theta_c) 
&= \sum\limits_{s \in S} d(s) \sum\limits_{a \in A} \nabla_{\theta_c}\pi_{\theta_c}(s,a) \mathcal{R}_{s,a}
\qquad \qquad \quad \; \mathbin{\#}\text{defn policy gradient} \\ 
&=  \sum\limits_{s \in S} d(s) \sum\limits_{a \in A} \pi_{\theta_c}(s,a) \nabla \log \pi_{\theta_c}(s,a) R_{s,a} 
\quad \mathbin{\#} \text{log derivative trick (Eqn \ref{eqn:ld_identity})
}  \\      
& =  \mathbb{E}_{\pi_{\theta_c}} \big [ \nabla_{\theta_c} \log \pi_{\theta_c}(s,a)R\big]
\qquad \qquad \qquad \quad \;  \mathbin{\#} \text{defn expectation}  \\
&= \mathbb{E}_{a_i \sim  P} \big [\nabla_{\theta_c}  \log P(a_t | a_{(t-1):1};\theta_c) R \big] \qquad \quad  \mathbin{\#} \pi_\theta(s,a) = P(a_{1:T};\theta_c) \\     
& = \sum\limits_{t = 1}^{T} P_{(a_1:T;\theta_c)} \big [ \nabla_{\theta_c} \log P(a_t | a_{(t-1):1};\theta_c) R \big]  
\;  \mathbin{\#} \text{REINFORCE pg}
\end{align}






\end{document}  