\documentclass[11pt, oneside]{article}   	% use "amsart" instead of "article" for AMSLaTeX format


%\usepackage{draftwatermark}
% \SetWatermarkText{Confidential}
% \SetWatermarkScale{5}
% \SetWatermarkLightness {0.85} 
% \SetWatermarkColor[rgb]{0.7,0,0}


\usepackage{geometry}                		% See geometry.pdf to learn the layout options. There are lots.
\geometry{letterpaper}                   		% ... or a4paper or a5paper or ... 
%\geometry{landscape}                		% Activate for for rotated page geometry
%\usepackage[parfill]{parskip}    		% Activate to begin paragraphs with an empty line rather than an indent
\usepackage{graphicx}				% Use pdf, png, jpg, or eps� with pdflatex; use eps in DVI mode
								% TeX will automatically convert eps --> pdf in pdflatex		
\usepackage{amssymb}
\usepackage{mathrsfs}
\usepackage{hyperref}
\usepackage{url}
\usepackage{subcaption}
\usepackage{authblk}
\usepackage{amsmath}
\usepackage{mathtools}
\usepackage{graphicx}
\usepackage{fixltx2e}
\usepackage{hyperref}
\usepackage{alltt}
\usepackage{color}
\usepackage[utf8]{inputenc}
\usepackage[english]{babel}
 
\newtheorem{theorem}{Theorem}[section]
\newtheorem{corollary}{Corollary}[theorem]
\newtheorem{lemma}[theorem]{Lemma}

\newcommand{\argmax}{\operatornamewithlimits{argmax}}
\newcommand{\argmin}{\operatornamewithlimits{argmin}}



\title{A Few Notes on Trust Region Policy Optimization}
\author{David Meyer \\ dmm@\{1-4-5.net,uoregon.edu,...\}}

\date{Last update: \today}							% Activate to display a given date or no date


\begin{document}
\maketitle

\section{Introduction} 
\label{sec:intro}

One of the goals of this note is to prove Theorem 1 of Schulman, J. et al., "Trust Region Policy Optimization \cite{DBLP:journals/corr/SchulmanLMJA15}.  This is the proof of the \emph{Policy Improvement Bound} described in \cite{DBLP:journals/corr/SchulmanLMJA15}, The proof begins with a  lemma from Kakade \& Langford \cite{Kakade+Langford:2002} that shows that the difference in policy performance $\eta(\tilde{\pi}) - \eta(\pi)$ can be decomposed as a sum of per-timestep \emph{advantages}.

\bigskip
\noindent
First, define $\eta(\pi)$ as follows. Let $\pi$ denote a stochastic policy $\pi :  \mathcal{S} \times \mathcal{A} \rightarrow  [0,1]$, and let $\eta(\pi)$ be the expected 
discounted reward under $\pi$. Then define $\eta(\pi)$:
\begin{flalign*}
\eta(\pi) &= \mathbb{E}_{s_0, a_0, s_1, a_1, \cdots} \Bigg [\sum\limits_{t = 0}^\infty \gamma^t r(s_t) \Bigg ] \text{ where} \\
& s_0 \sim \rho(s_0), a_t \sim \pi(a_t \mid s_t), s_{t + 1} \sim P(s_{t + 1} \mid s_t, a_t)
\end{flalign*}


\begin{lemma}
\label{lemma:advantages}
Given two policies $\pi$ and $\tilde{\pi}$

\begin{flalign*}
\eta(\tilde{\pi})= \eta(\pi) +  \mathbb{E}_{\tau \sim \tilde{\pi}} \Bigg [\sum\limits_{t = 0}^{\infty} \gamma^t A_{\pi} (s_t,a_t) \Bigg ]
\end{flalign*}
\end{lemma}

\bigskip
\noindent 
where the expectation is taken over trajectories $\tau \coloneqq (s_0,a_0,s_1,a_1,\cdots)$ and the notation $\mathbb{E}_{\tau \sim \tilde{\pi}} \big [\cdots\big]$ means that actions are sampled from $\tilde{\pi}$ to generate $\tau$. 

\bigskip
\noindent
\emph{Proof}. First, recall that the advantage $A_{\pi}(s,a)$ of an action $a$ in state $s$ is defined as follows:

\begin{flalign*}
A_{\pi} (s,a) =  \mathbb{E}_{s^\prime \sim P(s^\prime \mid s,a)}  \big [r(s) + \gamma V_\pi(s^\prime) - V_\pi(s) \big ]
\end{flalign*}

\bigskip
\noindent 
From here we can just work out the result:
\begin{flalign*}
% \label{eqn:defnA:1}
\mathbb{E}_{\tau \mid \tilde{\pi}}  \Bigg [\sum\limits_{t = 0}^{\infty} \gamma^t  A_{\pi}(s_t,a_t) \Bigg ] 
&= \mathbb{E}_{\tau \mid \tilde{\pi}}  \Bigg [\sum\limits_{t = 0}^{\infty} \Big [ \gamma^t (r(s_t) +  \gamma V_{\pi}(s_{t+1})  -  V_{\pi}(s_t)) \Big ] \Bigg ] \\
&= \mathbb{E}_{\tau \mid \tilde{\pi}}  \Bigg [\sum\limits_{t = 0}^{\infty} \Big [\gamma^t r(s_t) + 
\gamma^{t + 1} V_{\pi}(s_{t+1})  -  \gamma^t V_{\pi}(s_t) \Big ] \Bigg ] \\
% \label{eqn:defnA:2}
&= \mathbb{E}_{\tau \mid \tilde{\pi}}  \Bigg [\sum\limits_{t = 0}^{\infty} \gamma^t r(s_t) + \sum\limits_{t = 0}^{\infty} \gamma^{t + 1} V_{\pi}(s_{t+1})  
-  \sum\limits_{t = 0}^{\infty} \gamma^t V_{\pi}(s_t) \Bigg ] \\
&= \mathbb{E}_{\tau \mid \tilde{\pi}}  \Bigg [\sum\limits_{t = 0}^{\infty} \gamma^t r(s_t) + \sum\limits_{t = 1}^{\infty} \gamma^{t} V_{\pi}(s_t) 
-  \sum\limits_{t = 0}^{\infty} \gamma^t V_{\pi}(s_t) \Bigg ] \\
&= \mathbb{E}_{\tau \mid \tilde{\pi}}  \Bigg [\sum\limits_{t = 0}^{\infty} \gamma^t r(s_t)  + \sum\limits_{t = 1}^{\infty} \gamma^{t} V_{\pi}(s_t) 
-  \Bigg (V_{\pi}(s_0)  +\sum\limits_{t = 1}^{\infty} \gamma^t V_{\pi}(s_t) \Bigg)\Bigg ] \\
&= \mathbb{E}_{\tau \mid \tilde{\pi}}  \Bigg [\sum\limits_{t = 0}^{\infty} \gamma^t r(s_t) + \Bigg (\sum\limits_{t = 1}^{\infty} \gamma^{t} V_{\pi}(s_t) 
- \sum\limits_{t = 1}^{\infty} \gamma^{t} V_{\pi}(s_t)  \Bigg) -  V_{\pi}(s_0) \Bigg ] \\
&= \mathbb{E}_{\tau \mid \tilde{\pi}} \Bigg [ \sum\limits_{t = 0}^{\infty} \gamma^t r(s_t) -V_{\pi}(s_0) \Bigg ]  \\
&= -\mathbb{E}_{s_0} \Bigg [ V_{\pi}(s_0) + \mathbb{E}_{\tau \mid \tilde{\pi}}  \bigg [\sum\limits_{t = 0}^{\infty} \gamma^t r(s_t) \bigg ] \Bigg ] \\
&= -\eta(\pi)+ \eta(\tilde{\pi}) \qquad \qquad \qquad \qquad  \quad \qquad \mathrel{\#} \text{Definition $\eta(\pi)$}
\end{flalign*}

\subsection{A Bit of Intuition}
Another way to see this result: we can see that $\gamma V_{\pi}(s_{t+1})  -  V_{\pi}(s_t) = -V_{\pi}(s_0)$ by expanding the first few terms:

\bigskip
\begin{center}
\captionof{table}{Expansion of terms} \label{table:expansion}
\begin{tabular} {|| c | l ||}
 \hline \hline
 t  & $\gamma^t (r(s_t) + \gamma V_{\pi}(s_{t+1})  -  V_{\pi}(s_t))$  \\ 
 \hline \hline
 0 & $r(s_0) + {\color{red} \gamma V_{\pi}(s_1)}  - V_{\pi}(s_0)$  \\
 \hline
 1 & $ \gamma r(s_1) + {\color{red} \gamma^2 V_{\pi}(s_2)}  - {\color{blue}\gamma V_{\pi}(s_1)}$ \\
 \hline
 2 & $ \gamma^2 r(s_2) + {\color{red} \gamma^3 V_{\pi}(s_3)}  - {\color{blue}\gamma^2 V_{\pi}(s_2)}$ \\
\hline
 3 & $ \gamma^3 r(s_3) + {\color{red} \gamma^4 V_{\pi}(s_4)}  - {\color{blue}\gamma^3 V_{\pi}(s_3)}$ \\
\hline
\vdots &  \multicolumn{1}{c||}{\vdots} \\
\hline
$\infty$ & \multicolumn{1}{c||}{$\sum\limits_{t = 0}^{\infty} \gamma^t r(s_t)  -  V_{\pi}(s_0)$} \\
 \hline
 \hline
\end{tabular}
\end{center}

\bigskip
\noindent
Notice that in Table \ref{table:expansion} the {\color{red} red} term at time {\color{red} $t$} minus the {\color{blue} blue} term at time {\color{blue} $t+1$} equals zero. 
For example, at times {\color{red} $t = 0$} and {\color{blue}  $t = 1$}  we have  ${\color{red} \gamma V_{\pi}(s_1)} - {\color{blue}\gamma V_{\pi}(s_1)} = 0$.  More 
generally, at time {\color{red} $t$}  we have the term  ${\color{red} \gamma^t V_{\pi}(s_t)}$ and at time {\color{blue} $t+1$} we have the term  
${\color{blue}\gamma^t V_{\pi}(s_t)}$ whose difference is again zero.  So in the limit we are left with $\sum\limits_{t = 0}^{\infty} \gamma^t r(s_t) -  V_{\pi}(s_0)$.

\bigskip
\noindent
Armed with this result we can now define the expected advantage of $\tilde{\pi}$ over $\pi$ at state $s$ as 
$\bar{A}(s) = \mathbb{E}_{a \sim \tilde{\pi} (\cdot \mid s)} \Big [A_{\pi}(s,a) \Big]$. Now Lemma \ref{lemma:advantages} can be written as follows:



\begin{flalign*}
\eta(\tilde{\pi})= \eta(\pi) +  \mathbb{E}_{\tau \sim \tilde{\pi}} \Bigg [\sum\limits_{t = 0}^{\infty} \gamma^t \bar{A}_{\pi} (s_t,a_t) \Bigg ]
\end{flalign*}

\bigskip
\noindent
and that $L_\pi(\tilde{\pi})$ can be written as 
\begin{flalign*}
L_{\pi}(\tilde{\pi})= \eta(\pi) +  \mathbb{E}_{\tau \sim \tilde{\pi}} \Bigg [\sum\limits_{t = 0}^{\infty} \gamma^t \bar{A}_{\pi} (s_t,a_t) \Bigg ]
\end{flalign*}
\bigskip
\noindent
The difference between $\eta(\tilde{\pi})$ and $L_{\pi}(\tilde{\pi})$ is whether the states are sampled using $\pi$ or $\tilde{\pi}$. To bound the difference between $\eta(\tilde{\pi})$ and $L_{\pi}(\tilde{\pi})$ we need to bound the difference arising at each timestep. To do this, we first need to introduce a measure of how much $\pi$ and $\tilde{\pi}$ agree. 
The approach taken in this paper is to \emph{couple} the policies so that they define a joint distribution over pairs of actions.

\newpage
\bibliographystyle{ieeetr}
\bibliography{/Users/dmm/papers/bib/ml}



\end{document} 
