\documentclass{article}
%
%
%	pt.tex -- draw Pascal's triangle
%
%	David Meyer
%	dmm613@gmail.com
%	08 Oct 2023
%
%
%   get various packages
%
\usepackage[margin=1.0in]{geometry}                                     % adjust margins
\geometry{letterpaper}                                                  % or a4paper or a5paper or ... 
\usepackage{url}                                                        % need this to use URLs in bibtex
\usepackage{setspace}                                                   % need this for \setstrech{...}
\usepackage{scrextend}                                                  % need this for addmargin
\usepackage[export]{adjustbox}                                          % need this to get frame for includegraphics
%
%   tikz et al
%
\usepackage{tikz}
\usetikzlibrary{calc,patterns,angles,quotes,shapes,math,decorations,
                through,intersections,lindenmayersystems,backgrounds,
                hobby}
\tikzset{>=latex}                                                       % default to LaTeX arrow head
\usepackage{circuitikz}                                                 % draw circuits    
\usepackage{pgfplots}
%
%	more math stuff
%
\usepackage{amsmath,amsfonts,amssymb,amsthm}
\usepackage{bm}
\usepackage{mathtools}
\usepackage{commath}                                                    % get \norm{x}
\usepackage{fixmath}                                                    % get \mathbold
\usepackage{gensymb}                                                    % get \degree
\usepackage{mathrsfs}
\usepackage{hyperref}
\usepackage{subcaption}
\usepackage{authblk}                                                    % comment out if using beamer (stops \author{} from working in beamer)
\usepackage{graphicx}
\usepackage{hyperref}
\usepackage{alltt}
\usepackage{xcolor}
\usepackage{float}
\usepackage{braket}
\usepackage{siunitx}
\usepackage{relsize}
\usepackage{multirow}
\usepackage{esvect}
\usepackage{enumitem}                                                   % use characters instead of numbers in enumerate
\usepackage{changepage}                                                 % needed for \begin{adjustwidth}{-3.25em}{-2.0em} (left justify)
%
%	Describe floating point parameters, \fpeval
%
\ExplSyntaxOn
\cs_set_eq:NN \fpeval \fp_eval:n
\ExplSyntaxOff
%
%	Get the x and y components out of a coordinate, e.g.
%
%	\coordinate (EP) at (8,5);
%	\gettikzxy{(EP)}{\slopex}{\slopey}
%
\makeatletter
\newcommand{\gettikzxy}[3]{%
  \tikz@scan@one@point\pgfutil@firstofone#1\relax
  \edef#2{\the\pgf@x}%
  \edef#3{\the\pgf@y}%
}
\makeatother
%
%
%	Watermarks
%
% \usepackage{draftwatermark}
% \SetWatermarkText{Draft}
% \SetWatermarkScale{5}
% \SetWatermarkLightness {0.9} 
% \SetWatermarkColor[rgb]{0.7,0,0}
%
%
%	theorems, definitions, etc
%
\theoremstyle{definition}
\newtheorem{theorem}{Theorem}[section]
\newtheorem{definition}{Definition}[section]
\newtheorem{proposition}{Proposition}[section]
\newtheorem{lemma}{Lemma}[section]
\newtheorem{example}{Example}[section]
\newtheorem{remark}{Remark}[section]
%
%	For drawing matrix products 
%
\newcommand*{\vertbar}{\rule[-1ex]{0.5pt}{2.5ex}}
\newcommand*{\horzbar}{\rule[.5ex]{2.5ex}{0.5pt}}

%
%	The following code allows you to do
%
%	\begin{bmatrix}[r] (or [c] or [l])
%
\makeatletter
\renewcommand*\env@matrix[1][c]{\hskip -\arraycolsep
  \let\@ifnextchar\new@ifnextchar
  \array{*\c@MaxMatrixCols #1}}
\makeatother
%
%	make \arg{min,max}_{n \to \infty} work nicely
%
\newcommand{\argmax}{\operatornamewithlimits{argmax}}
\newcommand{\argmin}{\operatornamewithlimits{argmin}}
%
%	handy commands
%
\newcommand*{\Scale}[2][4]{\scalebox{#1}{$#2$}}
\DeclareMathOperator{\E}{\mathbb{E}}
\DeclareMathOperator{\bda}{\Big \downarrow}						% big down arrow
\newcommand{\veq}{\mathrel{\rotatebox{90}{$=$}}}
%
%	Title, author and date
%
\title{Pascal's Triangle Tests: This blows up when $\backslash\text{nrows} > 7$}
\author{David Meyer \\ \href{mailto:dmm613@gmail.com}
                            {dmm613@gmail.com}}
\date{Last update: \today}
%
%
%
\begin{document}
\maketitle
%
%
%
\section{Introduction}
\label{sec:introduction}

{\footnotesize
\begin{verbatim}
%
%	I started looking at this because the below code
%	blows up when \nrows > 7.
%
%
\def \nrows {7}                                                                         % number of rows
\pgfmathtruncatemacro \rows {\nrows + 1}                                                % we count from zero
%                                                                                       %
%       Draw the picture, wrapped in a figure                                           % draw the picture
%                                                                                       %
\begin{figure}[H]                                                                       % build the figure
  \centering                                                                            % center everything
  \resizebox{0.50 \textwidth}{!} {                                                      % resize figure if you want
    \fbox {                                                                             % put a box around the figure
       \begin{tikzpicture}[rotate=-90]                                                  % draw the triangle
         \foreach \x in {0,1,...,\nrows} {                                              % loop over the rows
            \foreach \y in {0,...,\x} {                                                 % loop over the columns
               \pgfmathsetmacro \binom {factorial(\x)/(factorial(\y)*factorial(\x-\y))}	% calculate the value
               \pgfmathsetmacro \shift {\x/2}                                          	%
               \node[xshift=-\shift cm] at (\x,\y) {\pgfmathprintnumber \binom};        % draw the node here
            }                                                                           % \end \foreach \x
         }                                                                              % \end \foreach \y
       \end{tikzpicture}                                                                %
    }                                                                                   % end \fbox
  }                                                                                     % end \resizebox
  \caption{Pascal's triangle with $\rows$ rows}                                         % \caption
  \label{fig:pascals_triangle}                                                          % \label{}
\end{figure}
\end{verbatim}
}

\smallskip
%
%
%	Parameters
%
%	I started looking at this because the below code
%	blows up when \nrows > 7.
%
%
\def \nrows {7}                                                                         % number of rows; this blows up when \nrows > 7
\pgfmathtruncatemacro \rows {\nrows + 1}                                                % we count from zero
%                                                                                       %
%       Draw the picture, wrapped in a figure                                           % draw the picture, wrapped in a figure
%                                                                                       %
\begin{figure}[H]                                                                       % build the figure
  \centering                                                                            % center everything
  \resizebox{0.35 \textwidth}{!} {                                                      % resize figure if you want
    \fbox {                                                                             % put a box around the figure
       \begin{tikzpicture}[rotate=-90]                                                  % draw the triangle
         \foreach \x in {0,1,...,\nrows} {                                              % loop over the rows
            \foreach \y in {0,...,\x} {                                                 % loop over the columns
               \pgfmathsetmacro \binom {factorial(\x)/(factorial(\y)*factorial(\x-\y))}	% calculate the value
               \pgfmathsetmacro \shift {\x/2}                                          	%
               \node[xshift=-\shift cm] at (\x,\y) {\pgfmathprintnumber \binom};        % draw the node here
            }                                                                           % \end \foreach \x
         }                                                                              % \end \foreach \y
       \end{tikzpicture}                                                                %
    }                                                                                   % end \fbox
  }                                                                                     % end \resizebox
  \caption{Pascal's triangle with $\rows$ rows}                                         % \caption
  \label{fig:pascals_triangle}                                                          % \label{}
\end{figure}



%
%
%	show a derivation/proof
%
% \begin{equation*}
% \begin{array}{llll}
% x
% &=& y				&\qquad \mathrel{\#} \text{comment or} \\
% &=& y				&\hspace{2em} \mathrel{\#} \text{comment} \\
% [10pt]
% \end{array}
% \end{equation*}
%
%
%
%	include an image
%
% \begin{figure}
% \center{\includegraphics[frame, scale=0.50] {images/X.png}}
% \caption{X}
% \label{fig:X}
% \end{figure}
%
%	How to put a box around a figure
%
% \begin{figure}[H]
%  \centering
%   \fbox{\begin{minipage}{34em}
%    \begin{equation*}
%     \begin{array}{rlrlrlr}
%       9^2      &=& 81           &\longrightarrow& 8 + 1           &=& 9	  \\
%       45^2     &=& 2025         &\longrightarrow& 20 + 25         &=& 45     \\
%       703^2    &=& 494209       &\longrightarrow& 494 + 209       &=& 703    \\
%       7777^2   &=& 60481729     &\longrightarrow& 6048 + 1729     &=& 7777   \\
%       857143^2 &=& 734694122449 &\longrightarrow& 734694 + 122449 &=& 857143 \\
%     \end{array}
%    \end{equation*}
%  \end{minipage}}
%  \caption{A Few Example Kaprekar Numbers}
%\end{figure}
%
%	
%	How to wrap (and resize) your tikzpicture in a figure
%
% \begin{figure}[H]
% \centering
%  \resizebox{0.40 \textwidth}{!} {		% resize the figure if you want
%    \begin{tikzpicture} 
%     .....
%    \end{tikzpicture}					% end tikzpicture
%  }									% end resizebox
% \caption{Some Figure}
% \label{fig:some_figure}
% \end{figure}
%
%
%
% \section{Conclusions}
% \label{sec:conclusions}
%
%
%
\vspace{0.50 cm}
%
%
\section{Acknowledgements}
\label{sec:acknowledgements}

\begin{verbatim}
From @psu_13@mathstodon.xyz: 

@dmm you are probably overflowing LaTeX's poor macro engine... 
which was never built to do this kind of heavy lifting.

if you make a small file with just this in it

\tikzmath{
    \x1 = factorial(8)/(factorial(7)*factorial(8-7));
}

you get an arithmetic overflow error when you run TeX.
\end{verbatim}

\bigskip
\noindent
Expanding a bit on @psu\_13@mathstodon.xyz's example, 
note that 

\begin{verbatim}
    \tikzmath {factorial(8);} 
\end{verbatim}

\noindent
is enough to get LaTeX to throw an arithmetic overflow error,
while 

\begin{verbatim}
    \tikzmath {factorial(7);} 
\end{verbatim}

\noindent
does not throw an error.

\bigskip
\noindent
In addition, Steve VanDevender pointed out that 7! = 5040 and 
8! = 40320, and that $40320 > 32767$, the largest number you 
can fit into a 16-bit integer. TeX, due to its age, could 
be using 16-bit integers, which would explain what we're
seeing above.
%
%	LaTeX source on overleaf.com
%
\section*{\LaTeX \hspace{0.025 mm} Source}
\url{https://www.overleaf.com/read/tgkjgjnjhfjm}
%
%	get a bibliography
%
%	Note:.bib files go in ~/Library/texmf/bibtex/bib with TeXShop (MacTeX).
%	You can also use an absolute path, e.g. \bibliography{/Users/dmm/papers/bib/qc}
%
% \bibliographystyle{plain}
% \bibliography{qc,ml}
%
%	done
%
\end{document} 

