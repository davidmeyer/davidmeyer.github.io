% !TEX TS-program = LuaLaTeX
\documentclass{article}
%
%
%	LaTeX template
%
%	David Meyer
%	dmm613@gmail.com
%	08 Oct 2023
%
%
%   get various packages
%
\usepackage[margin=1.0in]{geometry}                                     % adjust margins
\geometry{letterpaper}                                                  % or a4paper or a5paper or ... 
\usepackage{url}                                                        % need this to use URLs in bibtex
\usepackage{setspace}                                                   % need this for \setstrech{...}
\usepackage{scrextend}                                                  % need this for addmargin
\usepackage[export]{adjustbox}                                          % need this to get frame for includegraphics
%
%   tikz et al
%
\usepackage{tikz}
\usetikzlibrary{calc,patterns,angles,quotes,shapes,math,decorations,
                through,intersections,lindenmayersystems,backgrounds,
                hobby}
\tikzset{>=latex}                                                       % default to LaTeX arrow head
\usepackage{circuitikz}                                                 % draw circuits    
\usepackage{pgfplots}
%
%	more math stuff
%
\usepackage{amsmath,amsfonts,amssymb,amsthm}
\usepackage{bm}
\usepackage{mathtools}
\usepackage{commath}                                                    % get \norm{x}
\usepackage{fixmath}                                                    % get \mathbold
\usepackage{gensymb}                                                    % get \degree
\usepackage{mathrsfs}
\usepackage{hyperref}
\usepackage{subcaption}
\usepackage{authblk}													% comment out if using beamer (stops \author{} from working in beamer)
\usepackage{graphicx}
\usepackage{hyperref}
\usepackage{alltt}
\usepackage{xcolor}
\usepackage{float}
\usepackage{braket}
\usepackage{siunitx}
\usepackage{relsize}
\usepackage{multirow}
\usepackage{esvect}
\usepackage{enumitem}													% use characters instead of numbers in enumerate
\usepackage{changepage}													% needed for \begin{adjustwidth}{-3.25em}{-2.0em} (left justify)
%
%	lualatex
%
%	Put this before \documentclass
%
%	!TEX TS-program = LuaLaTeX
%
%
\usepackage{luatex85,luamplib}
% \usepackage{luacode}
\mplibnumbersystem{double}
\mplibtextextlabel{enable}
%
%
%
%	Describe floating point parameters, \fpeval
%
\ExplSyntaxOn
\cs_set_eq:NN \fpeval \fp_eval:n
\ExplSyntaxOff
%
%	Get the x and y components out of a coordinate, e.g.
%
%	\coordinate (EP) at (8,5);
%	\gettikzxy{(EP)}{\slopex}{\slopey}
%
\makeatletter
\newcommand{\gettikzxy}[3]{%
  \tikz@scan@one@point\pgfutil@firstofone#1\relax
  \edef#2{\the\pgf@x}%
  \edef#3{\the\pgf@y}%
}
\makeatother
%
%
%	Watermarks
%
% \usepackage{draftwatermark}
% \SetWatermarkText{Draft}
% \SetWatermarkScale{5}
% \SetWatermarkLightness {0.9} 
% \SetWatermarkColor[rgb]{0.7,0,0}
%
%
%	theorems, definitions, etc
%
\theoremstyle{definition}
\newtheorem{theorem}{Theorem}[section]
\newtheorem{definition}{Definition}[section]
\newtheorem{proposition}{Proposition}[section]
\newtheorem{lemma}{Lemma}[section]
\newtheorem{example}{Example}[section]
\newtheorem{remark}{Remark}[section]
%
%	For drawing matrix products 
%
\newcommand*{\vertbar}{\rule[-1ex]{0.5pt}{2.5ex}}
\newcommand*{\horzbar}{\rule[.5ex]{2.5ex}{0.5pt}}

%
%	The following code allows you to do
%
%	\begin{bmatrix}[r] (or [c] or [l])
%
\makeatletter
\renewcommand*\env@matrix[1][c]{\hskip -\arraycolsep
  \let\@ifnextchar\new@ifnextchar
  \array{*\c@MaxMatrixCols #1}}
\makeatother
%
%	make \arg{min,max}_{n \to \infty} work nicely
%
\newcommand{\argmax}{\operatornamewithlimits{argmax}}
\newcommand{\argmin}{\operatornamewithlimits{argmin}}
%
%	handy commands
%
\newcommand*{\Scale}[2][4]{\scalebox{#1}{$#2$}}
\DeclareMathOperator{\E}{\mathbb{E}}
\DeclareMathOperator{\bda}{\Big \downarrow}						% big down arrow
\newcommand{\veq}{\mathrel{\rotatebox{90}{$=$}}}

%
%	Title, author and date
%
\title{template.tex}
\author{David Meyer \\ \href{mailto:dmm613@gmail.com}
                            {dmm613@gmail.com}}
\date{Last update: \today}
%
%
%
\begin{document}
\maketitle
%
%
%
\section{Introduction}
\label{sec:introduction}

%
%	Draw Pascal's Triangle with binomial coefficients. The triangular numbers
%	are shown in christmasgreen.
%

%
%       Parameters
%
\newcommand \nrows {14}                                                 % number of rows
%
%
%	Draw the picture
%
\begin{figure}[H]                                                       % build the triangle
  \centering                                                            % center everything
   \resizebox{0.75 \textwidth}{!} {                                     % resize the figure if you want
    \begin{tikzpicture} [framed,scale=0.90]                             % put a frame on the tikzpicture 
       \foreach \n in {0,...,\nrows} {                                  % iterate over rows
          \foreach \k in {0,...,\n} {                                   % iterate over columns
             \node at (\k-\n/2,-\n) {                                   % put the coefficient here
             \ifnum \k = 2 {\color{blue} ${\binom{\n}{\k}}$}            % k = 2 => triangular number
                \else                                                   % else
                  \ifnum \k = 3 {\color{red}${\binom{\n}{\k}}$}         % k =3 => tetrahedral number
                    \else {\color{black}${\binom{\n}{\k}}$}             % default to black
                   \fi                                                  % end \ifnum \k = 3
              \fi};                                                     % end \ifnum \k = 2
          }                                                             % end \foreach \k ...
        }                                                               % end \foreach \n ...
    \end{tikzpicture}                                                   % end tikzpicture
   }                                                                    % end resizebox
  \caption{Pascal's Triangle with Binomial Coefficients}
  \label{fig:pascals_triangle_with_binomial_coefficients}
\end{figure}


%
%       Parameters
%
\def \nrows {14}
%
%	Build the figure
%
\begin{figure}[H]
  \centering
  \fbox {                                                                       % put a box around the figure
    \resizebox{0.75 \textwidth}{!} {                                            % resize the figure if you want
    \begin{mplibcode}                                                           %
        vardef Pascal_triangle(expr n, u, v) =                                  % define the function
        save b; numeric b[][]; clearxy;
        % defaultscale := 1.60;
        defaultscale := 16pt/fontsize bold;
        b[0][0] = 1; b[0][1] = 0; label("1", origin);                           % special case the "ones"
        for i = 1 upto n:                                                       % loop over the rows
          x := -u*i/2; y := -i*v;                                               % draw the result at (x,y)
          b[i][0] = 1; label("1", z);                                           % need one more "one"
          for k = 1 upto i:                                                     % loop over the columnss
              x := x + u;                                                       % coordinates
              b[i][k] = b[i-1][k-1] + b[i-1][k]; 
              if k = 2:    
                label(decimal(b[i][k]),z) withcolor blue;                       % triangular numbers 
              elseif k=3:                                                       % 
                label(decimal(b[i][k]),z) withcolor red;                  		% tetrahedral numbers in red
  		      else:                                                           	% default
  		        label(decimal(b[i][k]),z) withcolor black;                  	% default to black
              fi
           endfor 
           b[i][i+1]=0;
         endfor
       enddef;
%
%	Draw the triangle
%
    beginfig(1);
      Pascal_triangle(\nrows, 1.4cm, 1cm);
    endfig;
   \end{mplibcode}
   }                                                                            % end resizebox
 }                                                                              % end fbox
 \caption{Pascal's Triangle}           											% caption the figure
 \label{fig:pascals_triangle}                                                   % label the figure
\end{figure}                                                                    % end figure




\begin{figure}[H]
  \centering
  \fbox {                                                                       % put a box around the figure
    \resizebox{0.75 \textwidth}{!} {                                              % resize the figure if you want
      \begin{math}
\begin{array}{c}
1 \\
1 \qquad 1 \\ \noalign{\smallskip\smallskip}
1 \qquad 2  \qquad {\bf \color{blue}{1}} \\ \noalign{\smallskip\smallskip}
1 \qquad 3  \qquad {\bf \color{blue}{3}}  \qquad {\bf \color{red}{1}} \\ \noalign{\smallskip\smallskip}
1 \qquad 4  \qquad {\bf \color{blue}{6}}  \qquad {\bf \color{red}{4}}   \qquad 1 \\ \noalign{\smallskip\smallskip}
1 \qquad 5  \qquad {\bf \color{blue}{10}} \qquad {\bf \color{red}{10}}  \qquad 5    \qquad 1 \\ \noalign{\smallskip\smallskip}
1 \qquad 6  \qquad {\bf \color{blue}{15}} \qquad {\bf \color{red}{20}}  \qquad 15   \qquad 6   \qquad 1 \\ \noalign{\smallskip\smallskip}
1 \qquad 7  \qquad {\bf \color{blue}{21}} \qquad {\bf \color{red}{35}}  \qquad 35   \qquad 21  \qquad 7   \qquad 1 \\ \noalign{\smallskip\smallskip}
1 \qquad 8  \qquad {\bf \color{blue}{28}} \qquad {\bf \color{red}{56}}  \qquad 70   \qquad 56  \qquad 28  \qquad 8       \qquad 1 \\ \noalign{\smallskip\smallskip}
1 \qquad 9  \qquad {\bf \color{blue}{36}} \qquad {\bf \color{red}{84}}  \qquad 126  \qquad 126 \qquad 84  \qquad 36     \qquad 9   \qquad 1 \\ \noalign{\smallskip\smallskip}
1 \qquad 10 \qquad {\bf \color{blue}{45}} \qquad {\bf \color{red}{120}} \qquad 210  \qquad 252 \qquad 210 \qquad 120    \qquad 45  \qquad 10 \qquad 1 \\ \noalign{\smallskip\smallskip}
1 \qquad 11 \qquad {\bf \color{blue}{55}} \qquad {\bf \color{red}{165}} \qquad 330  \qquad 462 \qquad 462 \qquad 330    \qquad 165 \qquad 55 \qquad 11 \qquad 1 \\ \noalign{\smallskip\smallskip}
1 \qquad 12 \qquad {\bf \color{blue}{66}} \qquad {\bf \color{red}{220}} \qquad 495  \qquad 792 \qquad 924 \qquad 792    \qquad 495 \qquad 220 \qquad 66 \qquad 12 \qquad 1 \\ \noalign{\smallskip\smallskip}
1 \qquad 13 \qquad {\bf \color{blue}{78}} \qquad {\bf \color{red}{286}} \qquad 715  \qquad 1287 \qquad 1716 \qquad 1716 \qquad 1287 \qquad 715 \qquad 286 \qquad 78 \qquad 13 \qquad 1 \\ \noalign{\smallskip\smallskip}
1 \qquad 14 \qquad {\bf \color{blue}{91}} \qquad {\bf \color{red}{364}} \qquad 1001 \qquad 2002 \qquad 3003 \qquad 3432 \qquad 3003 \qquad 2002 \qquad 1001 \qquad 364 \qquad 91 \qquad 14 \qquad 1
\end{array}
     \end{math}
   }
  }
  \caption{Pascal's Triangle}
  \label{fig:pascals_triangle}
\end{figure}



%
%
%	show a derivation/proof
%
% \begin{equation*}
% \begin{array}{llll}
% x
% &=& y				&\hspace{2em} \mathrel{\#} \text{comment or} \\
% &=& y				&\hspace{2em} \mathrel{\#} \text{comment} \\
% [10pt]
% \end{array}
% \end{equation*}
%
%
%
%	include an image
%
% \begin{figure}
% \center{\includegraphics[frame, scale=0.50] {images/X.png}}
% \caption{X}
% \label{fig:X}
% \end{figure}
%
%	How to put a box around a figure
%
% \begin{figure}[H]
%  \centering
%   \fbox{\begin{minipage}{34em}
%    \begin{equation*}
%     \begin{array}{rlrlrlr}
%       9^2      &=& 81           &\longrightarrow& 8 + 1           &=& 9	  \\
%       45^2     &=& 2025         &\longrightarrow& 20 + 25         &=& 45     \\
%       703^2    &=& 494209       &\longrightarrow& 494 + 209       &=& 703    \\
%       7777^2   &=& 60481729     &\longrightarrow& 6048 + 1729     &=& 7777   \\
%       857143^2 &=& 734694122449 &\longrightarrow& 734694 + 122449 &=& 857143 \\
%     \end{array}
%    \end{equation*}
%  \end{minipage}}
%  \caption{A Few Example Kaprekar Numbers}
%\end{figure}
%
%	
%	How to wrap (and resize) your tikzpicture in a figure
%
% \begin{figure}[H]
% \centering
%  \resizebox{0.40 \textwidth}{!} {		% resize the figure if you want
%    \begin{tikzpicture} 
%     .....
%    \end{tikzpicture}					% end tikzpicture
%  }									% end resizebox
% \caption{Some Figure}
% \label{fig:some_figure}
% \end{figure}
%
%
%
\section{Conclusions}
\label{sec:conclusions}
%
%
%
\section{Acknowledgements}
\label{sec:acknowledgements}
%
%	LaTeX source on overleaf.com
%
\section*{\LaTeX \hspace{0.025 mm} Source}
\url{https://www.overleaf.com/read/jyppzhgdtrvg}
%
%	get a bibliography
%
%	Note:.bib files go in ~/Library/texmf/bibtex/bib with TeXShop (MacTeX).
%	You can also use an absolute path, e.g. \bibliography{/Users/dmm/papers/bib/qc}
%
\bibliographystyle{plain}
\bibliography{qc,ml}
%
%
%
% \section*{Appendix A}
%
%	done
%
\end{document} 

